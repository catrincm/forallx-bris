To do this, we provide a \define{symbolization key}, such as the following:
	\begin{ekey}
		\item[A] It is raining outside
		\item[C] Jenny is miserable
	\end{ekey}
In doing this, we are not fixing this symbolization \emph{once and for all}.

\section{Biconditional}
Consider these sentences:
	\begin{earg}
		\item[\ex{iff1}] Laika is a dog only if she is a mammal
		\item[\ex{iff2}] Laika is a dog if she is a mammal
		\item[\ex{iff3}] Laika is a dog if and only if she is a mammal
	\end{earg}
We will use the following symbolization key:
	\begin{ekey}
		\item[D] Laika is a dog
		\item[M] Laika is a mammal
	\end{ekey}
Sentence \ref{iff1}, for reasons discussed above, can be symbolized by $D \eif M$.

Sentence \ref{iff2} is importantly different. It can be paraphrased as, `If Laika is a mammal then Laika is a dog'. So it can be symbolized by $M \eif D$.

Sentence \ref{iff3} says something stronger than either \ref{iff1} or \ref{iff2}. It can be paraphrased as `Laika is a dog if Laika is a mammal, and Laika is a dog only if Laika is a mammal'. This is just the conjunction of sentences \ref{iff1} and \ref{iff2}. So we can symbolize it as $((D \eif M) \eand (M \eif D))$. We call this a \define{biconditional}, because it entails the conditional in both directions.

\newglossaryentry{biconditional}
{
name=biconditional,
description={The symbol \eiff, used to represent words and phrases that function like the English phrase ``if and only if''; or a sentence formed using this connective.}
}

We could treat every biconditional this way. So, just as we do not need a new TFL symbol to deal with \emph{exclusive or}, we do not really need a new TFL symbol to deal with biconditionals. Because the biconditional occurs so often, however, we will use the symbol `\eiff' for it. We can then symbolize sentence \ref{iff3} with the TFL sentence $(D \eiff M)$.

The expression `if and only if' occurs a lot especially in philosophy, mathematics, and logic. For brevity, we can abbreviate it with the snappier word `iff'. We will follow this practice. So `if' with only \emph{one} `f' is the English conditional. But `iff' with \emph{two} `f's is the English biconditional. Armed with this we can say:
	\factoidbox{
If a sentence can be paraphrased as `\metaX if and only if \metaY'\\ it can be symbolised as $\metaX\eiff\metaY$.
	}
A word of caution. Ordinary speakers of English often use `if \ldots, then\ldots' when they really mean to use something more like `\ldots if and only if \ldots'. Perhaps your parents told you, when you were a child: `if you don't eat your greens, you won't get any dessert'. Suppose you ate your greens, but that your parents refused to give you any dessert, on the grounds that they were only committed to the \emph{conditional} (roughly `if you get dessert, then you will have eaten your greens'), rather than the biconditional (roughly, `you get dessert iff you eat your greens'). Well, a tantrum would rightly ensue. So, be aware of this when interpreting people; but in your own writing, make sure you use the biconditional iff you mean to.

\chapter{Ambiguity}\label{s:AbmbiguityTFL}

In English, sentences can be \define{ambiguous}, i.e., they can have more than one meaning.  There are many sources of ambiguity. One is \emph{lexical ambiguity:} a sentence can contain words which have more than one meaning.  For instance, `bank' can mean the bank of a river, or a financial institution. So I might say that `I went to the bank' when I took a stroll along the river, or when I went to deposit a check.  Depending on the situation, a different meaning of `bank' is intended, and so the sentence, when uttered in these different contexts, expresses different meanings.

A different kind of ambiguity is \emph{structural ambiguity}.  This arises when a sentence can be interpreted in different ways, and depending on the interpretation, a different meaning is selected.  A famous example due to Noam Chomsky is the following:
\begin{earg}
	\prem Flying planes can be dangerous.
\end{earg}
There is one reading in which `flying' is used as an adjective which modifies `planes'. In this sense, what's claimed to be dangerous are airplanes which are in the process of flying.  In another reading, `flying' is a gerund: what's claimed to be dangerous is the act of flying a plane.  In the first case, you might use the sentence to warn someone who's about to launch a hot air baloon.  In the second case, you might use it to counsel someone against becoming a pilot.

When the sentence is uttered, usually only one meaning is intended. Which of the possible meanings an utterance of a sentence intends is determined by context, or sometimes by how it is uttered (which parts of the sentence are stressed, for instance). Often one interpretation is much more likely to be intended, and in that case it will even be difficult to ``see'' the unintended reading.  This is often the reason why a joke works, as in this example from Groucho Marx:
\begin{earg}
	\prem One morning I shot an elephant in my pajamas.
	\prem How he got in my pajamas, I don't know.
\end{earg}

Ambiguity is related to, but not the same as, vagueness. An adjective, for instance `rich' or `tall,' is \define{vague} when it is not always possible to determine if it applies or not.  For instance, a person who's 6~ft 4~in (1.9~m) tall is pretty clearly tall, but a building that size is tiny.  Here, context has a role to play in determining what the clear cases and clear non-cases are (`tall for a person,' `tall for a basketball player,' `tall for a building'). Even when the context is clear, however, there will still be cases that fall in a middle range.

In TFL, we generally aim to avoid ambiguity. We will try to give our symbolization keys in such a way that they do not use ambiguous words or  disambiguate them if a word has different meanings. So, e.g., your symbolization key will need two different sentence letters for `Rebecca went to the (money) bank' and `Rebecca went to the (river) bank.' Vagueness is harder to avoid. Since we have stipulated that every case (and later, every valuation) must make every basic sentence (or sentence letter) either true or false and nothing in between, we cannot accommodate borderline cases in TFL.

It is an important feature of sentences of TFL that they \emph{cannot} be structurally ambiguous. Every sentence of TFL can be read in one, and only one, way. This feature of TFL is also a strength. If an English sentence is ambiguous, TFL can help us make clear what the different meanings are.  Although we are pretty good at dealing with ambiguity in everyday conversation, avoiding it can sometimes be terribly important. Logic can then be usefully applied: it helps philosopher express their thoughts clearly, mathematicians to state their theorems rigorously, and software engineers to specify loop conditions, database queries, or verification criteria unambiguously.

Stating things without ambiguity is of crucial importance in the law as well. Here, ambiguity can, without exaggeration, be a matter of life and death. Here is a famous example of where a death sentence hinged on the interpretation of an ambiguity in the law. Roger Casement (1864--1916) was a British diplomat who was famous in his time for publicizing human-rights violations in the Congo and Peru (for which he was knighted in 1911). He was also an Irish nationalist. In 1914--16, Casement secretly travelled to Germany, with which Britain was at war at the time, and tried to recruit Irish prisoners of war to fight against Britain and for Irish independence. Upon his return to Ireland, he was captured by the British and tried for high treason.

The law under which Casement was tried is the \emph{Treason Act of 1351}. That act specifies what counts as treason, and so the prosecution had to establish at trial that Casement's actions met the criteria set forth in the Treason Act. The relevant passage stipulated that someone is guilty of treason
\begin{quote}
	if a man is adherent to the King's enemies in his
realm, giving to them aid and comfort in the realm, or elsewhere.
\end{quote}
Casement's defense hinged on the last comma in this sentence, which is not present in the original French text of the law from 1351.  It was not under dispute that Casement had been `adherent to the King's enemies', but the question was whether being adherent to the King's enemies constituted treason only when it was done in the realm, or also when it was done abroad. The defense argued that the law was ambiguous. The claimed ambiguity hinged on whether `or elsewhere' attaches only to `giving aid and comfort to the King's enemies' (the natural reading without the comma), or to both `being adherent to the King's enemies' and `giving aid and comfort to the King's enemies' (the natural reading with the comma).  Although the former interpretation might seem far fetched, the argument in its favor was actually not unpersuasive. Nevertheless, the court decided that the passage should be read with the comma, so Casement's antics in Germany were treasonous, and he was sentenced to death. Casement himself wrote that he was `hanged by a comma'.

We can use TFL to symbolize both readings of the passage, and thus to provide a disambiguiation. First, we need a symbolization key:
\begin{ekey}
	\item[A] Casement was adherent to the King's enemies in the realm.
	\item[G] Casement gave aid and comfort to the King's enemies in the realm.
	\item[B] Casement was adherent to the King's enemies abroad.
	\item[H] Casement gave aid and comfort to the King's enemies abroad.
\end{ekey}
The interpretation according to which Casement's behavior was not treasonous is this:
\begin{earg}
	\prem $A \lor (G \lor H)$
\end{earg}
The interpretation which got him executed, on the other hand, can be symbolized by:
\begin{earg}
	\prem $(A \lor B) \lor (G \lor H)$
\end{earg}
Remember that in the case we're dealing with Casement, was adherent to the King's enemies abroad ($B$ is true), but not in the realm, and he did not give the King's enemies aid or comfort in or outside the realm ($A$, $G$, and~$H$ are false).

One common source of structural ambiguity in English arises from its lack of parentheses. For instance, if I say `I like movies that are not long and boring', you will most likely think that what I dislike are movies that are long and boring. A less likely, but possible, interpretation is that I like movies that are both (a) not long and (b) boring. The first reading is more likely because who likes boring movies? But what about `I like dishes that are not sweet and flavorful'? Here, the more likely interpretation is that I like savory, flavorful dishes.  (Of course, I could have said that better, e.g., `I like dishes that are not sweet, yet flavorful'.) Similar ambiguities result from the interaction of `and' with `or'. For instance, suppose I ask you to send me a picture of a small and dangerous or stealthy animal.  Would a leopard count? It's stealthy, but not small. So it depends whether I'm looking for small animals that are dangerous or stealthy (leopard doesn't count), or whether I'm after either a small, dangerous animal or a stealthy animal (of any size).

These kinds of ambiguities are called \emph{scope ambiguities}, since they depend on whether or not a connective is in the scope of another. For instance, the sentence, `\emph{Avengers: Endgame} is not long and boring' is ambiguous between:
\begin{earg}
	\item[\ex{scamb1}] \emph{Avengers: Endgame} is not: both long and boring.
	\item[\ex{scamb2}] \emph{Avengers: Endgame} is both: not long and boring.
\end{earg}
Sentence~\ref{scamb2} is certainly false, since \emph{Avengers: Endgame} is over three hours long. Whether you think~\ref{scamb1} is true depends on if you think it is boring or not. We can use the symbolization key:
\begin{ekey}
	\item[B] \emph{Avengers: Endgame} is boring.
	\item[L] \emph{Avengers: Endgame} is long.
\end{ekey}
Sentence~\ref{scamb1} can now be symbolized as `$\enot(L \eand B)$', whereas sentence~\ref{scamb2} would be `$\enot L \eand B$'. In the first case, the `\eand' is in the scope of `\enot', in the second case `\enot' is in the scope of `\eand'.

The sentence `Tai Lung is small and dangerous or stealthy' is ambiguous between:
\begin{earg}
	\item[\ex{scamb3}] Tai Lung is either both small and dangerous or stealthy.
	\item[\ex{scamb4}] Tai Lung is both small and either dangerous or stealthy.
\end{earg}
We can use the following symbolization key:
\begin{ekey}
	\item[D] Tai Lung is dangerous.
	\item[S] Tai Lung is small.
	\item[T] Tai Lung is stealthy.
\end{ekey}
The symbolization of sentence~\ref{scamb3} is `$(S \eand D) \eor T$' and that of sentence~\ref{scamb4} is `$S \eand (D \eor T)$'. In the first, \eand is in the scope of \eor, and in the second \eor is in the scope of \eand.

\begin{practiceproblems}
\solutions
\problempart The following sentences are ambiguous. Give symbolization keys for each and symbolize the different readings.
\begin{earg}
	\item Haskell is a birder and enjoys watching cranes.
	\item The zoo has lions or tigers and bears.
	\item The flower is not red or fragrant.
\end{earg}
\end{practiceproblems}

\chapter{Symbolising complex sentences}\label{s:SymbolisingComplexTFL}
In \S\ref{s:TFLConnectives} we discussed how to symbolise sentences. But we mostly focused on simple sentences. Sentences of English, though, might end up having much more complex structure.

