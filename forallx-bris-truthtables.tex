%!TEX root = forallxbris.tex
\part{Truth tables}
\label{ch.TruthTables}
\addtocontents{toc}{\protect\mbox{}\protect\hrulefill\par}

\chapter{Truth rules for the connectives of TFL}
\label{s:CharacteristicTruthTables}

We now move to looking at when sentences of TFL are true or false. We will give precise rules for determining this. The important feature of truth functional logic is that the truth value of a complex sentence, such as $A\eor(B\eand C)$ is determined just by the truths of is component parts, that is $A$, $B$ and $C$. If we're told whether $A$, $B$ and $C$ are true or false, then we will be able to say whether $A\eor(B\eand C)$ is true or false.

To be able to do this, we need to describe how the truth values are to be combined. We work through each of our connectives describing the rules governing it.

\newglossaryentry{truth value}
                 {
                   name = truth value,
                   description = {One of the two logical values sentences can have: True and False}
                   }


\section{Negation}

Consider:
\begin{earg}
\item[\ex{neg-f}] Bristol is not in France.
\item[\ex{neg-f}] Bristol is not in England.
\end{earg}
`Bristol is in France' is false, so `Bristol is not in France' is true.
`Bristol is in England' is true, so `Bristol is not in England' is false.

In general, to know whether a sentence of the form $\enot \metaX$ is true. This depends on whether $\metaX$ is true or not in the way:
\begin{itemize}
\item If $\metaX$ is true, then $\enot \metaX$ is false.
\item If $\metaX$ is false, then $\enot \metaX$ is true.
\end{itemize}
We record this in shorthand:
\begin{highlighted}
\begin{center}
\begin{tabular}{ccc}
If \metaX is: &&then \enot \metaX is:\\
T &$\leadsto$& F\\
F &$\leadsto$& T
\end{tabular}
\end{center}
\end{highlighted}
%or even more concisely (to help the memory):
%\begin{highlighted}
%\begin{center}$\enot$ :
%\begin{tabular}{c@{ $\leadsto$ }c}
%T & F\\
%F & T
%\end{tabular}
%\end{center}
%\end{highlighted}
%
%in an easy to read form:
%a \define{characteristic truth table}:
%\newglossaryentry{characteristic truth table}
%                 {
%                   name = characteristic truth table,
%                   description = {A table describing the truth-rules for a connective}
%                   }
%\begin{center}
%\begin{tabular}{c|c}
%If \metaX is:& then \enot\metaX is:\\\hline
%T & F\\
%F & T
%\end{tabular}
%\end{center}
We have abbreviated `True' with `T' and `False' with `F'. (But just to be clear, the two truth values are True and False; the truth values are not \emph{letters}!)



\section{Conjunction}
Recall that $\metaX\eand\metaY$ was used to symbolise `$\metaX$ and $\metaY$'.


Consider:
\begin{earg}
\item[\ex{conj}] She can speak German and French.
\end{earg}
If she can speak German and she can speak French, then this is true, but otherwise it is false.

More generally, the rule governing $\eand$ is:
\begin{itemize}
\item If $\metaX$ and $\metaY$ are both true, then $\metaX\eand\metaY$ is true.
\item Otherwise, $\metaX\eand \metaY$ is false.
\end{itemize}
Which we summarise
\begin{highlighted}
\begin{center}
\begin{tabular}{cccc}
If \metaX is:&and \metaY is:&&then $\metaX\eand\metaY$ is:\\
T & T &$\leadsto$& T\\
T & F &$\leadsto$& F\\
F & T &$\leadsto$& F\\
F & F &$\leadsto$& F
\end{tabular}
\end{center}
\end{highlighted}
%in the characteristic truth table:
%\begin{center}
%\begin{tabular}{c c |c}
%\metaX & \metaY & $\metaX\eand\metaY$\\
%\hline
%T & T & T\\
%T & F & F\\
%F & T & F\\
%F & F & F
%\end{tabular}
%\end{center}
Note that conjunction is \emph{symmetrical}. The truth value for $\metaX \eand \metaY$ is always the same as the truth value for $\metaY \eand \metaX$.

\section{Disjunction}
Disjunction is a bit more subtle.
Consider:
\begin{earg}
\item[\ex{conj}] She can speak German or French.
\end{earg}
If she cannot speak either German or French, then this is false. If she can speak German but not French, then it is true, and if she can speak French but not German it is also true.
We have the general rules:
\begin{itemize}
\item If $\metaX$ and $\metaY$ are both false, then $\metaX\eor\metaY$ is false.
\item  If $\metaX$ is true and $\metaY$ are false, then $\metaX\eor\metaY$ is true.
\item  If $\metaX$ is false and $\metaY$ are true, then $\metaX\eor\metaY$ is true.
\end{itemize}

But what if she can speak both? Is it true or false?
We have already pointed out that in English there are two kinds of disjunctions: an \emph{inclusive} and an \emph{exclusive} one.

For the inclusive or, we might whisper a ``or both'' after it; whereas for the exclusive or, we'd want to whisper a ``but not both'':
\begin{earg}
\item [\ex{inclor}] She speaks German or
French (or both).
\item [\ex{exclor}] She speaks German or
French (but not both).
\end{earg}

In logic there can be no ambiguity. We choose that $\eor$ stands for the \emph{inclusive or}.
That is, we give the final rule:
\begin{itemize}
\item If $\metaX$ and $\metaY$ are both true, then $\metaX\eor\metaY$ is true.
\end{itemize}

So, when doing symbolisations, one should only symbolise a sentence as $\metaX\eor \metaY$ if it is to be read as the \emph{inclusive or}. To symbolise the exclusive or, you need to use the more complex sentence: $(\metaX\eor\metaY)\eand \enot(\metaX\eand\metaY)$, which essentially makes explicit the whispered ``but not both''. Sometimes the English is ambiguous, in which case one should point that out when symbolising and give the two alternative symbolisations.

Consider:
\begin{earg}
\item[\ex{exclor}]She either ate pizza or pasta.
\end{earg}
Maybe this is an exclusive or, though an alternative treatment is to take this to be an inclusive or, but note that there's a implicit, or missing, premise: that she didn't eat both.

To summarise the rules for $\eor$:
%\begin{center}
%\begin{tabular}{c c |c}
%\metaX & \metaY & $\metaX\eor\metaY$\\
%\hline
%T & T & T\\
%T & F & T\\
%F & T & T\\
%F & F & F
%\end{tabular}
%\end{center}
\begin{highlighted}
\begin{center}
\begin{tabular}{cccc}
If \metaX is:&and \metaY is:&&then $\metaX\eor\metaY$ is:\\
T & T &$\leadsto$& T\\
T & F &$\leadsto$& T\\
F & T &$\leadsto$& T\\
F & F &$\leadsto$& F
\end{tabular}
\end{center}
\end{highlighted}
Like conjunction, disjunction is symmetrical.



\section{Conditional}
Suppose you are a bartender considering the conditional
\begin{earg}
\item[\ex{bartender}] If she is drinking a beer, then she is over eighteen.
\end{earg}

Here is what we'll say about this conditional:

\begin{itemize}
\item If she's drinking beer and is sixteen, then it is false. (You should kick her out of the bar.)
\item If she is drinking beer and is nineteen, then it's true.
\item If she is drinking coke, then it's true. (No need to check her age)
\end{itemize}

%We get the characteristic truth table:
%\begin{center}
%\begin{tabular}{c c|c}
%\metaX & \metaY & $\metaX\eif\metaY$\\
%\hline
%T & T & T\\
%T & F & F\\
%F & T & T\\
%F & F & T
%\end{tabular}
%\end{center}
We summarise these rules:
\begin{highlighted}
\begin{center}
\begin{tabular}{cccc}
If \metaX is:&and \metaY is:&&then $\metaX\eif\metaY$ is:\\
T & T &$\leadsto$& T\\
T & F &$\leadsto$& F\\
F & T &$\leadsto$& T\\
F & F &$\leadsto$& T
\end{tabular}
\end{center}
\end{highlighted}
In this case, it's very important to remember which way around it goes. The TF-line is different to the FT-line.

This is why the terms `antecedent' and `consequent' are so useful.  \begin{highlighted}
In $\metaX\eif\metaY$, \metaX is called the \define{antecedent}, and \metaY the \define{consequent}.
\end{highlighted}
We can redescribe this rule:
If the antecedent is true and the consequent false, then the conditional sentence is false, otherwise it is true.
\newglossaryentry{antecedent}
{
name=antecedent,
description={The sentence on the left side of a \gls{conditional}}
}


\newglossaryentry{consequent}
{
name=consequent,
description={The sentence on the right side of a \gls{conditional}}
}

The TFL connective $\eif$ is \emph{stipulated} to be governed by these rules. It is sometimes called the \define{material conditional} to highlight that it is governed by these rules.

\subsection{Subjunctive conditionals}\label{s:IndicativeSubjunctive}
Some English sentences that have the form `if\ldots then\ldots' do not fit with these rules.
Consider
\begin{earg}
\item[\ex{brexit}] If the UK had voted to remain in the EU, then the moon would be made of cheese.
\end{earg}
This has a false antecedent (we voted to exit), and a false consequent (I certainly can't fly). But the whole sentence would be judged to be true. So this doesn't fit with the rule we've given for $\eif$.

There are two kinds of sentences of the form `if\ldots, then\ldots' in English: indicative and subjunctive ones.
Consider the two sentences:
\begin{earg}
\item[\ex{indicative}] If Oswald didn't kill Kennedy, someone else did.
\item[\ex{subjunctive}] If Oswald hadn't killed Kenney, someone else would have.
\end{earg}
The former uses the \define{indicative conditional} and is plausibly true. After all, we know that Kennedy was assassinated. The latter uses the \define{subjunctive conditional} and is plausibly false. Unless there was a conspiracy plot, if Oswald hadn't have killed Kennedy then Kennedy wouldn't have been assassinated. The term \define{counterfactual conditional} is also often used.\footnote{Strictly speaking, we can use subjunctive conditionals with true antecedents, whereas a counterfactual conditional has to have a false antecedent (to be ``counter to the fact'').}

Whilst we use the English `if\ldots then\ldots' for both sentences \ref{indicative} and \ref{subjunctive}, they are very different kinds of conditionals. The conditional of TFL does a terrible job at symbolising the subjunctive conditional which aren't governed by these rules. In fact TFL just doesn't have the resources to consider subjunctive conditionals, they are not ``truth functional''. We will discuss this in the next chapter, but for now we simply not that if you have to symbolise a subjunctive conditional in TFL it is then a matter of judgement whether to do it as $\metaX\eif \metaY$, or to simply use an atomic sentence.

The material conditional does a much better job at symbolising the indicative conditionals, though it is a matter of substantial philosophical debate about exactly how well they do.
We briefly mention some of the worries in \S\ref{s:ParadoxesOfMaterialConditional}.


\section{Biconditional}
Consider
\begin{earg}
\item[\ex{bicond}] Sue is coming to the party if and only if Maya is coming.
\end{earg}
If Sue and Maya are both coming, then this is true. If they're both not coming, then it is also true. If only one of them is coming, then it is false.


%\begin{center}
%\begin{tabular}{c c|c}
%\metaX & \metaY & $\metaX\eiff\metaY$\\
%\hline
%T & T & T\\
%T & F & F\\
%F & T & F\\
%F & F & T
%\end{tabular}
%%\end{center}


\begin{highlighted}
\begin{center}
\begin{tabular}{cccc}
If \metaX is:&and \metaY is:&&then $\metaX\eiff\metaY$ is:\\
T & T &$\leadsto$& T\\
T & F &$\leadsto$& F\\
F & T &$\leadsto$& F\\
F & F &$\leadsto$& T
\end{tabular}
\end{center}

\end{highlighted}

\bigskip
You can see a summary of the rules for all the connectives, for ease of reference, in the `Quick Reference Appendix' \S\ref{app.CharacteristicTTs}

\chapter{Truth-functional connectives}
\label{s:TruthFunctionality}
In this chapter, we reflect on truth-functional logic and the connectives we've used.
\section{Non truth-functional connectives}

Let's introduce an important idea.
	\factoidbox{
		A connective is \define{truth-functional} iff the truth value of a sentence with that connective as its main connective is uniquely determined by the truth value(s) of the constituent sentence(s).
	}
\newglossaryentry{truth-functional connective}
{
name=truth-functional connective,
description={an operator that builds larger sentences out of smaller ones and fixes the \gls{truth value} of the resulting sentence based only on the truth value of the component sentences}
}

Every connective in TFL is truth-functional. We were able to give rules to determine what the truth value of a sentence $\enot\metaX$ is depending only on the truth value of $\metaX$. The truth value of $\metaX$ uniquely determines the truth value of $\enot \metaX$. The same is true for all the other connectives of TFL ($\eand,\eor,\eif,\eiff$). This is what gives TFL its name: it is \emph{truth-functional logic}.

%The truth value of a negation is uniquely determined by the truth value of the unnegated sentence. The truth value of a conjunction is uniquely determined by the truth value of both conjuncts. The truth value of a disjunction is uniquely determined by the truth value of both disjuncts, and so on.

This then means that to determine the truth value of any TFL sentence, we only need to know the truth value of the atomic sentences it includes. We will see exactly how to do this in \S\ref{s:CompleteTruthTables}.


%The truth value of a non-atomic sentence of TFL, such as $A\eand B$ depends on the truth values of $A$ and $B$. But once the truth values of $A$, $B$ and $C$ are provided, the truth value of $A\eand B$ is fixed. Indeed, this is the characteristic feature of \emph{truth functional logic}.


In plenty of languages there are connectives that are not truth-functional. We here describe just a few:

\subsection{Necessarily}

In English, for example, we can form a new sentence from any simpler sentence by prefixing it with `It is necessarily the case that\ldots'. The truth value of this new sentence is not fixed solely by the truth value of the original sentence. For consider two true sentences:
	\begin{earg}
		\item[\ex{nec-math}] $2 + 2 = 4$
		\item[\ex{nec-music}] Shostakovich wrote fifteen string quartets
	\end{earg}
Whereas it is necessarily the case that $2 + 2 = 4$, it is not \emph{necessarily} the case that Shostakovich wrote fifteen string quartets. If Shostakovich had died earlier, he would have failed to finish Quartet no.\ 15; if he had lived longer, he might have written a few more. So `It is necessarily the case that\ldots' is a connective of English, but it is not \emph{truth-functional}.



\subsection{Subjunctive conditionals}\label{s:IndicativeSubjunctive}
%We want to bring home the point that TFL can \emph{only} deal with truth functions by considering the case of the conditional.
%We gave the characteristic truth table for the conditional in \S\ref{s:CharacteristicTruthTables}.
%
%The kind of justification we offered there made an important assumption: that what we said about the bartender sentence could form a general rule. In the bartender case, we said that if the antecedent is false and the consequent false then the conditional is true. (The nineteen year old coke drinker.) And we then concluded that whenever the antecedent and consequent are both false, then the conditional is true. This is required if we are to give a \emph{truth functional conditional}. Our conditional, `$\eif$', is stipulated to be governed by the rules:
%\begin{center}$\eif$ :
%\begin{tabular}{c@{, }c@{ $\leadsto$ }c}
%T & T & T\\
%T & F & F\\
%F & T & T\\
%F & F & T
%\end{tabular}
%\end{center}
%This connective is often more precisely called the \emph{material conditional} (we simply call it the `conditional' as it's the only one we use). It is stipulated to act according to these rules.
%It is the {best} candidate for a truth-functional conditional. Otherwise put, \emph{it is the best conditional that TFL can provide}.


%
%When we introduced the characteristic truth table for the material conditional in \S\ref{s:CharacteristicTruthTables}, we did not say anything to justify it. Let's now offer a justification, which follows Dorothy Edgington.\footnote{Dorothy Edgington, `Conditionals', 2006, in the \emph{Stanford Encyclopedia of Philosophy} (\url{http://plato.stanford.edu/entries/conditionals/}).}
%
%Suppose that Lara has drawn some shapes on a piece of paper, and coloured some of them in. We have not seen them, but nevertheless claim:
%	\begin{quote}
%		If any shape is grey, then that shape is also circular.
%	\end{quote}
%As it happens, Lara has drawn the following:
%\begin{center}
%\begin{tikzpicture}
%	\node[circle, grey_shape] (cat1) {A};
%	\node[right=10pt of cat1, diamond, phantom_shape] (cat2)  { } ;
%	\node[right=10pt of cat2, circle, white_shape] (cat3)  {C} ;
%	\node[right=10pt of cat3, diamond, white_shape] (cat4)  {D};
%\end{tikzpicture}
%\end{center}
%In this case, our claim is surely true.  Shapes C and D are not grey, and so can hardly present \emph{counterexamples} to our claim. Shape A \emph{is} grey, but fortunately it is also circular. So my claim has no counterexamples. It must be true. That means that each of the following \emph{instances} of our claim must be true too:
%	\begin{ebullet}
%		\item If A is grey, then it is circular \hfill (true antecedent, true consequent)
%		\item If C is grey, then it is circular\hfill (false antecedent, true consequent)
%		\item If D is grey, then it is circular \hfill (false antecedent, false consequent)
%	\end{ebullet}
%However, if Lara had drawn a fourth shape, thus:
%\begin{center}
%\begin{tikzpicture}
%	\node[circle, grey_shape] (cat1) {A};
%	\node[right=10pt of cat1, diamond, grey_shape] (cat2)  {B};
%	\node[right=10pt of cat2, circle, white_shape] (cat3)  {C};
%	\node[right=10pt of cat3, diamond, white_shape] (cat4)  {D};
%\end{tikzpicture}
%\end{center}
%then our claim would be false. So it must be that this claim is false:
%	\begin{ebullet}
%		\item If B is grey, then it is circular \hfill (true antecedent, false consequent)
%	\end{ebullet}
%Now, recall that every connective of TFL has to be truth-functional. This means that merely the truth values of the antecedent and consequent must uniquely determine the truth value of the conditional as a whole. Thus, from the truth values of our four claims---which provide us with all possible combinations of truth and falsity in antecedent and consequent---we can read off the truth table for the material conditional.


We said that $\eif$ was pretty bad at capturing \emph{subjunctive conditionals} of English. The problem is that a subjunctive conditional is not truth functional.
Consider the two sentences:
	\begin{earg}
		\item[\ex{brownwins1}] If Mitt Romney had won the 2012 election, then he would have been the 45th President of the USA.
		\item[\ex{brownwins2}] If Mitt Romney had won the 2012 election, then he would have turned into a helium-filled balloon and floated away into the night sky.
	\end{earg}
Sentence \ref{brownwins1} is true; sentence \ref{brownwins2} is false, but both have false antecedents and false consequents. So the truth value of the whole sentence is not uniquely determined by the truth value of the parts.

$\eif$ is the best that can be done at symbolising subjunctive conditionals of English in TFL. TFL just doesn't have the required resources as the subjunctive conditional is not truth functional.


%We have to be a bit more careful when symbolising. Sometimes you cannot adequately symbolize an English `if \dots, then \dots' with TFL's `$\eif$'.
%
%Sentences \ref{brownwins1} and \ref{brownwins2} employ \emph{subjunctive} conditionals, rather than \emph{indicative} conditionals. They ask us to imagine something contrary to fact---Mitt Romney lost the 2012 election---and then ask us to evaluate what \emph{would} have happened in that case. Such considerations just cannot be tackled using `$\eif$'.
%So, $\eif$ is not at all good at symbolising \emph{subjunctive} conditionals, though it is the best that can be done in truth functional logic.
%
%$\eif$ is better at symbolising \emph{indicative} conditionals.
%There is a grammatical difference between the two conditionals. Consider:
%\begin{earg}
%\item[\ex{indicative}] If Oswald didn't kill Kennedy, someone else did.
%\item[\ex{subjunctive}] If Oswald hadn't have killed Kenney, someone else would have.
%\end{earg}
%The former is an \emph{indicative} conditional (and is true; after all, we know he was killed), the latter is \emph{subjunctive} (and is probably false, unless there was a conspiracy plot).
%
%Exactly how well the material conditional does at representing \emph{indicative} conditionals, however, is a philosophically contentious issue, and not one we can delve into in this course.
%\todo{move some of the paradoxes stuff back in... ??}
%%We will say more about the difficulties with conditionals in \S\ref{s:ParadoxesOfMaterialConditional}. For now, we will content ourselves with the observation that `$\eif$' is the only candidate for a truth-functional conditional for TFL, but that many English conditionals cannot be represented adequately using `$\eif$'. TFL is an intrinsically limited language.
%
%%In truth functional logic we work with the \emph{material conditional}, which is stipulated to given by these truth rules. How well it does at capturing the English `if\ldots, then\ldots'.

\section{Symbolizing versus translating}
All of the connectives of TFL are truth-functional, but more than that: they really do nothing \emph{but} map us between truth values.

When we symbolize a sentence or an argument in TFL, we ignore everything \emph{besides} the contribution that the truth values of a component might make to the truth value of the whole. There are subtleties to our ordinary claims that far outstrip their mere truth values. Sarcasm; poetry; snide implicature; emphasis; these are important parts of everyday discourse, but none of this is retained in TFL. As remarked in \S\ref{s:TFLConnectives}, TFL cannot capture the subtle differences between the following English sentences:
	\begin{earg}
		\item Dana is a logician and Dana is a nice person
		\item Although Dana is a logician, Dana is a nice person
		\item Dana is a logician despite being a nice person
		\item Dana is a nice person, but also a logician
		\item Dana's being a logician notwithstanding, he is a nice person
	\end{earg}
All of the above sentences will be symbolized with the same TFL sentence, perhaps `$L \eand N$'.

We keep saying that we use TFL sentences to \emph{symbolize} English sentences. Many other textbooks talk about \emph{translating} English sentences into TFL. However, a good translation should preserve certain facets of meaning, and---as we have just pointed out---TFL just cannot do that. This is why we will speak of \emph{symbolizing} English sentences, rather than of \emph{translating} them.

This affects how we should understand our symbolization keys. Consider a key like:
	\begin{ekey}
		\item[L] Dana is a logician.
		\item[N] Dana is a nice person.
	\end{ekey}
Other textbooks will understand this as a stipulation that the TFL sentence `$L$' should \emph{mean} that Dana is a logician, and that the TFL sentence `$N$' should \emph{mean} that Dana is a nice person, but TFL just is totally unequipped to deal with \emph{meaning}. The preceding symbolization key is doing no more and no less than stipulating that the TFL sentence `$L$' should take the same truth value as the English sentence `Dana is a logician' (whatever that might be), and that the TFL sentence `$N$' should take the same truth value as the English sentence `Dana is a nice person' (whatever that might be).
	\factoidbox{
		When we treat a TFL sentence as \emph{symbolizing} an English sentence, we are stipulating that the TFL sentence is to take the same truth value as that English sentence.
	}



\chapter{Complete truth tables}
\label{s:CompleteTruthTables}

In \S\ref{s:CharacteristicTruthTables} we described how the truth values of two sentences, \metaX and \metaY, should combine to determine the truth of a sentence such as $\metaX\eor\metaY$.

We will now describe how to extend this reasoning to determine the truths of more complex sentences such as $(\enot I\eif H)\eand H$.

So far, we have considered assigning truth values to TFL sentences indirectly. We have said, for example, that a TFL sentence such as `$B$' is to take the same truth value as the English sentence `Ben is happy' (whatever that truth value may be), but we can also assign truth values \emph{directly}. We can simply stipulate that `$B$' is to be true, or stipulate that it is to be false.
	\factoidbox{
		A \define{valuation} is any assignment of truth values to particular atomic sentences of TFL.
	}

        \newglossaryentry{valuation}
{
name=valuation,
description={An assignment of \glspl{truth value} to particular atomic \glspl{sentence of TFL}}
}

We describe how to fill out \define{truth tables}.
Each row of a truth table is a valuation. The entire truth table represents all possible valuations; thus the truth table provides us with a means to calculate the truth values of complex sentences, on each possible valuation. This is easiest to explain by example.

\section{A worked example}

%Consider the sentence `$A\eor B$'.
%We will give a \emph{truth table} which lists all the valuations and says whether the sentence, `$A\eor B$' is true or false on each of them.
%We start by listing the valuations. There are four possible ways to assign True and False to the atomic sentences $A$ and $B$:
%\begin{center}
%\begin{tabular}{c c|d e f}
%$A$&$B$&$A$&\eif&$B$\\
%\hline
% T & T\\
% T & F\\
% F & T\\
% F & F
%\end{tabular}
%\end{center}
%Now, we look at our rule for $\eor$:
%\begin{center}$\eor$ :
%\begin{tabular}{c@{ }c@{ $\leadsto$ }c}
%%1&
%T & T & T\\
%%2&
%T & F & T\\
%%3&
%F & T & T\\
%%4&
% F & F & F
%\end{tabular}
%\end{center}
%And to work out what goes in each blank space in our truth table for $A\eor B$, we just follow the rules given here. In fact here, the lines nicely match up to the different rules to follow, and we can complete the truth table by simply copying it over:
%\begin{center}
%\begin{tabular}{c c|d e f}
%$A$&$B$&$A$&\eor&$B$\\
%\hline
% T & T&&T\\
% T & F&&T\\
% F & T&&T\\
% F & F&&F
%\end{tabular}
%\end{center}
%Often, though, it won't be quite so simple and we'll have to carefully go through which line of the connective-rule to follow.


Consider the sentence $(\enot I\eand H)\eif H$. We will give a \emph{truth table} which lists all the valuations and says whether this sentence is true or false on each of them.
The valuations assign either True or False to each atomic sentence. In this case we have two atomic sentences, $I$ and $H$, so we have four possible valuations, each of which is a line in the truth table:
\begin{center}
\begin{tabular}{cc|c}
$I$&$H$&$(\enot I\eand H)\eif H$\\\hline
T&T&\\
T&F&\\
F&T&\\
F&F&
\end{tabular}
\end{center}
Our job is to fill out the truth values of $(\enot I\eand H)\eif H$.

Here the formation tree will help us know what to do (see \S\ref{TFLsentences}):
\begin{center}
\begin{forest}
[$(\enot I\eand H)\eif H$
	[$(\enot I\eand H)$
		[$\enot I$
			[$I$]
		]
		[$H$]
	]
	[$H$]
]
\end{forest}
\end{center}
The truth rule for $\enot$ tells us how the truth value of $\enot I$ depends on the truth of $I$. Then the rule for $\eand$ tells us how the truth value of $\enot I\eand H$ depends on the truths of $\enot I$ and $H$; and finally, the rule for $\eif$ tells us how the truth value of $(\enot I\eand H)\eif H$ depends on those of $\enot I\eand H$ and $H$.

So to work out the truth values of $(\enot I\eand H)\eif H$ we first need to work out the truth values of $\enot I$ and $\enot I\eand H$.

So, we expand our truth table with columns for each of these.
\begin{center}
\begin{tabular}{cc|c|c|c}
$I$&$H$&$\enot I$&$(\enot I\eand H)$&$(\enot I\eand H)\eif H$\\\hline
T&T&&\\
T&F&&\\
F&T&&\\
F&F&&
\end{tabular}
\end{center}
The first step is $\enot I$.
To be able to do this, we will use the truth rule we specified for $\enot$:
\begin{center}
\begin{tabular}{ccc}
If \metaX is & & then $\enot X$ is\\
T&$\leadsto$&F\\
F&$\leadsto$&T
\end{tabular}
\end{center}
Now, we can fill out:
\begin{center}
\begin{tabular}{cc|c|c|c}
$I$&$H$&$\enot I$&$(\enot I\eand H)$&$(\enot I\eand H)\eif H$\\\hline
T&T&a=F&&\\
T&F&b=F&&\\
F&T&c=T&&\\
F&F&d=T&&\\
$\bigstar$
\end{tabular}
\end{center}
We worked these out by:
\begin{itemize}
\item For `a': Look at the column for $I$, and see it's T, so by our rule (T-line), we fill out a=F.
\item For `b': Look at the column for $I$, and see it's T, so by our rule (T-line), we fill out b=F.
\item For `c': Look at the column for $I$, and see it's F, so by our rule (F-line), we fill out c=T.
\item For `d': Look at the column for $I$, and see it's F, so by our rule (F-line), we fill out d=T.
\end{itemize}

The next step is to consider $\enot I\eand H$. For this we will use the truth rule for $\eand$:
\begin{center}
\begin{tabular}{ccccc}
If \metaX is&and \metaY is  && then $\metaX\eand \metaY$ is\\
T&T&$\leadsto$&T\\
T&F&$\leadsto$&F\\
F&T&$\leadsto$&F\\
F&F&$\leadsto$&F
\end{tabular}
\end{center}
Now, we can fill out:
\begin{center}
\begin{tabular}{cc|c|c|c}
$I$&$H$&$\enot I$&$(\enot I\eand H)$&$(\enot I\eand H)\eif H$\\\hline
T&T&F&a=F&\\
T&F&F&b=F&\\
F&T&T&c=T&\\
F&F&T&d=F&\\
&$\bigstar$&$\bigstar$
\end{tabular}
\end{center}
We worked these out by:
\begin{itemize}
\item For `a': Look at the column for $\enot I$ and the column for $H$, we have F and T; so by our rule (FT-line), we fill out a=F.
\item For `b': Look at the column for $\enot I$ and the column for $H$, we have F and F; so by our rule (FF-line), we fill out a=F.
\item For `c': Look at the column for $\enot I$ and the column for $H$, we have T and T; so by our rule (TT-line), we fill out a=T.
\item For `d': Look at the column for $\enot I$ and the column for $H$, we have T and F; so by our rule (TF-line), we fill out a=F.
\end{itemize}

Now, finally, we need to look at $(\enot I\eand H)\eif H$, and will use the truth rule for $\eif$:

\begin{center}
\begin{tabular}{cccc}
If \metaX is&and \metaY is  && then $\metaX\eif \metaY$ is\\
T&T&$\leadsto$&T\\
T&F&$\leadsto$&F\\
F&T&$\leadsto$&T\\
F&F&$\leadsto$&T
\end{tabular}
\end{center}
Now, we can fill out:
\begin{center}
\begin{tabular}{cl|c|l|c}
$I$&$H$&$\enot I$&$(\enot I\eand H)$&$(\enot I\eand H)\eif H$\\\hline
T&T&F&F&a=T\\
T&F&F&F&b=T\\
F&T&T&T&c=T\\
F&F&T&F&d=T\\
&$\bigstar_{\text{consequent}}$&&$\bigstar_{\text{antecedent}}$
\end{tabular}
\end{center}
We worked these out by:
\begin{itemize}
\item For `a': In the column for $(\enot I\eand H)$ (which is our antecedent) we have F; and in the column for $H$ (our consequent)  we have T. For the conditional, it's very important to bear in mind which order they come in. We are looking at antecedent then consequent, so the rule line we are looking at is FT (the antecedent is False and the consequent is True rather than visa versa). And we get a=T.
\item For `b': In the column for $(\enot I\eand H)$ we have F; and in the column for $H$ we have F. So by our rule for $\eif$ (FF-line) we have b=T.
\item For `c': In the column for $(\enot I\eand H)$ we have T; and in the column for $H$ we have T. So by our rule for $\eif$ (TT-line) we have b=T.
\item For `d': In the column for $(\enot I\eand H)$ we have F; and in the column for $H$ we have F. So by our rule for $\eif$ (FF-line) we have d=T.
\end{itemize}




When we do these in practice, our sentences can have many subsentences, consider, e.g.~$$\enot (B\eand(\enot B\eiff \enot A)))$$ A tool for fitting our truth tables on a page is not to put all the subsentences out as separate entire columns to calculate, but to simply list the values underneath the main connective of the sentence. This is simply to keep the truth table more concise.

So instead of writing
\begin{center}
\begin{tabular}{cc|cccc|c}
$A$&$B$&$\enot B$&$\enot A$&$\enot B\eiff \enot A$&$(B\eand(\enot B\eiff \enot A)))$&$\enot (B\eand(\enot B\eiff \enot A)))$\\\hline
T&T&F&F&T&T&F\\
T&F&T&F&F&F&T\\
F&T&F&T&F&F&T\\
F&F&T&T&T&F&T
\end{tabular}
\end{center}
I can write:\begin{center}
\begin{tabular}{cc|deeeef}
$A$&$B$&\textbf{$\enot$}&$(B$&$\eand$&$(\enot B$&$\eiff$&$ \enot A)))$\\\hline
T&T&\textbf{F}&&T&F&T&F\\
T&F&\textbf{T}&&F&T&F&F\\
F&T&\textbf{T}&&F&F&F&T\\
F&F&\textbf{T}&&F&T&T&T
\end{tabular}
\end{center}and highlight the column under the main connective, which provides our final answer. You should feel free to use whichever method you find easier.

There are tricks that mean you can miss out some of the gaps by using strategies such as: knowing that $\metaX$ is false is already enough to see that $\metaX\eand\metaY$ is false, so we don't need to continue to work out the truth value of $\metaY$. These ``shortcuts'' are discussed in \S\ref{s:PartialTruthTable}, but we won't read it in this course.



%A \define{complete truth table} has a line for every possible assignment of True and False to the relevant atomic sentences. Each line represents a \emph{valuation}, and a complete truth table has a line for all the different valuations.

\newglossaryentry{truth table}
{
name=truth table,
description={A table that gives all the possible \glspl{truth value} for a \gls{sentence of TFL} or sentences in TFL, with a line for every possible \gls{valuation} of all atomic sentences}
}

\section{The possible valuations}


The size of the complete truth table depends on the number of different atomic sentences in the table. A sentence that contains only one atomic sentence requires only two rows, as in the characteristic truth table for negation. This is true even if the same letter is repeated many times, as in the sentence
`$[(C\eiff C) \eif C] \eand \enot(C \eif C)$'.
The complete truth table requires only two lines because there are only two possibilities: `$C$' can be true or it can be false. The truth table for this sentence looks like this:
\begin{center}
\begin{tabular}{c| d e e e e e e e e e e e e e e f}
$C$&$[($&$C$&\eiff&$C$&$)$&\eif&$C$&$]$&\eand&\enot&$($&$C$&\eif&$C$&$)$\\
\hline
 T &    & T &  T  & T &   & T  & T & &\TTbf{F}&  F& &   T &  T  & T &   \\
 F &    & F &  T  & F &   & F  & F & &\TTbf{F}&  F& &   F &  T  & F &   \\
\end{tabular}
\end{center}
Looking at the column underneath the main logical operator, we see that the sentence is false on both rows of the table; i.e., the sentence is false regardless of whether `$C$' is true or false. It is false on every valuation.

A sentence that contains two atomic sentences requires four lines for a complete truth table, as in the characteristic truth tables for our binary connectives, and as in the complete truth table for `$(H \eand I)\eif H$'.

A sentence that contains three atomic sentences requires eight lines:
\begin{center}
\begin{tabular}{c c c|d e e e f}
$M$&$N$&$P$&$M$&\eand&$(N$&\eor&$P)$\\
\hline
%           M        &     N   v   P
T & T & T & T & \TTbf{T} & T & T & T\\
T & T & F & T & \TTbf{T} & T & T & F\\
T & F & T & T & \TTbf{T} & F & T & T\\
T & F & F & T & \TTbf{F} & F & F & F\\
F & T & T & F & \TTbf{F} & T & T & T\\
F & T & F & F & \TTbf{F} & T & T & F\\
F & F & T & F & \TTbf{F} & F & T & T\\
F & F & F & F & \TTbf{F} & F & F & F
\end{tabular}
\end{center}
From this table, we know that the sentence `$M\eand(N\eor P)$' can be true or false, depending on the truth values of `$M$', `$N$', and `$P$'.

A complete truth table for a sentence that contains four different atomic sentences requires 16 lines. Five letters, 32 lines. Six letters, 64 lines. And so on. To be perfectly general: If a complete truth table has $n$ different atomic sentences, then it must have $2^n$ lines.

In order to fill in the columns of a complete truth table, begin with the right-most atomic sentence and alternate between `T' and `F'. In the next column to the left, write two `T's, write two `F's, and repeat. For the third atomic sentence, write four `T's followed by four `F's. This yields an eight line truth table like the one above. For a 16 line truth table, the next column of atomic sentences should have eight `T's followed by eight `F's. For a 32 line table, the next column would have 16 `T's followed by 16 `F's, and so on.


\section{More about brackets}\label{s:MoreBracketingConventions}
Consider these two sentences:
	\begin{align*}
		((A \eand B) \eand C)\\
		(A \eand (B \eand C))
	\end{align*}
These are truth functionally equivalent. Consequently, it will never make any difference from the perspective of truth value -- which is all that TFL cares about (see \S\ref{s:TruthFunctionality}) -- which of the two sentences we assert (or deny). Even though the order of the brackets does not matter as to their truth, we should not just drop them. The expression
	\begin{align*}
		A \eand B \eand C
	\end{align*}
is ambiguous between the two sentences above.  The same observation holds for disjunctions. The following sentences are logically equivalent:
	\begin{align*}
		((A \eor B) \eor C)\\
		(A \eor (B \eor C))
	\end{align*}
But we should not simply write:
	\begin{align*}
		A \eor B \eor C
	\end{align*}
In fact, it is a specific fact about the characteristic truth table of $\eor$ and $\eand$ that guarantees that any two conjunctions (or disjunctions) of the same sentences are truth functionally equivalent, however you place the brackets. \emph{But be careful}. These two sentences have \emph{different} truth tables:
	\begin{align*}
		((A \eif B) \eif C)\\
		(A \eif (B \eif C))
	\end{align*}
So if we were to write:
	\begin{align*}
		A \eif B \eif C
	\end{align*}
it would be dangerously ambiguous. So we must not do the same with conditionals. Equally, these sentences have different truth tables:
	\begin{align*}
		((A \eor B) \eand C)\\
		(A \eor (B \eand C))
	\end{align*}
So if we were to write:
	\begin{align*}
		A \eor B \eand C
	\end{align*}
it would be dangerously ambiguous. \emph{Never write this.} The moral is: never drop brackets.

\begin{practiceproblems}\label{pr.TT.TTorC}
\problempart
\label{pr.TT.TTorC}
\label{pr.TT.TTorC}
Complete truth tables for each of the following:
\begin{earg}
\item $A \eif A$ %taut
\myanswer{\begin{center}
\begin{tabular}{c | def}
$A$ & $A$&$\eif$&$A$\\
\hline
 T & T&\TTbf{T}&T\\
F & F&\TTbf{T}&F
\end{tabular}
\end{center}
}

\item $C \eif\enot C$ %contingent
\myanswer{\begin{center}
\begin{tabular}{c | d e e f}
$C$ & $C$&$\eif$&$\enot$&$C$\\
\hline
 T & T & \TTbf{F}& F& T\\
F & F & \TTbf{T}& T& F\\
\end{tabular}
\end{center}}
\item $(A \eiff B) \eiff \enot(A\eiff \enot B)$ %tautology

\myanswer{\begin{center}
\begin{tabular}{c c | d e e e e e e e f}
$A$ & $B$&$(A$&$\eiff$&$B)$&$\eiff$&$\enot$&$(A$&$\eiff$&$\enot$&$B)$ \\
\hline
T & T & T & T & T & \TTbf{T} & T & T & F & F & T\\
T & F & T & F & F & \TTbf{T} & F & T & T & T & F\\
F & T & F & F & T & \TTbf{T} & F & F & T & F & T\\
F & F & F & T & F & \TTbf{T} & T & F & F & T & F
 \end{tabular}
\end{center}}
\item $(A \eif B) \eor (B \eif A)$ % taut

\myanswer{
\begin{center}
\begin{tabular}{c c | d e e e e e f}
$A$ & $B$&$(A$&$\eif$&$B)$&$\eor$&$(B$&$\eif$&$A)$ \\
\hline
T & T & T & T & T & \TTbf{T} & T & T & T\\
T & F & T & F & F & \TTbf{T} & F & T & T\\
F & T & F & T & T & \TTbf{T} & T & F & F \\
F & F & F & T & F & \TTbf{T} & F & T & F
 \end{tabular}
\end{center}}
\item $(A \eand B) \eif (B \eor A)$  %taut

\myanswer{
\begin{center}
\begin{tabular}{c c | d e e e e e f}
$A$ & $B$&$(A$&$\eand$&$B)$&$\eif$&$(B$&$\eor$&$A)$ \\
\hline
T & T & T & T & T & \TTbf{T} & T & T & T\\
T & F & T & F & F & \TTbf{T} & F & T & T\\
F & T & F & F & T & \TTbf{T} & T & T & F \\
F & F & F & F & F & \TTbf{T} & F & F & F
 \end{tabular}
\end{center}}
\item $\enot(A \eor B) \eiff (\enot A \eand \enot B)$ %taut

\myanswer{\begin{center}
\begin{tabular}{c c | d e e e e e e e e f}
$A$ & $B$&$\enot$&$(A$&$\eor$&$B)$&$\eiff$&$(\enot$&$A$&$\eand$&$\enot$&$B)$\\
\hline
T & T & F & T & T & T & \TTbf{T} & F & T & F & F & T\\
T & F & F & T& T & F & \TTbf{T} & F & T & F & T & F\\
F & T & F & F & T & T & \TTbf{T} & T & F & F & F & T\\
F & F & T & F & F & F & \TTbf{T} & T & F & T & T & F
 \end{tabular}
\end{center}}
\item $\bigl[(A\eand B) \eand\enot(A\eand B)\bigr] \eand C$ %contradiction

\myanswer{\begin{center}
\begin{tabular}{c c c | d e e e e e e e e f}
$A$ & $B$&$C$&$\bigl[(A$&$\eand$&$B)$&$ \eand$&$\enot$&$(A$&$\eand$&$B)\bigr]$&$\eand$&$C$\\
\hline
T & T & T & T & T & T & F & F & T & T & T & \TTbf{F} & T\\
T & T & F & T& T & T & F & F & T & T & T & \TTbf{F}& F\\
T & F & T & T & F & F & F & T & T & F & F & \TTbf{F} & T\\
T & F & F & T & F & F & F & T & T & F & F & \TTbf{F} & F\\
F & T & T & F & F & T & F & T & F & F & T & \TTbf{F} & T\\
F & T & F & F & F & T & F & T & F & F & T & \TTbf{F} & F\\
F & F & T & F & F & F & F & T & F & F & F & \TTbf{F} & T\\
F & F & F & F & F & F & F & T & F & F & F & \TTbf{F} & F
\end{tabular}
\end{center}}
\item $[(A \eand B) \eand C] \eif B$ %taut

\myanswer{\begin{center}
\begin{tabular}{c c c | d e e e e e f}
$A$ & $B$&$C$&$[(A$&$\eand$&$B)$&$\eand$&$C]$&$\eif$&$B$\\
\hline
T & T & T & T & T & T & T & T & \TTbf{T} & T\\
T & T & F & T & T & T & F & F & \TTbf{T} & T\\
T & F & T & T & F & F & F & T & \TTbf{T} & F\\
T & F & F & T & F & F & F & F & \TTbf{T} & F\\
F & T & T & F & F & T & F & T & \TTbf{T} & T\\
F & T & F & F & F & T & F & F & \TTbf{T} & T\\
F & F & T & F & F & F & F & T & \TTbf{T} & F\\
F & F & F & F & F & F & F & F & \TTbf{T} & F\\
\end{tabular}
\end{center}}
\item $\enot\bigl[(C\eor A) \eor B\bigr]$ %contingent

\myanswer{\begin{center}
\begin{tabular}{c c c | d e e e e f}
$A$ & $B$&$C$&$\enot\bigl[($&$C$&$\eor$&$A)$&$\eor$&$B\bigr]$\\
\hline
T & T & T & \TTbf{F} & T & T & T & T & T\\
T & T & F & \TTbf{F} & F & T & T & T & T\\
T & F & T & \TTbf{F} & T & T & T & T & F\\
T & F & F & \TTbf{F} & F & T & T & T & F\\
F & T & T & \TTbf{F} & T & T & F & T & T\\
F & T & F & \TTbf{F} & F & F & F & T & T\\
F & F & T & \TTbf{F} & T & T & F & T & F\\
F & F & F & \TTbf{T} & F & F & F & F & F
\end{tabular}
\end{center}}
\end{earg}


\problempart
Check all the claims made in introducing the new notational conventions in \S10.3, i.e.\ show that:
\begin{earg}

	\item `$((A \eand B) \eand C)$' and `$(A \eand (B \eand C))$' have the same truth table
\myanswer{\begin{center}
\begin{tabular}{c c c | d e e e f | d e e e f }
$A$ & $B$ & $C$ & $(A$&$\eand$& $B)$ &$ \eand$ & $C$ & $A$ & $\eand$ & $(B$&$\eand$&$C)$\\
\hline
T & T & T & T & T & T &  \TTbf{T} & T &T & \TTbf{T} & T & T& T \\
T & T & F & T& T & T &  \TTbf{F} & F & T& \TTbf{F} & T & F& F\\
T & F & T & T & F & F &  \TTbf{F} & T & T & \TTbf{F} & F & F & T \\
T & F & F &  T & F & F &  \TTbf{F} & F & T& \TTbf{F} & F & F & F\\
F & T & T & F & F & T & \TTbf{F} & T & F& \TTbf{F} & T & T & T\\
F & T & F & F & F & T & \TTbf{F} & F & F& \TTbf{F} &  T & F & F\\
F & F & T & F & F & F & \TTbf{F} & T & F& \TTbf{F} & F& F & T\\
F & F & F & F & F & F & \TTbf{F} & F & F& \TTbf{F} &  F& F & F\\
\end{tabular}
\end{center}}

	\item `$((A \eor B) \eor C)$' and `$(A \eor (B \eor C))$' have the same truth table
\myanswer{\begin{center}
\begin{tabular}{c c c | d e e e f | d e e e f }
$A$ & $B$ & $C$ & $(A$&$\eor$& $B)$ &$ \eor$ & $C$ & $A$ & $\eor$ & $(B$&$\eor$&$C)$\\
\hline
T & T & T & T & T & T &  \TTbf{T} & T &T & \TTbf{T} & T & T& T \\
T & T & F & T& T & T &  \TTbf{T} & F & T& \TTbf{T} & T & T& F\\
T & F & T & T & T & F &  \TTbf{T} & T & T & \TTbf{T} & F & T & T \\
T & F & F &  T & T& F &  \TTbf{T} & F & T& \TTbf{T} & F & F & F\\
F & T & T & F & T & T & \TTbf{T} & T & F& \TTbf{T} & T & T & T\\
F & T & F & F & T & T & \TTbf{T} & F & F& \TTbf{T} &  T & T & F\\
F & F & T & F & F & F & \TTbf{T} & T & F& \TTbf{T} & F& T & T\\
F & F & F & F & F & F & \TTbf{F} & F & F& \TTbf{F} &  F& F & F\\
\end{tabular}
\end{center}}

	\item `$((A \eor B) \eand C)$' and `$(A \eor (B \eand C))$' do not have the same truth table
\myanswer{\begin{center}
\begin{tabular}{c c c | d e e e f | d e e e f }
$A$ & $B$ & $C$ & $(A$&$\eor$& $B)$ &$ \eand$ & $C$ & $A$ & $\eor$ & $(B$&$\eand$&$C)$\\
\hline
T & T & T & T & T & T &  \TTbf{T} & T &T & \TTbf{T} & T & T& T \\
T & T & F & T& T & T &  \TTbf{F} & F & T& \TTbf{T} & T & F& F\\
T & F & T & T & T & F &  \TTbf{T} & T & T & \TTbf{T} & F & F & T \\
T & F & F &  T & T& F &  \TTbf{F} & F & T& \TTbf{T} & F & F & F\\
F & T & T & F & T & T & \TTbf{T} & T & F& \TTbf{T} & T & T & T\\
F & T & F & F & T & T & \TTbf{F} & F & F& \TTbf{F} &  T & F & F\\
F & F & T & F & F & F & \TTbf{F} & T & F& \TTbf{F} & F& F & T\\
F & F & F & F & F & F & \TTbf{F} & F & F& \TTbf{F} &  F& F & F\\
\end{tabular}
\end{center}}

	\item `$((A \eif B) \eif C)$' and `$(A \eif (B \eif C))$' do not have the same truth table
\myanswer{\begin{center}
\begin{tabular}{c c c | d e e e f | d e e e f }
$A$ & $B$ & $C$ & $(A$&$\eif$& $B)$ &$ \eif$ & $C$ & $A$ & $\eif$ & $(B$&$\eif$&$C)$\\
\hline
T & T & T & T & T & T &  \TTbf{T} & T &T & \TTbf{T} & T & T& T \\
T & T & F & T& T & T &  \TTbf{F} & F & T& \TTbf{F} & T & F& F\\
T & F & T & T & F & F &  \TTbf{T} & T & T & \TTbf{T} & F & T & T \\
T & F & F &  T & F& F &  \TTbf{T} & F & T& \TTbf{T} & F & T & F\\
F & T & T & F & T & T & \TTbf{T} & T & F& \TTbf{T} & T & T & T\\
F & T & F & F & T & T & \TTbf{F} & F & F& \TTbf{T} &  T & F & F\\
F & F & T & F & T & F & \TTbf{T} & T & F& \TTbf{T} & F& T & T\\
F & F & F & F & T & F & \TTbf{F} & F & F& \TTbf{T} &  F& T & F\\
\end{tabular}
\end{center}}
\end{earg}
Also, check whether:
\begin{earg}

	\item[5.] `$((A \eiff B) \eiff C)$' and `$(A \eiff (B \eiff C))$' have the same truth table
		\\\myanswer{Indeed they do:
\begin{center}
\begin{tabular}{c c c | d e e e f | d e e e f }
$A$ & $B$ & $C$ & $(A$&$\eiff$& $B)$ &$ \eiff$ & $C$ & $A$ & $\eiff$ & $(B$&$\eiff$&$C)$\\
\hline
T & T & T & T & T & T &  \TTbf{T} & T &T & \TTbf{T} & T & T& T \\
T & T & F & T& T & T &  \TTbf{F} & F & T& \TTbf{F} & T & F& F\\
T & F & T & T & F & F &  \TTbf{F} & T & T & \TTbf{F} & F & F & T \\
T & F & F &  T & F& F &  \TTbf{T} & F & T& \TTbf{T} & F & T & F\\
F & T & T & F & F & T & \TTbf{F} & T & F& \TTbf{F} & T & T & T\\
F & T & F & F & F & T & \TTbf{T} & F & F& \TTbf{T} &  T & F & F\\
F & F & T & F & T & F & \TTbf{T} & T & F& \TTbf{T} & F& F & T\\
F & F & F & F & T & F & \TTbf{F} & F & F& \TTbf{F} &  F& T & F\\
\end{tabular}
\end{center}}
\end{earg}

\problempart
Write complete truth tables for the following sentences and mark the column that represents the possible truth values for the whole sentence.

\begin{earg}

\item $\enot (S \eiff (P \eif S))$
\myanswer{
\begin{center}
\begin{tabular}{c|c|ccccc}
\cline{2-2}
~	&	\enot 	&	(S 	&	\eiff	&	(P 	&	\eif	&	S))	\\ 
\cline{2-7}
	& 	F 		&	T	&	T	&	T	&	T	&	T	\\
	& 	F 		&	T	&	T	&	F	&	T	&	T	\\
	& 	F 		&	F	&	T	&	T	&	F	&	F	\\
	& 	T 		&	F	&	F	&	F	&	T	&	F	\\
\cline{2-2}
\end{tabular}
\end{center}
}


 \item $\enot [(X \eand Y) \eor (X \eor Y)]$

\myanswer{
\begin{center}
\begin{tabular}{c|c|ccccccc}
\cline{2-2}
~	&	\enot	&	 [(X 	&	\eand& 	Y) 	&	\eor 	&	(X 	&	\eor 	&	Y)] \\
\cline{2-9}
	&	F	&	T	&	T	&	T	&	T	&	T	&	T	&	T	\\
	&	F	&	T	&	F	&	F	&	T	&	T	&	T	&	F	\\
	&	F	&	F	&	F	&	T	&	T	&	F	&	T	&	T	\\
	&	T	&	F	&	F	&	F	&	F	&	F	&	F	&	F	\\
\cline{2-2}
\end{tabular}
\end{center}
}


\item $(A \eif B) \eiff (\enot B\eiff \enot A)$

\myanswer{
\begin{center}
\begin{tabular}{cccc|c|ccccc}
\cline{5-5}
~	&	(A 	&	\eif	&	B)	&	 \eiff 	&	(\enot&	B 	&	\eiff 	&	 \enot 	& 	 A) \\
\cline{2-10}
	&	T	&	T	&	T	&	T		&	F	 &	T	&	T	&	F		&	T	\\	
	&	T	&	F	&	F	&	T		&	T	 &	F	&	F	&	F		&	T	\\
	&	F	&	T	&	T	&	F		&	F	 &	T	&	F	&	T		&	F	\\
	&	F	&	T	&	F	&	T		&	T	 &	F	&	T	&	T		&	F	\\
\cline{5-5}
\end{tabular}
\end{center}
}

\item $[C \eiff (D \eor E)] \eand \enot C$

\myanswer{
\begin{center}
\begin{tabular}{cccccc|c|cc}
\cline{7-7}
~	&	[C 	&	\eiff 	&	(D 	&	\eor 	&	E)] 	&	\eand 	&	 \enot 	&	 C \\
\cline{2-9}
	&	T	&	T	&	T	&	T	&	T	&	F		&	F		&	T	\\
	&	T	&	T	&	T	&	T	&	F	&	F		&	F		&	T	\\
	&	T	&	T	&	F	&	T	&	T	&	F		&	F		&	T	\\
	&	T	&	F	&	F	&	F	&	F	&	F		&	F		&	T	\\
	&	F	&	F	&	T	&	T	&	T	&	F		&	T		&	F	\\
	&	F	&	F	&	T	&	T	&	F	&	F		&	T		&	F	\\
	&	F	&	F	&	F	&	T	&	T	&	F		&	T		&	F	\\
	&	F	&	T	&	F	&	F	&	F	&	T		&	T		&	F	\\
\cline{7-7}
\end{tabular}
\end{center}
}

\item $\enot(G \eand (B \eand H)) \eiff (G \eor (B \eor H))$

\myanswer{
\begin{center}
\begin{tabular}{ccccccc|c|ccccc}
\cline{8-8}
~	&\enot&	(G 	&\eand &	(B 	&	 \eand 	&	 H))	&	\eiff 	&	(G 	& \eor 	& (B 	& \eor	& H))	\\
\cline{2-13}
	&F	   &	T	&	  T &	T	&	T		&	T	&	F	&	T	&	T	&	T	&	T	&	T	\\
	&T	   &	T	&	  F &	T	&	F		&	F	&	T	&	T	&	T	&	T	&	T	&	F	\\	
	&T	   &	T	&	 F  &	F	&	F		&	T	&	T	&	T	&	T	&	F	&	T	&	T	\\
	&T	   &	T	&	 F  &	F	&	F		&	F	&	T	&	T	&	T	&	F	&	F	&	F	\\
	&T	   &	F	&	F   &	T	&	T		&	T	&	T	&	F	&	T	&	T	&	T	&	T	\\
	&T	   &	F	&	F   &	T	&	F		&	F	&	T	&	F	&	T	&	T	&	T	&	F	\\
	&T	   &	F	&	F   &	F	&	F		&	T	&	T	&	F	&	T	&	F	&	T	&	T	\\
	&T	   &	F	&	F   &	F	&	F		&	F	&	F	&	F	&	F	&	F	&	F	&	F	\\
\cline{8-8}
\end{tabular}
\end{center}
}

\vspace{1em}

\end{earg}

\problempart
Write complete truth tables for the following sentences and mark the column that represents the possible truth values for the whole sentence.

\begin{earg}

\item	$(D \eand \enot D) \eif G $


\myanswer{\vspace{1em}
\begin{center}
\begin{tabular}{ccccc|c|c}
\cline{6-6}
	&	(D 	&	 \eand 	& 	 \enot	&	 D) 	&	 \eif 	&	 G \\
 \hline
	&	T	&	F		&	F		&	T	&	T	&	T	\\
	&	T	&	F		&	F		&	T	&	T	&	F	\\
	&	F	&	F		&	T		&	F	&	T	&	T	\\
	&	F	&	F		&	T		&	F	&	T	&	F	\\
\cline{6-6}
\end{tabular}
\end{center}
}
\vspace{1em}


\item	$(\enot P \eor \enot M) \eiff M $

\myanswer{
\begin{center}
\begin{tabular}{cccccc|c|c}
\cline{7-7}
	&	(\enot 	&	P 	&	\eor 	&	\enot 	& 	 M) 	& 	\eiff 	&	 M \\
 \hline
	&	F		&	T	&	F	&	F		&	T	&	F	&	T	\\
	&	F		&	T	&	T	&	T		&	F	&	F	&	F	\\
	&	T		&	F	&	T	&	F		&	T	&	T	&	T	\\
	&	T		&	F	&	T	&	T		&	F	&	F	&	F	\\
\cline{7-7}
\end{tabular}
\end{center}
}
\vspace{1em}



\item	$\enot \enot (\enot A \eand \enot B)  $

\myanswer{
\begin{center}
\begin{tabular}{c|c|cccccc}
\cline{2-2}
	&	\enot		&	 \enot 	&	(\enot 	& 	 A 	& \eand 	& 	\enot 	&	 B)  \\
 \hline
	&	F		&	T		&	F		&	T	&	F	&	F		&	T	\\
	&	F		&	T		&	F		&	T	&	F	&	T		&	F	\\
	&	F		&	T		&	T		&	F	&	F	&	F		&	T	\\
	&	T		&	F		&	T		&	F	&	T	&	T		&	F	\\
\cline{2-2}
\end{tabular}
\end{center}
}
\vspace{1em}



\item 	$[(D \eand R) \eif I] \eif \enot(D \eor R) $

\myanswer{
\begin{center}
\begin{tabular}{cccccc|c|cccc}
\cline{7-7}
	&	[(D 	& 	 \eand 	& 	 R)	& 	\eif 	&	I] 	&	\eif 	&	 \enot 	&	(D 	&	 \eor 	& R) \\
	 \hline
	&	T	&	T		&	T	&	T	&	T	&	F	&	F		&	T	&	T		&T	\\
	&	T	&	T		&	T	&	F	&	F	&	T	&	F		&	T	&	T		&T	\\
	&	T	&	F		&	F	&	T	&	T	&	F	&	F		&	T	&	T		&F	\\
	&	T	&	F		&	F	&	T	&	F	&	F	&	F		&	T	&	T		&F	\\
	&	F	&	F		&	T	&	T	&	T	&	F	&	F		&	F	&	T		&T	\\
	&	F	&	F		&	T	&	T	&	F	&	F	&	F		&	F	&	T		&T	\\
	&	F	&	F		&	F	&	T	&	T	&	T	&	T		&	F	&	F		&F	\\
	&	F	&	F		&	F	&	T	&	F	&	T	&	T		&	F	&	F		&F	\\
\cline{7-7}
\end{tabular}
\end{center}
}
	
\vspace{1em}


\item	$\enot [(D \eiff O) \eiff A] \eif (\enot D \eand O) $

\myanswer{
\begin{center}
\begin{tabular}{ccccccc|c|cccc}
\cline{8-8}
	&	\enot 	&	[(D 	&	\eiff 	&	O) 	&	\eiff 	&	 A]	& 	\eif 	 &	(\enot 	& 	D 	 & 	 \eand &O) \\ 
	\hline
	&	F		&	T	&	T	&	T	&	T	&	T	&	T	&	F		&	T	&	F	&T	\\
	&	T		&	T	&	T	&	T	&	F	&	F	&	F	&	F		&	T	&	F	&T	\\
	&	T		&	T	&	F	&	F	&	F	&	T	&	F	&	F		&	T	&	F	&F	\\
	&	F		&	T	&	F	&	F	&	T	&	F	&	T	&	F		&	T	&	F	&F	\\
	&	T		&	F	&	F	&	T	&	F	&	T	&	T	&	T		&	F	&	T	&T	\\
	&	F		&	F	&	F	&	T	&	T	&	F	&	T	&	T		&	F	&	T	&T	\\
	&	F		&	F	&	T	&	F	&	T	&	T	&	T	&	T		&	F	&	F	&F	\\
	&	T		&	F	&	T	&	F	&	F	&	F	&	F	&	T		&	F	&	F	&F	\\
\cline{8-8}
\end{tabular}
\end{center}
}
\vspace{1em}
\end{earg}



\problempart
 Can you think of sentences with the following truth table:
\begin{enumerate}
\item \begin{tabular}{cc|c}
$A$&$B$&?\\\hline
T&T&T\\
T&F&T\\
F&T&T\\
F&F&T
\end{tabular}\myanswer{$A\eor\enot A$}
\item \begin{tabular}{cc|c}
$A$&$B$&?\\\hline
T&T&F\\
T&F&F\\
F&T&T\\
F&F&F
\end{tabular}\myanswer{$\enot A\eand B$}
\item \begin{tabular}{cc|c}
$A$&$B$&?\\\hline
T&T&T\\
T&F&F\\
F&T&T\\
F&F&T
\end{tabular}\myanswer{\quad\begin{minipage}{.75\textwidth}
$\enot( A\eand \enot B)$;\\ or the more systematic answer following G.2.: $$(A\eand B)\eor (\enot A\eand B)\eor (\enot A\eand\enot B)$$
\end{minipage}}

\end{enumerate}
\problempart
There are in fact sixteen different possible columns if there are two atomic sentences.
\begin{enumerate}
\item Can you explain why?\myanswer{\\ The first line might have T or F, the second line might have T or F, etc. There are 4 valuations, so there are $2\times 2\times 2\times 2=2^4=16$ different ways of putting Ts and Fs to these 4 lines. I.e.~16 different columns. }
\item Can you show for each of these combinations that there is a sentence of TFL with that column describing its truth.
\myanswer{\\
See section 37.2 of forall$x$:Bristol for the general argument for this.
}
\item {} Can you show there's always a formula just using $\enot$ and $\eand$ with that column describing its truth.
\myanswer{\\
The previous answer had that every truth table can be given by a sentence with $\eand, \eor$ and $\enot$. We can replace any instances of $\eor$ by $\eand$ using: $$\metaX\eor\metaY\text{ is logically equivalent to }\enot(\enot\metaX\eand\enot\metaY)$$
See 37.4 of forall$x$:Bristol for more details.
}
\end{enumerate}

If you want additional practice, you can construct truth tables for any of the sentences and arguments in the exercises for the previous chapter.


\end{practiceproblems}

\chapter{Validity in TFL}
\label{s:ValidityTFL}

In the previous section, we introduced the idea of a valuation and showed how to determine the truth value of any TFL sentence, on any valuation, using a truth table. In this section, we will introduce some related ideas, and show how to use truth tables to test whether or not they apply.

\section{Validity}
Logic is particularly useful for evaluating arguments, in particular to help us see if an argument is \emph{valid}: logic can help us see if an argument is valid in virtue of its form. \todo{replace the $\therefore$ symbol?}

Now, since we are going to be talking about arguments a lot, we will introduce some abbreviation. An argument had the form:
\begin{earg}
\prem $\metaX_1$
\prem $\metaX_2$
\prem $\ldots$
\prem $\metaX_n$
\conc $\metaY$
\end{earg}
We can write this more concisely as: $$\metaX_1, \metaX_2, \ldots, \metaX_n\therefore \metaY$$ The $\therefore$ symbol can be read out loud as `therefore'. (When typing any answers out in word, you can replace it with `Therefore'.)

This is an argument, which may be valid or not.

In \S\ref{s:Valid} we introduce the notion of validity: an argument is valid if it's impossible for all the premises to be true and the conclusion false. We investigated validity as it applies to English language arguments, but we also have TFL arguments. The logical substitute notion of validity replaces `impossible' by `\emph{logically} impossible', where logical impossibility is characterised by there being no valuations with the property of interest.
\factoidbox{
	The TFL argument `$\metaX_1, \metaX_2, \ldots, \metaX_n\therefore \metaY$' is \define{valid} iff there are no valuations where the premises are true and the conclusion false.
}
\newglossaryentry{logically valid in TFL}
{
	name=logical validity (in TFL),
	text = logically valid,
	description={A property held by arguments if and only if no \gls{valuation} makes all premises true and the conclusion false}
}



Validity is a property of an \emph{argument}.
If we want to talk about the conclusion being a consequence of the premises, or \emph{following from}, we use the term `\emph{entails}'.
\factoidbox{
	The TFL sentences $\metaX_1, \metaX_2, \ldots, \metaX_n$ \define{entail} the sentence $\metaY$ iff there are no valuations where all of $\metaX_1, \metaX_2, \ldots, \metaX_n$ true and $\metaY$ false.
}
\newglossaryentry{entailment in TFL}
{
	name= entailment (in TFL),
	text = entailment,
	description={Some sentences entail another if and only if no \gls{valuation} makes all the former sentences true and the latter false, i.e.~if the corresponding argument is valid.}
}
%
%We can see that entailment and validity are sister-notions:
%\factoidbox{
%		$\metaX_1, \metaX_2, \ldots, \metaX_n$ entail $\metaY$ iff $\metaX_1, \metaX_2, \ldots, \metaX_n \therefore \metaY$ is valid.
%	}
An argument is valid if and only if its premises entail its conclusion.

It is easy to check for this with a truth table.
Consider the argument:
$$\enot L \eif (J \eor L),\enot L\therefore J$$
We need to check whether there is any valuation which makes both $\enot L \eif (J \eor L)$ and $\enot L$ true whilst making $J$ false. So we use a truth table:
\begin{center}
	\begin{tabular}{c c|d e e e e f|d f| c}
		$J$&$L$&\enot&$L$&\eif&$(J$&\eor&$L)$&\enot&$L$&$J$\\
		\hline
		%J   L   -   L      ->     (J   v   L)
		T & T & F &  & \TTbf{T} &  & T &  & \TTbf{F} &  & \TTbf{T}\\
		T & F & T &  & \TTbf{T} &  & T &  & \TTbf{T} &  & \TTbf{T}\\
		F & T & F &  & \TTbf{T} &  & T &  & \TTbf{F} &  & \TTbf{F}\\
		F & F & T &  & \TTbf{F} &  & F &  & \TTbf{T} &  & \TTbf{F}
	\end{tabular}
\end{center}
The only row on which both `$\enot L \eif (J \eor L)$' and `$\enot L$' are true is the second row, and that is a row on which `$J$' is also true. So `$\enot L \eif (J \eor L)$' and `$\enot L$' entail `$J$'.

%We now make an important observation:
%\factoidbox{
%	$\metaX_1, \metaX_2, \ldots, \metaX_n$ entail $\metaY$ iff $\metaX_1, \metaX_2, \ldots, \metaX_n \therefore \metaY$ is valid.
%}
%Here's why. If $\metaX_1, \metaX_2, \ldots, \metaX_n$ entail $\metaY$, then there is no valuation which makes all of $\metaX_1, \metaX_2, \ldots, \metaX_n$ true whilst making $\metaY$ false. This means that it is \emph{logically impossible} for $\metaX_1, \metaX_2, \ldots, \metaX_n$ all to be true whilst $\metaY$ is false. But this is just what it takes for an argument, with premises $\metaX_1, \metaX_2, \ldots, \metaX_n$ and conclusion $\metaY$, to be valid!

In short, we have a way to test for the validity of English arguments. First, we symbolize them in TFL, as having premises $\metaX_1, \metaX_2, \ldots, \metaX_n$, and conclusion $\metaY$. Then we check whether they are valid using truth tables.


\subsection{`Entails' versus `$\eif$'}
We now want to compare and contrast `entails' and `$\eif$'.

Observe: $\metaX$ entails $\metaY$ iff there is no valuation of the atomic sentences that makes $\metaX$ true and $\metaY$ false.

Observe: $\metaX \eif \metaY$ is a tautology iff there is no valuation of the atomic sentences that makes $\metaX \eif \metaY$ false. Since a conditional is true except when its antecedent is true and its consequent false, $\metaX \eif \metaY$ is a tautology iff there is no valuation that makes $\metaX$ true and $\metaY$ false.

Combining these two observations, we see that $\metaX \eif \metaY$  is a tautology iff  $\metaX$ entails $\metaY$. But there is a really, really important difference between entailment and `$\eif$': `$\eif$' is a sentential connective of TFL.\\ `Entails' is a word in English.
%\factoidbox{`$\eif$' is a sentential connective of TFL.\\ `$\entails$' is a symbol of augmented English.
%}
When `$\eif$' is flanked with two TFL sentences, the result is a longer TFL sentence. By contrast, when we use `entails', we are expressing that there is a relationship between the surrounding TFL sentences.



\section{Formal validity and English arguments}\label{s:TFLvsEngl}

We can now use the tool of TFL to investigate whether an argument of English is formally valid.

Consider an argument like:
\begin{quote}
If Jones signed the contract under duress, then the contract is void. But since it was not signed under duress, the contract is not void.\footnote{
This sort of example is discussed in \url{https://lawpublications.barry.edu/cgi/viewcontent.cgi?article=1026&context=barrylrev}.}
\end{quote}

We are interested in determining if this is valid or not. Here's a general strategy to help us:
\begin{highlighted}\begin{enumerate}
\item Find the structure of the argument. \\Identify the premises and conclusion.
\item \label{itm:validity-symbolise}Symbolise the argument in TFL.
\item \label{itm:validity-TTs} Check if the TFL argument is valid.\begin{itemize}\item Using truth tables to look for a valuation providing a counter example. If there is no such valuation, then it is valid.
\item Or, use natural deduction to show that it is valid.
\end{itemize}
%\item Use the insight you've now got from TFL to consider the original English argument and if that is valid.
\end{enumerate}
\end{highlighted}
\label{sec:checking validity}

%
%
%
%
%\chapter{English and TFL validity}
%We have now seen two notions of validity. Arguments are valid if there are no ``counter-examples'', where a counter example is a situation where the premises are true and the conclusion is false. The difference between English and TFL validity is that TFL cashes the notion of `situation' out in a clear and careful way: a possible situation is given by a \emph{valuation}. The English notion, however relies on ``possibility'', which we didn't \emph{define} and there's no clear way of checking what is possible or not.
%
%We are often interested in validity in English, but TFL is a tool that we can use to see if an English argument is valid or not.
%
%\section{Using TFL to investigate validity in English}
%
%Consider an argument like:
%\begin{itemize}
%\premIf Jones signed the contract under duress, then the contract is void. But since it was not signed under duress, the contract is not void.\footnote{
%This sort of example is discussed in \url{https://lawpublications.barry.edu/cgi/viewcontent.cgi?article=1026&context=barrylrev}.}
%\end{itemize}
%
%We are interested in determining if this is valid or not. Here's a general strategy to help us:
%\begin{highlighted}\begin{enumerate}
%\item Find the structure of the argument. \\Identify the premises and conclusion.
%\item \label{itm:validity-symbolise}Symbolise the argument in TFL.
%\item \label{itm:validity-TTs} Check if the TFL argument is valid.\begin{itemize}\item Using truth tables to look for a valuation providing a counter example. If there is no such valuation, then it is valid.
%\item Or, use natural deduction to show that it is valid.
%\end{itemize}
%%\item Use the insight you've now got from TFL to consider the original English argument and if that is valid.
%\end{enumerate}\end{highlighted}

Following this procedure, then, we need to first find the structure of the argument. It has two premises and a conclusion:
\begin{earg}
\prem If Jones signed the contract under duress, then the
contract is void.
\prem Jones did not sign the contract under duress.
\conc The contract is not void.
\end{earg}

We are interested in whether it is valid. I.e.~is there a situation where the premises are true and the conclusion is false. Perhaps in this case you can work it out, but if not TFL provides us a tool for thinking through all the possibilities. We therefore move to stage \ref{itm:validity-symbolise} and symbolise the sentences of the argument in TFL as well as we can.

\begin{earg}
 \prem $D\eif V$
 \prem $\enot D$
 \conc $\enot V$
 \end{earg}

And moving to stage \ref{itm:validity-TTs}, we check whether the TFL argument is valid. We do this with truth tables. Later in the course we'll see Natural Deduction which will provide an alternative method for checking validity.

So we need to fill in truth table with all the sentences involved in the argument. In this case each sentence has a simple structure, if we draw out the formation trees they just have one step, e.g.~\begin{center}
\begin{forest}
[$D\eif V$
	[$D$]
	[$V$]
]
\end{forest}
\end{center}So our truth table doesn't need any additional ``calculation'' lines. We can directly fill it in, following our truth rules
\begin{center}
\begin{tabular}{cc|cccl}
&&Premise&Premise&Concln\\
$D$&$V$&$D\eif V$&$\enot D$&$\enot V$\\\hline
T&T&T&F&F\\
T&F&F&F&T\\
F&T&T&T&F&$\leftsquigarrow$ counterexample\\
F&F&T&T&T
\end{tabular}
\end{center}

And we can see that this argument is invalid. The valuation providing the counter example is line 3: \begin{tabular}{cc}
$D$&$V$\\\hline
F&T
\end{tabular}

This allows us to read of a possibility where the premises are true but the conclusion false:  It might be that the contract is void without having been made under duress. E.g.~mental illness, misrepresentation,\ldots.

This has then allowed us to see that the original English argument is invalid. In fact, the argument followed an invalid pattern that is commonly found. It is a common fallacy called ``denying the consequent''.
%We will see some more common fallacies soon, but first let's work through another example where the strategy of moving to TFL will help us check if the original English argument is valid.

This argument is invalid. So we can accept the premises and still reject the conclusion. If we have a valid argument, then the conclusion can only be rejected if the premises are rejected. In a valid argument our discussion needs to be focused on the premises.

 In fact in this case we can give a fixed-up, related, valid argument \begin{earg}\prem $V\eif D$\prem $\enot D$\conc$\enot V$\end{earg} This uses the $\eif$ the other way around and is valid. It is an an argument form called Modus Tollens. However, presenting this argument and showing it's valid isn't enough to convince me of the conclusion that the contract is void: the argument uses a false premise: $V\eif D$. Or at least, this is a premise that I won't accept without further justification: look, if there was misrepresentation in the trial then $V\eif D$ is false. Give me a way of ruling out misrepresentation if you want to get to the conclusion that the contract is not void.


\subsubsection*{Another example: disjunctive reasoning}
\begin{quote}
The murder weapon was the dagger or the rope. So Mrs.~Peacock is guilty. After all, if the murder weapon was the dagger then she's guilty. And if it was the rope, she's guilty.
\end{quote}

First we find the argument structure:
\begin{earg}
\prem The murder weapon was the dagger or the rope.
\prem If the murder weapon was the dagger then Mrs.~Peacock  is guilty.
\prem If the murder weapon was the rope then Mrs.~Peacock  is guilty.
\conc Mrs.~Peacock is guilty.
\end{earg}
We then symbolise each constituent sentence into TFL:
\begin{earg}
\prem $D\eor R$
\prem $D\eif G$
\prem$R\eif G$
\conc $G$
\end{earg}

Now, we check if the TFL argument is valid using truth tables:
\begin{center}
\begin{tabular}{ccc|cccc}
&&&Premise&Premise&Premise&Concln\\
$D$&$G$&$R$&$D\eor G$&$D\eif G$&$R\eif G$&$G$\\\hline
T&T&T&T&T&T&T\\
T&T&F&T&T&T&T\\
T&F&T&T&F&F&F\\
T&F&F&T&F&T&F\\
F&T&T&T&T&T&T\\
F&T&F&T&T&T&T\\
F&F&T&F&T&F&F\\
F&F&F&F&T&T&F
\end{tabular}
\end{center}
And we see that this argument is valid: there is no valuation (=line) where the premises are all true and the conclusion false.

%This allows us to conclude that the original English argument is valid.


%\section{Fallacies}
%
%
%There are a number of arguments that are commonly used which are fallacies that provide no support for the conclusion. They have a mistaken logical form. Let's see a few arguments:
%\begin{highlighted}
%
%\begin{tabular}{ll}
%\textbf{Valid}&\textbf{Invalid. Fallacies}\\[1em]
%\begin{minipage}{4cm}
%Modus Ponens:
%% $\metaX\eif\metaZ,\;\metaX\;\therefore\;\metaZ$
%\begin{earg}
%\prem $\metaX\eif\metaZ$
%\prem $\metaX$
%\conc$\metaZ$
%\end{earg}
%\end{minipage}&
%\begin{minipage}{5cm}
%Affirming the consequent:
%\begin{earg}
%\prem $\metaZ\eif\metaX$
%\prem $\metaX$
%\conc$\metaZ$
%\end{earg}
%\end{minipage}\\[4em]
%
%\begin{minipage}{3cm}
%Modus Tollens:
%\begin{earg}
%\prem $\metaX\eif\metaZ$
%\prem $\enot\metaZ$
%\conc$\enot\metaX$
%\end{earg}
%\end{minipage}&
%\begin{minipage}{5cm}
%Denying the antecedent:
%\begin{earg}
%\prem $\metaZ\eif\metaX$
%\prem $\enot\metaZ$
%\conc$\enot\metaX$
%\end{earg}
%\end{minipage}
%\end{tabular}\end{highlighted}
%
%We can check these with truth tables. We have already shown that Denying the Antecedent is invalid, that was our example of duress and the void contract.
%
%Consider Affirming the Consequent:
%\begin{center}
%\begin{tabular}{cc|cccc}
%&&Premise&Premise&Concln\\
%$\metaX$&\metaZ&$\metaX\eif\metaZ$&\metaZ&\metaX\\\cline{1-5}
%\Tstrut
%T&T&T&T&T\\
%T&F&F&F&T\\
%F&T&T&T&F& $\leadsto$ This valuation shows invalid.\\
%F&F&T&F&F
%\end{tabular}
%\end{center}
%An example of an English argument with this structure is:\begin{itemize}
%\item [] If the contract was signed under duress then it is void. It is void. Therefore it was signed under duress.
%\end{itemize}Again we have the same sort of scenario showing it is invalid: it might be void because of misrepresentation.
%
%
%Sometimes we see people who are plausibly providing the fallacious argument. We should think of counter examples such as misrepresentation case, and see if they can add further plausible premises that will allow us to rule out such cases or if they can fix up their argument.
%
%%We should help them by directing them towards the premise that would actually help them in their argument.
%
%Consider the following argument: \begin{itemize}
%\item []If each man had a definite set of rules of conduct by which he regulated his
%life he would be a machine. But there are no such rules, so men
%cannot be machines.\footnote{Close to an argument discussed in Alan Turing ``Computing Machinery and Intelligence''. Turing points out that the argument is invalid. }
%\end{itemize}
%We might symbolise this argument as:
%\begin{earg}
%\item $R\eif M$\item $\enot R$\conc$\enot M$
%\end{earg}This is denying the antecedent; and is invalid.
%%We walked through an example of denying the antecedent at the beginning of this chapter.
%The situation showing its invalidity is a case where $M$ is true but $R$ false. This is a situation where men are machines but do not have a definite set of rules of conduct.
%
%
%
%Each fallacy has a very closely connected valid argument. From the invalid argument \begin{earg}
%\item $R\eif M$\item $\enot R$\conc$\enot M$
%\end{earg} we can construct a valid argument by reversing the ``if then''\begin{earg}
%\item $M\eif R$\item $\enot R$\conc$\enot M$
%\end{earg}
%
%
%The argument would be valid if the first premise ``If each man had a definite set of rules of conduct by which he regulated his life he would be a machine.'' was symbolised not as $R\eif M$, but as $M\eif R$: ``If man is a machine, then he would have a definite set of rules of conduct by which he regulated his life'' (``iff'' would do too: $R\eiff M$). It would then be an instance of Modus Tollens: a valid argument form. So it can be interpreted as a valid argument, but we then should make sure that if we are trying to use this as an argument we have the premise explicitly stated the way that we need it for the argument: $M\eif R$ instead of $R\eif M$; and we can direct our discussion towards the correct premise and think about whether it is true that if man is a machine then man is governed by a definite set of rules of conduct.
%
%
%
%
%Here we list just a few more fallacies. Though there are many more!
%\begin{itemize}
%\item Begging the question.  An argument begs the question when the conclusion is implicit in premises. This is actually a valid argument: draw truth tables for $A\therefore A$ and you see there is no valuation where the premise is true but conclusion false: the premise just \emph{is} the conclusion. But it is a fallacious argument because it gives us no reason to believe the conclusion. Noone will accept the premises unless they already accept the conclusion. \begin{quote}
%You should believe in God because He loves
%you. He spoke to me and asked me to talk to
%you about His love for you.
%\end{quote}
%%[Does induction beg the question?]
%\item False dilemma (of false dichotomy).  This implicitly assumes that these are the only two options. \begin{quote}
%we can't
%stand by and do nothing, therefore we have to
%carry out my preferred policy
%\end{quote}We can make this a valid argument by making the implicit premise $A\eor B$ (``my way or no way'') explicit. But this valid argument has a premise that we don't want to accept, so doesn't lead us to accepting the conclusion.
%\item Irrelevant conclusion.\begin{quote}The prosecution will show that the defendant in the dock was
%indeed the murderer. The crime was heinous,
%committed by a callous killer who is a danger to society, and
%such a crime cannot be allowed to go unpunished.\end{quote}This is an argument of the form $A\therefore B$. The premise and conclusion are unconnected and the premise provides no support for the conclusion.
%%\item Equivocation.
%%\item Lots more
%%%\item ambiguity in the conclusion
%\end{itemize}
%A fallacy is an argument that fails to provide a reason to
%believe the conclusion. Usually because the relationship between premises and
%conclusion is poor. Sometimes because the premises are not an independent
%reason to believe the conclusion.
%

\section{Other kinds of validity}

We have reached an important milestone: a test for the validity of arguments! However, we should not get carried away just yet. It is important to understand the {limits} of our achievement.

This allows us to investigate the formal validity of an argument. More precisely, whether its TFL-form is a valid form.

If the TFL-form of an argument is valid, then the argument is also conceputually valid, nomologically valid, and valid in any other (plausible) sense. We cannot conceive of possibilities that do not follow the laws of logic.

\begin{highlighted}
If the TFL symbolisation of an argument is valid (in TFL), then the original argument is formally valid, and valid in all the other senses.
\end{highlighted}

However, some arguments may have their TFL forms being invalid but are nonetheless conceptually valid. For example:
\begin{earg}
	\prem That is a triangle.
	\conc That has three sides.
\end{earg}
This argument is conceptually valid. But it would be symbolised in TFL simply as:
\begin{earg}
\prem $A$
\conc $B$
\end{earg}and looking at the truth tables (which are very easy to do!) we see:
\begin{center}
\begin{tabular}{cc|ccl}
&&Premise&Concln\\
$A$&$B$&$A$&$B$\\\cline{1-4}
T&T&T&T\\
T&F&T&F&$\leftsquigarrow$ \begin{minipage}{.4\linewidth}
 counterexample valuation, \\but not conceptually possible
\end{minipage}\\
F&T&F&T\\
F&F&F&F
\end{tabular}
\end{center}
The valuation in line 2 gives us a counterexample: it makes the premise, $A$, true, and the conclusion, $B$, false.
This valuation, however, violates the conceptual connections between words in English. Triangles simply have to have three sides. The ``case'' being described by this valuation makes it true that it is a triangle but false that it has three sides. That is not a conceptual possibility.
The argument is conceptually valid, but its TFL symbolisation is invalid.

If the TFL symbolisation is valid, the argument must be conceptually valid too. But some conceptually valid arguments have invalid TFL symbolisations.

Consider:
%
%However, it gives us a good tool. It can tell us where to look to think about our conceptual counterexample. For example in our argument:
\begin{earg}
\prem If Jones signed the contract under duress, then the
contract is void.
\prem Jones did not sign the contract under duress.
\conc The contract is not void.
\end{earg}which we symbolised in TFL as:
\begin{center}
$D\eif V$,\; $\enot D$\;\therefore\;$\enot V$
\end{center}
which is invalid. The valuation providing the counterexample is:
\begin{tabular}{cc}
$D$&$V$\\\hline
F&T
\end{tabular} To see if the argument is \emph{conceptually} valid, one should then investigate whether the valuations providing counterexamples to the formal validity are conceivable possibilities. Here, we're probably interested in something like legal validity, which governs conditions under which a contract is void.  We now know we should investigate whether the possibility of it being void but not signed under duress is compatible with the legal laws.
This will depend on what the legal laws are.
Actually, a contract can be void without being signed under duress, for example due to misrepresentation or mental incapacity. So the argument is in fact legally invalid as well.
%But were the laws different, and the only way to be void was to be signed under duress, then the argument would have been legally valid despite being formally invalid.


There is one loose end to tie up. In fact, TFL is in't all there is to formal validity.
If the TFL symbolisation of an argument is valid, then the original argument is formally valid. But consider the following argument:
Consider
\begin{earg}
\prem Socrates is a man.
\prem All men are mortal.
\conc Socrates is mortal.
\end{earg}
When we symbolise this in TFL, each of the sentences is simply an atomic sentence, so we get:
\begin{center}
$A$, $B$\; \therefore \; $C$
\end{center}
which is invalid.
The problem isn't the symbolisation we gave. It is the best we could do \emph{in TFL}. The problem is that TFL is limited in the ``form'' it can see, after all. We will improve on this when we move to First Order Logic in \S\ref{s:FOLBuildingBlocks}.
This argument is formally valid in virtue of its FOL form rather than TFL form.

\begin{practiceproblems}

\problempart
\label{pr.TT.valid}
\label{pr.TT.valid}
Use truth tables to determine whether each argument is valid or invalid.
\begin{earg}
\item $A\eif A \therefore A$  \hfill \myanswer{Invalid (see line 2)}
\myanswer{\begin{center}
\begin{tabular}{c | d e f | c}
$A$ &$A$&$\eif$&$A$&$A$\\
\hline
 T & T & \TTbf{T} & T & T\\
 F & F & \TTbf{T} & F & F
 \end{tabular}
\end{center}}
\item $A\eif(A\eand\enot A) \therefore \enot A$  \hfill \myanswer{Valid}
\myanswer{\begin{center}
\begin{tabular}{c | d e e e e f | df}
$A$&$A$&$\eif$&$(A$&$\eand$&$\enot$&$A)$&$\enot$&$A$\\
\hline
 T & T & \TTbf{F} & T & F& F&T&\TTbf{F}&T\\
 F & F & \TTbf{T} & F & F&T&F&\TTbf{T}&F
 \end{tabular}
\end{center}}
\item $A\eor(B\eif A) \therefore \enot A \eif \enot B$  \hfill \myanswer{Valid}
\myanswer{\begin{center}
\begin{tabular}{c c | d e e e f | d e e e f}
$A$ & $B$ & $A$&$\eor$&$(B$&$\eif$&$A)$&$\enot$&$A$&$\eif$&$\enot$&$B$\\
\hline
T & T & T & \TTbf{T} & T & T & T & F & T & \TTbf{T} & F & T \\
T & F & T & \TTbf{T} & F & T & T & F & T & \TTbf{T} & T & F \\
F & T & F & \TTbf{F} & T & F & F & T & F & \TTbf{F} & F & T \\
F & F & F & \TTbf{T} & F & T & F & T & F & \TTbf{T} & T & F
\end{tabular}
\end{center}}
\item $A\eor B, B\eor C, \enot A \therefore B \eand C$  \hfill \myanswer{Invalid (see line 6)}
\myanswer{\begin{center}
\begin{tabular}{c c c | d e f | d e f | d f | d e f}
$A$ & $B$ & $C$ & $A$&$\eor$&$B$&$B$&$\eor$&$C$&$\enot$&$A$&$B$&$\eand$&$C$\\
\hline
T & T & T & T & \TTbf{T} & T & T & \TTbf{T} & T & \TTbf{F} & T & T & \TTbf{T} & T \\
T & T & F & T & \TTbf{T} & T & T & \TTbf{T} & F & \TTbf{F} & T & T &\TTbf{F} & F \\
T & F & T & T & \TTbf{T} & F & F & \TTbf{T} & T & \TTbf{F} & T & F & \TTbf{F} & T \\
T & F & F & T & \TTbf{T} & F & F & \TTbf{F} & F & \TTbf{F} & T & F & \TTbf{F} & F\\
T & T & T & F & \TTbf{T} & T & T & \TTbf{T} & T & \TTbf{T} & F & T & \TTbf{T} & T \\
T & T & F & F & \TTbf{T} & T & T & \TTbf{T} & F & \TTbf{T} & F & T &\TTbf{F} & F \\
T & F & T & F & \TTbf{F} & F & F & \TTbf{T} & T & \TTbf{T} & F & F & \TTbf{F} & T \\
T & F & F & F & \TTbf{F} & F & F & \TTbf{F} & F & \TTbf{T} & F & F & \TTbf{F} & F
\end{tabular}
\end{center}}
\item $(B\eand A)\eif C, (C\eand A)\eif B \therefore (C\eand B)\eif A$  \hfill \myanswer{Invalid (see line 5)}
\myanswer{\begin{center}
\begin{tabular}{c c c | d e e e f | d e e e f | d e e e f}
$A$ & $B$ & $C$ & $(B$&$\eand$&$A)$&$\eif$&$C$&$(C$&$\eand$&$A)$&$\eif$&$B$&$(C$&$\eand$&$ B)$&$\eif$&$A$\\
\hline
T & T & T & T & T & T & \TTbf{T} & T & T & T & T & \TTbf{T} & T & T & T & T & \TTbf{T} & T\\
T & T & F & T & T & T & \TTbf{F} & F & F & F & T & \TTbf{T} & T & F & F & T & \TTbf{T} & T\\
T & F & T & F & F & T & \TTbf{T} & T & T & T & T & \TTbf{F} & F & T & F & F & \TTbf{T} & T\\
T & F & F & F & F & T & \TTbf{T} & F & F & F & T & \TTbf{T} & F & F & F & F & \TTbf{T} & T\\
F & T & T & T & F & F & \TTbf{T} & T & T & F & F & \TTbf{T} & T & T & T & T & \TTbf{F} & F\\
F & T & F & T & F & F & \TTbf{T} & F & F & F & F & \TTbf{T} & T & F & F & T & \TTbf{T} & F\\
F & F & T & F & F & F & \TTbf{T} & T & T & F & F & \TTbf{T} & F & T & F & F & \TTbf{T} & F\\
F & F & F & F & F & F & \TTbf{T} & F & F & F & F & \TTbf{T} & F & F & F & F & \TTbf{T} & F
\end{tabular}
\end{center}}
\end{earg}

\problempart Determine whether each sentence is a tautology, a contradiction, or a contingent sentence, using a complete truth table.
\begin{earg}
\item $\enot B \eand B$ \vspace{.5ex} \hfill \myanswer{Contradiction}


\item $\enot D \eor D$ \vspace{.5ex} \hfill \myanswer{Tautology}


\item $(A\eand B) \eor (B\eand A)$\vspace{.5ex} \hfill \myanswer{Contingent}


\item $\enot[A \eif (B \eif A)]$\vspace{.5ex} \hfill \myanswer{Contradiction}


\item $A \eiff [A \eif (B \eand \enot B)]$ \vspace{.5ex} \hfill \myanswer{Contradiction}


\item $[(A \eand B) \eiff B] \eif (A \eif B)$ \vspace{.5ex} \hfill \myanswer{Contingent}


\end{earg}

\noindent\problempart
\label{pr.TT.equiv}
Determine whether each the following sentences are logically equivalent using complete truth tables. If the two sentences really are logically equivalent, write ``equivalent.'' Otherwise write, ``Not equivalent.'' 
\begin{earg}
\item $A$ and $\enot A$
\item $A \eand \enot A$ and $\enot B \eiff B$
\item $[(A \eor B) \eor C]$ and $[A \eor (B \eor C)]$
\item $A \eor (B \eand C)$ and $(A \eor B) \eand (A \eor C)$
\item $[A \eand (A \eor B)] \eif B$ and $A \eif B$\end{earg}


\problempart
\label{pr.TT.equiv2}
Determine whether each the following sentences are logically equivalent using complete truth tables. If the two sentences really are equivalent, write ``equivalent.'' Otherwise write, ``not equivalent.''
\begin{earg}
\item $A\eif A$ and $A \eiff A$
\item $\enot(A \eif B)$ and $\enot A \eif \enot B$
\item $A \eor B$ and $\enot A \eif B$
\item$(A \eif B) \eif C$ and $A \eif (B \eif C)$
\item $A \eiff (B \eiff C)$ and $A \eand (B \eand C)$
\end{earg}

\problempart
\label{pr.TT.satisfiable2}
Determine whether each collection of sentences is jointly satisfiable or jointly unsatisfiable using a complete truth table.

\begin{earg}

\item $A \eand \enot B$, $\enot(A \eif B)$, $B \eif A$ %Consistent

\myanswer{
\begin{center}
\begin{tabular}{ccccccccccccccc} 
~ 	&	A 	& \eand	&  \enot & B &  & \enot &  (A &  \eif & B)	 & 	 & 	 B	 & 	\eif  & A  & Consistent \\ 
\cline{2-5} \cline{7-10}\cline{12-14} 
	& 	T   & F     &   F	 & T &  &  F	& 	T &   T	  & T 	 & 	 & 	 T	 & 	 T	  & T  &	  \\ 
\cline{2-14}
	& \multicolumn{1}{|r}{T}& 	\textbf{T}	 & T	 & F & & \textbf{T}	 & 	 T	 & 	 F	 	 & 	 F	 	 & 	 & 	 F	 	 & 	 \textbf{T}	 	 & 	 \multicolumn{1}{r|}{T}	 	 & 	  \\ 
\cline{2-14}
	& 	 F	 				 & 	 F	 & 	 F	 & T & 	& 	 F	 & 	 F	 & 	 T	 	 & 	 T	 	 & 	  & 	 T	 	 & 	 F	 	 & 	 F	 	 & 	  \\ 
	& 	 F	  				& 	 F	 & 	 T	 & 	F&  & 	 F	 & 	 F	 & 	 T	 	 & 	 F	 	 & 	  & 	 F	 	 & 	 T	 	 & 	 F	 	 & 	  \\ 
\end{tabular}
\end{center}
}

\item $A \eor B$, $A \eif \enot A$, $B \eif \enot B$ %unsatisfiable.

\myanswer{
\begin{center}
\begin{tabular}{ccccccccccccccc} 
  & A	 & \eor 	 & B 	 & 	 	 & A 	 & \eif 	 & 	\enot & A 	 & 	 	 & B 	 & \eif 	 & \enot	 & 	B 	 & 	Insatisfiable \\ 
\cline{2-4}\cline{6- 9} \cline{11-14}
   &	 T	 & 	 T	 &T  	 & 	 	 & T	 & 	 F	 & 	F 	 & T 	 & 	 	 & 	T 	 & 	F 	 & 	 F	 & 	T 	 & 	 \\ 
   &	 T	& 	 T	 & F 	 & 	 	 & 	T 	 & 	 F	 & 	 F	 & 	 T	 & 	 	 & 	F 	 & 	 T	 & 	 T	 & 	 F	 & 	 \\ 
   &	 F	& 	 T	 & 	 T	 & 	 	 & 	F 	 & 	 T	 & 	 T	 & 	F 	 & 	 	 & 	 T	 & 	 F	 & 	 F	 & 	 T	 & 	 \\ 
   &	 F	& 	 F	 & 	 F	 & 	 	 & 	 F	 & 	 T	 & 	 T	 & 	 F	 & 	 	 & 	 F	 & 	 T	 & 	 T	 & 	 F	 & 	 \\ 
\end{tabular}
\end{center}
}

\item $\enot(\enot A \eor B) $, $A \eif \enot C$, $A \eif (B \eif C)$ \hfill \myanswer{Consistent}

\myanswer{
\begin{center}
\begin{tabular}{ccccccccccccccccc}
   \enot & (\enot & A & \eor & B) &  & A  & \eif 	 & \enot 	 & C & 	 & A & \eif 	& (B & \eif & C) &  \\ 
 \cline{1-5}\cline{7-10} \cline{12-16} 
	F 	& 	F	 & 	T & T	 & T & 	  & T & F	 & 	 F & T 	 & 	 & T & T	 & T	 & T 	 & T 	 & \\ 
   	 F	& 	F	 & 	T & T	 & T & 	  & T & T	 & 	 T & F	 & 	 & T & F	 & T	 & F	 & F 	 & \\ 
   	 T & 	F 	& 	T & F	 & F & 	  & T & F	 & 	 F & T	 & 	 & T & T	 & F	 & T	 & T 	 & \\ 
\cline{1-16}
   	 \multicolumn{1}{|r}{\TTbf{T}}		&  F	 & 	T & F	 & 	F &  & 	T & \TTbf{T}	 & 	 T & F 	& 	 & T & \TTbf{T}	 & F	 & T	 & \multicolumn{1}{r|}{F} 	 & \\ 
\cline{1-16}
   	 F	& 	T	 & 	F & T	 & 	T &  & 	F & T	 & 	 F & T	 & 	 & F	 & F	 & T	 & T	 & T 	 & \\ 
   	 F	& 	 T	& 	F & T	 & 	T &  & 	F & T	 & 	T & F 	& 	 & F	 & T	 & T	 & F 	 & F 	 & \\ 
   	 F	& 	 T	& 	F & T	 & 	F &  & 	F & T	 & 	F & T	 & 	 & F	 & T	 & F	 & T	 & T 	 & \\ 
   	 F	& 	 T	& 	F & T	 & 	F &  & 	F & T	 & 	T & F	 & 	 & F	 & T	 & F	 & T	 & F 	 & \\ 
\end{tabular}
\end{center}
}



\item $A \eif B$, $A \eand \enot B$ \hfill \myanswer{Insatisfiable}

\item $A \eif (B \eif C)$, $(A \eif B) \eif C$, $A \eif C$ \hfill \myanswer{ Consistent} 

\end{earg}

\noindent\problempart
\label{pr.TT.satisfiable3}
Determine whether each collection of sentences is jointly satisfiable or jointly unsatisfiable, using a complete truth table.
\begin{earg}
\item $\enot B$, $A \eif B$, $A$ \vspace{.5ex} \hfill \myanswer{Insatisfiable}
\item $\enot(A \eor B)$, $A \eiff B$, $B \eif A$\vspace{.5ex} \hfill \myanswer{Consistent}
\item $A \eor B$, $\enot B$, $\enot B \eif \enot A$\vspace{.5ex} \hfill \myanswer{Insatisfiable}
\item $A \eiff B$, $\enot B \eor \enot A$, $A \eif B$\vspace{.5ex} \hfill \myanswer{Consistent} 
\item $(A \eor B) \eor C$, $\enot A \eor \enot B$, $\enot C \eor \enot B$\vspace{.5ex} \hfill \myanswer{Consistent}
\end{earg}

\noindent\problempart
\label{pr.TT.valid2}
Determine whether each argument is valid or invalid, using a complete truth table.
\begin{earg}
\item $A\eif B$, $B \therefore  A$ \hfill \myanswer{Invalid}

\item $A\eiff B$, $B\eiff C \therefore A\eiff C$ \hfill \myanswer{Valid}

\item $A \eif B$, $A \eif C\therefore B \eif C$ \hfill \myanswer{Invalid}.

\item $A \eif B$, $B \eif A\therefore A \eiff B$ \hfill \myanswer{Valid} 
\end{earg}

\noindent\problempart
\label{pr.TT.valid3}
Determine whether each argument is valid or invalid, using a complete truth table.
\begin{earg}
\item $A\eor\bigl[A\eif(A\eiff A)\bigr] \therefore  A $\vspace{.5ex} \hfill \myanswer{Invalid}
\item $A\eor B$, $B\eor C$, $\enot B \therefore A \eand C$\vspace{.5ex} \hfill \myanswer{Valid}
\item $A \eif B$, $\enot A\therefore \enot B$ \vspace{.5ex} \hfill \myanswer{Invalid}
\item $A$, $B\therefore \enot(A\eif \enot B)$ \vspace{.5ex} \hfill \myanswer{Valid}
\item $\enot(A \eand B)$, $A \eor B$, $A \eiff B\therefore C$ \vspace{.5ex} \hfill \myanswer{Valid}
\end{earg}


\end{practiceproblems}


\chapter{Other logical notions}

\section{Tautologies and contradictions}\label{s:TautologiesAndContradictions}
We have similar substitutes for the notions we introduced in \S\ref{s:BasicNotions}. There we explained \emph{necessary truth} and \emph{necessary falsity}. Both notions have surrogates in TFL. We will start with a surrogate for necessary truth.
	\factoidbox{
		$\metaX$ is a \define{tautology} iff it is true on every valuation.
	}

\newglossaryentry{tautology}
{
name=tautology,
description={A sentence that is true on every \gls{valuation}}
}

We can determine whether a sentence is a tautology just by using truth tables. If the sentence is true on every line of a complete truth table, then it is true on every valuation, so it is a tautology. In the example of \S\ref{s:CompleteTruthTables}, `$(H \eand I) \eif H$' is a tautology.

Our main example of a tautology is $A\eor\enot A$. Whatever the weather is like you know it's either raining or not. Similarly here, whether or not $A$ is true, we already know $A\eor \enot A$. We can check this with a truth table.
\begin{center}
	\begin{tabular}{c|deef}
	$A$&$A$&$\eor$&$\enot$&$A$\\\hline
	T&T&\textbf{T}&F&T\\
	F&F&\textbf{T}&T&F
\end{tabular}
\end{center}
Since there is a $T$ under the main connective on every line of the truth table, $A\eor \enot A$ is true, whatever the valuation is, i.e.~whether $A$ is true or false.


This is only, though, a \emph{surrogate} for necessary truth. There are some necessary truths that we cannot adequately symbolize in TFL. An example is `$2 + 2 = 4$'. This \emph{must} be true, but if we try to symbolize it in TFL, the best we can offer is an atomic sentence, and no atomic sentence is a tautology. Still, if we can adequately symbolize some English sentence using a TFL sentence which is a tautology, then that English sentence expresses a necessary truth.

We have a similar surrogate for necessary falsity:
	\factoidbox{
		$\metaX$ is a \define{contradiction} iff it is false on every valuation.
	}
\newglossaryentry{contradiction of TFL}
{
  name=contradiction (of TFL),
  text = contradiction,
description={A sentence that is false on every \gls{valuation}}
}

We can determine whether a sentence is a contradiction just by using truth tables. If the sentence is false on every line of a complete truth table, then it is false on every valuation, so it is a contradiction.
%In the example of \S\ref{s:CompleteTruthTables}, `$[(C\eiff C) \eif C] \eand \enot(C \eif C)$' is a contradiction.

Our core example of a contradiction is $A\eand\enot A$. Whether $A$ is true or false, $A\eand\enot A$ is false. This can again be checked using truth tables, observing that there is an F under the main connective in each line of the truth table.


\section{Logical equivalence}
Here is a similar useful notion:
	\factoidbox{
		$\metaX$ and $\metaY$ are \define{logically equivalent} iff, for every valuation, their truth values agree, i.e.\ if there is no valuation in which they have opposite truth values.
	}
\newglossaryentry{logically equivalent}
{
  name=logical equivalence (in TFL),
  text = logically equivalent,
description={A property held by pairs of sentences if and only if the sentences have the same truth value on every valuation}
}
We have already made use of this notion, in effect, in \S\ref{s:MoreBracketingConventions}; the point was that `$(A \eand B) \eand C$' and  `$A \eand (B \eand C)$' are logically equivalent. Again, it is easy to test for logical equivalence using truth tables. Consider the sentences `$\enot(P \eor Q)$' and `$\enot P \eand \enot Q$'. Are they logically equivalent? To find out, we construct a truth table.
\begin{center}
\begin{tabular}{c c|d e e f |d e e e f}
$P$&$Q$&\enot&$(P$&\eor&$Q)$&\enot&$P$&\eand&\enot&$Q$\\
\hline
 T & T & \TTbf{F} & T & T & T & F & T & \TTbf{F} & F & T\\
 T & F & \TTbf{F} & T & T & F & F & T & \TTbf{F} & T & F\\
 F & T & \TTbf{F} & F & T & T & T & F & \TTbf{F} & F & T\\
 F & F & \TTbf{T} & F & F & F & T & F & \TTbf{T} & T & F
\end{tabular}
\end{center}
Look at the columns for the main logical operators; negation for the first sentence, conjunction for the second. On the first three rows, both are false. On the final row, both are true. Since they match on every row, the two sentences are logically equivalent.


\section{Consistency}
In \S\ref{s:BasicNotions}, we said that sentences are jointly possible iff it is possible for all of them to be true at once. We can offer a surrogate for this notion too:
	\factoidbox{
		$\metaX_1, \metaX_2, \ldots, \metaX_n$ are \define{consistent} iff there is some valuation which makes them all true.
	}

        \newglossaryentry{consistency in TFL}
{
  name=consistency (in TFL),
  text=consistent,
description={A property held by sentences if and only if there is some \gls{valuation} that makes all the sentences true}
}

Derivatively, sentences are jointly logically inconsistent if there is no valuation that makes them all true. Again, it is easy to test for joint logical consistency using truth tables.

\section{These notions and the English variants}
Just as in \S\ref{s:TFLvsEngl} we said we can use the logical notion of validity in TFL to help with various notions of validity of English, the same can  be said for these other logical notions.

For example, if you show that the TFL symbolisation of an English sentence is a tautology, you can conclude that it is a necessary truth. However, there are different kinds of `necessary truths'. Some sentences like `A triangle has three sides' are necessary truths, but are not tautologies. But the tool of TFL can help.


\begin{practiceproblems}
\problempart
Revisit your answers to \S\ref{s:CompleteTruthTables}\textbf{A}. Determine which sentences were tautologies, which were contradictions, and which were neither tautologies nor contradictions.
\solutions

\

\problempart
\label{pr.TT.consistent}
Use truth tables to determine whether these sentences are jointly consistent, or jointly inconsistent:
\begin{earg}
\item $A\eif A$, $\enot A \eif \enot A$, $A\eand A$, $A\eor A$ %consistent
\item $A\eor B$, $A\eif C$, $B\eif C$ %consistent
\item $B\eand(C\eor A)$, $A\eif B$, $\enot(B\eor C)$  %inconsistent
\item $A\eiff(B\eor C)$, $C\eif \enot A$, $A\eif \enot B$ %consistent
\end{earg}



\problempart Determine whether each sentence is a tautology, a contradiction, or a contingent sentence, using a complete truth table.
\begin{earg}
\item $\enot B \eand B$ \vspace{.5ex}%contra


\item $\enot D \eor D$ \vspace{.5ex}%taut


\item $(A\eand B) \eor (B\eand A)$\vspace{.5ex} %contingent


\item $\enot[A \eif (B \eif A)]$\vspace{.5ex} %contra


\item $A \eiff [A \eif (B \eand \enot B)]$ \vspace{.5ex}%contra


\item $[(A \eand B) \eiff B] \eif (A \eif B)$ \vspace{.5ex}% contingent.

\end{earg}



\noindent\problempart
\label{pr.TT.equiv}
Determine whether each the following sentences are logically equivalent using complete truth tables. If the two sentences really are logically equivalent, write ``equivalent.'' Otherwise write, ``Not equivalent.''
\begin{earg}
\item $A$ and $\enot A$
\item $A \eand \enot A$ and $\enot B \eiff B$
\item $[(A \eor B) \eor C]$ and $[A \eor (B \eor C)]$
\item $A \eor (B \eand C)$ and $(A \eor B) \eand (A \eor C)$
\item $[A \eand (A \eor B)] \eif B$ and $A \eif B$\end{earg}


\problempart
\label{pr.TT.equiv2}
Determine whether each the following sentences are logically equivalent using complete truth tables. If the two sentences really are equivalent, write ``equivalent.'' Otherwise write, ``not equivalent.''
\begin{earg}
\item $A\eif A$ and $A \eiff A$
\item $\enot(A \eif B)$ and $\enot A \eif \enot B$
\item $A \eor B$ and $\enot A \eif B$
\item$(A \eif B) \eif C$ and $A \eif (B \eif C)$
\item $A \eiff (B \eiff C)$ and $A \eand (B \eand C)$
\end{earg}


\problempart
\label{pr.TT.consistent2}
Determine whether each collection of sentences is jointly consistent or jointly inconsistent using a complete truth table.
\begin{earg}
\item $A \eand \enot B$, $\enot(A \eif B)$, $B \eif A$\vspace{.5ex} %Consistent

%\begin{tabular}{ccccccccccccccc}
%1. 	&	A 					 & \eand 		&  \enot & B & & \enot  		& 	 (A	  & 	 \eif	 	 & 	 B)		 & 	 & 	 B	 	 & 	\eif 	 	 & 	A 	 	 & 	 Consistent \\
%\cline{2-5} \cline{7-10}\cline{12-14}
%	& 	T 					 & 	 F	 		&  F	 & T & & F	 		& 	 T	  & 	 T	 	 & 	T 	 	 & 	 & 	 T	 	 & 	 T	 	 & T	 	 	&	  \\
%\cline{2-14}
%	& \multicolumn{1}{|r}{T}& 	\textbf{T}	 & T	 & F & & \textbf{T}	 & 	 T	 & 	 F	 	 & 	 F	 	 & 	 & 	 F	 	 & 	 \textbf{T}	 	 & 	 \multicolumn{1}{r|}{T}	 	 & 	  \\
%\cline{2-14}
%	& 	 F	 				 & 	 F	 & 	 F	 & T & 	& 	 F	 & 	 F	 & 	 T	 	 & 	 T	 	 & 	  & 	 T	 	 & 	 F	 	 & 	 F	 	 & 	  \\
%	& 	 F	  				& 	 F	 & 	 T	 & 	F&  & 	 F	 & 	 F	 & 	 T	 	 & 	 F	 	 & 	  & 	 F	 	 & 	 T	 	 & 	 F	 	 & 	  \\
%\end{tabular}

\item $A \eor B$, $A \eif \enot A$, $B \eif \enot B$ \vspace{.5ex}%inconsistent.

%\begin{tabular}{ccccccccccccccc}
%2. &A	 & \eor 	 & B 	 & 	 	 & A 	 & \eif 	 & 	\enot & A 	 & 	 	 & B 	 & \eif 	 & \enot	 & 	B 	 & 	Inconsistent \\
%\cline{2-4}\cline{6- 9} \cline{11-14}
%   &	T	 & 	 T	 &T  	 & 	 	 & T	 & 	 F	 & 	F 	 & T 	 & 	 	 & 	T 	 & 	F 	 & 	 F	 & 	T 	 & 	 \\
%   &	 T	& 	 T	 & F 	 & 	 	 & 	T 	 & 	 F	 & 	 F	 & 	 T	 & 	 	 & 	F 	 & 	 T	 & 	 T	 & 	 F	 & 	 \\
%   &	 F	& 	 T	 & 	 T	 & 	 	 & 	F 	 & 	 T	 & 	 T	 & 	F 	 & 	 	 & 	 T	 & 	 F	 & 	 F	 & 	 T	 & 	 \\
%   &	 F	& 	 F	 & 	 F	 & 	 	 & 	 F	 & 	 T	 & 	 T	 & 	 F	 & 	 	 & 	 F	 & 	 T	 & 	 T	 & 	 F	 & 	 \\
%\end{tabular}

\item $\enot(\enot A \eor B) $, $A \eif \enot C$, $A \eif (B \eif C)$\vspace{.5ex} %Inconsistent

%3. &\enot & (\enot & A & \eor &B) &  &A  & \eif 	 &\enot 	 &C & 	 & A &\eif 	& (B 	 &\eif 	& C)	 &Consistent \\
%\cline{2-6}\cline{8-11} \cline{13-17}
%   &	F 	& 	F	 & 	T & T	 & T & 	  & T & F	 & 	 F&T 	 & 	 &T & T	 & T	 &T 	 &T 	 & \\
%   &	 F	& 	F	 & 	T & T	 & T & 	  & T & T	 & 	 T& F	 & 	 &T & F	 & T	 & F	 &F 	 & \\
%
%  &	 T & 	F 	& 	T & F	 & F & 	  & T & F	 & 	 F& T	 & 	 &T & T	 & F	 & T	 &T 	 & \\
%\cline{2-17}
%   &	 \multicolumn{1}{|r}{{\color{red}T}}		&  F	 & 	T & F	 & 	F &  & 	T & {\color{red}T}	 & 	 T&F 	& 	 &T & {\color{red}T}	 & F	 & T	 &\multicolumn{1}{r|}{F} 	 & \\
%\cline{2-17}
%   &	 F	& 	T	 & 	F & T	 & 	T &  & 	F & T	 & 	 F& T	 & 	 &F	 & F	 & T	 & T	 &T 	 & \\
%   &	 F	& 	 T	& 	F & T	 & 	T &  & 	F & T	 & 	T & F 	& 	 &F	 & T	 & T	 &F 	 &F 	 & \\
%   &	 F	& 	 T	& 	F & T	 & 	F &  & 	F & T	 & 	F & T	 & 	 &F	 & T	 & F	 & T	 &T 	 & \\
%   &	 F	& 	 T	& 	F & T	 & 	F &  & 	F & T	 & 	T & F	 & 	 &F	 & T	 & F	 & T	 &F 	 & \\
%\end{tabular}
%


\item $A \eif B$, $A \eand \enot B$\vspace{.5ex} %Inconsistent

\item $A \eif (B \eif C)$, $(A \eif B) \eif C$, $A \eif C$\vspace{.5ex} % consistent.

\end{earg}

\noindent\problempart
\label{pr.TT.consistent3}
Determine whether each collection of sentences is jointly consistent or jointly inconsistent, using a complete truth table.
\begin{earg}
\item $\enot B$, $A \eif B$, $A$ \vspace{.5ex}%inconsistent.
\item $\enot(A \eor B)$, $A \eiff B$, $B \eif A$\vspace{.5ex} %Consistent
\item $A \eor B$, $\enot B$, $\enot B \eif \enot A$\vspace{.5ex} %Inconsistent
\item $A \eiff B$, $\enot B \eor \enot A$, $A \eif B$\vspace{.5ex} %consistent.
\item $(A \eor B) \eor C$, $\enot A \eor \enot B$, $\enot C \eor \enot B$\vspace{.5ex} %consistent
\end{earg}




\solutions
\problempart
\label{pr.TT.concepts}
Answer each of the questions below and justify your answer.
\begin{earg}
\item Suppose $\metaX$ is true and $\metaY$ is true\\Is $\metaX\eor\metaY$ true.\myanswer{\\true}
\item Suppose $\metaX$ is false \\Is $\metaX\eor\metaY$ true?\myanswer{\\false. If $\metaY$ is also false, then $\metaX\eor\metaY$ is false. }
\item Suppose $\metaX$ is false \\Is $\metaX\eif\metaY$ true. \myanswer{\\true}
%\item Suppose $\entails \metaX$.\\ What can you say about $\metaX$?\myanswer{\\There are no valuations where the premises (in this case nothing) are true and the conclusion, \metaX, is false. So there are no valuations where the conclusion is false. I.e.~$\metaX$ is true on all valuations. It is a tautology.}
%\item Suppose that \metaX and \metaY are logically equivalent.\\ What can you say about $\metaX\eiff\metaY$?
%\myanswer{\item[] \metaX and \metaY have the same truth value on every line of a complete truth table, so $\metaX\eiff\metaY$ is true on every line. It is a tautology.}
%\item Suppose $\metaX\entails\metaY$. \\Is $\metaX\therefore\metaY$ valid?\myanswer{\\Yes}
\item Suppose that $\metaX$, $\metaY$ and $\metaZ$  are jointly inconsistent. \\What can you say about $\metaX\eand\metaY\eand\metaY$?
\myanswer{\item[] Since the sentences are jointly inconsistent, there is no valuation on which they are all true. So their conjunction is false on every valuation. It is a contradiction}
\item Suppose $\metaX$ is a tautology and  $\metaX\therefore\metaY$ is a valid argument. \\What can you say about $\metaY$?
\item Suppose that $(\metaX\eand\metaY)\eif\metaZ$ is not a tautology.\\ What can you say about whether $\metaX, \metaY \therefore\metaZ$ is valid?
\myanswer{\item[] Since the sentence $(\metaX\eand\metaY)\eif\metaZ$ is not a tautology, there is some line on which it is false. Since it is a conditional, on that line, \metaX and \metaY are true and \metaZ is false. So the argument is invalid.}
\item Suppose that \metaX is a contradiction.\\ What can you say about whether $\metaX, \metaY \entails \metaZ$?
\myanswer{\item[] Since \metaX is false on every line of a complete truth table, there is no line on which \metaX and \metaY are true and \metaZ is false. So the entailment holds.}
%%\item Suppose that \metaZ is a tautology.\\ What can you say about whether $\metaX, \metaY\entails \metaZ$?
%%\myanswer{\item[] Since \metaZ is true on every line of a complete truth table, there is no line on which \metaX and \metaY are true and \metaZ is false. So the entailment holds.}
\item Suppose that \metaX and \metaY are logically equivalent.\\ What can you say about $(\metaX\eor\metaY)$?
\myanswer{\item[] Not much. Since $\metaX$ and $\metaY$ are true on exactly the same lines of the truth table, their disjunction is true on exactly the same lines. So, their disjunction is logically equivalent to them.}
\item Suppose that \metaX and \metaY are \emph{not} logically equivalent.\\ What can you say about $\metaX\eor\metaY$?
\myanswer{\item[] \metaX and \metaY have different truth values on at least one line of a complete truth table, and $(\metaX\eor\metaY)$ will be true on that line. On other lines, it might be true or false. So $\metaX\eor\metaY$ is either a tautology or it is contingent; it is \emph{not} a contradiction.}
\end{earg}
\problempart
Consider the following principle:
	\begin{ebullet}
		\item Suppose $\metaX$ and $\metaY$ are logically equivalent. Suppose an argument contains $\metaX$ (either as a premise, or as the conclusion). The validity of the argument would be unaffected, if we replaced $\metaX$ with $\metaY$.
	\end{ebullet}
Is this principle correct? Explain your answer.

\end{practiceproblems}

\chapter{Truth table shortcuts}\teachingnote{Not taught in the course. This knowledge will not be needed for the exam, though you can use these techniques in the exam if you wish.\\}

Sometimes shortcuts can be taken when doing truth tables.
%With practice, you will quickly become adept at filling out truth tables.
In this section, we want to give you some permissible shortcuts to help you along the way.
These will never be {required} but they can speed things up.




\section{Working through truth tables}
You will quickly find that you do not need to copy the truth value of each atomic sentence, but can simply refer back to them. So you can speed things up by writing:
\begin{center}
\begin{tabular}{c c|d e e e e f}
$P$&$Q$&$(P$&\eor&$Q)$&\eiff&\enot&$P$\\
\hline
 T & T &  & T &  & \TTbf{F} & F\\
 T & F &  & T &  & \TTbf{F} & F\\
 F & T &  & T & & \TTbf{T} & T\\
 F & F &  & F &  & \TTbf{F} & T
\end{tabular}
\end{center}
You also know for sure that a disjunction is true whenever one of the disjuncts is true. So if you find a true disjunct, there is no need to work out the truth values of the other disjuncts. Thus you might offer:
\begin{center}
\begin{tabular}{c c|d e e e e e e f}
$P$&$Q$& $(\enot$ & $P$&\eor&\enot&$Q)$&\eor&\enot&$P$\\
\hline
 T & T & F & & F & F& & \TTbf{F} & F\\
 T & F &  F & & T& T& &  \TTbf{T} & F\\
 F & T & & &  & & & \TTbf{T} & T\\
 F & F & & & & & &\TTbf{T} & T
\end{tabular}
\end{center}
Equally, you know for sure that a conjunction is false whenever one of the conjuncts is false. So if you find a false conjunct, there is no need to work out the truth value of the other conjunct. Thus you might offer:
\begin{center}
\begin{tabular}{c c|d e e e e e e f}
$P$&$Q$&\enot &$(P$&\eand&\enot&$Q)$&\eand&\enot&$P$\\
\hline
 T & T &  &  & &  & & \TTbf{F} & F\\
 T & F &   &  &&  & & \TTbf{F} & F\\
 F & T & T &  & F &  & & \TTbf{T} & T\\
 F & F & T &  & F & & & \TTbf{T} & T
\end{tabular}
\end{center}
A similar short cut is available for conditionals. You immediately know that a conditional is true if either its consequent is true, or its antecedent is false. Thus you might present:
\begin{center}
\begin{tabular}{c c|d e e e e e f}
$P$&$Q$& $((P$&\eif&$Q$)&\eif&$P)$&\eif&$P$\\
\hline
 T & T & &  & & & & \TTbf{T} & \\
 T & F &  &  & && & \TTbf{T} & \\
 F & T & & T & & F & & \TTbf{T} & \\
 F & F & & T & & F & &\TTbf{T} &
\end{tabular}
\end{center}
So `$((P \eif Q) \eif P) \eif P$' is a tautology. In fact, it is an instance of \emph{Peirce's Law}, named after Charles Sanders Peirce.

\section{Testing for validity and entailment}
When we use truth tables to test for validity or entailment, we are checking for \emph{bad} lines: lines where the premises are all true and the conclusion is false. Note:
	\begin{earg}
		\item[\textbullet] Any line where the conclusion is true is not a bad line.
		\item[\textbullet] Any line where some premise is false is not a bad line.
	\end{earg}
Since \emph{all} we are doing is looking for bad lines, we should bear this in mind. So: if we find a line where the conclusion is true, we do not need to evaluate anything else on that line: that line definitely isn't bad. Likewise, if we find a line where some premise is false, we do not need to evaluate anything else on that line.

With this in mind, consider how we might test the following for validity:
	$$\enot L \eif (J \eor L), \enot L \therefore J$$
The \emph{first} thing we should do is evaluate the conclusion. If we find that the conclusion is \emph{true} on some line, then that is not a bad line. So we can simply ignore the rest of the line. So at our first stage, we are left with something like:
\begin{center}
\begin{tabular}{c c|d e e e e f |d f|c}
$J$&$L$&\enot&$L$&\eif&$(J$&\eor&$L)$&\enot&$L$&$J$\\
\hline
%J   L   -   L      ->     (J   v   L)
 T & T & &&&&&&&& {T}\\
 T & F & &&&&&&&& {T}\\
 F & T & &&?&&&&?&& {F}\\
 F & F & &&?&&&&?&& {F}
\end{tabular}
\end{center}
where the blanks indicate that we are not going to bother doing any more investigation (since the line is not bad) and the question-marks indicate that we need to keep investigating.

The easiest premise to evaluate is the second, so we next do that:
\begin{center}
\begin{tabular}{c c|d e e e e f |d f|c}
$J$&$L$&\enot&$L$&\eif&$(J$&\eor&$L)$&\enot&$L$&$J$\\
\hline
%J   L   -   L      ->     (J   v   L)
 T & T & &&&&&&&& {T}\\
 T & F & &&&&&&&& {T}\\
 F & T & &&&&&&{F}&& {F}\\
 F & F & &&?&&&&{T}&& {F}
\end{tabular}
\end{center}
Note that we no longer need to consider the third line on the table: it will not be a bad line, because (at least) one of premises is false on that line. Finally, we complete the truth table:
\begin{center}
\begin{tabular}{c c|d e e e e f |d f|c}
$J$&$L$&\enot&$L$&\eif&$(J$&\eor&$L)$&\enot&$L$&$J$\\
\hline
%J   L   -   L      ->     (J   v   L)
 T & T & &&&&&&&& {T}\\
 T & F & &&&&&&&& {T}\\
 F & T & &&&&&&{F}& & {F}\\
 F & F & T &  & \TTbf{F} &  & F & & {T} & & {F}
\end{tabular}
\end{center}
The truth table has no bad lines, so the argument is valid. (Any valuation on which all the premises are true is a valuation on which the conclusion is true.)

It might be worth illustrating the tactic again. Let us check whether the following argument is valid
$$A\eor B, \enot (A\eand C), \enot (B \eand \enot D) \therefore (\enot C\eor D)$$
At the first stage, we determine the truth value of the conclusion. Since this is a disjunction, it is true whenever either disjunct is true, so we can speed things along a bit. We can then ignore every line apart from the few lines where the conclusion is false.
\begin{center}
\begin{tabular}[t]{c c c c | c|c|c|d e e f }
$A$ & $B$ & $C$ & $D$ & $A\eor B$ & $\enot (A\eand C)$ & $\enot (B\eand \enot D)$ & $(\enot$ &$C$& $\eor$ & $D)$\\
\hline
T & T & T & T & & & & &  &  \TTbf{T} & \\
T & T & T & F & ? & ? & ? & F & &  \TTbf{F} & \\
T & T & F & T &  & &   & & &  \TTbf{T} & \\
T & T & F & F &  &  &   & T & &  \TTbf{T} &\\
T & F & T & T &  &  &  & & &  \TTbf{T} & \\
T & F & T & F & ? & ? & ?  & F &  &  \TTbf{F} &\\
T & F & F & T & & & & & & \TTbf{T} &\\
T & F & F & F & & & & T &  & \TTbf{T} & \\
F & T & T & T & & & & & & \TTbf{T} & \\
F & T & T & F & ? & ? & ? & F &  & \TTbf{F} &\\
F & T & F & T & & &  & & & \TTbf{T} & \\
F & T & F & F & & & &T & & \TTbf{T} & \\
F & F & T & T & & & & & & \TTbf{T} & \\
F & F & T & F & ? & ? & ? & F & & \TTbf{F} & \\
F & F & F & T & & & & & & \TTbf{T} & \\
F & F & F & F & & & & T& & \TTbf{T} & \\
\end{tabular}
\end{center}
We must now evaluate the premises. We use shortcuts where we can:
\begin{center}
\begin{tabular}[t]{c c c c | d e f |d e e f |d e e e f |d e e f }
$A$ & $B$ & $C$ & $D$ & $A$ & $\eor$ & $B$ & $\enot$ & $(A$ &$\eand$ &$ C)$ & $\enot$ & $(B$ & $\eand$ & $\enot$ & $D)$ & $(\enot$ &$C$& $\eor$ & $D)$\\
\hline
T & T & T & T & & && & && & && & & & &  &  \TTbf{T} & \\
T & T & T & F & &\TTbf{T}& & \TTbf{F}& &T& & & & & & & F & &  \TTbf{F} & \\
T & T & F & T & & && & && & &&  & &   & & &  \TTbf{T} & \\
T & T & F & F & & && & && & &&  &  &   & T & &  \TTbf{T} & \\
T & F & T & T & & && & && & &&  &  &  & & &  \TTbf{T} & \\
T & F & T & F & &\TTbf{T}& &\TTbf{F}& &T& &  && & & & F & & \TTbf{F} & \\
T & F & F & T & & && & && & && & & & & & \TTbf{T} & \\
T & F & F & F & & && & && & && & & & T &  & \TTbf{T} & \\
F & T & T & T& & && & && & & & & & & & & \TTbf{T} & \\
F & T & T & F & &\TTbf{T}& & \TTbf{T}& & F& & \TTbf{F}& & T& T&  & F &  & \TTbf{F} & \\
F & T & F & T & & && & && & && & &  & & & \TTbf{T} & \\
F & T & F & F& & && & && & && & & &T & & \TTbf{T} & \\
F & F & T & T & & && & && & && & & & & & \TTbf{T} & \\
F & F & T & F & & \TTbf{F} & & & & & & &&  &  &  & F & & \TTbf{F} & \\
F & F & F & T & & && & && & && & & & & & \TTbf{T} & \\
F & F & F & F & & && & && & && & & & T& & \TTbf{T} & \\
\end{tabular}
\end{center}
If we had used no shortcuts, we would have had to write 256 `T's or `F's on this table. Using shortcuts, we only had to write 37. We have saved ourselves a \emph{lot} of work.

We have been discussing shortcuts in testing for logically validity, but exactly the same shortcuts can be used in testing for entailment. By employing a similar notion of \emph{bad} lines, you can save yourself a huge amount of work.

\begin{practiceproblems}
\problempart
\label{pr.TT.TTorC2}
Using shortcuts, determine whether each sentence is a tautology, a contradiction, or neither.
\begin{earg}

\item $\enot B \eand B$ %contra
\myanswer{ \hfill Contradiction\begin{center}
\begin{tabular}{c | d e e f }
$B$ & $\enot$&$B$&$\eand$&$B$\\
\hline
T & F & & \TTbf{F}\\
F & & & \TTbf{F} & 
\end{tabular}
\end{center}}
\item $\enot D \eor D$ %taut
\myanswer{\hfill Tautology
\begin{center}
\begin{tabular}{c | d e e f }
$D$ & $\enot$&$D$&$\eor$&$D$\\
\hline
T &  & & \TTbf{T}\\
F & T & & \TTbf{T}
\end{tabular}
\end{center}}
\item $(A\eand B) \eor (B\eand A)$ %contingent
\myanswer{\hfill Neither
\begin{center}
\begin{tabular}{c c | d e e e e e f }
$A$ & $B$ & $(A$&$\eand$&$B)$&$\eor$&$(B$&$\eand$&$A)$\\
\hline
T & T & & T & & \TTbf{T}\\
T & F & & F & & \TTbf{F} & & F \\
F & T & & F & & \TTbf{F} & & F \\
F & F & & F & & \TTbf{F} & & F 
\end{tabular}
\end{center}}
\item $\enot[A \eif (B \eif A)]$ %contra
\myanswer{\hfill Contradiction
\begin{center}
\begin{tabular}{c c | d e e e e f }
$A$ & $B$ & $\enot[$&$A$&$\eif$&$(B$&$\eif$&$A)]$\\
\hline
T & T & \TTbf{F} &  & T & & T\\
T & F & \TTbf{F} &  & T & & T \\
F & T & \TTbf{F} &  & T & &  \\
F & F & \TTbf{F} &  & T & &  
\end{tabular}
\end{center}}
\item $A \eiff [A \eif (B \eand \enot B)]$ %contra
\myanswer{\hfill Contradiction
\begin{center}
\begin{tabular}{c c | d e e e e e e f }
$A$ & $B$ & $A$&$\eiff$&$[A$&$\eif$&$(B$&$\eand$&$\enot$&$B)]$\\
\hline
T & T & & \TTbf{F} &  & F & & F& F\\
T & F & & \TTbf{F} &  & F & & F \\
F & T & & \TTbf{F} &  & T & &  \\
F & F & & \TTbf{F} &  & T & &  
\end{tabular}
\end{center}}
\item $\enot(A\eand B) \eiff A$ %contingent
\myanswer{\hfill Neither
\begin{center}
\begin{tabular}{c c | d e e e e f }
$A$ & $B$ & $\enot$&$(A$&$\eand$&$B)$&$\eiff$&$A$\\
\hline
T & T & F & & T & & \TTbf{F} \\
T & F & T & & F & & \TTbf{T} \\
F & T & T & & F & & \TTbf{F} \\
F & F & T & & F & & \TTbf{F} \\
\end{tabular}
\end{center}}
\item $A\eif(B\eor C)$ %contingent
\myanswer{\hfill Neither
\begin{center}
\begin{tabular}{c c c | d e e e f }
$A$ & $B$ & $C$ & $A$&$\eif$&$(B$&$\eor$&$C)$\\
\hline
T & T & T & & \TTbf{T} & & T \\
T & T & F & & \TTbf{T} & & T \\
T & F & T & & \TTbf{T} & & T \\
T & F & F & & \TTbf{F} & & F \\
F & T & T & & \TTbf{T} & &  \\
F & T & F & & \TTbf{T} & &  \\
F & F & T & & \TTbf{T} & &  \\
F & F & F & & \TTbf{T} & &  
\end{tabular}
\end{center}}
\item $(A \eand\enot A) \eif (B \eor C)$ %tautology
\myanswer{\hfill Tautology
\begin{center}
\begin{tabular}{c c c | d e e e e e e f }
$A$ & $B$ & $C$ & $(A$&$\eand$&$\enot$&$A)$&$\eif$&$(B$&$\eor$&$C)$\\
\hline
T & T & T & & \TTbf{F} & F& & T \\
T & T & F & & \TTbf{F} &F & & T \\
T & F & T & & \TTbf{F} &F & & T \\
T & F & F & & \TTbf{F} & F& & T \\
F & T & T & & \TTbf{F} & & & T \\
F & T & F & & \TTbf{F} & & & T \\
F & F & T & & \TTbf{F} & & & T \\
F & F & F & & \TTbf{F} & &  & T \\ 
\end{tabular}
\end{center}}
\item $(B\eand D) \eiff [A \eiff(A \eor C)]$%contingent
\myanswer{\hfill Neither
\begin{center}
\begin{tabular}{c c c c | d e e e e e e e f }
$A$ & $B$ & $C$ & $D$ & $(B$&$\eand$&$D)$&$\eiff $&$[A$&$\eiff$&$(A$&$\eor$&$C)]$\\
\hline
T & T & T & T & & T & & \TTbf{T} &  & T & &  T\\
T & T & T & F & & F & & \TTbf{F} &  & T & &  T\\
T & T & F & T & & T & & \TTbf{T} &  & T & &  T\\
T & T & F & F & & F & & \TTbf{F} &  & T & &  T\\
T & F & T & T & & F & & \TTbf{F} &  & T & &  T\\
T & F & T & F & & F & & \TTbf{F} &  & T & &  T\\
T & F & F & T & & F & & \TTbf{F} &  & T & &  T\\
T & F & F & F & & F & & \TTbf{F} &  & T & &  T\\
F & T & T & T & & T & & \TTbf{F} &  & F & &  T\\
F & T & T & F & & F & & \TTbf{T} &  & F & &  T\\
F & T & F & T & & T & & \TTbf{T} &  & T & &  F\\
F & T & F & F & & F & & \TTbf{F} &  & T & &  F\\
F & F & T & T & & F & & \TTbf{T} &  & F & &  T\\
F & F & T & F & & F & & \TTbf{T} &  & F & &  T\\
F & F & F & T & & F & & \TTbf{F} &  & T & &  F\\
F & F & F & F & & F & & \TTbf{F} &  & T & &  F
\end{tabular}
\end{center}}
\end{earg}
\end{practiceproblems}

\chapter{Partial truth tables}\label{s:PartialTruthTable}\teachingnote{Not taught in course. This knowledge will not be needed for the exam. (Although you can use these techniques in the exam if you wish.)}

Sometimes, we do not need to know what happens on every line of a truth table. Sometimes, just a line or two will do.

\paragraph{Tautology.}
In order to show that a sentence is a tautology, we need to show that it is true on every valuation. That is to say, we need to know that it comes out true on every line of the truth table. So we need a complete truth table.

To show that a sentence is \emph{not} a tautology, however, we only need one line: a line on which the sentence is false. Therefore, in order to show that some sentence is not a tautology, it is enough to provide a single valuation---a single line of the truth table---which makes the sentence false.

Suppose that we want to show that the sentence `$(U \eand T) \eif (S \eand W)$' is \emph{not} a tautology. We set up a \define{partial truth table}:
\begin{center}
\begin{tabular}{c c c c |d e e e e e f}
$S$&$T$&$U$&$W$&$(U$&\eand&$T)$&\eif    &$(S$&\eand&$W)$\\
\hline
   &   &   &   &    &   &    &\TTbf{F}&    &   &
\end{tabular}
\end{center}
We have only left space for one line, rather than 16, since we are only looking for one line on which the sentence is false. For just that reason, we have filled in `F' for the entire sentence.

The main logical operator of the sentence is a conditional. In order for the conditional to be false, the antecedent must be true and the consequent must be false. So we fill these in on the table:
\begin{center}
\begin{tabular}{c c c c |d e e e e e f}
$S$&$T$&$U$&$W$&$(U$&\eand&$T)$&\eif    &$(S$&\eand&$W)$\\
\hline
   &   &   &   &    &  T  &    &\TTbf{F}&    &   F &
\end{tabular}
\end{center}
In order for the `$(U\eand T)$' to be true, both `$U$' and `$T$' must be true.
\begin{center}
\begin{tabular}{c c c c|d e e e e e f}
$S$&$T$&$U$&$W$&$(U$&\eand&$T)$&\eif    &$(S$&\eand&$W)$\\
\hline
   & T & T &   &  T &  T  & T  &\TTbf{F}&    &   F &
\end{tabular}
\end{center}
Now we just need to make `$(S\eand W)$' false. To do this, we need to make at least one of `$S$' and `$W$' false. We can make both `$S$' and `$W$' false if we want. All that matters is that the whole sentence turns out false on this line. Making an arbitrary decision, we finish the table in this way:
\begin{center}
\begin{tabular}{c c c c|d e e e e e f}
$S$&$T$&$U$&$W$&$(U$&\eand&$T)$&\eif    &$(S$&\eand&$W)$\\
\hline
 F & T & T & F &  T &  T  & T  &\TTbf{F}&  F &   F & F
\end{tabular}
\end{center}
We now have a partial truth table, which shows that `$(U \eand T) \eif (S \eand W)$' is not a tautology. Put otherwise, we have shown that there is a valuation which makes `$(U \eand T) \eif (S \eand W)$' false, namely, the valuation which makes `$S$' false, `$T$' true, `$U$' true and `$W$' false.

\paragraph{Contradiction.}
Showing that something is a contradiction requires a complete truth table: we need to show that there is no valuation which makes the sentence true; that is, we need to show that the sentence is false on every line of the truth table.

However, to show that something is \emph{not} a contradiction, all we need to do is find a valuation which makes the sentence true, and a single line of a truth table will suffice. We can illustrate this with the same example.
\begin{center}
\begin{tabular}{c c c c|d e e e e e f}
$S$&$T$&$U$&$W$&$(U$&\eand&$T)$&\eif    &$(S$&\eand&$W)$\\
\hline
  &  &  &  &   &   &   &\TTbf{T}&  &  &
\end{tabular}
\end{center}
To make the sentence true, it will suffice to ensure that the antecedent is false. Since the antecedent is a conjunction, we can just make one of them false. For no particular reason, we choose to make `$U$' false; and then we can assign whatever truth value we like to the other atomic sentences.
\begin{center}
\begin{tabular}{c c c c|d e e e e e f}
$S$&$T$&$U$&$W$&$(U$&\eand&$T)$&\eif    &$(S$&\eand&$W)$\\
\hline
 F & T & F & F &  F &  F  & T  &\TTbf{T}&  F &   F & F
\end{tabular}
\end{center}

\paragraph{Truth functional equivalence.}
To show that two sentences are logically equivalent, we must show that the sentences have the same truth value on every valuation. So this requires a  complete truth table.

To show that two sentences are \emph{not} logically equivalent, we only need to show that there is a valuation on which they have different truth values. So this requires only a one-line partial truth table: make the table so that one sentence is true and the other false.

\paragraph{Consistency.}
To show that some sentences are jointly consistent, we must show that there is a valuation which makes all of the sentences true,so this requires only a partial truth table with a single line.

To show that some sentences are jointly inconsistent, we must show that there is no valuation which makes all of the sentence true. So this requires a complete truth table: You must show that on every row of the table at least one of the sentences is false.

\paragraph{Validity.}
To show that an argument is valid, we must show that there is no valuation which makes all of the premises true and the conclusion false. So this  requires a complete truth table.  (Likewise for entailment.)

To show that argument is \emph{invalid}, we must show that there is a valuation which makes all of the premises true and the conclusion false. So this requires only a one-line partial truth table on which all of the premises are true and the conclusion is false. (Likewise for a failure of entailment.)


\
\\This table summarises what is required:

\begin{center}
\begin{tabular}{l l l}
%\cline{2-3}
 & \textbf{Yes} & \textbf{No}\\
 \hline
%\cline{2-3}
tautology? & complete truth table & one-line partial truth table\\
contradiction? &  complete truth table  & one-line partial truth table\\
%contingent? & two-line partial truth table & complete truth table\\
equivalent? & complete truth table & one-line partial truth table\\
consistent? & one-line partial truth table & complete truth table\\
valid? & complete truth table & one-line partial truth table\\
entailment? & complete truth table & one-line partial truth table\\
\end{tabular}
\end{center}
\label{table.CompleteVsPartial}


\begin{practiceproblems}
\problempart
\label{pr.TT.equiv3}
Use complete or partial truth tables (as appropriate) to determine whether these pairs of sentences are logically equivalent:
\begin{earg}
\item $A$, $\enot A$
\myanswer{\hfill Not logically equivalent
\begin{center}
\begin{tabular}{c | c | c}
$A$ &$A$&$\enot A$\\
\hline
 T & T & F
 \end{tabular}
\end{center}}
\item $A$, $A \eor A$
\myanswer{\hfill Logically equivalent
\begin{center}
\begin{tabular}{c | c | c}
$A$ &$A$&$A \eor A$\\
\hline
 T & T & T\\
 T & T & T
 \end{tabular}
\end{center}}
\item $A\eif A$, $A \eiff A$ %Yes
\myanswer{\hfill Logically equivalent
\begin{center}
\begin{tabular}{c | c | c}
$A$ &$A \eif A$&$A \eiff A$\\
\hline
T & T & T\\
F & T & T
\end{tabular}
\end{center}}
\item $A \eor \enot B$, $A\eif B$ %No
\myanswer{\hfill Not logically equivalent
\begin{center}
\begin{tabular}{c c | c | c}
$A$ & $B$ &$A\eor \enot B$&$A \eif B$\\
\hline
T & F & {T} & F
\end{tabular}
\end{center}}
\item $A \eand \enot A$, $\enot B \eiff B$ %Yes
\myanswer{\hfill Logically equivalent
\begin{center}
\begin{tabular}{c c | d e e f | d e e f}
$A$ & $B$ &$A$&$\eand$&$\enot$&$A$&$\enot$&$B$&$\eiff$&$B$\\
\hline
T & T & & \TTbf{F} & F & & F & & \TTbf{F} & \\
T & F & & \TTbf{F} & F & & T & & \TTbf{F} & \\
F & T & & \TTbf{F} &  & & F & & \TTbf{F} & \\
F & F & & \TTbf{F} &  & &  T & & \TTbf{F} & \\
\end{tabular}
\end{center}}

\item $\enot(A \eand B)$, $\enot A \eor \enot B$ %Yes
\myanswer{\hfill Logically equivalent
\begin{center}
\begin{tabular}{c c | d e e f | d e e e f}
$A$ & $B$ &$\enot$&$(A$&$\eand$&$B)$&$\enot$&$A$&$\eor$&$\enot$&$B$\\
\hline
T & T &\TTbf{F}& & T & & F & & \TTbf{F} & F\\
T & F & \TTbf{T} & & F & & F & & \TTbf{T} & T\\
F & T & \TTbf{T} & & F & & T & & \TTbf{T} & F\\
F & F & \TTbf{T} & & F & &  T& & \TTbf{T} & T\\
\end{tabular}
\end{center}}
\item $\enot(A \eif B)$, $\enot A \eif \enot B$ %No
\myanswer{\hfill Not logically equivalent
\begin{center}
\begin{tabular}{c c | d e e f | d e e e f}
$A$ & $B$ &$\enot$&$(A$&$\eif$&$B)$&$\enot$&$A$&$\eif$&$\enot$&$B$\\
\hline
T & T &\TTbf{F}& & T & & F & & \TTbf{T} & F\\
\end{tabular}
\end{center}}
\item $(A \eif B)$, $(\enot B \eif \enot A)$ %Yes
\myanswer{\hfill Logically equivalent
\begin{center}
\begin{tabular}{c c | c | d e e e f}
$A$ & $B$ &$(A \eif B)$&$(\enot$&$B$&$\eif$&$\enot$&$A)$\\
\hline
T & T &T& F & & \TTbf{T} &  \\
T & F &F & T & & \TTbf{F} & F  \\
F & T &T& F & & \TTbf{T} &  \\
F & F &T& T & & \TTbf{T} & T 
\end{tabular}
\end{center}}
\end{earg}

\solutions
\problempart
\label{pr.TT.satisfiable4}
Use complete or partial truth tables (as appropriate) to determine whether these sentences are jointly satisfiable, or jointly unsatisfiable:
\begin{earg}
\item $A \eand B$, $C\eif \enot B$, $C$ %unsatisfiable
\myanswer{\hfill Jointly unsatisfiable
\begin{center}
\begin{tabular}{c c c | c | d e f | c c c c  }
$A$ & $B$ & $C$ & $A \eand B$ & $C$&$\eif$&$\enot B$&$C$\\
\hline
T & T & T & T & & \TTbf{F} & F& T \\
T & T & F & T & & \TTbf{T} & & F \\
T & F & T & F & & \TTbf{T} &T & T \\
T & F & F & F & & \TTbf{T} & & F \\
F & T & T & F & & \TTbf{F} &F & T \\
F & T & F & F & & \TTbf{T} & & F \\
F & F & T & F & & \TTbf{T} & T& T \\
F & F & F & F & & \TTbf{T} & &  F \\ 
\end{tabular}
\end{center}}
\item $A\eif B$, $B\eif C$, $A$, $\enot C$ %unsatisfiable
\myanswer{\hfill Jointly unsatisfiable
\begin{center}
\begin{tabular}{c c c | c | c | c | c }
$A$ & $B$ & $C$ & $A \eif B$ & $B \eif C$ & $A$ & $\enot C$\\
\hline
T & T & T & T & T & T & F\\
T & T & F & T & F & T & T\\
T & F & T & F & T & T & F\\
T & F & F & F & T & T & T\\
F & T & T & T & T & F & F\\
F & T & F & T & F & F & T\\
F & F & T & T & T & F & F\\
F & F & F & T & T & F & T
\end{tabular}
\end{center}}
\item $A \eor B$, $B\eor C$, $C\eif \enot A$ %satisfiable
\myanswer{\hfill Jointly satisfiable
\begin{center}
\begin{tabular}{c c c | c | c | d e f }
$A$ & $B$ & $C$ & $A \eor B$ & $B \eor C$ & $C$&$\eif$& $\enot A$\\
\hline
F & T & T & T & T & & \TTbf{T} & T\\
\end{tabular}
\end{center}}
\item $A$, $B$, $C$, $\enot D$, $\enot E$, $F$ %satisfiable
\myanswer{\hfill Jointly satisfiable
\begin{center}
\begin{tabular}{c c c c c c| c | c | c | c |c | c }
$A$ & $B$ & $C$ & $D$ & $E$ & $F$ & $A$ & $B$ & $C$ & $\enot D$ & $\enot E$ & $F$\\
\hline
T & T & T & F & F & T & T & T& T& T& T& T
\end{tabular}
\end{center}}
% solutions to last two problems omitted
\end{earg}

\solutions
\problempart
\label{pr.TT.valid4}
Use complete or partial truth tables (as appropriate) to determine whether each argument is valid or invalid:
\begin{earg}
\item $A\eor\bigl[A\eif(A\eiff A)\bigr] \therefore A$ %invalid
\myanswer{\hfill Invalid
\begin{center}
\begin{tabular}{c |  d e e e e e f | c}
$A$ & $A$&$\eor$&$\bigl[A$&$\eif$&$(A$&$\eiff$&$A)\bigr]$ & $A$\\
\hline
F & & \TTbf{T} & & T & & & & F
\end{tabular}
\end{center}}
\item $A\eiff\enot(B\eiff A) \therefore A$ %invalid
\myanswer{\hfill Invalid
\begin{center}
\begin{tabular}{c c | d e e f | c}
$A$&$B$ & $A$&$\eiff$&$\enot$&$(B\eiff A)$ & $A$\\
\hline
F & F & & \TTbf{T} & F & T & F
\end{tabular}
\end{center}}

\item $A\eif B, B \therefore A$ %invalid
\myanswer{\hfill Invalid
\begin{center}
\begin{tabular}{c c | c | c | c}
$A$&$B$ & $A \eif B$ & $B$ & $A$\\
\hline
F & T & T & T & F
\end{tabular}
\end{center}}

\item $A\eor B, B\eor C, \enot B \therefore A \eand C$ %valid
\myanswer{\hfill Valid
\begin{center}
\begin{tabular}{c c c | c | c | c | c }
$A$ & $B$ & $C$ & $A \eor B$ & $B \eor C$ & $\enot B$ & $A \eand C$\\
\hline
T & T & T &  &  &  & T\\
T & T & F &  &  & F & F\\
T & F & T &  &  &  & T\\
T & F & F & T & F & T & F\\
F & T & T &  &  & F & F\\
F & T & F &  &  & F & F\\
F & F & T & F &  & T & F\\
F & F & F & F &  & T & F
\end{tabular}
\end{center}}

\item $A\eiff B, B\eiff C \therefore A\eiff C$ %valid
\myanswer{\hfill Valid
\begin{center}
\begin{tabular}{c c c | c | c | c }
$A$ & $B$ & $C$ & $A \eiff B$ & $B \eiff C$ & $A \eiff C$\\
\hline
T & T & T &  &  & T\\
T & T & F & T & F & F\\
T & F & T &  &  & T\\
T & F & F & F &  & F\\
F & T & T & F &  & F\\
F & T & F &  &  & T\\
F & F & T & T & F & F\\
F & F & F &  &  & T
\end{tabular}
\end{center}}
\end{earg}

\problempart
\label{pr.TT.TTorC3}
Determine whether each sentence is a tautology, a contradiction, or a contingent sentence. Justify your answer with a complete or partial truth table where appropriate.

% truth tables in LaTeX generated by http://www.curtisbright.com/logic/. Be sure to give him a shout out.

\begin{earg}
\item  $A \eif \enot A$ 
\myanswer{\hfill Contingent
 \[
 \begin{array}{c|cccc}
 A&A&\eif&\enot&A\\\hline
 T&T&\mathbf{F}&F&T\\
 F&F&\mathbf{T}&T&F
 \end{array}
 \]
}
%	T letter, 2 connectives
\item $A \eif (A \eand (A \eor B))$ 
\myanswer{ \hfill Tautology
\[
\begin{array}{cc|ccc@{}ccc@{}ccc@{}c@{}c}
A&B&A&\eif&(&A&\eand&(&A&\eor&B&)&)\\\hline
T&T&T&\mathbf{T}&&T&T&&T&T&T&&\\
T&F&T&\mathbf{T}&&T&T&&T&T&F&&\\
F&T&F&\mathbf{T}&&F&F&&F&T&T&&\\
F&F&F&\mathbf{T}&&F&F&&F&F&F&&
\end{array}
\]
\vspace{6pt}
}
%			2 letters, 3 connectives

\item $(A \eif B) \eiff (B \eif A)$
\myanswer{ \hfill Contingent
\[
\begin{array}{cc|c@{}ccc@{}ccc@{}ccc@{}c}
A&B&(&A&\rightarrow&B&)&\leftrightarrow&(&B&\rightarrow&A&)\\\hline
T&T&&T&T&T&&\mathbf{T}&&T&T&T&\\
T&F&&T&F&F&&\mathbf{F}&&F&T&T&\\
F&T&&F&T&T&&\mathbf{F}&&T&F&F&\\
F&F&&F&T&F&&\mathbf{T}&&F&T&F&
\end{array}
\]}
%		2 letters, 3 connectives

\item $A \eif \enot(A \eand (A \eor B)) $
\myanswer{ \hfill Contingent
\[
\begin{array}{cc|cccc@{}ccc@{}ccc@{}c@{}c}
A&B&A&\rightarrow&\enot&(&A&\eand&(&A&\eor&B&)&)\\\hline
T&T&T&\mathbf{F}&F&&T&T&&T&T&T&&\\
T&F&T&\mathbf{F}&F&&T&T&&T&T&F&&\\
F&T&F&\mathbf{T}&T&&F&F&&F&T&T&&\\
F&F&F&\mathbf{T}&T&&F&F&&F&F&F&&
\end{array}
\]
}

% 2 letters, 4 connectives

\item $\enot B \eif [(\enot A \eand A) \eor B]$
\myanswer{ \hfill Contingent
\[
\begin{array}{cc|cccc@{}c@{}cccc@{}ccc@{}c}
A&B&\enot&B&\rightarrow&(&(&\enot&A&\eand&A&)&\eor&B&)\\\hline
T&T&F&T&\mathbf{T}&&&F&T&F&T&&T&T&\\
T&F&T&F&\mathbf{F}&&&F&T&F&T&&F&F&\\
F&T&F&T&\mathbf{T}&&&T&F&F&F&&T&T&\\
F&F&T&F&\mathbf{F}&&&T&F&F&F&&F&F&
\end{array}
\]}
%	2 letters, 5 connectives

\item $\enot(A \eor B) \eiff (\enot A \eand \enot B)$
\myanswer{ \hfill Tautology
\[
\begin{array}{cc|cc@{}ccc@{}ccc@{}ccccc@{}c}
A&B&\enot&(&A&\eor&B&)&\leftrightarrow&(&\enot&A&\eand&\enot&B&)\\\hline
T&T&F&&T&T&T&&\mathbf{T}&&F&T&F&F&T&\\
T&F&F&&T&T&F&&\mathbf{T}&&F&T&F&T&F&\\
F&T&F&&F&T&T&&\mathbf{T}&&T&F&F&F&T&\\
F&F&T&&F&F&F&&\mathbf{T}&&T&F&T&T&F&
\end{array}
\]}
%2 letters, 6 connectives

\item $[(A \eand B) \eand C] \eif B$
\myanswer{ \hfill Tautology
\[
\begin{array}{ccc|c@{}c@{}ccc@{}ccc@{}ccc}
A&B&C&(&(&A&\eand&B&)&\eand&C&)&\rightarrow&B\\\hline
T&T&T&&&T&T&T&&T&T&&\mathbf{T}&T\\
T&T&F&&&T&T&T&&F&F&&\mathbf{T}&T\\
T&F&T&&&T&F&F&&F&T&&\mathbf{T}&F\\
T&F&F&&&T&F&F&&F&F&&\mathbf{T}&F\\
F&T&T&&&F&F&T&&F&T&&\mathbf{T}&T\\
F&T&F&&&F&F&T&&F&F&&\mathbf{T}&T\\
F&F&T&&&F&F&F&&F&T&&\mathbf{T}&F\\
F&F&F&&&F&F&F&&F&F&&\mathbf{T}&F
\end{array}
\]}

%3 letters, 3 connectives

\item $\enot\bigl[(C\eor A) \eor B\bigr]$
\myanswer{ \hfill Contingent
\[
\begin{array}{ccc|cc@{}c@{}ccc@{}ccc@{}c}
A&B&C&\enot&(&(&C&\eor&A&)&\eor&B&)\\\hline
T&T&T&\mathbf{F}&&&T&T&T&&T&T&\\
T&T&F&\mathbf{F}&&&F&T&T&&T&T&\\
T&F&T&\mathbf{F}&&&T&T&T&&T&F&\\
T&F&F&\mathbf{F}&&&F&T&T&&T&F&\\
F&T&T&\mathbf{F}&&&T&T&F&&T&T&\\
F&T&F&\mathbf{F}&&&F&F&F&&T&T&\\
F&F&T&\mathbf{F}&&&T&T&F&&T&F&\\
F&F&F&\mathbf{T}&&&F&F&F&&F&F&
\end{array}
\]}
%	 	3 letters, 3 connectives

\item $\bigl[(A\eand B) \eand\enot(A\eand B)\bigr] \eand C$
\myanswer{ \hfill Contradiction
\[
\begin{array}{ccc|c@{}c@{}ccc@{}cccc@{}ccc@{}c@{}ccc}
A&B&C&(&(&A&\eand&B&)&\eand&\enot&(&A&\eand&B&)&)&\eand&C\\\hline
T&T&T&&&T&T&T&&F&F&&T&T&T&&&\mathbf{F}&T\\
T&T&F&&&T&T&T&&F&F&&T&T&T&&&\mathbf{F}&F\\
T&F&T&&&T&F&F&&F&T&&T&F&F&&&\mathbf{F}&T\\
T&F&F&&&T&F&F&&F&T&&T&F&F&&&\mathbf{F}&F\\
F&T&T&&&F&F&T&&F&T&&F&F&T&&&\mathbf{F}&T\\
F&T&F&&&F&F&T&&F&T&&F&F&T&&&\mathbf{F}&F\\
F&F&T&&&F&F&F&&F&T&&F&F&F&&&\mathbf{F}&T\\
F&F&F&&&F&F&F&&F&T&&F&F&F&&&\mathbf{F}&F
\end{array}
\]}

% 	3 letters, 5 connectives

\item $(A \eand B) ]\eif[(A \eand C) \eor (B \eand D)]$
\myanswer{ \hfill Contingent
\[
\begin{array}{cccc|c@{}c@{}ccc@{}c@{}ccc@{}c@{}ccc@{}ccc@{}ccc@{}c@{}c}
A&B&C&D&(&(&A&\eand&B&)&)&\eif&(&(&A&\eand&C&)&\eor&(&B&\eand&D&)&)\\\hline
T&T&T&T&&&T&T&T&&&\mathbf{T}&&&T&T&T&&T&&T&T&T&&\\
T&T&F&F&&&T&T&T&&&\mathbf{F}&&&T&F&F&&F&&T&F&F&&\\
\end{array}
\]}

%	4 letters, 5 connectives
\end{earg}

\noindent\problempart
\label{pr.TT.TTorC4}
Determine whether each sentence is a tautology, a contradiction, or a contingent sentence. Justify your answer with a complete or partial truth table where appropriate.
\begin{earg}
\item  $\enot (A \eor A)$\vspace{.5ex}		\hfill			\myanswer{		Contradiction}	%	1 letter, 2 connectives
\item $(A \eif B) \eor (B \eif A)$\vspace{.5ex}	\hfill		\myanswer{			Tautology	}	%	2 letters, 2 connectives
\item $[(A \eif B) \eif A] \eif A$\vspace{.5ex}	\hfill			\myanswer{	  	Tautology		 }   %2 letters, 3 connectives
\item $\enot[( A \eif B) \eor (B \eif A)]$\vspace{.5ex}	\hfill		\myanswer{		Contradiction}	%	2 letters, 4 connectives
\item $(A \eand B) \eor (A \eor B)$\vspace{.5ex} 	\hfill			\myanswer{		Contingent	}	%2 letters, 5 connectives
\item $\enot(A\eand B) \eiff A$\vspace{.5ex} 		\hfill			\myanswer{	Contingent		}	%2 letters, 3 connectives
\item $A\eif(B\eor C)$\vspace{.5ex} 				\hfill			\myanswer{	Contingent		}	%3 letters, 2 connectives
\item $(A \eand\enot A) \eif (B \eor C)$\vspace{.5ex} 	\hfill		\myanswer{	Tautology		}	%3 letters, 4 connectives 
\item $(B\eand D) \eiff [A \eiff(A \eor C)]$\vspace{.5ex}\hfill		\myanswer{		Contingent	}	%	4 letters, 4 connectives
\item $\enot[(A \eif B) \eor (C \eif D)]$\vspace{.5ex} 	\hfill		\myanswer{	 Contingent 	}	%4 letters, 4 connectives
\end{earg}



\noindent\problempart
Determine whether each the following pairs of sentences are logically equivalent using complete truth tables. If the two sentences really are logically equivalent, write ``equivalent.'' Otherwise write, ``not equivalent.''
\begin{earg}
\item $A$ and $A \eor A$
\item $A$ and $A \eand A$
\item $A \eor \enot B$ and $A\eif B$
\item $(A \eif B)$ and $(\enot B \eif \enot A)$
\item $\enot(A \eand B)$ and $\enot A \eor \enot B$
\item $ ((U \eif (X \eor X)) \eor U)$ and $\enot (X \eand (X \eand U))$
\item $ ((C \eand (N \eiff C)) \eiff C)$ and $(\enot \enot \enot N \eif C)$
\item $[(A \eor B) \eand C]$ and $[A \eor (B \eand C)]$
\item $((L \eand C) \eand I)$ and $L \eor C$
\end{earg}


\noindent\problempart
\label{pr.TT.satisfiable5}
Determine whether each collection of sentences is jointly satisfiable or jointly unsatisfiable. Justify your answer with a complete or partial truth table where appropriate.
\begin{earg}
\item $A\eif A$, $\enot A \eif \enot A$, $A\eand A$, $A\eor A$ \vspace{.5ex} \hfill \myanswer{Consistent}
\item $A \eif \enot A$, $\enot A \eif A$\vspace{.5ex} \hfill \myanswer{Insatisfiable }
\item $A\eor B$, $A\eif C$, $B\eif C$\vspace{.5ex} \hfill \myanswer{Consistent}
\item $A \eor B$, $A \eif C$, $B \eif C$, $\enot C$\vspace{.5ex} \hfill \myanswer{	Insatisfiable}
\item $B\eand(C\eor A)$, $A\eif B$, $\enot(B\eor C)$\vspace{.5ex}  \hfill \myanswer{Insatisfiable}
\item $(A \eiff B) \eif B$,  $B \eif \enot (A \eiff B)$, $A \eor B$ \vspace{.5ex} \hfill \myanswer{Consistent}
\item $A\eiff(B\eor C)$, $C\eif \enot A$, $A\eif \enot B$\vspace{.5ex} \hfill \myanswer{Consistent}
\item  $A \eiff B$,  $\enot B \eor \enot A$,  $A \eif  B$ \vspace{.5ex} \hfill \myanswer{ Consistent}
\item $A \eiff B$, $A \eif C$, $B \eif D$, $\enot(C \eor D)$\vspace{.5ex} \hfill \myanswer{Consistent}
\item $\enot (A \eand \enot B)$,  $B \eif \enot A$, $\enot B$  \vspace{.5ex} \hfill \myanswer{Consistent}
\end{earg}

\noindent\problempart Determine whether each argument is valid or invalid. Justify your answer with a complete or partial truth table where appropriate.
\label{pr.TT.valid5} 
\begin{enumerate}
\item $A\eif(A\eand\enot A)\therefore \enot A$ \hfill \myanswer{ Valid}
\item $A \eor B$, $A \eif B$, $B \eif A \therefore  A \eiff B$  \hfill \myanswer{ Valid}
\item $A\eor(B\eif A)\therefore \enot A \eif \enot B$ \hfill \myanswer{Valid}
\item $A \eor B$, $A \eif B$, $ B \eif A \therefore  A \eand B$ \hfill \myanswer{Valid}
\item $(B\eand A)\eif C$, $(C\eand A)\eif B\therefore (C\eand B)\eif A$ \hfill \myanswer{Invalid}
\item $\enot (\enot A \eor \enot B)$, $A \eif \enot C \therefore  A \eif (B \eif C)$ \hfill \myanswer{ Invalid}
\item $A \eand (B \eif C)$, $\enot C \eand (\enot B \eif \enot A)\therefore C \eand \enot C$ \hfill \myanswer{ Valid}
\item $A \eand B$, $\enot A \eif \enot C$, $B \eif \enot D \therefore  A \eor B$ \hfill \myanswer{ Invalid}
\item $A \eif B\therefore (A \eand B) \eor (\enot A \eand \enot B)$ \hfill \myanswer{ Invalid}
\item $\enot A \eif B$,$ \enot B \eif C $,$ \enot C \eif A \therefore  \enot A \eif (\enot B \eor \enot C)$ \hfill \myanswer{Invalid}

\end{enumerate}

\noindent\problempart Determine whether each argument is valid or invalid. Justify your answer with a complete or partial truth table where appropriate.
\label{pr.TT.valid6} 
\begin{enumerate}
\item $A\eiff\enot(B\eiff A)\therefore A$ \hfill \myanswer{ Invalid}
\item $A\eor B$, $B\eor C$, $\enot A\therefore B \eand C$ \hfill \myanswer{ Invalid}
\item $A \eif C$, $E \eif (D \eor B)$, $B \eif \enot D\therefore (A \eor C) \eor (B \eif (E \eand D))$ \hfill \myanswer{ Invalid}
\item $A \eor B$, $C \eif A$, $C \eif B\therefore A \eif (B \eif C)$ \hfill \myanswer{ Invalid}
\item $A \eif B$, $\enot B \eor A\therefore A \eiff B$ \hfill \myanswer{ Valid}
\end{enumerate}



\end{practiceproblems}


\chapter{TFL vs English connectives }\label{s:ParadoxesOfMaterialConditional}
\teachingnote{Not covered in this course. This is just a taster for students who are interested.\\}
Consider the sentence:
	\begin{earg}
		\item[\ex{n:JanBald}] Jan is neither bald nor not-bald.
	\end{earg}
To symbolize this sentence in TFL, we would offer something like `$\enot J \eand \enot \enot J$'. This a contradiction (check this with a truth-table), but sentence \ref{n:JanBald} does not itself seem like a contradiction; for we might have happily go on to add `Jan is on the borderline of baldness'!

Third, consider the following sentence:
	\begin{earg}
		\item[\ex{n:GodParadox}]	It's not the case that, if God exists, She answers malevolent prayers.
%	Aaliyah wants to kill Zebedee. She knows that, if she puts chemical A into Zebedee's water bottle, Zebedee will drink the contaminated water and die. Equally, Bathsehba wants to kill Zebedee. She knows that, if she puts chemical B into Zebedee's water bottle, then Zebedee will drink the contaminated water and die. But chemicals A and B neutralize each other; so that if both are added to the water bottle, then Zebedee will not die.
	\end{earg}
        Symbolizing this in TFL, we would offer something like `$\enot (G \eif M)$'. Now, `$\enot (G \eif M)$' entails `$G$' (again, check this with a truth table). So if we symbolize sentence \ref{n:GodParadox} in TFL, it seems to entail that God exists. But that's strange: surely even an atheist can accept sentence \ref{n:GodParadox}, without contradicting herself!

        One lesson of this is that the symbolization of \ref{n:GodParadox} as `$\enot(G \eif M)$' shows that \ref{n:GodParadox} does not express what we intend. Perhaps we should rephrase it as
        	\begin{earg}
                \item[\ex{n:GodParadox2}] If God exists, She does not answer malevolent prayers.
  \end{earg}
and symbolize \ref{n:GodParadox2} as `$G \eif \enot M$'.  Now, if atheists are right, and there is no God, then `$G$' is false and so `$G \eif \enot M$' is true, and the puzzle disappears. However, if `$G$' is false, `$G \eif M$', i.e.\ `If God exists, She answers malevolent prayers', is \emph{also} true!

In different ways, these  examples highlight some of the limits of working with a language (like TFL) that can \emph{only} handle truth-functional connectives. Moreover, these limits give rise to some interesting questions in philosophical logic. The case of Jan's baldness (or otherwise) raises the general question of what logic we should use when dealing with \emph{vague} discourse. The case of the atheist raises the question of how to deal with the (so-called) \emph{paradoxes of the material conditional}. Part of the purpose of this course is to equip you with the tools to explore these questions of \emph{philosophical logic}. But we have to walk before we can run; we have to become proficient in using TFL, before we can adequately discuss its limits, and consider alternatives.
