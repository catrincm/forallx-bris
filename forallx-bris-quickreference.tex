%!TEX root = forallxbris.tex
\chapter{Quick reference}\label{glossary:quick ref}
%\pagestyle{plain}
\section{Sentences of TFL}
Definition of being a sentence of TFL:
	\begin{enumerate}
		\item Every atomic sentence is a sentence.
		\begin{itemize}
		\item $A,B,C\ldots,W$, or with subscripts $A_1, B_3, A_{100}, J_{375}$
		\end{itemize}
		\item If \metaX is a sentence, then $\enot\metaX$ is a sentence.
		\item If \metaX and \metaY are sentences, then $(\metaX\eand\metaY)$ is a sentence.
		\item If \metaX and \metaY are sentences, then $(\metaX\eor\metaY)$ is a sentence.
		\item If \metaX and \metaY are sentences, then $(\metaX\eif\metaY)$ is a sentence.
		\item If \metaX and \metaY are sentences, then $(\metaX\eiff\metaY)$ is a sentence.
		\item Nothing else is a sentence.
	\end{enumerate}
	
\section{Truth Rules for Connectives in TFL}
\label{app.CharacteristicTTs}
%\begin{center}
%			\begin{minipage}{.25\textwidth}
%				\begin{tabular}{c|c}
%				$\metaX$&$\enot\metaX$\\\hline
%				T&F\\
%				F&T
%				\end{tabular}
%			\end{minipage}
%			\begin{minipage}{.35\textwidth}
%				\begin{tabular}{cc|c}
%				$\metaX$&$\metaY$&$\metaX\eand\metaY$\\\hline
%				T&T&T\\
%				T&F&F\\
%				F&T&F\\
%				F&F&F
%				\end{tabular}
%			\end{minipage}
%			\begin{minipage}{.35\textwidth}
%				\begin{tabular}{cc|c}
%				$\metaX$&$\metaY$&$\metaX\eor\metaY$\\\hline
%				T&T&T\\
%				T&F&T\\
%				F&T&T\\
%				F&F&F
%				\end{tabular}
%			\end{minipage}
%			
%			\medskip 
%			\begin{minipage}{.35\textwidth}
%				\begin{tabular}{cc|c}
%				$\metaX$&$\metaY$&$\metaX\eif\metaY$\\\hline
%				T&T&T\\
%				T&F&F\\
%				F&T&T\\
%				F&F&T
%				\end{tabular}
%			\end{minipage}
%			\begin{minipage}{.35\textwidth}
%				\begin{tabular}{cc|c}
%				$\metaX$&$\metaY$&$\metaX\eiff\metaY$\\\hline
%				T&T&T\\
%				T&F&F\\
%				F&T&F\\
%				F&F&T
%				\end{tabular}
%			\end{minipage}
%	\end{center}
\begin{center}
			\begin{minipage}{.25\textwidth}$\enot$:\,
				\begin{tabular}{c@{ $\leadsto$ }c}
%				$\metaX$&$\enot\metaX$\\\hline
				T&F\\
				F&T
				\end{tabular}
			\end{minipage}
			\begin{minipage}{.35\textwidth}$\eand$:\,
				\begin{tabular}{c@{, }c@{ $\leadsto$ }c}
%				$\metaX$&$\metaY$&$\metaX\eand\metaY$\\\hline
				T&T&T\\
				T&F&F\\
				F&T&F\\
				F&F&F
				\end{tabular}
			\end{minipage}
			\begin{minipage}{.35\textwidth}$\eor$:\,
				\begin{tabular}{c@{, }c@{ $\leadsto$ }c}
%				$\metaX$&$\metaY$&$\metaX\eor\metaY$\\\hline
				T&T&T\\
				T&F&T\\
				F&T&T\\
				F&F&F
				\end{tabular}
			\end{minipage}
			
			\medskip 
			\begin{minipage}{.35\textwidth}$\eif$:\,
				\begin{tabular}{c@{, }c@{ $\leadsto$ }c}
%				$\metaX$&$\metaY$&$\metaX\eif\metaY$\\\hline
				T&T&T\\
				T&F&F\\
				F&T&T\\
				F&F&T
				\end{tabular}
			\end{minipage}
%			\begin{minipage}{.35\textwidth}$\eiff$:\,
%				\begin{tabular}{c@{, }c@{ $\leadsto$ }c}
%%				$\metaX$&$\metaY$&$\metaX\eiff\metaY$\\\hline
%				T&T&T\\
%				T&F&F\\
%				F&T&F\\
%				F&F&T
%				\end{tabular}
%			\end{minipage}
	\end{center}


\vspace*{2cm}
\section{Symbolization}
\subsubsection*{Rough Meaning of the TFL Connectives}
		\begin{tabular}{cll}
		\textbf{symbol}&\textbf{name}&\textbf{rough meaning}\\
		\hline
		\enot&negation&`It is not the case that$\ldots$'\\
		\eand&conjunction&`$\ldots$\ and $\ldots$'\\
		\eor&disjunction&`$\ldots$\ or $\ldots$'\\
		\eif&conditional&`If $\ldots$\ then $\ldots$'\\
		\eiff&biconditional&`$\ldots$ if and only if $\ldots$'\\
		\end{tabular}
		
		

\label{app.symbolization}
\subsubsection*{Sentential Connectives}
\begin{center}\begin{tabular*}{\textwidth}{ll}
It is not the case that P & $\enot P$\\
P or Q & $(P \eor Q)$\\
P and Q & $(P \eand Q)$\\
If P, then Q & $(P \eif Q)$\\
P if and only if Q & $(P \eiff Q)$\\
\end{tabular*}\end{center}Further symbolisation help:\begin{center}
\begin{tabular*}{\textwidth}{ll}
Neither P nor Q & $\enot(P \eor Q)$\ or \ $(\enot P \eand \enot Q)$\\
P but Q & $(P \eand Q)$\\
P unless Q & $(P \eor Q)$\\
P only if Q & $(P \eif Q)$
\end{tabular*}
\end{center}
\subsubsection*{Predicates}
\begin{center}
\begin{tabular*}{\textwidth}{ll}\label{SymbolizingPredicates}
All Fs are Gs & $\forall x(Fx \eif Gx)$\\
Some Fs are Gs & $\exists x(Fx \eand Gx)$\\
Not all Fs are Gs & $\enot\forall x(Fx \eif Gx)$\ or\ $\exists x(Fx \eand \enot Gx)$\\
No Fs are Gs & $\forall x(Fx \eif\enot Gx)$\ or\ $\enot\exists x(Fx \eand Gx)$\\
\end{tabular*}
\end{center}
\subsubsection*{Identity}
\begin{center}
\begin{tabular*}{\textwidth}{ll}
Only c is G & $\forall x(Gx \eif x\eid c)$ or perhaps $\eiff$.  \\
Everything besides c is G & $\forall x(\enot \,x \eid  c \eif Gx)$\\
%$j$ is more $R$ than anyone else. & $\forall x(xotEeq j \eif Rjx)$\\
The F is G & $\exists x(Fx \eand \forall y(Fy \eif x\eid y) \eand Gx)$\\
It is not the case that the F is G & $\enot\exists x(Fx \eand \forall y(Fy \eif x\eid y) \eand Gx)$\\
The F is non-G & $\exists x(Fx \eand \forall y(Fy \eif x\eid y) \eand \enot Gx)$
\end{tabular*}
\end{center}






% BEGIN: symbolizing cardinality

\section{Using identity to symbolize quantities}

\subsection*{There are at least \blank\ Fs.}
\label{summary.atleast}

\begin{ekey}
\item[\text{one}] $\exists xFx$
\item[\text{two}] $\exists x_1\exists x_2(Fx_1 \eand Fx_2 \eand \enot x_1  \eid  x_2)$
\item[\text{three}] $\exists x_1\exists x_2\exists x_3(Fx_1 \eand Fx_2 \eand Fx_3 \eand \enot x_1 \eid  x_2 \eand\enot x_1 \eid  x_3 \eand \enot x_2 \eid  x_3)$
\item[\text{four}] $\exists x_1\exists x_2\exists x_3\exists x_4 (Fx_1 \eand Fx_2 \eand Fx_3 \eand Fx_4 \eand \phantom{x}$\\
\phantom{$\exists x_1\exists x_2$}$\enot x_1 \eid  x_2 \eand \enot x_1 \eid  x_3 \eand \enot x_1 \eid  x_4 \eand \enot x_2 \eid  x_3 \eand \enot x_2 \eid  x_4 \eand \enot x_3 \eid  x_4)$
\item[n] $\exists x_1\ldots\exists x_n(Fx_1 \eand\ldots\eand Fx_n \eand \enot x_1 \eid  x_2 \eand\ldots\eand \enot x_{n-1} \eid  x_n)$ 
\end{ekey}

\subsubsection*{There are at most \blank\ Fs.}
\label{summary.atmost}

One way to say `there are at most $n$ Fs' is to put a negation sign in front of the symbolization for `there are at least $n+1$ Fs'. Equivalently, we can offer:
\begin{ekey}
\item[\text{one}] $\forall x_1\forall x_2\bigl[(Fx_1 \eand Fx_2) \eif x_1\eid x_2\bigr]$
\item[\text{two}] $\forall x_1\forall x_2\forall x_3\bigl[(Fx_1 \eand Fx_2 \eand Fx_3) \eif (x_1\eid x_2 \eor x_1\eid x_3 \eor x_2\eid x_3)\bigr]$
\item[\text{three}] $\forall x_1\forall x_2\forall x_3\forall x_4\bigl[(Fx_1 \eand Fx_2 \eand Fx_3 \eand Fx_4) \eif \phantom{.}$\\
\phantom{$\exists x_1 \exists x_2$}$(x_1\eid x_2 \eor x_1\eid x_3 \eor x_1\eid x_4 \eor x_2\eid x_3 \eor x_2\eid x_4 \eor x_3\eid x_4)\bigr]$
\item[n]$\forall x_1\ldots\forall x_{n+1}
\bigl[(Fx_1\eand \ldots \eand Fx_{n+1}) \eif (x_1\eid x_2 \eor \ldots \eor x_n\eid x_{n+1})\bigr]$ 
\end{ekey}

\subsubsection*{There are exactly \blank\ Fs.}
\label{summary.exactly}

One way to say `there are exactly $n$ Fs' is to conjoin two of the symbolizations above and say `there are at least $n$ Fs and there are at most $n$ Fs.' The following equivalent formulae are shorter:
\begin{ekey}
\item[\text{zero}] $\forall x\enot Fx$
\item[\text{one}] $\exists x\bigl[Fx \eand \forall y(Fy \eif x\eid  y)\bigr]$
\item[\text{two}] $\exists x_1\exists x_2\bigl[Fx_1 \eand Fx_2 \eand \enot x_1 \eid  x_2 \eand \forall y\bigl(Fy \eif (y\eid  x_1 \eor y \eid  x_2)\bigr) \bigr]$
\item[\text{three}] $\exists x_1\exists x_2\exists x_3\bigl[Fx_1 \eand Fx_2 \eand Fx_3 \eand \enot x_1 \eid   x_2 \eand \enot  x_1 \eid  x_3 \eand \enot x_2 \eid  x_3 \eand \phantom{.}$\\
\phantom{$\exists x_1 \exists x_2$}$\forall y\bigl(Fy \eif (y \eid  x_1 \eor y \eid  x_2 \eor y \eid   x_3)\bigr) \bigr]$
\item[n] $\exists x_1\ldots\exists x_n\bigl[Fx_1 \eand\ldots\eand Fx_n  \eand \enot x_1 \eid  x_2 \eand\ldots\eand \enot x_{n-1}\eid  x_n \eand \phantom{.}$\\
\phantom{$\exists x_1\exists x_2$}$\forall y\bigl(Fy \eif (y\eid  x_1 \eor \ldots \eor y\eid  x_n)\bigr)\bigr]$ 
%\item[one] $\exists x\forall y\bigl[Fx \eand (Fy \eif y \eid  x)\bigr]$
%\item[two] $\exists x\exists y\forall z\Bigl(Fx \eand Fy \eand \bigl[Fz \eif (z\eid x \eor z\eid y)\bigr] \eand x \neq y\Bigr)$
%\item[three] $\exists x_1\exists x_2\exists x_3\forall y\Bigl(Fx_1 \eand Fx_2 \eand Fx_3 \eand [Fy \eif (y\eid x_1 \eor y\eid x_2 \eor y\eid x_3)] \eand x_1 \neq x_2 \eand x_1 \neq x_3 \eand x_2 \neq x_3\Bigr)$
%\item[n] $\exists x_1\cdots\exists x_n\forall y\Bigl(Fx_1 \eand\cdots\eand Fx_n \eand \bigl[Fy \eif (y\eid x_1 \eor \cdots \eor y\eid x_n)\bigr] \eand x_1 \neq x_2 \eand\cdots\eand x_{n-1}\neq x_n\Bigr)$ 
\end{ekey}


\label{ProofRules}
\newpage
\section{Basic deduction rules for TFL}\label{glossary:basic rules TFL}
\renewenvironment{fitchproof}
	{\noindent\par\noindent\small$\begin{nd}}
	{\end{nd}$\noindent\normalsize\ignorespacesafterend}
	
\vspace{-5pt}
%{\LARGE \textbf{Basic Rules of Proof}}
\newcommand{\separatepfs}{\vspace{-10pt}\hrulefill\vspace{-5pt}}
% Conjunction section with two-column layout
\begin{multicols}{2}
\subsubsection*{Conjunction}

\begin{fitchproof}
	\have[m]{x}{\metaX}
	\have[n]{y}{\metaY}
	\have[\ ]{xy}{\metaX \eand \metaY} \andI{x,y}
\end{fitchproof}

\vfill\null
\columnbreak

\begin{fitchproof}
	\have[m]{xy}{\metaX \eand \metaY}
	\have[\ ]{x}{\metaX} \andE{xy}
\end{fitchproof}
\smallskip
\begin{fitchproof}
	\have[m]{xy}{\metaX \eand \metaY}
	\have[\ ]{y}{\metaY} \andE{xy}
\end{fitchproof}
\end{multicols}

\separatepfs
% Disjunction section with two-column layout
\begin{multicols}{2}
\subsubsection*{Disjunction}

\begin{fitchproof}
	\have[m]{x}{\metaX}
	\have[\ ]{xy}{\metaX \eor \metaY}\orI{x}
\end{fitchproof}
\smallskip
\begin{fitchproof}
	\have[m]{x}{\metaX}
	\have[\ ]{xy}{\metaY \eor \metaX}\orI{x}
\end{fitchproof}

\vfill\null
\columnbreak

\begin{fitchproof}
	\have[m]{xy}{\metaX \eor \metaY}
	\open
		\hypo[i]{x}{\metaX}
		\ellipsesline
		\have[j]{z1}{\metaZ}
	\close
	\open
		\hypo[k]{y}{\metaY}
		\ellipsesline
		\have[l]{z2}{\metaZ}
	\close
	\have[\ ]{z}{\metaZ} \orE{xy,x-z1,y-z2}
\end{fitchproof}
\end{multicols}

\separatepfs
% Conditional section with two-column layout
\begin{multicols}{2}
\subsubsection*{Conditional}

\begin{fitchproof}
	\open
		\hypo[m]{x}{\metaX}
		\ellipsesline
		\have[n]{y}{\metaY}
	\close
	\have[\ ]{xy}{\metaX \eif \metaY}\ifI{x-y}
\end{fitchproof}

\vfill\null
\columnbreak

\begin{fitchproof}
	\have[m]{xy}{\metaX \eif \metaY}
	\have[n]{x}{\metaX}
	\have[\ ]{y}{\metaY} \ifE{xy,x}
\end{fitchproof}
\end{multicols}

\separatepfs
% Negation section with two-column layout
\begin{multicols}{2}
\subsubsection*{Negation}

\begin{fitchproof}
\open
	\hypo[m]{x}{\metaX}
	\ellipsesline
	\have[k]{red}{\metaY\eand\enot\metaY}
\close
\have[\ ]{nx}{\enot\metaX}\notI{x-red}
\end{fitchproof}

\columnbreak

\begin{fitchproof}
\open
	\hypo[m]{nx}{\enot\metaX}
	\ellipsesline
	\have[k]{red}{\metaY\eand\enot\metaY}
\close
\have[\ ]{x}{\metaX}\notE{nx-red}
\end{fitchproof}
\end{multicols}

\separatepfs
\begin{multicols}{2}
	\subsubsection*{Biconditional}
	
	
	\begin{fitchproof}
		\ellipsesline
		\have[m]{a}{(\metaX\eif\metaY)\eand (\metaY\eif\metaX)}
		\ellipsesline
		\have[\ ]{ab}{\metaX\eiff\metaY}\iffI{a}
	\end{fitchproof}
	
	\vfill\null
	\columnbreak
	
	
	
	\begin{fitchproof}
		\ellipsesline
		\have[m]{ab}{\metaX\eiff\metaY}
		\ellipsesline
		\have[\ ]{b}{(\metaX\eif\metaY)\eand(\metaY\eiff\metaX)}\iffE{ab}
	\end{fitchproof}
\end{multicols}

\separatepfs
% Reiteration and Law of Excluded Middle section with two-column layout
\begin{multicols}{2}
\subsubsection*{Reiteration}

\begin{fitchproof}
	\have[m]{x}{\metaX}
	\have[\ ]{x2}{\metaX} \Reiteration{x}
\end{fitchproof}

\subsubsection*{Law of Excluded Middle}
\begin{fitchproof}
	\have[\ ]{lem}{\metaX \eor \enot\metaX}\LEM
\end{fitchproof}
\end{multicols}



\newpage
\section{Derived rules for TFL}
\begin{multicols}{2}
\subsubsection*{Disjunctive syllogism}
\begin{fitchproof}
	\have[m]{ab}{\metaX \eor \metaY}
	\have[n]{nb}{\enot \metaX}
	\have[\ ]{con}{\metaY}\by{DS}{ab, nb}

	\have[m]{ab}{\metaX \eor \metaY}
\\	\have[n]{nb}{\enot \metaY}
	\have[\ ]{con}{\metaX}\by{DS}{ab, nb}
\end{fitchproof}

%\subsubsection*{Reiteration}
%
%\begin{fitchproof}
%	\have[m]{a}{\metaX}
%	\have[\ ]{c}{\metaX} \by{R}{a}
%\end{fitchproof}

\subsubsection*{Modus Tollens}

\begin{fitchproof}
	\have[m]{ab}{\metaX\eif\metaY}
	\have[n]{a}{\enot\metaY}
	\have[\ ]{b}{\enot\metaX} \by{MT}{ab,a}
\end{fitchproof}

\subsubsection*{Double-negation elimination}
	\begin{fitchproof}
		\have[m]{dna}{\enot \enot \metaX}
		\have[ ]{a}{\metaX}\dne{dna}
	\end{fitchproof}


%\subsubsection*{Proof by Contradiction}
%	\begin{fitchproof}
%		\open
%			\hypo[m]{na}{\enot\metaX}
%			\ellipsesline
%			\have[n]{red}{\ered}
%		\close
%		\have[ ]{a}{\metaX}\by{IP}{na-red}
%	\end{fitchproof}



\subsubsection*{Explosion}
	\begin{fitchproof}
	\have[m]{ab}{\metaX}
	\have[n]{a}{\enot\metaX}
	\have[\ ]{b}{\metaY} \by{Explosion}{ab,a}
	\end{fitchproof}


%
%\subsubsection*{Hypothetical Syllogism}
%
%\begin{fitchproof}
%	\have[m]{ab}{\metaX\eif\metaY}
%	\have[n]{bc}{\metaY\eif\metaZ}
%	\have[\ ]{ac}{\metaX\eif\metaZ}\by{HS}{ab,bc}
%\end{fitchproof}

\subsubsection*{De Morgan Rules}
\begin{fitchproof}
	\have[m]{ab}{\enot (\metaX \eor \metaY)}
	\have[\ ]{dm}{\enot \metaX \eand \enot \metaY}\dem{ab}

	\have[m]{ab}{\enot \metaX \eand \enot \metaY}
\\	\have[\ ]{dm}{\enot (\metaX \eor \metaY)}\dem{ab}

	\have[m]{ab}{\enot (\metaX \eand \metaY)}
\\	\have[\ ]{dm}{\enot \metaX \eor \enot \metaY}\dem{ab}

	\have[m]{ab}{\enot \metaX \eor \enot \metaY}
\\	\have[\ ]{dm}{\enot (\metaX \eand \metaY)}\dem{ab}
\end{fitchproof}
\end{multicols}

\newpage

\section{Basic deduction rules for FOL}

\begin{multicols}{2}
\subsubsection*{Universal elimination}

\begin{fitchproof}
	\have[m]{a}{\forall \meta{x}\metaX(\ldots \meta{x} \ldots \meta{x}\ldots)}
	\have[\ ]{c}{\metaX(\ldots \meta{c} \ldots \meta{c}\ldots)} \forallE{a}
\end{fitchproof}

\subsubsection*{Universal introduction}

\begin{fitchproof}
	\have[m]{a}{\metaX(\ldots \meta{c} \ldots \meta{c}\ldots)}
	\have[\ ]{c}{\forall \meta{x}\metaX(\ldots \meta{x} \ldots \meta{x}\ldots)} \forallI{a}
\end{fitchproof}

\noindent 	\meta{c} must not occur in any undischarged assumption\\ 
\meta{x} must not occur in $\metaX(\ldots \meta{c} \ldots \meta{c}\ldots)$

%\vfill

\subsubsection*{Existential introduction}

\begin{fitchproof}
	\have[m]{a}{\metaX(\ldots \meta{c} \ldots \meta{c}\ldots)}
	\have[\ ]{c}{\exists \meta{x}\metaX(\ldots \meta{x} \ldots \meta{c}\ldots)} \Ei{a}
\end{fitchproof}

\noindent \meta{x} must not occur in $\metaX(\ldots \meta{c} \ldots \meta{c}\ldots)$
%\noindent You can replace one or more instance of \meta{c} with \meta{x}.

\subsubsection*{Existential elimination}

\begin{fitchproof}
	\have[m]{a}{\exists \meta{x}\metaX(\ldots \meta{x} \ldots \meta{x}\ldots)}
	\open	
		\hypo[i]{b}{\metaX(\ldots \meta{c} \ldots \meta{c}\ldots)}
		\ellipsesline
		\have[j]{c}{\metaY}
	\close
	\have[\ ]{d}{\metaY} \Ee{a,b-c}
\end{fitchproof}

\noindent \meta{c} must not occur in \\any undischarged assumption, \\in $\exists \meta{x}\metaX(\ldots \meta{x} \ldots \meta{x}\ldots)$, \\or in \metaY
%\vfill\columnbreak
\end{multicols}

\bigskip 

\subsubsection*{Identity introduction}

\begin{fitchproof}
	\have[\ \,\,\,]{x}{\meta{c}\eid \meta{c}} \by{= I}{}
\end{fitchproof}


\subsubsection*{Identity elimination}

\begin{multicols}{2}
\begin{fitchproof}
	\have[m]{e}{\meta{a}\eid \meta{b}}
	\have[n]{a}{\metaX(\ldots \meta{a} \ldots \meta{a}\ldots)}
	\have[\ ]{ea1}{\metaX(\ldots \meta{b} \ldots \meta{a}\ldots)} \by{= E}{e,a}
\end{fitchproof}
\begin{fitchproof}
	\have[m]{e}{\meta{a}\eid \meta{b}}
	\have[n]{a}{\metaX(\ldots \meta{b} \ldots \meta{b}\ldots)}
	\have[\ ]{ea2}{\metaX(\ldots \meta{a} \ldots \meta{b}\ldots)} \by{= E}{e,a}
\end{fitchproof}
\end{multicols}


\bigskip 
\section{Derived rules for FOL}
\begin{multicols}{2}
\begin{fitchproof}
	\have[m]{ab}{\forall \meta{x}\enot \metaX}
	\have[\ ]{ac}{\enot \exists \meta{x} \metaX}\cq{ab}

	\have[m]{ab}{\enot \exists \meta{x}  \metaX}
\\	\have[\ ]{ac}{\forall \meta{x}\enot\metaX}\cq{ab}
\end{fitchproof}
\begin{fitchproof}
	\have[m]{ab}{\exists \meta{x}\enot\metaX}
	\have[\ ]{ac}{\enot \forall \meta{x} \metaX}\cq{ab}

	\have[m]{ab}{\enot \forall \meta{x}  \metaX}
\\	\have[\ ]{ac}{\exists \meta{x}\enot \metaX}\cq{ab}
\end{fitchproof}
\end{multicols}

