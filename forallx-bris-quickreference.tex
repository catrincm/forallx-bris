%!TEX root = forallxbris.tex
\chapter{Quick reference}\label{glossary:quick ref}
%\pagestyle{plain}
\section{Characteristic Truth Tables}
\label{app.CharacteristicTTs}

\begin{tabular}{c|c}
\meta{A} & \enot\meta{A}\\
\hline
T & F\\
F & T \\
\end{tabular}
\hfill
\begin{tabular}{c c|c|c|c|c}
\meta{A} & \meta{B} & $\meta{A}\eand\meta{B}$ & $\meta{A}\eor\meta{B}$ & $\meta{A}\eif\meta{B}$ & $\meta{A}\eiff\meta{B}$\\
\hline
T & T & T & T & T & T\\
T & F & F & T & F & F\\
F & T & F & T & T & F\\
F & F & F & F & T & T
\end{tabular}


\vfill

\section{Symbolization}

\label{app.symbolization}
\subsubsection*{Sentential Connectives}
\begin{center}
\begin{tabular*}{\textwidth}{ll}
It is not the case that P & $\enot P$\\
P or Q & $(P \eor Q)$\\
Neither P nor Q & $\enot(P \eor Q)$\ or \ $(\enot P \eand \enot Q)$\\
Both P and Q & $(P \eand Q)$\\
If P, then Q & $(P \eif Q)$\\
P only if Q & $(P \eif Q)$\\
P if and only if Q & $(P \eiff Q)$\\
P unless Q & $(P \eor Q)$\\
\end{tabular*}
\end{center}
\subsubsection*{Predicates}
\begin{center}
\begin{tabular*}{\textwidth}{ll}\label{SymbolizingPredicates}
All Fs are Gs & $\forall x(Fx \eif Gx)$\\
Some Fs are Gs & $\exists x(Fx \eand Gx)$\\
Not all Fs are Gs & $\enot\forall x(Fx \eif Gx)$\ or\ $\exists x(Fx \eand \enot Gx)$\\
No Fs are Gs & $\forall x(Fx \eif\enot Gx)$\ or\ $\enot\exists x(Fx \eand Gx)$\\
\end{tabular*}
\end{center}
\subsubsection*{Identity}
\begin{center}
\begin{tabular*}{\textwidth}{ll}
Only c is G & $\forall x(Gx \eif x\eid c)$ or perhaps $\eiff$.  \\
Everything besides c is G & $\forall x(\enot \,x \eid  c \eif Gx)$\\
%$j$ is more $R$ than anyone else. & $\forall x(x\neq j \eif Rjx)$\\
The F is G & $\exists x(Fx \eand \forall y(Fy \eif x\eid y) \eand Gx)$\\
It is not the case that the F is G & $\enot\exists x(Fx \eand \forall y(Fy \eif x\eid y) \eand Gx)$\\
The F is non-G & $\exists x(Fx \eand \forall y(Fy \eif x=y) \eand \enot Gx)$
\end{tabular*}
\end{center}






% BEGIN: symbolizing cardinality

\newpage
\section{Using identity to symbolize quantities}

\subsection*{There are at least \blank\ Fs.}
\label{summary.atleast}

\begin{ekey}
\item[\text{one}] $\exists xFx$
\item[\text{two}] $\exists x_1\exists x_2(Fx_1 \eand Fx_2 \eand \enot x_1  = x_2)$
\item[\text{three}] $\exists x_1\exists x_2\exists x_3(Fx_1 \eand Fx_2 \eand Fx_3 \eand \enot x_1 = x_2 \eand\enot x_1 = x_3 \eand \enot x_2 = x_3)$
\item[\text{four}] $\exists x_1\exists x_2\exists x_3\exists x_4 (Fx_1 \eand Fx_2 \eand Fx_3 \eand Fx_4 \eand \phantom{x}$\\
\phantom{$\exists x_1\exists x_2$}$\enot x_1 = x_2 \eand \enot x_1 = x_3 \eand \enot x_1 = x_4 \eand \enot x_2 = x_3 \eand \enot x_2 = x_4 \eand \enot x_3 = x_4)$
\item[n] $\exists x_1\ldots\exists x_n(Fx_1 \eand\ldots\eand Fx_n \eand \enot x_1 = x_2 \eand\ldots\eand \enot x_{n-1} = x_n)$ 
\end{ekey}

\subsubsection*{There are at most \blank\ Fs.}
\label{summary.atmost}

One way to say `there are at most $n$ Fs' is to put a negation sign in front of the symbolization for `there are at least $n+1$ Fs'. Equivalently, we can offer:
\begin{ekey}
\item[\text{one}] $\forall x_1\forall x_2\bigl[(Fx_1 \eand Fx_2) \eif x_1=x_2\bigr]$
\item[\text{two}] $\forall x_1\forall x_2\forall x_3\bigl[(Fx_1 \eand Fx_2 \eand Fx_3) \eif (x_1=x_2 \eor x_1=x_3 \eor x_2=x_3)\bigr]$
\item[\text{three}] $\forall x_1\forall x_2\forall x_3\forall x_4\bigl[(Fx_1 \eand Fx_2 \eand Fx_3 \eand Fx_4) \eif \phantom{.}$\\
\phantom{$\exists x_1 \exists x_2$}$(x_1=x_2 \eor x_1=x_3 \eor x_1=x_4 \eor x_2=x_3 \eor x_2=x_4 \eor x_3=x_4)\bigr]$
\item[n]$\forall x_1\ldots\forall x_{n+1}
\bigl[(Fx_1\eand \ldots \eand Fx_{n+1}) \eif (x_1=x_2 \eor \ldots \eor x_n=x_{n+1})\bigr]$ 
\end{ekey}

\subsubsection*{There are exactly \blank\ Fs.}
\label{summary.exactly}

One way to say `there are exactly $n$ Fs' is to conjoin two of the symbolizations above and say `there are at least $n$ Fs and there are at most $n$ Fs.' The following equivalent formulae are shorter:
\begin{ekey}
\item[\text{zero}] $\forall x\enot Fx$
\item[\text{one}] $\exists x\bigl[Fx \eand \forall y(Fy \eif x= y)\bigr]$
\item[\text{two}] $\exists x_1\exists x_2\bigl[Fx_1 \eand Fx_2 \eand \enot x_1 = x_2 \eand \forall y\bigl(Fy \eif (y= x_1 \eor y = x_2)\bigr) \bigr]$
\item[\text{three}] $\exists x_1\exists x_2\exists x_3\bigl[Fx_1 \eand Fx_2 \eand Fx_3 \eand \enot x_1 =  x_2 \eand \enot  x_1 = x_3 \eand \enot x_2 = x_3 \eand \phantom{.}$\\
\phantom{$\exists x_1 \exists x_2$}$\forall y\bigl(Fy \eif (y = x_1 \eor y = x_2 \eor y =  x_3)\bigr) \bigr]$
\item[n] $\exists x_1\ldots\exists x_n\bigl[Fx_1 \eand\ldots\eand Fx_n  \eand \enot x_1 = x_2 \eand\ldots\eand \enot x_{n-1}= x_n \eand \phantom{.}$\\
\phantom{$\exists x_1\exists x_2$}$\forall y\bigl(Fy \eif (y= x_1 \eor \ldots \eor y= x_n)\bigr)\bigr]$ 
%\item[one] $\exists x\forall y\bigl[Fx \eand (Fy \eif y = x)\bigr]$
%\item[two] $\exists x\exists y\forall z\Bigl(Fx \eand Fy \eand \bigl[Fz \eif (z=x \eor z=y)\bigr] \eand x \neq y\Bigr)$
%\item[three] $\exists x_1\exists x_2\exists x_3\forall y\Bigl(Fx_1 \eand Fx_2 \eand Fx_3 \eand [Fy \eif (y=x_1 \eor y=x_2 \eor y=x_3)] \eand x_1 \neq x_2 \eand x_1 \neq x_3 \eand x_2 \neq x_3\Bigr)$
%\item[n] $\exists x_1\cdots\exists x_n\forall y\Bigl(Fx_1 \eand\cdots\eand Fx_n \eand \bigl[Fy \eif (y=x_1 \eor \cdots \eor y=x_n)\bigr] \eand x_1 \neq x_2 \eand\cdots\eand x_{n-1}\neq x_n\Bigr)$ 
\end{ekey}


\label{ProofRules}
\newpage\section{Basic deduction rules for TFL}\label{glossary:basic rules TFL}
\renewenvironment{proof}
	{\noindent\par\noindent\small$\begin{nd}}
	{\end{nd}$\noindent\normalsize\ignorespacesafterend}
	

%{\LARGE \textbf{Basic Rules of Proof}}
\begin{multicols}{2}
\subsubsection*{Reiteration}

\begin{proof}
	\have[m]{a}{\meta{A}}
	\have[\ ]{c}{\meta{A}} \by{R}{a}
\end{proof}


\subsubsection*{Conjunction}

\begin{proof}
	\have[m]{a}{\meta{A}}
	\have[n]{b}{\meta{B}}
	\have[\ ]{c}{\meta{A}\eand\meta{B}} \ai{a, b}

	\have[m]{ab}{\meta{A}\eand\meta{B}}
\\	\have[\ ]{a}{\meta{A}} \ae{ab}

	\have[m]{ab}{\meta{A}\eand\meta{B}}
\\	\have[\ ]{b}{\meta{B}} \ae{ab}
\end{proof}

\subsubsection*{Conditional}

\begin{proof}
	\open
		\hypo[m]{a}{\meta{A}}
		\ellipsesline
		\have[n]{b}{\meta{B}}
	\close
	\have[\ ]{ab}{\meta{A}\eif\meta{B}}\ci{a-b}

	\have[m]{ab}{\meta{A}\eif\meta{B}}
\\	\have[n]{a}{\meta{A}}
	\have[\ ]{b}{\meta{B}} \ce{ab,a}
\end{proof}

\subsubsection*{Biconditional}

\begin{proof}
\have[m]{ab}{\meta{A}\eiff\meta{B}}
\have[\ ]{b}{(\meta{A}\eif\meta{B})\eand(\meta{B}\eif\meta{A})}\iffE{ab}

\have[m]{b}{(\meta{A}\eif\meta{B})\eand(\meta{B}\eif\meta{A})}
\\\have[\ ]{ab}{\meta{A}\eiff\meta{B}}\iffI{b}
\end{proof}

%\begin{proof}
%	\open
%		\hypo[i]{a1}{\meta{A}} 
%		\have[j]{b1}{\meta{B}}
%	\close
%	\open
%		\hypo[k]{b2}{\meta{B}}
%		\have[l]{a2}{\meta{A}}
%	\close
%	\have[\ ]{ab}{\meta{A}\eiff\meta{B}}\bi{a1-b1,b2-a2}
%
%	\have[m]{ab}{\meta{A}\eiff\meta{B}}
%\\	\have[n]{a}{\meta{A}}
%	\have[\ ]{b}{\meta{B}} \be{ab,a}
%
%	\have[m]{ab}{\meta{A}\eiff\meta{B}}
%\\	\have[n]{a}{\meta{B}}
%	\have[\ ]{b}{\meta{A}} \be{ab,a}
%\end{proof}


\subsubsection*{Contradiction}

\begin{proof}
\have[m]{a}{\meta{A}}
\have[n]{na}{\enot\meta{A}}
\have[ ]{bot}{\ered}\ri{a, na}

\have[m]{bot}{\ered}
\\\have[ ]{}{\meta{A}}\re{bot}
\end{proof}


\subsubsection*{Negation}
\begin{proof}
\open
	\hypo[m]{a}{\meta{A}}
	\ellipsesline
	\have[n]{nb}{\ered}
\close
\have[\ ]{na}{\enot\meta{A}}\ni{a-nb}
\end{proof}



\subsubsection*{Disjunction}

\begin{proof}
	\have[m]{a}{\meta{A}}
	\have[\ ]{ab}{\meta{A}\eor\meta{B}}\oi{a}

	\have[m]{a}{\meta{A}}
\\	\have[\ ]{ba}{\meta{B}\eor\meta{A}}\oi{a}

	\have[m]{ab}{\meta{A}\eor\meta{B}}
\\	\open
		\hypo[i]{a}{\meta{A}}
		\ellipsesline
		\have[j]{c1}{\meta{C}}
	\close
	\open
		\hypo[k]{b}{\meta{B}}
		\ellipsesline
		\have[l]{c2}{\meta{C}}
	\close
	\have[\ ]{c}{\meta{C}} \oe{ab,a-c1, b-c2}
\end{proof}

\subsubsection*{Law of Excluded Middle}
\begin{proof}
	\have[\ ]{lem}{\meta{A}\eor\enot\meta{A}}\LEM
\end{proof}


\end{multicols}

\newpage
\section{Derived rules for TFL}
\begin{multicols}{2}
\subsubsection*{Disjunctive syllogism}
\begin{proof}
	\have[m]{ab}{\meta{A} \eor \meta{B}}
	\have[n]{nb}{\enot \meta{A}}
	\have[\ ]{con}{\meta{B}}\by{DS}{ab, nb}

	\have[m]{ab}{\meta{A} \eor \meta{B}}
\\	\have[n]{nb}{\enot \meta{B}}
	\have[\ ]{con}{\meta{A}}\by{DS}{ab, nb}
\end{proof}

%\subsubsection*{Reiteration}
%
%\begin{proof}
%	\have[m]{a}{\meta{A}}
%	\have[\ ]{c}{\meta{A}} \by{R}{a}
%\end{proof}

\subsubsection*{Modus Tollens}

\begin{proof}
	\have[m]{ab}{\meta{A}\eif\meta{B}}
	\have[n]{a}{\enot\meta{B}}
	\have[\ ]{b}{\enot\meta{A}} \by{MT}{ab,a}
\end{proof}

\subsubsection*{Double-negation elimination}
	\begin{proof}
		\have[m]{dna}{\enot \enot \meta{A}}
		\have[ ]{a}{\meta{A}}\dne{dna}
	\end{proof}


\subsubsection*{Proof by Contradiction}
	\begin{proof}
		\open
			\hypo[m]{na}{\enot\meta{A}}
			\ellipsesline
			\have[n]{red}{\ered}
		\close
		\have[ ]{a}{\meta{A}}\by{IP}{na-red}
	\end{proof}



%
%\subsubsection*{Hypothetical Syllogism}
%
%\begin{proof}
%	\have[m]{ab}{\meta{A}\eif\meta{B}}
%	\have[n]{bc}{\meta{B}\eif\meta{C}}
%	\have[\ ]{ac}{\meta{A}\eif\meta{C}}\by{HS}{ab,bc}
%\end{proof}

\subsubsection*{De Morgan Rules}
\begin{proof}
	\have[m]{ab}{\enot (\meta{A} \eor \meta{B})}
	\have[\ ]{dm}{\enot \meta{A} \eand \enot \meta{B}}\dem{ab}

	\have[m]{ab}{\enot \meta{A} \eand \enot \meta{B}}
\\	\have[\ ]{dm}{\enot (\meta{A} \eor \meta{B})}\dem{ab}

	\have[m]{ab}{\enot (\meta{A} \eand \meta{B})}
\\	\have[\ ]{dm}{\enot \meta{A} \eor \enot \meta{B}}\dem{ab}

	\have[m]{ab}{\enot \meta{A} \eor \enot \meta{B}}
\\	\have[\ ]{dm}{\enot (\meta{A} \eand \meta{B})}\dem{ab}
\end{proof}
\end{multicols}

\newpage

\section{Basic deduction rules for FOL}

\begin{multicols}{2}
\subsubsection*{Universal elimination}

\begin{proof}
	\have[m]{a}{\forall \meta{x}\meta{A}(\ldots \meta{x} \ldots \meta{x}\ldots)}
	\have[\ ]{c}{\meta{A}(\ldots \meta{c} \ldots \meta{c}\ldots)} \Ae{a}
\end{proof}

\subsubsection*{Universal introduction}

\begin{proof}
	\have[m]{a}{\meta{A}(\ldots \meta{c} \ldots \meta{c}\ldots)}
	\have[\ ]{c}{\forall \meta{x}\meta{A}(\ldots \meta{x} \ldots \meta{x}\ldots)} \Ai{a}
\end{proof}

\noindent 	\meta{c} must not occur in any undischarged assumption\\ 
\meta{x} must not occur in $\meta{A}(\ldots \meta{c} \ldots \meta{c}\ldots)$

%\vfill

\subsubsection*{Existential introduction}

\begin{proof}
	\have[m]{a}{\meta{A}(\ldots \meta{c} \ldots \meta{c}\ldots)}
	\have[\ ]{c}{\exists \meta{x}\meta{A}(\ldots \meta{x} \ldots \meta{c}\ldots)} \Ei{a}
\end{proof}

\noindent \meta{x} must not occur in $\meta{A}(\ldots \meta{c} \ldots \meta{c}\ldots)$
%\noindent You can replace one or more instance of \meta{c} with \meta{x}.

\subsubsection*{Existential elimination}

\begin{proof}
	\have[m]{a}{\exists \meta{x}\meta{A}(\ldots \meta{x} \ldots \meta{x}\ldots)}
	\open	
		\hypo[i]{b}{\meta{A}(\ldots \meta{c} \ldots \meta{c}\ldots)}
		\ellipsesline
		\have[j]{c}{\meta{B}}
	\close
	\have[\ ]{d}{\meta{B}} \Ee{a,b-c}
\end{proof}

\noindent \meta{c} must not occur in \\any undischarged assumption, \\in $\exists \meta{x}\meta{A}(\ldots \meta{x} \ldots \meta{x}\ldots)$, \\or in \meta{B}
%\vfill\columnbreak
\end{multicols}

\bigskip 

\subsubsection*{Identity introduction}

\begin{proof}
	\have[\ \,\,\,]{x}{\meta{c}=\meta{c}} \by{=I}{}
\end{proof}


\subsubsection*{Identity elimination}

\begin{multicols}{2}
\begin{proof}
	\have[m]{e}{\meta{a}=\meta{b}}
	\have[n]{a}{\meta{A}(\ldots \meta{a} \ldots \meta{a}\ldots)}
	\have[\ ]{ea1}{\meta{A}(\ldots \meta{b} \ldots \meta{a}\ldots)} \by{=E}{e,a}
\end{proof}
\begin{proof}
	\have[m]{e}{\meta{a}=\meta{b}}
	\have[n]{a}{\meta{A}(\ldots \meta{b} \ldots \meta{b}\ldots)}
	\have[\ ]{ea2}{\meta{A}(\ldots \meta{a} \ldots \meta{b}\ldots)} \by{=E}{e,a}
\end{proof}
\end{multicols}


\bigskip 
\section{Derived rules for FOL}
\begin{multicols}{2}
\begin{proof}
	\have[m]{ab}{\forall \meta{x}\enot \meta{A}}
	\have[\ ]{ac}{\enot \exists \meta{x} \meta{A}}\cq{m}

	\have[m]{ab}{\enot \exists \meta{x}  \meta{A}}
\\	\have[\ ]{ac}{\forall \meta{x}\enot\meta{A}}\cq{m}
\end{proof}
\begin{proof}
	\have[m]{ab}{\exists \meta{x}\enot\meta{A}}
	\have[\ ]{ac}{\enot \forall \meta{x} \meta{A}}\cq{m}

	\have[m]{ab}{\enot \forall \meta{x}  \meta{A}}
\\	\have[\ ]{ac}{\exists \meta{x}\enot \meta{A}}\cq{m}
\end{proof}
\end{multicols}

