%!TEX root = forallxbris.tex
\part{Key notions of logic}
\label{ch.intro}
\addtocontents{toc}{\protect\mbox{}\protect\hrulefill\par}


\chapter{Arguments}
\label{s:Arguments}

Logic is the business of evaluating arguments; sorting the good from the bad. 

In everyday language, we sometimes use the word `argument' to talk about belligerent shouting matches.  If you and a friend have an argument in this sense, things are not going well between the two of you. Logic is not concerned with such teeth-gnashing and hair-pulling. They are not arguments, in our sense; they are disagreements.

An argument, as we will understand it, is something more like this:
	\begin{earg}\label{argRaining}
		\item[] It will rain today.
		\item[\therefore] You should take an umbrella.
	\end{earg}
We here have a series of sentences. The three dots on the third line of the argument are read `therefore.' They indicate that the final sentence expresses the \emph{conclusion} of the argument. The two sentences before that are the \emph{premises} of the argument. The premisses are supposed to support for the conclusion. If you believe the premises then the argument should lead you to believing the conclusion.



This is the sort of thing that logicians are interested in. We will say that an argument is any collection of premises, together with a conclusion. 

This Part discusses some basic logical notions that apply to arguments in a natural language like English. It is important to begin with a clear understanding of what arguments are and of what it means for an argument to be valid. Later we will translate arguments from English into a formal language. We want formal validity, as defined in the formal language, to have at least some of the important features of natural-language validity. 





When approaching an argument, we want to know whether or not the conclusion follows from the premises. So the first thing to do is to separate out the conclusion from the premises. The premises and arguments aren't always as nicely packaged up as they were in our rain argument. Sometimes a single sentence can contain a complete argument. Consider:
	\begin{quote}
		 I was wearing my sunglasses; so it must have been sunny.
	\end{quote}
This argument has one premise (I was wearing my sunglasses) followed by a conclusion (it was sunny). 

Many arguments start with premises, and end with a conclusion, but not all of them. The argument with which this section began might equally have been presented with the conclusion at the beginning, like so:
	\begin{quote}
		\underline{You should take an umbrella.} After all, it is raining heavily. And if you do not take an umbrella, you will get soaked. 
	\end{quote}
Equally, it might have been presented with the conclusion in the middle:
	\begin{quote}
		It is raining heavily. Accordingly, \underline{you should take an umbrella}, given that if you do not take an umbrella, you will get soaked.
	\end{quote}

 As a guideline, the following words are often used to indicate an argument's conclusion:
	\begin{center}
		so, therefore, hence, thus, accordingly, consequently
	\end{center}
Similarly, these expressions often indicate that we are dealing with a premise, rather than a conclusion:
	\begin{center}
		since, because, given that
	\end{center}



Sometimes it can take some work to pick out what the conclusion of the argument is. 
Consider: 
%\begin{center}
%You have already said that you love me and that you can't imagine spending the rest of your life without me. Once you even tried to propose to me. And now you claim that you need time to think about whether we should be married. Well, everything you've told me regarding our relationship has been a lie. In some of your letters to a friend you admitted that you were misleading me. You've been telling everyone that we are just friends, not lovers. And worst of all, you've been secretly dating someone else. Why are you doing this? It's all been a farce, and I'm outta here. 
%\citet{vaughn}
%\end{center}
\begin{quote}
Virgin would then dominate the rail system. Is that something the government should worry about? Not necessarily. The industry is regulated, and one powerful company might at least offer a more coherent schedule of services than the present arrangement has produced. The reason the industry was broken up into more than 100 companies at privatisation was not operational, but political: the Conservative government thought it would thus be harder to renationalise. \emph{The Economist 16.12.2000; used on \href{https://philosophy.hku.hk/think/arg/arg.php}{critical thinking web}}
\end{quote}
This is an argument that we shouldn't be worried if Virgin dominates the rail system. 

Often people aren't giving arguments but are simply presenting facts or stating their opinion.
For example, the following do not contain arguments, they're not trying to convince us of anything. 
\begin{itemize}
\item I don't like cats. They're evil.
\item Hundreds of vulnerable children as young as 10, who have spent most of their lives in the UK, are having their applications for British citizenship denied for failing to pass the government’s controversial `good character' test.
\end{itemize}
What about:\begin{quote}
 If the murder weapon was a gun, then Prof. Plum did it.
\end{quote}These conditional, or ``if-then'', statements might look like it expresses the argument, but in itself it does not. It's just stating a fact, albeit a conditional fact. It might be used in an argument:
\begin{earg}
\item If the murder weapon was a gun, then Prof.~Plum did it.
\item The murder weapon was a gun.
\item[\therefore] Prof.~Plum did it.
\end{earg}


In analysing an argument, there is no substitute for a good nose. 

\newglossaryentry{premise indicator word}
{
name=premise indicator,
description={a word or phrase such as ``because'' used to indicate that what follows is the premise of an argument}
}

\newglossaryentry{conclusion indicator word}
{
name=conclusion indicator,
description={a word or phrase such as ``therefore'' used to indicate that what follows is the conclusion of an argument}
}

\newglossaryentry{argument}
{
name=argument,
description={a connected series of sentences, divided into \gls{premise}s and \gls{conclusion}}
}

\newglossaryentry{premise}
{
name=premise,
description={a sentence in an \gls{argument} other than the \gls{conclusion}}
}

\newglossaryentry{conclusion}
{
name=conclusion,
description={the last sentence in an \gls{argument}}
}


\section{Sentences}
\label{intro.sentences}

To be perfectly general, we can define an \define{argument} as a series of sentences. The sentences at the beginning of the series are premises. The final sentence in the series is the conclusion. If the premises are true and the argument is a good one, then you have a reason to accept the conclusion.

In logic, we are only interested in sentences that can figure as a premise or conclusion of an argument. So we will say that a \define{sentence} is something that can be true or false.

You should not confuse the idea of a sentence that can be true or false with the difference between fact and opinion. Often, sentences in logic will express things that would count as facts--- such as `Kierkegaard was a hunchback' or `Kierkegaard liked almonds.' They can also express things that you might think of as matters of opinion--- such as, `Almonds are tasty.'

Also, there are things that would count as `sentences' in a linguistics or grammar course that we will not count as sentences in logic.

\paragraph{Questions} In a grammar class, `Are you sleepy yet?' would count as an interrogative sentence. Although you might be sleepy or you might be alert, the question itself is neither true nor false. For this reason, questions will not count as sentences in logic. Suppose you answer the question: `I am not sleepy.' This is either true or false, and so it is a sentence in the logical sense. Generally, \emph{questions} will not count as sentences, but \emph{answers} will. 

`What is this course about?' is not a sentence (in our sense). `No one knows what this course is about' is a sentence.

\paragraph{Imperatives} Commands are often phrased as imperatives like `Wake up!', `Sit up straight', and so on. In a grammar class, these would count as imperative sentences. Although it might be good for you to sit up straight or it might not, the command is neither true nor false. Note, however, that commands are not always phrased as imperatives. `You will respect my authority' \emph{is} either true or false--- either you will or you will not--- and so it counts as a sentence in the logical sense.

\paragraph{Exclamations} `Ouch!' is sometimes called an exclamatory sentence, but it is neither true nor false. We will treat `Ouch, I hurt my toe!' as meaning the same thing as `I hurt my toe.' The `ouch' does not add anything that could be true or false.


\practiceproblems
At the end of some chapters, there are problems that review and explore the material covered in the chapter. There is no substitute for actually working through some problems, because logic is more about a way of thinking than it is about memorizing facts.

\medskip

Highlight the phrase which expresses the conclusion of each of these arguments:
\begin{earg}
	\item It is sunny. So I should take my sunglasses.
	\item It must have been sunny. I did wear my sunglasses, after all.
	\item No one but you has had their hands in the cookie-jar. And the scene of the crime is littered with cookie-crumbs. You're the culprit!
	\item Miss Scarlett and Professor Plum were in the study at the time of the murder. Reverend Green had the candlestick in the ballroom, and we know that there is no blood on his hands. Hence Colonel Mustard did it in the kitchen with the lead-piping. Recall, after all, that the gun had not been fired.
	\item There are no hard numbers, but the evidence from Asia's expatriate community is unequivocal. Three years after its handover from Britain to China, Hong Kong is unlearning English. The city's gweilos (Cantonese for "ghost men") must go to ever greater lengths to catch the oldest taxi driver available to maximize their chances of comprehension. Hotel managers are complaining that they can no longer find enough English- speakers to act as receptionists. Departing tourists, polled at the airport, voice growing frustration at not being understood. \\ \emph{The Economist 20.1.2001}, used in \href{https://philosophy.hku.hk/think/arg/arg.php}{Critical Thinking Web}
\end{earg}


\chapter{Valid arguments}
\label{s:Valid}

In \S\ref{s:Arguments}, we gave a very permissive account of what an argument is. To see just how permissive it is, consider the following:
	\begin{earg}
		\item[] There is a bassoon-playing dragon in the \emph{Cathedra Romana}.
		\item[\therefore] Salvador Dali was a poker player.
	\end{earg}
We have been given a premise and a conclusion. So we have an argument. Admittedly, it is a \emph{terrible} argument, but it is still an argument.

\section{Two ways that arguments can go wrong}

It is worth pausing to ask what makes the argument so weak. In fact, there are two sources of weakness. First: the argument's (only) premise is obviously false. The Pope's throne is only ever occupied by a hat-wearing man. Second: the conclusion does not follow from the premise of the argument. Even if there were a bassoon-playing dragon in the Pope's throne, we would not be able to draw any conclusion about Dali's predilection for poker.

What about the main argument discussed in \S\ref{s:Arguments}? The premises of this argument might well be false. It might be sunny outside; or it might be that you can avoid getting soaked without taking an umbrella. But even if both premises were true, it does not necessarily show you that you should take an umbrella. Perhaps you enjoy walking in the rain, and you would like to get soaked. So, even if both premises were true, the conclusion might nonetheless be false.

%Consider a third argument:
%	\begin{earg}
%		\item[] You are reading this book.
%		\item[] This is a logic book.
%		\item[So:] You are a logic student.
%	\end{earg}
%This is not a terrible argument. Both of the premises are true. And most people who read this book are logic students. Yet, it is possible for someone besides a logic student to read this book. If your roommate picked up the book and thumbed through it, they would not immediately become a logic student. So the premises of this argument, even though they are true, do not guarantee the truth of the conclusion.
%

The general point is as follows. For any argument, there are two ways that it might go wrong:
	\begin{ebullet}
		\item One or more of the premises might be false. 
		\item The conclusion might not follow from the premises.
	\end{ebullet}
To determine whether or not the premises of an argument are true is often a very important matter. However, that is normally a task best left to experts in the field: as it might be, historians, scientists, or whomever. In our role as \emph{logicians}, we are more concerned with arguments \emph{in general}. So we are (usually) more concerned with the second way in which arguments can go wrong.

So: we are interested in whether or not a conclusion \emph{follows from} some premises. Don't, though, say that the premises \emph{infer} the conclusion. Entailment is a relation between premises and conclusions; inference is something we do. (So if you want to mention inference when the conclusion follows from the premises, you could say that \emph{one may infer} the conclusion from the premises.)

\section{Validity}

As logicians, we want to be able to determine when the conclusion of an argument follows from the premises. One way to put this is as follows. We want to know whether, if all the premises were true, the conclusion would also have to be true. This motivates a definition:
	\factoidbox{
		An argument is \define{valid} if and only if it is impossible for all of the premises to be true and the conclusion false.
	}

\newglossaryentry{valid}
{
name=valid,
description={A property of arguments where it is impossible for the premises to be true and the conclusion false}
}

\newglossaryentry{invalid}
{
name=invalid,
description={A property of arguments that holds when it is possible for the premises to be true without the conclusion being true; the opposite of \gls{valid}}
}
        
The crucial thing about a valid argument is that it is impossible for the premises to be true whilst the conclusion is false. Consider this example:
	\begin{earg}
		\item[] Oranges are either fruits or musical instruments.
		\item[] Oranges are not fruits.
		\item[\therefore] Oranges are musical instruments.
	\end{earg}
The conclusion of this argument is ridiculous. Nevertheless, it follows from the premises. \emph{If} both premises were true, \emph{then} the conclusion just has to be true. So the argument is valid. 

This highlights that valid arguments do not need to have true premises or true conclusions. Conversely, having true premises and a true conclusion is not enough to make an argument valid. Consider this example:
	\begin{earg}
		\item[] London is in England.
		\item[] Beijing is in China.
		\item[\therefore] Paris is in France.
	\end{earg}
The premises and conclusion of this argument are, as a matter of fact, all true, but the argument is invalid. If Paris were to declare independence from the rest of France, then the conclusion would be false, even though both of the premises would remain true. Thus, it is \emph{possible} for the premises of this argument to be true and the conclusion false. The argument is therefore invalid.

The important thing to remember is that validity is not about the actual truth or falsity of the sentences in the argument. It is about whether it is \emph{possible} for all the premises to be true and the conclusion false. Nonetheless, we will say that an argument is \define{sound} if and only if it is both valid and all of its premises are true.

\newglossaryentry{sound}
{
name=sound,
description={A property of arguments that holds if the argument is valid and has all true premises}
}

Sometimes arguments that seem plausible will be invalid. Consider:
\begin{earg}
\item[]	If I have freshers flu, then I have a sore throat.
\item[]	I have a sore throat.
\item[\therefore]	I have freshers flu.
\end{earg}
This is invalid because it is possible that I have a sore throat without having freshers flu; I might have just been shouting too loudly yesterday. This sort of fallacy is called `Affirming the Consequent'.


\section{Inductive arguments}
Many good arguments are invalid. Consider this one:
	\begin{earg}
		\item[] In January 1997, it rained in London.
		\item[] In January 1998, it rained in London.
		\item[] In January 1999, it rained in London.
		\item[] In January 2000, it rained in London.
	\item[\therefore] It rains every January in London.
\end{earg}
This argument generalises from observations about several cases to a conclusion about all cases. Such arguments are called \define{inductive} arguments. The argument could be made stronger by adding additional premises before drawing the conclusion: In January 2001, it rained in London; In January 2002\ldots. But, however many premises of this form we add, the argument will remain invalid. Even if it has rained in London in every January thus far, it remains \emph{possible} that London will stay dry next January.

The point of all this is that inductive arguments---even good inductive arguments---are not (deductively) valid. They are not \emph{watertight}. Unlikely though it might be, it is \emph{possible} for their conclusion to be false, even when all of their premises are true. In this book, we will set aside (entirely) the question of what makes for a good inductive argument. Our interest is simply in sorting the (deductively) valid arguments from the invalid ones.  




\practiceproblems
\problempart
Which of the following arguments are valid? Which are invalid?

\begin{earg}
\item Socrates is a man.
\item All men are carrots.
\item[\therefore] Socrates is a carrot.
\end{earg}

\begin{earg}
\item Abe Lincoln was either born in Illinois or he was once president.
\item Abe Lincoln was never president.
\item[\therefore] Abe Lincoln was born in Illinois.
\end{earg}

\begin{earg}
\item If I pull the trigger, Abe Lincoln will die.
\item I do not pull the trigger.
\item[\therefore] Abe Lincoln will not die.
\end{earg}

\begin{earg}
\item Abe Lincoln was either from France or from Luxemborg.
\item Abe Lincoln was not from Luxemborg.
\item[\therefore] Abe Lincoln was from France.
\end{earg}

\begin{earg}
\item If the world were to end today, then I would not need to get up tomorrow morning.
\item I will need to get up tomorrow morning.
\item[\therefore] The world will not end today.
\end{earg}

\begin{earg}
\item Joe is now 19 years old.
\item Joe is now 87 years old.
\item[\therefore] Bob is now 20 years old.
\end{earg}

\problempart
\label{pr.EnglishCombinations}
Could there be:
	\begin{earg}
		\item A valid argument that has one false premise and one true premise?
		\item A valid argument that has only false premises?
		\item A valid argument with only false premises and a false conclusion?
		\item An invalid argument that can be made valid by the addition of a new premise?
		\item A valid argument that can be made invalid by the addition of a new premise?
	\end{earg}
In each case: if so, give an example; if not, explain why not.


\chapter{Other logical notions}\label{s:BasicNotions}

In \S\ref{s:Valid}, we introduced the idea of a valid argument. We will want to introduce some more ideas that are important in logic.

%\section{Truth values}
%As we said in \S\ref{s:Arguments}, arguments consist of premises and a conclusion. Note that many kinds of English sentence cannot be used to express premises or conclusions of arguments. For example:
%	\begin{ebullet}
%		\item \textbf{Questions}, e.g.\ `are you feeling sleepy?'
%		\item \textbf{Imperatives}, e.g.\ `Wake up!'
%		\item \textbf{Exclamations}, e.g.\ `Ouch!'
%	\end{ebullet}
%The common feature of these three kinds of sentence is that they are not \emph{assertoric}: they cannot be true or false. It does not even make sense to ask whether a \emph{question} is true (it only makes sense to ask whether the \emph{answer} to a question is true).

%The general point is that, the premises and conclusion of an argument must be capable of having a \define{truth value}. The two truth values that concern us are just True and False. 

\section{Joint possibility}

Consider these two sentences:
	\begin{ebullet}
		\item[B1.] Jane's only brother is shorter than her.
		\item[B2.] Jane's only brother is taller than her.
	\end{ebullet}
Logic alone cannot tell us which, if either, of these sentences is true. Yet we can say that \emph{if} the first sentence (B1) is true, \emph{then} the second sentence (B2) must be false. Similarly, if B2 is true, then B1 must be false. It is impossible that both sentences are true together. These sentences are inconsistent with each other, they cannot all be true at the same time. This motivates the following definition:
	\factoidbox{
		Sentences are \define{jointly possible} if and only if it is possible for them all to be true together.
	}
Conversely, B1 and B2 are \emph{jointly impossible}.

\newglossaryentry{possibility}
{
name=joint possibility,
text={jointly possible},
description={A property possessed by some sentences when they can all be true at the same time}
}

We can ask about the possibility of any number of sentences. For example, consider the following four sentences:
	\begin{ebullet}	
		\item[G1.] \label{MartianGiraffes} There are at least four giraffes at the wild animal park.
		\item[G2.] There are exactly seven gorillas at the wild animal park.
		\item[G3.] There are not more than two martians at the wild animal park.
		\item[G4.] Every giraffe at the wild animal park is a martian.
	\end{ebullet}
G1 and G4 together entail that there are at least four martian giraffes at the park. This conflicts with G3, which implies that there are no more than two martian giraffes there. So the sentences G1--G4 are jointly impossible. They cannot all be true together. (Note that the sentences G1, G3 and G4 are jointly impossible. But if sentences are already jointly impossible, adding an extra sentence to the mix will not make them jointly possible!)

\section[Necessary truths, falsehoods, and contingency]{Necessary truths, necessary falsehoods, and contingency}

In assessing arguments for validity, we care about what would be true \emph{if} the premises were true, but some sentences just \emph{must} be true. Consider these sentences:
	\begin{earg}
		\item[\ex{Acontingent}] It is raining.
		\item[\ex{Atautology}] Either it is raining here, or it is not.
		\item[\ex{Acontradiction}] It is both raining here and not raining here.
	\end{earg}
In order to know if sentence \ref{Acontingent} is true, you would need to look outside or check the weather channel. It might be true; it might be false. A sentence which is capable of being true and capable of being false (in different circumstances, of course) is called \define{contingent}.

\newglossaryentry{contingent sentence}
{
name=contingent sentence,
description={A sentence that is neither a \gls{necessary truth} nor a \gls{necessary falsehood}; a sentence that in some situations is true and in others false}
}

Sentence \ref{Atautology} is different. You do not need to look outside to know that it is true. Regardless of what the weather is like, it is either raining or it is not. That is a \define{necessary truth}. 

\newglossaryentry{necessary truth}
{
name={necessary truth},
description={A sentence that must be true}
}

Equally, you do not need to check the weather to determine whether or not sentence \ref{Acontradiction} is true. It must be false, simply as a matter of logic. It might be raining here and not raining across town; it might be raining now but stop raining even as you finish this sentence; but it is impossible for it to be both raining and not raining in the same place and at the same time. So, whatever the world is like, it is not both raining here and not raining here. It is a \define{necessary falsehood}.

\newglossaryentry{necessary falsehood}
{
name={necessary falsehood},
description={A sentence that must be false}
}

%Something might \emph{always} be true and still be contingent. For instance, if there never were a time when the universe contained fewer than seven things, then the sentence `At least seven things exist' would always be true. Yet the sentence is contingent: the world could have been much, much smaller than it is, and then the sentence would have been false. 


\section{Necessary and sufficient conditions}
We can also ask about the logical relations \emph{between} two sentences. 

Consider the two sentences:
\begin{earg}
\item Fido is a bulldog.
\item Fido is a dog.
\end{earg}
Fido being a bulldog is \define{sufficient condition} for Fido being a dog.
%If we know that Fido is a bulldog, that's all the information we need to conclude that Fido is a dog. 
The argument 
\begin{earg}
\item Fido is a bulldog.
\item[\therefore] Fido is a dog.
\end{earg} is a valid argument. And 
\begin{earg}
\item[] If Fido is a bulldog, then Fido is a dog.
\end{earg}is necessarily true.

	\factoidbox{
		\meta{A} is \define{sufficient for} \meta{B} if it's impossible for \meta{A} to be true and \meta{B} false. 
	}

The other way around is a \define{necessary condition}. For Fido to be a bulldog it is \emph{necessary} that Fido is a dog. There's no way he could be a bulldog without being a dog. 

	\factoidbox{
		\meta{A} is \define{necessary for}  \meta{B} if it's impossible for \meta{B} to be true and \meta{A} false. 
	}




Sometimes a sentence is both necessary and sufficient for a second sentence. For example:
\begin{earg}
\item[] John went to the store after he washed the dishes.
\item[] John washed the dishes before he went to the store.
\end{earg}
The sentences must have the same truth value. 
%These two sentences are both contingent, since John might not have gone to the store or washed dishes at all. Yet they must have the same truth-value.
 If either of the sentences is true, then they both are; if either of the sentences is false, then they both are. We then say that they are \define{necessarily equivalent}.


\newglossaryentry{necessary equivalence}
{
name={necessary equivalence},
text={necessarily equivalent},
description={A property held by a pair of sentences that must always have the same truth value}
}

%\section{Necessary and Sufficient conditions}



\section*{Summary of logical notions}

\begin{itemize}
\item An argument is (deductively) \define{valid} if it is impossible for the premises to be true and the conclusion false; it is \define{invalid} otherwise.

\item A \define{necessary truth} is a sentence that must be true, that could not possibly be false.

\item A \define{necessary falsehood} is a sentence that must be false, that could not possibly be true.

\item A \define{contingent sentence} is neither a necessary truth nor a necessary falsehood. It may be true but could have been false, or vice versa.

\item A sentence is \define{suffient for} a second sentence if it's impossible for the first to be true and the second false. 

\item A sentence is \define{necessary for} a second sentence if it's impossible for the second to be true and the first false. 

\item Two sentences are \define{necessarily equivalent} if they must have the same truth value.

\item A collection of sentences is \define{jointly possible} if it is possible for all these sentences to be true together; it is \define{jointly impossible} otherwise.
\end{itemize}


\practiceproblems
\problempart
\label{pr.EnglishTautology2}
For each of the following: Is it a necessary truth, a necessary falsehood, or contingent?
\begin{earg}
\item Caesar crossed the Rubicon.
\item Someone once crossed the Rubicon.
\item No one has ever crossed the Rubicon.
\item If Caesar crossed the Rubicon, then someone has.
\item Even though Caesar crossed the Rubicon, no one has ever crossed the Rubicon.
\item If anyone has ever crossed the Rubicon, it was Caesar.
\end{earg}

\problempart
For each of the following: Is it a necessary truth, a necessary falsehood, or contingent?
\begin{earg}
\item Elephants dissolve in water.
\item Wood is a light, durable substance useful for building things.
\item If wood were a good building material, it would be useful for building things.
\item I live in a three story building that is two stories tall.
\item If gerbils were mammals they would nurse their young.
\end{earg}

\problempart Which of the following pairs of sentences are necessarily  equivalent? 

\begin{earg}
\item Elephants dissolve in water.	\\
	If you put an elephant in water, it will disintegrate.
\item All mammals dissolve in water.\\		
	If you put an elephant in water, it will disintegrate. 
\item George Bush was the 43rd president. \\
	 Barack Obama is the 44th president. 
\item Barack Obama is the 44th president. \\
	  Barack Obama was president immediately after the 43rd president. 
\item Elephants dissolve in water. 	\\	
	All mammals dissolve in water. 
\end{earg}
\problempart Which of the following pairs of sentences are necessarily equivalent? 

\begin{earg}
\item  Thelonious Monk played piano.	\\
	John Coltrane played tenor sax. 
\item  Thelonious Monk played gigs with John Coltrane.	\\
	John Coltrane played gigs with Thelonious Monk.
\item  All professional piano players have big hands.	\\
	Piano player Bud Powell had big hands.
\item  Bud Powell suffered from severe mental illness.	 \\
	All piano players suffer from severe mental illness.
\item John Coltrane was deeply religious.	 \\
John Coltrane viewed music as an expression of spirituality. 
\end{earg}

\noindent \problempart Consider the following sentences: 
\begin{enumerate}%[label=(\alph*)]
\item[G1] \label{itm:at_least_four}There are at least four giraffes at the wild animal park.
\item[G2] \label{itm:exactly_seven} There are exactly seven gorillas at the wild animal park.
\item[G3] \label{itm:not_more_than_two} There are not more than two Martians at the wild animal park.
\item[G4] \label{itm:martians} Every giraffe at the wild animal park is a Martian.
\end{enumerate}

Now consider each of the following collections of sentences. Which are jointly possible? Which are jointly impossible?
\begin{earg}
\item Sentences G2, G3, and G4
\item Sentences G1, G3, and G4
\item Sentences G1, G2, and G4
\item Sentences G1, G2, and G3
\end{earg}

\problempart Consider the following sentences.
\begin{enumerate}%[label=(\alph*)]
\item[M1] \label{itm:allmortal} All people are mortal.
\item[M2] \label{itm:socperson} Socrates is a person.
\item[M3] \label{itm:socnotdie} Socrates will never die.
\item[M4] \label{itm:socmortal} Socrates is mortal.
\end{enumerate}
Which combinations of sentences are jointly possible? Mark each ``possible'' or ``impossible.''
\begin{earg}
\item Sentences M1, M2, and M3
\item Sentences M2, M3, and M4
\item Sentences M2 and M3
\item Sentences M1 and M4
\item Sentences M1, M2, M3, and M4
\end{earg}

\problempart
\label{pr.EnglishCombinations2}
Which of the following is possible? If it is possible, give an example. If it is not possible, explain why.
\begin{earg}
\item A valid argument that has one false premise and one true premise

\item A valid argument that has a false conclusion

\item A valid argument, the conclusion of which is a necessary falsehood

\item An invalid argument, the conclusion of which is a necessary truth

\item A necessary truth that is contingent

\item Two necessarily equivalent sentences, both of which are necessary truths

\item Two necessarily equivalent sentences, one of which is a necessary truth and one of which is contingent

\item Two necessarily equivalent sentences that together are jointly impossible

\item A jointly possible collection of sentences that contains a necessary falsehood

\item A jointly impossible set of sentences that contains a necessary truth
\end{earg}

\problempart
Which of the following is possible? If it is possible, give an example. If it is not possible, explain why.

\begin{earg}
\item A valid argument, whose premises are all necessary truths, and whose conclusion is contingent
\item A valid argument with true premises and a false conclusion
\item A jointly possible collection of sentences that contains two sentences that are not necessarily equivalent
\item A jointly possible collection of sentences, all of which are contingent
\item A false necessary truth
\item A valid argument with false premises
\item A necessarily equivalent pair of sentences that are not jointly possible
\item A necessary truth that is also a necessary falsehood
\item A jointly possible collection of sentences that are all necessary falsehoods
\end{earg}
