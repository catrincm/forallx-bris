%!TEX root = forallxbris.tex
\part{Arguments}
\label{ch.intro}
\addtocontents{toc}{\protect\mbox{}\protect\hrulefill\par}


\chapter{Arguments}
\label{s:Arguments}
\section{Introduction and Overview}
In philosophy we think, reason, and argue. We intellectually scrutinize theories and view points. We reflect on arguments in support or in opposition of some philosophical position. In other words arguments and argumentation take centerstage in philosophy. But what is an argument and what are good arguments? 

An argument, as we will understand it, is something like this:
		\begin{ebullet}
	\item If I acted of my own free will, then I could have acted otherwise.
	\item I could not have acted otherwise.
	\item Therefore: I did not act of my own free will.
		\end{ebullet}\todo{add a reference to SEP compatibilism?}
We here have a series of sentences which may either be true or false. The final sentence, ``I did not act of my own free will.'' expresses the \emph{conclusion} of the argument. The two sentences before that are the \emph{premises} of the argument. In a good argument, the conclusion follows from the premises. If you believe the premises then the argument should lead you to believing the conclusion.

On the face of it, our sample argument appears to be a good argument, but can one develop a general theory of what a good argument is, that is, a theory that abstracts away from particular arguments and equips us with general tools for evaluating arguments? It turns out that one can and this general theory is what we call `logic'.

In logic we want to abstract away from specific arguments and their particular content, and focus on specific structural features of arguments. To achieve this  logic is spelled out in its own particular language like, say, latin.  This means that learning logic amounts to learning a new language and it can be helpful to approach the course with this mindset. As we want to make sure that there are no ambiguities or misunderstandings the language of logic happens to be a formal language. Now to learn a language one needs to
\begin{itemize}
\item[\textsc{Voc}:]\label{vocInt} learn its vocabulary (what are the basic expressions of language?);
\item[\textsc{Gra}:]\label{graInt} learn its grammar (what is a well-formed sentence of the language?);
\item[\textsc{Sem}:]\label{semInt} learn what the sentences of the new language say or mean;
\item[\textsc{PT}:]\label{ptInt} learn how to use the new language and how to reason with it.
\end{itemize}

This course is about learning logic and thus learning the language of logic. Accordingly, we will work ourselves through the different items of the above list. But in addition we need to think about how this helps us with assessing philosophical arguments and reasoning, which is carried out in English or some other natural language. To this effect we discuss how one can symbolize (or formalize) English arguments in the language of logic, which will allow us to evaluate English arguments in the logical language. Summing up, the structure of the course will be as follows: we first introduce the \define{syntax} of the language, which subsumes both \textsc{Voc} and \textsc{Gra}, and then turn to its \define{semantics} (\textsc{Sem}), that is, the meaning of the sentences of the language. The semantics of the language will also allow us to spell out whether an argument is a good argument from the logical perspective. Such an argument is called a \emph{valid} argument.\todo{valid=good?}  We then explain how to symbolize English sentences in the logical language, which enables us to check whether an English argument is valid or not. Finally, we turn to PT and introduce a reasoning system for the logical language, that makes it possible for us to argue or reason within the logical language. In the context of our formal language, this means that we give a proof system, that is, we specify a number of rules that allows us to derive a sentence (the conclusion) from other sentences (the premises) in the same way that in natural language a good argument licenses us to infer the conclusion of the argument from its premises. Importantly, the rules will only allow us to produce valid argument. If we apply the rules correctly our reasoning cannot go wrong (this doesn't meant that our premises were true however).

%Much of philosophical practice is about argument and
%analysis.
%Arguing in support of or against some position, or understanding someone else's argument.
%Logic is the study of the practice of argument and
%analysis, abstracted from the specific details of a
%particular case. 



%In everyday language, we sometimes use the word `argument' to talk about belligerent shouting matches.  If you and a friend have an argument in this sense, things are not going well between the two of you. Logic is not concerned with such teeth-gnashing and hair-pulling. They are not arguments, in our sense; they are disagreements.
%
%An argument, as we will understand it, is something more like this:
%		\begin{ebullet}
%	\item If I acted of my own free will, then I could have acted otherwise.
%	\item I could not have acted otherwise.
%	\item Therefore: I did not act of my own free will.
%		\end{ebullet}\todo{add a reference to SEP compatibilism?}
%We here have a series of sentences which may either be true or false. The final sentence, ``I did not act of my own free will.'' expresses the \emph{conclusion} of the argument. The two sentences before that are the \emph{premises} of the argument. In a good argument, the conclusion follows from the premises. If you believe the premises then the argument should lead you to believing the conclusion.
%
%Logic provides the ideal model of good argument:
%rational argument without rhetoric.
%The logical study of an argument can show whether it
%supports its conclusion or is flawed.
%Logic focuses only on the statements presented and the
%relationships between them.
%Extraneous factors are set aside: unspoken assumptions,
%additional connotations of words, appeal to emotions.
%
%Logical thinking can help us to work out the intended
%interpretation of a text, and to find alternative
%unintended interpretations.
%This can be helpful when reading someone else’s
%writing, and essential when we are trying to write
%unambiguously.
%Logical analysis can help us to find ambiguity and
%alternative interpretations, and to write in a precise and
%unambiguous way that can only be interpreted as we
%intend.
%These are vital skills used in all philosophy as well as in life more generally.
%


But we are getting ahead of ourselves and before we introduce the language of logic, we should first better understand what an argument is and which arguments are good ones. To this effect we discuss arguments in natural languages and more specifically arguments in English. As we mentioned at the beginning, understanding and evaluating arguments is a crucial aspect of philosophy. Reflecting and investigating arguments will thus help you to analyse some of the texts you are reading during your philosophy studies.


%Throughout this textbook we will also consider arguments in formal languages and say what it is for those to be valid or invalid. We want formal validity, as defined in the formal language, to have at least some of the important features of natural-language validity.

%some properties of argumentbasic logical notions that apply to arguments in a natural language like English. It is important to begin with a clear understanding of what arguments are and of what it means for an argument to be valid. Later we will translate arguments from English into a formal language. We want formal validity, as defined in the formal language, to have at least some of the important features of natural-language validity.




\section{Finding the components of an argument}



Arguments consist of a list of \emph{premises} along with a \emph{conclusion}. In a good argument, the conclusion will follow from the premises.
Our standard way to present them is:
\begin{earg}
\prem Premise 1
\prem Premise 2
\item[] \dots
\item[$n$.] Premise $n$
\conc Conclusion
\end{earg}
For	example
	\begin{earg}
	\prem If I acted of my own free will, then I could have acted otherwise.
	\prem I could not have acted otherwise.
	\conc I did not act of my own free will.
	\end{earg}

The three dots in this final line can be read ``therefore''. Really then we're duplicating things by also adding the word ``Therefore''. But we do this to really carefully highlight that this is the conclusion. %If you are writing your answers up on the computer and cannot use this symbol, that's OK. But make sure it is very clear what the conclusion of the argument is.
%. Bu. It is our concise way to identify the conclusion of the argument. \todo{should I delete it throughout?}


Often arguments are presented simply in a paragraph of text, or in a speech or article, and we first have to work out what the premises and conclusions are.
Sometimes it's easy, for example:
\begin{quote}
	If I acted of my own free will, then I could have acted otherwise.
	But, I could not have acted otherwise.
	So, I did not act of my own free will.
\end{quote}
But often it is a significant piece of work to work out the premises and conclusion of an argument.




%\begin{quote}
% Mrs.~White must have done it, because it was
% either her or Revd.~Green, and if was him, it would have been in the Kitchen, which it wasn’t.
%\end{quote}
%\begin{earg}
%\prem It was either Mrs.~White or Revd.~Green.
%\prem If it was Revd.~Green, it was in the kitchen.
%\prem It wasn't in the kitchen.
%\conc Mrs.~White did it it
%\end{earg}
%%This argument has one premise (I was wearing my sunglasses) followed by a conclusion (it was sunny).

Many arguments start with premises, and end with a conclusion, but not all of them. It might start with the conclusion:
	\begin{quote}\textbf{We should not have a second Brexit referendum}.
		A second Brexit referendum would erode the very basis of
		democracy by suggesting that rule by the majority is an insufficient
		condition for democratic legitimacy.
	\end{quote}
Or it might have been presented with the conclusion in the middle:
	\begin{quote}
		Since the first Brexit referendum was made under false pretences,
		\textbf{the voters deserve a further say on any final deal agreed with
		Brussels}. After all, decisions as big as this need to have the public
		support, which has to come from a referendum.
	\end{quote}


Sometimes premises or the conclusion may be clauses in a sentence.
A complete argument may even be contained in a single sentence:
\begin{quote}
The butler has an alibi; so they cannot have done it.
\end{quote}
This argument has one premise, followed immediately by its conclusion.

One particular kind of sentence can be confusing. Consider:
\begin{ebullet}
\item  If the murder weapon was a gun, then Prof. Plum did it.
\end{ebullet}
These conditional, or ``if-then'', statements might look like it expresses the argument, but in itself it does not. It's just stating a fact, albeit a conditional fact. It might also be used in an argument, even as the conclusion of the argument:
	\begin{earg}
	\prem  If I have free will, then there is some event that I could have caused to go differently.
	\prem  If determinism is true, then there is no event that I could have caused to go differently.
	\conc If determinism is true, I do not have free will.
	\end{earg}

As a guideline, there are some words you can look for which are often used to indicate whether something is a premise or conclusion:
	\begin{highlighted}
	Words often used to indicate an argument's conclusion:
		\begin{center}
			so, therefore, hence, thus, accordingly, consequently
		\end{center}
	Words often used to indicate a premise:
		\begin{center}
			since, because, as, given that, recalling that, after all
		\end{center}
	\end{highlighted}



%Sometimes it can take some work to pick out what the conclusion of the argument is.
%Consider:
%%\begin{center}
%%You have already said that you love me and that you can't imagine spending the rest of your life without me. Once you even tried to propose to me. And now you claim that you need time to think about whether we should be married. Well, everything you've told me regarding our relationship has been a lie. In some of your letters to a friend you admitted that you were misleading me. You've been telling everyone that we are just friends, not lovers. And worst of all, you've been secretly dating someone else. Why are you doing this? It's all been a farce, and I'm outta here.
%%\citet{vaughn}
%%\end{center}
%\begin{quote}
%Virgin would then dominate the rail system. Is that something the government should worry about? Not necessarily. The industry is regulated, and one powerful company might at least offer a more coherent schedule of services than the present arrangement has produced. The reason the industry was broken up into more than 100 companies at privatisation was not operational, but political: the Conservative government thought it would thus be harder to renationalise. \emph{The Economist 16.12.2000; used on \href{https://philosophy.hku.hk/think/arg/arg.php}{critical thinking web}}
%\end{quote}
%The conclusion of this argument is that we shouldn't be worried if Virgin dominates the rail system.




In analysing an argument, there is no substitute for a good nose.
Whenever you come across an argument in a
piece of philosophy you read, be it lecture notes,
primary text, or secondary text, or in a newspaper
article or on the internet, practice identifying the
premises and conclusion.

Sometimes, though, people aren't giving arguments but are simply presenting facts or stating their opinion.
For example, the following do not contain arguments, they're not trying to convince us of anything.
\begin{ebullet}
\item I don't like cats. I think they're evil.
\item Hundreds of vulnerable children as young as 10, who have spent most of their lives in the UK, are having their applications for British citizenship denied for failing to pass the government's controversial `good character' test.
\end{ebullet}


\section{Intermediate Conclusions}
%We are consideri said that an argument is just a list of premises and a single conclusion.
%
%In a good argument, the conclusion should follow from the premises, in a sense we will come to.

We said an argument is given by a collection of \emph{premises} along with a single \emph{conclusion}.
We might represent this as something like:
\begin{center}
 \begin{tikzpicture}

        \node[propn] (concln) {Conclusion};

        \coordinate (zJoin) at ([yshift=1cm]concln.north);

        \node[propn] (premise1) [above left=1.5cm and 1cm  of concln]{Premise 1};
        \node[propn] (premise2) [above right=1.5cm and 1cm of concln]{Premise 2};
                \draw[joiningargout] (zJoin)--(concln);
                \draw[joiningargin] (premise1)--(zJoin);
                \draw[joiningargin] (premise2)--(zJoin);
                  \end{tikzpicture}

\end{center}
The premises are working together to lead to the conclusion.

But sometimes in the process of someone making an argument someone will make use of \emph{intermediate conclusions}. Such arguments might have a structure more like:
\begin{center}
\begin{tikzpicture}
        \node[propn] (premise) {Premise 1};
          \node[propn] (premise2) [right=2cm of premise] {Premise 2};
        \node[propn] (intconclusion) [below right=1.5cm and 0cm of premise] {Intermediate Conclusion};
                  \coordinate (zJoin) at ([yshift=1cm]intconclusion.north);
        \draw[joiningargin] (premise)--(zJoin);
          \draw[joiningargin] (premise2)--(zJoin);
                  \draw[joiningargout] (zJoin)--(intconclusion);
          \node[propn] (premise3) [right=1cm of intconclusion] {Premise 3};
               \node[propn] (concln) [below right=1.5cm and 0cm of intconclusion] {Final Conclusion};
                     \coordinate (zJoin2) at ([yshift=1cm]concln.north);
        \draw[joiningargin] (intconclusion)--(zJoin2);
          \draw[joiningargin] (premise3)--(zJoin2);
                  \draw[joiningargout] (zJoin2)--(concln);
    \end{tikzpicture}
\end{center}

However, we say that an argument is only something of the first kind. So what do we say about the second kind of thing? We can consider it two ways. We could consider it as an argument from premise 1, 2 and 3 to the conclusion. Or alternatively we can think of it as two arguments of the first kind chained together, one from premise 1 and 2 to the intermediate conclusion, and the second from the intermediate conclusion and premise 3 to the final conclusion.

%In order to better understand arguments, though, we focus on arguments which have the simple structure with a series of premises and a single conclusion. We will understand the latter structure as multiple arguments chained together and we analyse each argument individually.

%
%
%But we will analyse this simple structure to give us the tools to understand the more complex arguments as we will come across them throughout philosophy.


\section{Sentences}
\label{intro.sentences}\todo{is anyone ever confused about this anyway?? Also Richard thinks that it's not sentences but propositions.}
What kinds of things are the premises and conclusions of arguments? They are sentences which can either be true or false.
%In fact, we can define arguments to be a series of sentences, called the premises, followed by a single sentence, the conclusion. These sentences should be the sorts of things which can either be true or false.
Such sentences are called \define{declarative sentences}.

There are many other kinds of sentences, for example:
\begin{description}
%\item[{Declarative Sentence}]
\item[{Questions}] `Are you sleepy yet?' is an interrogative sentence. Although you might be sleepy or you might be alert, the question itself is neither true nor false. For this reason, questions will not count as declarative sentences. Suppose you answer the question: `I am not sleepy.' This is either true or false, and so it is a declarative sentences. Generally, \emph{questions} will not count as declarative sentences, but \emph{answers} will.

`What is this course about?' is not a declarative sentence (in our sense). `No one knows what this course is about' is a declarative sentence.

\item[{Imperatives}] Commands are often phrased as imperatives like `Wake up!', `Sit up straight', and so on. These are imperative sentences. Although it might be good for you to sit up straight or it might not, the command is neither true nor false and it is thus not a declarative sentence. Note, however, that commands are not always phrased as imperatives. `You will respect my authority' \emph{is} either true or false--- either you will or you will not--- and so it counts as a declarative sentences.

\item[{Exclamations}] `Ouch!' is sometimes called an exclamatory sentence, but it is not the sort of thing which is true or false. `That hurt!', however, is a declarative sentence.
%We will treat `Ouch, I hurt my toe!' as meaning the same thing as `I hurt my toe.' The `ouch' does not add anything that could be true or false.

\end{description}

Arguments are formed of \emph{declarative sentences} --- those sentences which can be true or false --- for example `spiders have eight legs'.

\begin{highlighted}An \define{argument} consists of a collection of declarative sentences of which one is marked as the conclusion of the argument.\end{highlighted}


The unmarked declarative sentences are of course the premises of the argument. We typically drop the term `declarative' and simply call them sentences, but bear in mind that it is only these sorts of sentences that are relevant in this textbook.

You should not confuse the idea of a sentence that can be true or false with the difference between fact and opinion. Often, sentences in logic will express things that would count as facts--- such as `spiders have eight legs' or `Kierkegaard liked almonds.' They can also express things that you might think of as matters of opinion---such as, `Almonds are tasty.' In other words, a sentence is not disqualified from being part of an argument because we don't know if it is true or false, or because its truth or falsity seems to be a purely subjective matter. All that matters is whether what the sentence expresses it is the sort of thing that could be true or false. If it is, it can play the role of premise or conclusion.

When you are reading a text and putting it in our standard form you should make sure that your premises and conclusions are declarative sentences.
You should also make them as clear as possible. Each premise and the conclusion should be able to be read and understood independently. Any context from the original paragraph should be copied over to each of the premises and conclusions.
For example:
\begin{quote}
Donating to charity no strings attached is the most effective way to do so. So if you are going to donate to charity, you should do it this way.
\end{quote}
When presenting this we should fill out ``this way'' with the relevant way. So I'd write:
\begin{earg}
\prem	Donating to charity no strings attached is the most effective way to do so.
\conc  If you are going to donate to charity, you should do so no strings attached.
\end{earg}





%%Some sentences depend on context for whether they are true or false. For example the truth of `I have long hair' depends on who is uttering it. Sometimes we will use examples where the context does matter. Everything that we say, though, still holds when the context is held fixed throughout the course of the argument.
%In English, sentences can be \emph{ambiguous}. There are many sources of ambiguity. One is \emph{lexical ambiguity:} a sentence can contain words which have more than one meaning.  For instance, `bank' can mean the bank of a river, or a financial institution. So I might say that `Katie went to the bank' when she took a stroll along the river, or when she went to deposit a check.  When we talk about sentences, we assume that all words have been disambiguated and it is settled whether `bank' is somewhere you have a picnic or where you deposit your money.

\todo{Should I talk about ambiguity at all?}

%For example:
%
%\begin{earg}
%\end{earg}
\todo{Indexicals and context dependence?}





\newglossaryentry{premise indicator word}
{
name=premise indicator,
description={a word or phrase such as ``because'' used to indicate that what follows is the premise of an argument}
}

\newglossaryentry{conclusion indicator word}
{
name=conclusion indicator,
description={a word or phrase such as ``therefore'' used to indicate that what follows is the conclusion of an argument}
}

\newglossaryentry{argument}
{
name=argument,
description={a connected series of sentences, divided into \gls{premise}s and \gls{conclusion}}
}

\newglossaryentry{premise}
{
name=premise,
description={a sentence in an \gls{argument} other than the \gls{conclusion}}
}

\newglossaryentry{conclusion}
{
name=conclusion,
description={the last sentence in an \gls{argument}}
}






\begin{practiceproblems}

At the end of some chapters, there are exercises that review and
explore the material covered in the chapter.
%The problem sheet you need to complete is constructed from these exercises, but the book offers some additional practice if you want more.
There is no substitute for actually working through some problems. This course isn't about memorizing facts but about developing a way of thinking.

%\bigskip
So here’s the first exercise.
\problempart

\begin{earg}
\item Are arguments always presented in our standard form?
\item Do conclusions always come after the premises in arguments in texts?
\item Might premises and conclusions be clauses within sentences?
\item Can questions be premises?
\end{earg}

\problempart
Write down the conclusion of each of these arguments:
\begin{earg}
	\item It is sunny. So I should take my sunglasses.\myanswer{\\I should take my sunglasses.}
	\item It must have been sunny. I did wear my sunglasses, after all.\myanswer{\\It was sunny.}
	\item No one but you has had their hands in the cookie-jar. And the scene of the crime is littered with cookie-crumbs. You're the culprit!
	\item Miss Scarlett and Professor Plum were in the study at the time of the murder. Reverend Green had the candlestick in the ballroom, and we know that there is no blood on his hands. Hence Colonel Mustard did it in the kitchen with the lead-piping. Recall, after all, that the gun had not been fired.
	\item  Since I do not know that I am not under the spell of a malicious demon, I do not know that this table exists. After all, if I know that this table exists, then I know that I am not under the spell of a malicious demon.
	\item Cutting the interest rate will have no effect on the stock market this time round as people have been expecting a rate cut all along. This factor has already been reflected in the market.
%	\item {We should not have a second Brexit referendum}.
%	A second Brexit referendum would erode the very basis of
%		democracy by suggesting that rule by the majority is an insufficient
%			condition for democratic legitimacy.
%	\item
%	Since the first Brexit referendum was made under false pretences,
%			{the voters deserve a further say on any final deal agreed with
%			Brussels}. After all, decisions as big as this need to have the public
%			support, which has to come from a referendum.
	\item Virgin would then dominate the rail system. Is that something the government should worry about? Not necessarily. The industry is regulated, and one powerful company might at least offer a more coherent schedule of services than the present arrangement has produced. The reason the industry was broken up into more than 100 companies at privatisation was not operational, but political: the Conservative government thought it would thus be harder to renationalise. \emph{The Economist 16.12.2000; used on \href{https://philosophy.hku.hk/think/arg/arg.php}{critical thinking web}}
	\item The idea that being vegetarian is better for the environment has, over the
	last forty years, become a piece of conventional wisdom. But it is simply
	wrong. A paper from Carnegie Mellon University researchers published
	this week finds that the diets recommended by the Dietary Guidelines for
	Americans, which include more fruits and vegetables and less meat, exacts
	a greater environmental toll than the typical American diet. Shifting to
	the diets recommended by Dietary Guidelines for American would increase
	energy use by 38 percent, water use by ten percent and greenhouse gas
	emissions by six percent, according to the paper.
	\item There are no hard numbers, but the evidence from Asia's expatriate community is unequivocal. Three years after its handover from Britain to China, Hong Kong is unlearning English. The city's gweilos (Cantonese for ``ghost men") must go to ever greater lengths to catch the oldest taxi driver available to maximize their chances of comprehension. Hotel managers are complaining that they can no longer find enough English- speakers to act as receptionists. Departing tourists, polled at the airport, voice growing frustration at not being understood. \\ \emph{The Economist 20.1.2001}, used in \href{https://philosophy.hku.hk/think/arg/arg.php}{Critical Thinking Web}
\end{earg}

\problempart
Write each of the following arguments in the standard form.
\begin{enumerate}
	\item[x.] It might surprise you, but denoting to charity no strings attached is the most effective way to do so. So if you are going to donate to charity, you should do it this way.
	\prem Answer:
	\begin{earg}
	\prem	Denoting to charity no strings attached is the most effective way to do so.
	\conc	If you are going to donate to charity, you should do so no strings attached.
	\end{earg}

	\item It is sunny. So I should take my sunglasses.
	\myanswer{\\I should take my sunglasses}
	\item It must have been sunny. I did wear my sunglasses, after all.\myanswer{\\It was sunny}
	\item No one but you has had their hands in the cookie-jar. And the scene of the crime is littered with cookie-crumbs. You're the culprit! \myanswer{\\You're the culprit}
	\item Kate didn't write it. If Kate or David wrote it, it will be reliable; and it isn’t.
	\item  Since I do not know that I am not under the spell of a malicious demon, I do not know that this table exists. After all, if I know that this table exists, then I know that I am not under the spell of a malicious demon.
	\item Miss Scarlett and Professor Plum were in the study at the time of the murder. And Reverend Green had the candlestick in the ballroom, and we know that there is no blood on his hands. Hence Colonel Mustard did it in the kitchen with the lead-piping. Recall, after all, that the gun had not been fired.\myanswer{\\Colonel Mustard did it in the kitchen with the lead-piping}
%%CCM
%	\item I'll only bring the book tomorrow if you ring me to remind me or you Billy to ring me and remind me. But you never ring me because you don't have credit on your account. I know that Billy's busy this evening, so unless Billy rings me this afternoon I won't bring the book. \myanswer{\\unless Billy rings me this afternoon I won't bring the book} %%CCM
%	\item The Indigenous Australians travelled from New Guinea to Australia
%	by boat. So, they arrived less then 30,000 years
%	ago because if they arrived by boat, they must have arrived less
%	than 30,000 years ago.\myanswer{\\they arrived less then 30,000 years
%		ago}
%	\item I know that this table exists. Thus, I know that I am not under the spell of a malicious demon.\newpage
%	\item The idea that being vegetarian is better for the environment has, over the last forty years, become a piece of conventional wisdom. But it is simply wrong.
%	A paper from Carnegie Mellon University researchers published this week finds that the diets recommended by the Dietary Guidelines for Americans, which include more fruits and vegetables and less meat, exacts a greater environmental toll than the typical American diet. Shifting to the diets recommended by Dietary Guidelines for American  would increase energy use by 38 percent, water use by ten percent and greenhouse gas emissions by six percent, according to the paper.
%	\myanswer{\\Vegetarian is not better for the environment.}\newpage
%\item The Crown has been cancelled early. That's awful! It must have been really expensive to make. \newpage
%\item The idea of X is unpopular. But it what we should do.
\end{enumerate}

\end{practiceproblems}

\chapter{The scope of logic}
\label{s:Valid}




\section{Consequence and validity}

In \S\ref{s:Arguments}, we talked about arguments, i.e., a collection of sentences (the premises), followed by a single sentence (the conclusion). We said that some words, such as ``therefore,'' indicate which sentence is supposed to be the conclusion. ``Therefore,'' of course, suggests that there is a connection between the premises and the conclusion, namely that the conclusion \emph{follows from}, or \emph{is a consequence of}, the premises.

This notion of consequence is one of the primary things logic is concerned with and we will ultimately define a precise notion of consequence for our formal language. One might even say that logic is the science of what follows from what.  Logic develops a general account of and general tools that tell us when a sentence follows from some other sentences.

What about the following argument:
\begin{earg}
	\prem The butler or the gardener did it.
	\prem The butler did not do it.
	\conc The gardener did it.
\end{earg}
We don't have any context for what the sentences in this argument refer to. Perhaps you suspect that ``did it'' here means ``was the perpetrator'' of some unspecified crime. You might imagine that the argument occurs in a mystery novel or TV show, perhaps spoken by a detective working through the evidence. But even without having any of this information, you probably agree that the argument is a good one in the sense that whatever the premises refer to, if they are both true, the conclusion is guaranteed to be true as well. If the first premise is true, i.e., it's true that ``the butler did it or the gardener did it,'' then at least one of them ``did it,'' whatever that means. And if the second premise is true, then the butler did not ``do it.'' That leaves only one option: ``the gardener did it'' must be true. Here, the conclusion follows from the premises. We call arguments that have this property \define{valid}.

By way of contrast, consider the following argument:
\begin{earg}\label{argMaidDriver}
	\prem If the driver did it, the maid didn't do it.
	\prem The maid didn't do it.
	\conc The driver did it.
\end{earg}
We still have no idea what is being talked about here. But, again, you probably agree that this argument is different from the previous one in an important respect. If the premises are true, it is not guaranteed that the conclusion is also true. The premises of this argument do not rule out, by themselves, that someone other than the maid or the driver ``did it.'' In this second argument, the conclusion does not follow from the premises. If, like in this argument, the conclusion does not follow from the premises, we say it is \define{invalid}.

%\section{Logical Validity}\todo{consistency and logical truth}
We said the first argument was valid because if the premises will be true, we are guaranteed  that the conclusion is true. In fact in this argument the premises guarantee the truth independently of whether we are talking about butlers, crocodiles, murderers or cake thieves:
\begin{earg}\label{argMaidDriver}
	\prem The crocodile or the kangaroo stole the cake.
	\prem The crocodile did not steal the cake.
	\conc The kangaroo did it.
\end{earg}
 It is irrelevant for the validity of the argument what the premises and the conclusion are about. The argument is valid on all ways of interpreting the premises as long as we understand the sentential connectives `or' and `not'. To put it in less abstract terms an argument is valid if and only if no \define{counterexample} to the argument can be produced. 

\factoidbox{
		An argument is \define{valid} if and only if there is no interpretation such that all the premises are true and the conclusion false. Otherwise the argument is \define{invalid}.
	}	

\newglossaryentry{valid}
{
name=valid,
description={A property of arguments where there conclusion is a consequence of the premises}
}

\newglossaryentry{invalid}
{
name=invalid,
description={A property of arguments that holds when the conclusion is not a consequence of the premises; the opposite of \gls{valid}}
}

We said that the following argument was invalid.	
\begin{earg}\label{argMaidDriver}
	\prem If the driver did it, the maid didn't do it.
	\prem The maid didn't do it.
	\conc The driver did it.
\end{earg}	
Can we find a counterexample to the argument? The answer is yes. For example, if we understand `did it' as `mowed the lawn', then the premises of the argument are both true, but the conclusion is false, as it was the gardener who mowed the lawn (at least on one interpretation/scenario).

Earlier we introduced the idea that the conclusion of an argument is meant to follow from the premises; that it is a consequence of the premises. This motivates the following definition:

\factoidbox{
		A sentence \metaY is a \define{logical consequence} of sentences $\metaX_1$, \dots, $\metaX_n$ if and only if there is no interpretation such that $\metaX_1$, \dots, $\metaX_n$ are all true and \metaY is not true. (We then also say that \metaY \define{logically follows from} $\metaX_1$, \dots, $\metaX_n$.)
	}% or that $\metaX_1$, \dots, $\metaX_n$ \define{entail}~\metaY

Another way of saying that  \metaY is a logical consequence of sentences $\metaX_1$, \dots, $\metaX_n$ is to say that the argument with premises $\metaX_1$, \dots, $\metaX_n$ and conclusion \metaY is valid.

Valid arguments are arguments for which there is no interpretation such that all premises are true but the conclusion is not. It is irrelevant whether such an interpretation is ``reasonable'': no matter how unreasonable an interpretation is that can be used to give a counterexample to an argument, if there exists such an interpretation the argument will not be valid.

Is there a straightforward way of telling whether an argument is logical valid? Is there some feature that sets apart all valid arguments from other (possibly convincing) arguments? We have already seen that validity should not depend on the content of the premises and conclusion. Rather it should only depend on their (logical) form. For instance, consider the valid argument
\begin{earg}
	\prem Either Priya is an ophthalmologist or a dentist.
	\prem Priya isn't a dentist.
	\conc Priya is an ophthalmologist.
\end{earg}
We can describe the ``form'' of this argument as the following pattern:
\begin{earg}
	\prem Either $a$ is an $F$ or a $G$.
	\prem $a$ isn't an $F$.
	\conc $a$ is a $G$.
\end{earg}
Here, $a$, $F$, and $G$ are placeholders for appropriate expressions that, when substituted for $a$, $F$, and $G$, turn the pattern into an argument consisting of sentences (at a first approximation this is also one way of understanding the ``interpretation'' talk). For instance,
\begin{earg}
	\prem Either Mei is a mathematician or a botanist.
	\prem Mei isn't a botanist.
	\conc Mei is a mathematician.
\end{earg}
is an argument of the same form and it is also valid. However, the following argument is not of the same form:
\begin{earg}
	\prem Either Priya is an ophthalmologist or a dentist.
	\prem Priya isn't a dentist.
	\conc Priya is an eye doctor.
\end{earg}
we would have to replace $F$ by different expressions (once by ``ophthalmologist'' and once by ``eye doctor'') to obtain it from the pattern. This argument is not valid. To see that the conclusion follows from the premises we need the additional information that an ophthalmologist is indeed an eye doctor, that is, we need information that ``ophthalmologist'' and ``eye doctor'' mean the same thing.

To see more clearly that the latter argument cannot be deemed valid solely on the basis of its logical form let's consider \emph{its} form:
\begin{earg}
	\prem Either $a$ is an $F$ or a $G$.
	\prem $a$ isn't an $F$.
	\conc $a$ is a $H$.
\end{earg}
In this pattern we can replace $F$ by ``ophthalmologist'' and $H$ by ``eye doctor'' to obtain the original argument.  But here is another argument of the same form which can be obtained by replacing $F$ by ``is a mathematician'', $G$ by ``is a botanist'', and. $H$ by ``is an acrobat'':
\begin{earg}
	\prem Either Mei is a mathematician or a botanist.
	\prem Mei isn't a botanist.
	\conc Mei is an acrobat.
\end{earg}
This argument is clearly not valid. The conclusion does not follow from the premises of the argument.

In valid arguments the conclusion follows from the premises of the argument solely in virtue of its logical form, that is, the logical structure of the premises and the conclusion. This feature is an aspect of the so-called \define{formality} of logic. Much of the present logic course will be devoted to studying and determining valid argument forms and structures, and to make precise the idea of \emph{interpretation} we used in discussing the validity of arguments.

\section{Sound arguments}

Arguments in our sense, as conclusions which (supposedly) follow from premises, are of course used all the time in everyday reasoning, but also philosophical and scientific discourse. When they are, arguments are given to support or even prove their conclusions. Now, if an argument is valid, it will support its conclusion, but \emph{only if} its premises are all true: validity rules out that the premises are true and the conclusion false. It does not, by itself, rule out that the conclusion is false, as the premises can be false. An argument can be valid, but have false premises. In short, a  valid argument may have a conclusion that is not true!

Consider this example:
	\begin{earg}
		\prem Oranges are either fruit or musical instruments.
		\prem Oranges are not fruit.
		\conc Oranges are musical instruments.
	\end{earg}
The conclusion of this argument is ridiculous. Nevertheless, it logically follows from the premises due to the logical form of the argument. For what the argument is concerned, it is not relevant whether the oranges are musical instruments (of course, they are not!). What is relevant is that if according to a (weird) interpretation Oranges are either fruit or musical instruments, but not fruit, then oranges are musical instruments according to that interpretation: \emph{If} both premises are true, \emph{then} the conclusion just has to be true independently of the content of the premises and the conclusion. The argument is valid.

Conversely, having true premises and a true conclusion does not guarantee that the argument is valid. Consider this example:
	\begin{earg}
		\prem London is in England.
		\prem Beijing is in China.
		\conc Paris is in France.
	\end{earg}
The premises and conclusion of this argument are all true, but the argument is invalid. The logical form of premises and conclusion do not guarantee that the conclusion is true whenever the premises are true: on an interpretation on which  `France' is interpreted to mean Great Britain, the conclusion is not true, even though both of the premises would remain true. So the argument is invalid.

The important thing to remember is that validity is not about the truth or falsity of the sentences in the argument. It is about whether the conclusion follows from the premises of the argument in virtue of their logical form; about whether the conclusion is true whenever the premises are true, that is, whether for all interpretations on which all premises are true, the conclusion is true likewise. Nothing about the way things are---whether something is true or false---can by itself determine if an argument is valid. It is often said that logic doesn't care about feelings. Actually, it doesn't care about facts, either.

When we use an argument to prove that its conclusion \emph{is true}, then, we need two things. First, we need the argument to be valid, i.e., we need the conclusion to logically follow from the premises. But we also need the premises to be true. We will say that an argument is \define{sound} if and only if it is both valid and all of its premises are true.

\newglossaryentry{sound}
{
name=sound,
description={A property of arguments that holds if the argument is valid and has all true premises}
}

The flip side of this is that when you want to rebut an argument, you have two options: you can show that (one or more of) the premises are not true, or you can show that the argument is not valid.  Logic, however, will only help you with the latter!





\section{Missing premises}

%One way that an argument can fail to be valid is if one
%or more of its premises are missing, or suppressed, or
%left implicit.
%When someone is using an argument, they often leave premises out, usually when the premise is thought to be
%obvious and so doesn't need to be stated explicitly.

Most arguments we make and evaluate in everyday reasoning are not valid simpliciter. We are often interested in whether the conclusion follows from the premises given certain implicit or explicit background assumptions which the interlocutor has failed to explicitly mention. If the missing background assumption is explicitly added as a premise, the argument may turn out to be valid after all. For example, we already discussed that the argument
\todo{use the arguments above; not a new one}
\begin{earg}
	\prem Either Priya is an ophthalmologist or a dentist.
	\prem Priya isn't a dentist.
	\conc Priya is an eye doctor.
\end{earg}
is strictly speaking not a valid argument. Still it seems to be a good argument in the sense that the truth of the premises seems to guarantee the truth of the conclusion. One explanation for why we think it is a good argument, is that it can easily be turned into a valid argument by adding the (true) premise
\begin{earg}
\prem If Priya is an ophthalmologist, then Priya is an eye doctor.
\end{earg}
Arguably this premise is one we all implicitly assume, which explains why an interlocutor might not feel the need of mentioning it.

Sometimes it is not obvious to tell what kind of implicit underlying assumption are assumed in the formulation of an argument. If someone you disagree with makes an invalid
argument it’s often more useful (and more charitable) to consider
whether there are missing premises rather than to simply dismiss the argument. Perhaps the author or interlocutor was assuming that an additional
premise was so obvious that it didn’t need to be stated.

For example an author might make the following argument:
\begin{earg}
\prem I could not have acted otherwise.
\conc I did not act of my own free will.
\end{earg}
This argument is invalid. But, it can be made valid by addition of the premise:
\begin{earg}
\prem If I could not have acted otherwise, I did not act of my own free will.
\end{earg}

But be careful when you’re filling in ‘missing’ premises.
The aim is to help improve the argument, to make it
more convincing, so you can assess it fairly.
Only add extra premises that seem reasonable, or that you think
the original author would agree with.
There’s no point in adding absurd or unreasonable
premises, or premises that the author wouldn’t
endorse. Then you just create a \emph{strawman} argument –
a caricature of the original argument.


\begin{quotation}
“Just how charitable are you supposed to be when criticizing the views of an opponent? If there are obvious contradictions in the opponent’s case, then of course you should point them out, forcefully. If there are somewhat hidden contradictions, you should carefully expose them to view—and then dump on them. But the search for hidden contradictions often crosses the line into nitpicking, sea-lawyering, and—as we have seen—outright parody. The thrill of the chase and the conviction that your opponent has to be harboring a confusion somewhere encourages uncharitable interpretation, which gives you an easy target to attack. But such easy targets are typically irrelevant to the real issues at stake and simply waste everybody’s time and patience, even if they give amusement to your supporters.''\\
\emph{Daniel C. Dennett (2013). “Intuition Pumps And Other Tools for Thinking”. }
\end{quotation}


Dennett formulates the following four rules (named after
Anatol Rapoport) for “how to compose a successful critical
commentary”:
\begin{enumerate}
\item You should attempt to re-express your target’s position so
clearly, vividly, and fairly that your target says, “Thanks, I
wish I’d thought of putting it that way.”
\item You should list any points of agreement (especially if they
are not matters of general or widespread agreement).
\item You should mention anything you have learned from your
target.
\item Only then are you permitted to say so much as a word of
rebuttal or criticism
\end{enumerate}


%\section{Ambiguity}
%Some sentences depend on context for whether they are true or false. For example the truth of `I have long hair' depends on who is uttering it. Sometimes we will use examples where the context does matter. Everything that we say, though, still holds when the context is held fixed throughout the course of the argument.
%
%
%In English, sentences can be \emph{ambiguous}. There are many sources of ambiguity. One is \emph{lexical ambiguity:} a sentence can contain words which have more than one meaning.  For instance, `bank' can mean the bank of a river, or a financial institution. So I might say that `Katie went to the bank' when she took a stroll along the river, or when she went to deposit a check.  When we talk about sentences, we assume that all words have been disambiguated and it is settled whether `bank' is somewhere you have a picnic or where you deposit your money.
%
%
%Here are some examples of arguments whose validity depends on how we
%interpret ambiguous sentences. It is a good exercise to try to spot how
%one reading makes the argument deductively valid and the other reading
%does not.
%
%\begin{earg}
%\prem {Salvatore brought a hat from Italy.}
%\conc {Salvatore has been to Italy.}
%\end{earg}
%
%\begin{earg}
%\prem {Bill and Barb are married.}
%\conc {Bill is Barb's husband.}
%\end{earg}
%
%\begin{earg}
%\prem {Charlotte's Web is a children's novel about a pig named Wilbur who
%is saved from being slaughtered by an intelligent spider named
%Charlotte. }
%\conc {In C.W., Charlotte saves Wilbur.}
%\end{earg}
%
%\begin{earg}
%\prem {John saw the man on the mountain with a telescope.}
%\conc {The man on the mountain has a telescope.}
%\end{earg}
%
%If we are to get very far with formal logic, we will need to have a way
%of dealing with ambiguity. The key idea here come from a
%mathematician-philosopher named Gottlob Frege. Frege's idea was this.
%Because natural languages contain ambiguous sentences, we need a special
%artificial language for the purpose of studying logical consequence.
%Such a language should be free of ambiguity. Each sentence should have
%exactly one meaning, and exactly one logical form. If we could devise
%such a language, then we could say clearly and systematically which
%arguments are formally valid. Such a language is called a \emph{formal}
%language.
%
%In order to design an unambiguous formal language language, we need to
%get some grip on the sources of ambiguity in natural language. That way,
%we can make sure to prevent those sources from including ambiguity in
%our formal language. What are the sources of ambiguity? Let's consider
%another example.
%
%\begin{quote}
%``I shot an elephant in my pajamas''\footnote{: See
%  \href{http://www.youtube.com/watch?v=NfN_gcjGoJo}{The Marx Brothers
%  Video}.}
%\end{quote}
%
%What makes this sentence ambiguous is that it is not clear which words
%are meant to modify which other words. The sentence might be read in one
%of two ways, either as:
%
%%\includegraphics{/static/img/elephant1.svg}
%
%Or alternatively, as
%
%%\includegraphics{/static/img/elephant2.svg}
%
%The meaning of the sentence depends on how we read it. Of course, we can
%eliminate the ambiguity by specifying which of the trees above we
%intend. But this is pretty awkward. A better way to eliminate the
%ambiguity is to use \emph{parentheses} to stick together the units that
%are supposed to go together---the units that we do not unpack until
%further down the tree.
%
%So we might express the first reading of the sentence by writing ``I
%((shot an elephant) in my pajamas)'', and the second by writing ``I
%(shot (an elephant in my pajamas))''.\footnote{: This idea may seem
%  unfamiliar, but it is actually something that you have been doing for
%  a long time. If you know the difference between
%  ``\((2 + 2) \times 2\)'' and ``\(2 + (2 \times 2)\)'' then you know
%  how to disambiguate language by using parentheses.}
%
%Our formal language will make use of parentheses for the same purpose.
%

\section{Beyond Validity}
As we mentioned, many arguments we make and evaluate in everyday reasoning are not strictly speaking valid. As discussed sometimes important implicit premises are not made explicit. However, the fact that implicit premises are not made explicit point to a more general phenomenon, namely, that in everyday reasoning we take certain conceptual or meaning relations for granted. Going back to the argument involving Priya, we found the the conclusion ``Priya is an eye doctor'' not to be a logical consequence of the premises despite the fact that intuitively one might be inclined to say the the conclusion follows from the premises of the argument. While the argument is not strictly speaking valid, the conclusion follows from the premises of the argument once we acknowledge that `ophthalmologist' is just a fancy word for an eye doctor. More generally, there is no interpretation that respects all conceptual relations between expressions of the language on which all premises are true but the conclusion false, that is, there is no counterexample to the argument involving Priya which acknowledges that `ophthalmologist' and `eye doctor' mean the same thing.

Arguments for which there is no interpretation that respects all conceptual/meaning connections between the various words of our language are sometimes called \define{conceptually valid} and sometimes you'll find the term `validity' used in this sense in the literature . For example, the arguments

\begin{earg}
		\prem Priya is an ophthalmologist.
		\conc Priya is an eye doctor.
	\end{earg}

	\begin{earg}
		\prem Jonas is a bachelor.
		\conc Jonas is an unmarried man.
	\end{earg}

	are both conceptually valid but not (logically) valid according to our definition of validity. All (logically) valid arguments are also conceptually valid, but not the other way around.

Perhaps in everyday reasoning we are often judging arguments whether they are conceptually valid as opposed to valid simpliciter. However, while the cases of conceptual validity discussed have been reasonably clear, it is sometimes not that easy to make precise and agree on the exact underlying conceptual relations. Consequently, while, as we shall see, it is relatively straightforward to decide whether an argument is (logically), this becomes much more tricky turning to conceptual validity. For this reason it is preferable to focus on (logical) validity and focus on what additional premises are needed to turn an intuitively convincing argument into a valid argument. In a second step one can then ask whether the additional premises are conceptual truths. If the answer is yes, then we can deem the argument conceptually valid despite being invalid in the strict sense of validity.

%We could list further categories of validity, but the point is that when we discuss whether an argument is valid or not we need to make sure that we agree on the category of validity we are interested in: in the natural sciences we might be interested in nomological validity. In logic we are interested in logical validity.

%When we consider cases of various kinds in order to evaluate the validity of an argument, we will make a few assumptions. The first assumption is that every case makes every sentence true or false---at least, every sentence in the argument under consideration.
%So imagined scenarios have to specify all relevant facts.
%Any imagined scenario which leaves it undetermined if a sentence in our argument is true will not be considered as a potential counterexample.
%For instance, a scenario where Priya is a dentist but not an ophthalmologist will count as a case to be considered in the first few arguments in this section, but not as a case to be considered in the last two: it doesn't tell us if Mei is a mathematician, a botanist, or an acrobat.
%If a case doesn't make a sentence true, we say it makes it \define{false}. We'll thus assume that cases make sentences true or false but never both.\footnote{Even if these assumptions seem common-sensical to you, they are controversial among philosophers of logic. First of all, there are logicians who want to consider cases where sentences are neither true nor false, but have some kind of intermediate level of truth. More controversially, some philosophers think we should allow for the possibility of sentences to be both true and false at the same time. There are systems of logic in which sentences can be neither true nor false, or both, but we will not discuss them in this book.}

%Depending on what kinds of cases we consider as potential counterexamples, then, we arrive at different notions of consequence and validity. We might call an argument \define{nomologically valid} if there are no counterexamples that don't violate the laws of nature, and an argument \define{conceptually valid} if there are no counterexamples that don't violate conceptual connections between words.
%For both of these notions of validity, aspects of the world (e.g., what the laws of nature are) and aspects of the meaning of the sentences in the argument (e.g., that ``ophthalmologist'' just means a kind of eye doctor) figure into whether an argument is valid.

%\section{Formal validity}
\section{Ampliative Arguments}
There is further reason why many arguments of everyday reasoning are not strictly speaking valid: not all arguments of everyday reasoning are so-called \define{deductive} arguments. In deductive arguments the truth of the premises is supposed to guarantee the truth of the conclusion. Not all good arguments are deductive and sometimes there are no plausible missing premises you could add to someone's argument to make it valid.

However, this doesn't necessarily mean that the interlocutor was wrong or mistaken.
Deductively valid arguments with plausible premises are good arguments, but they aren't the only good arguments there are. This is just as well, since many arguments we give in our everyday lives are not deductively valid, even after filling in plausible missing premises. Here's an example:
	\begin{earg}
		\prem In January 1997, it rained in London.
		\prem In January 1998, it rained in London.
		\prem In January 1999, it rained in London.
		\prem In January 2000, it rained in London.
	\conc It rains every January in London.
\end{earg}

This argument generalises from observations about several cases to a conclusion about all cases---in each year listed, it rained in January, so it does in every year. Such arguments are called \define{inductive} arguments. The argument could be made stronger by adding additional premises before drawing the conclusion: In January 2001, it rained in London; In January 2002\ldots. But, however many premises of this form we add, the argument will remain invalid. Even if it has rained in London in every January thus far, it remains possible that London will stay dry next January. The point of all this is that inductive arguments—even good inductive arguments—are not (deductively) valid. They are not watertight. The premises might make the conclusion very likely, but they don't absolutely guarantee its truth. Unlikely though it might be, it is possible for their conclusion to be false, even when all of their premises are true.

Inductive arguments of the sort just given belong to a species of argument called \define{ampliative arguments}. This means that the conclusion goes beyond what you find in the premises. That is, the premises don't guarantee, or entail, the conclusion. They do, however, provide some support for it. These arguments are deductively invalid. They may be good and useful, however it is important to know the difference.

In this book, we will set aside the question of what makes for a good ampliative argument and focus instead on sorting the deductively valid arguments from the deductively invalid ones.
But we pause here to mention some further forms of ampliative argument.

Inductive arguments, like the one we saw above, \todo{At end of 2.1. I said we're not allowed this terminology!}allow one to infer from a series of observed cases to a generalization that covers them: from all observed $F$s have been $G$s, we infer all $F$s are $G$s. We use these all the time. Every time I've drunk water from my tap, it's quenched my thirst; therefore, every time I ever drink water from my tap, it will quench my thirst. Every time I've stroked my neighbour's cat, it hasn't bitten me; therefore, every time I ever stroke my neighbour's cat, it won't bite me. And it's a form of arguments much beloved by scientists. Every time we've measured the acceleration of a body falling, it's matched Newton's theory, therefore, all bodies are governed by Newton's theory. The premises of these argument seem to make their conclusions likely without guaranteeing them. The areas of philosophy called inductive logic or confirmation theory try to make precise what that means and why it's true. And of course inductive arguments can go wrong. Before I visited Australia, every swan I'd seen was white, and so I concluded that all swans were white; but when I visited Australia, I realised my conclusion was wrong, because some swans there are black.

A closely related, but different form of argument, is \define{statistical}. Here, we start with an observation about the proportion of Fs that are Gs in a sample that we've observed, and we infer that the same proportion of Fs are Gs in general. So, for instance, if I poll 1,000 people in Scotland eligible to vote in a second independence referendum, and 600 say that they'll vote yes, I might infer that 60\% of all eligible voters will vote yes. Or if I test 1,000,000 people in England for an active infection, and 20,000 test positive, I might infer that 2\% of the whole population has an active infection. How good these argument are depends on a number of things, and these are studied by statisticians. For instance, suppose you picked the 1,000 Scottish voters entirely at random from an anonymised version of the electoral register. But suppose that, when you deanonymised, you learned that, by chance, all of the people you'd picked were over 65, or they all lived on the Isle of Skye. Then you might worry that your sample, though random, was unrepresentative of the population as a whole. This question is a genuine concern for randomised controlled trials in medicine.

Abductive arguments provide an inference from a phenomenon you've observed to the \emph{best explanation} of that phenomenon: from $E$, and the best explanation of $E$ is $H$, you might conclude $H$. Again, this is extremely widespread. A classic sort of example would be the inferences that detectives draw during their investigations. They look at the evidence and the possible explanations of it, and they tend to conclude in favour of the best one. And similarly for doctors looking at a patient's suite of symptoms and trying to discover what ails them. Another important example comes from science. Here is Charles Darwin explaining what convinces him of his theory of natural selection:
\begin{quotation}
``It can hardly be supposed that a false theory would explain, in so satisfactory a manner as does the theory of natural selection, the several large classes of facts above specified. It has recently been objected that this is an unsafe method of arguing; but it is a method used in judging of the common events of life, and has often been used by the greatest natural philosophers."\\ (Charles Darwin, On the origin of species by means of natural selection (6th ed.).
London: John Murray)
\end{quotation}








\begin{practiceproblems}
\problempart
\begin{enumerate}
\item What kind of things are valid or invalid?
\item When is an argument said to be valid?
\item When is an argument said to be sound?
\end{enumerate}

\problempart
Are the following valid? If it is invalid, describe a counterexample.
\begin{enumerate}
\item[x.]
\begin{earg}
\prem Every good zoo has a giraffe.
\prem It is a zoo.
\conc It has a giraffe.
\end{earg}
\prem Invalid. It is a zoo, but not a good one. (And has a giraffe.)
\item
\begin{earg}
\prem Everyone in group 1 handed in their homework.
\prem Jenny is in group 1.
\conc Jenny handed in her homework.
\end{earg}
\myanswer{valid}
\begin{earg}
\prem If she won the lottery then she is rich.
\prem She is rich.
\conc She won the lottery.
\end{earg}
\myanswer{Invalid, there are many ways to get rich\ldots}
%\item
%\begin{earg}
%\prem The law is unfair.
%\conc The law must be changed.
%\end{earg}
\item
\begin{earg}
\prem Most people are scared of spiders.
\conc Oscar is scared of spiders.
\end{earg}\myanswer{Invalid}
\item
\begin{earg}
\prem She is a donkey.
\conc She does not talk.
\end{earg}
%\item
%\begin{earg}
%\prem That was a terrible crime.
%\conc He must be punished.
%\end{earg}
\end{enumerate}


\problempart
Which of the following arguments is valid? Which is invalid?

\begin{earg}
\item Socrates is a man.
\item All men are carrots.
\item[$\therefore$]  Socrates is a carrot. \hfill \myanswer{Valid}
\end{earg}

\begin{earg}
\item Abe Lincoln was either born in Illinois or he was president.
\item Abe Lincoln was not president.
\item[$\therefore$] Abe Lincoln was born in Illinois. \hfill \myanswer{Valid}
\end{earg}

\begin{earg}
\item If I pull the trigger, Abe Lincoln will die.
\item I do not pull the trigger.
\item[$\therefore$] Abe Lincoln will not die. \hfill \myanswer{Invalid \\ Abe Lincoln might die for some other reason: someone else might pull the trigger; he might die of old age.}
\end{earg}

\begin{earg}
\item Abe Lincoln was either from France or from Luxembourg.
\item Abe Lincoln was not from Luxembourg.
\item[$\therefore$] Abe Lincoln was from France. \hfill \myanswer{Valid}
\end{earg}

\begin{earg}
\item If the world were to end today, then I would not need to get up tomorrow morning.
\item I will need to get up tomorrow morning.
\item[$\therefore$] The world will not end today. \hfill \myanswer{Valid}
\end{earg}

\begin{earg}
\item Joe is now 19 years old.
\item Joe is now 87 years old.
\item[$\therefore$] Bob is now 20 years old. \hfill \myanswer{Not valid (although arguably conceptually valid. To make it valid one would need to add the premise that one cannot be both 19 years old and 87 years old at the same time.}
%\\myanswer{An argument is valid if and only if it is impossible for all the premises to be true and the conclusion false. It is impossible for all the premises to be true; so it is certainly impossible that the premises are all true and the conclusion is false.}
\end{earg}



\problempart
\label{pr.EnglishCombinations}
Could there be:
	\begin{earg}
		\item A valid argument that has one false premise and one true premise? \hfill \myanswer{Yes. \\Example: the first argument, above.}
		\item A valid argument that has only false premises? \hfill \myanswer{Yes.\\Example: Socrates is a frog, all frogs are excellent pianists, therefore Socrates is an excellent pianist.}
		\item A valid argument with only false premises and a false conclusion? \hfill \myanswer{Yes. \\The same example will suffice.}
		\item An invalid argument that can be made valid by the addition of a new premise? \hfill\myanswer{Yes.\\ Plenty of examples, but let me offer a more general observation. We can \emph{always} make an invalid argument valid, by adding a contradiction into the premises. For an argument is valid if and only if it is impossible for all the premises to be true and the conclusion false. If the premises are contradictory, then it is impossible for them all to be true (and the conclusion false).}
		\item A valid argument that can be made invalid by the addition of a new premise? \hfill \myanswer{No.\\ An argument is valid if and only if it is impossible for all the premises to be true and the conclusion false. Adding another premise will only make it harder for the premises all to be true together.}
	\end{earg}
In each case: if so, give an example; if not, explain why not.

\medskip

%\problempart
%\label{pr.Ampliative}
\todo{!!!}
\end{practiceproblems}


%\chapter{Other logical notions}\label{s:BasicNotions}

%In \S\ref{s:Valid}, we introduced the ideas of consequence and of valid argument.  This is one of the most important ideas in logic. In this section, we will introduce are some similarly important ideas. They all rely, as did validity, on the idea that sentences are true (or not) in cases. For the rest of this section, we'll take cases in the sense of conceivable scenario, i.e., in the sense in which we used them to define conceptual validity. The points we made about different kinds of validity can be made about our new notions along similar lines: if we use a different idea of what counts as a ``case'' we will get different notions.  And as logicians we will, eventually, consider a more permissive definition of case than we do here.

%\section{Truth values}
%As we said in \S\ref{s:Arguments}, arguments consist of premises and a conclusion. Note that many kinds of English sentence cannot be used to express premises or conclusions of arguments. For example:
%	\begin{ebullet}
%		\item \textbf{Questions}, e.g.\ `are you feeling sleepy?'
%		\item \textbf{Imperatives}, e.g.\ `Wake up!'
%		\item \textbf{Exclamations}, e.g.\ `Ouch!'
%	\end{ebullet}
%The common feature of these three kinds of sentence is that they are not \emph{assertoric}: they cannot be true or false. It does not even make sense to ask whether a \emph{question} is true (it only makes sense to ask whether the \emph{answer} to a question is true).

%The general point is that, the premises and conclusion of an argument must be capable of having a \define{truth value}. The two truth values that concern us are just True and False.

%\section{Joint possibility}

%Consider these two sentences:
	%\begin{ebullet}
		%\item[B1.] Jane's only brother is shorter than her.
		%\item[B2.] Jane's only brother is taller than her.
	%\end{ebullet}
%Logic alone cannot tell us which, if either, of these sentences is true. Yet we can say that \emph{if} the first sentence (B1) is true, \emph{then} the second sentence (B2) must be false. Similarly, if B2 is true, then B1 must be false. There is no possible scenario where both sentences are true together. These sentences are incompatible with each other, they cannot all be true at the same time. This motivates the following definition:
%	\factoidbox{
%		Sentences are \define{jointly possible} if and only if there is a case where they are all true together.
%	}
%B1 and B2 are \emph{jointly impossible}, while, say, the following two sentences are jointly possible:
%	\begin{ebullet}
%		\item[B1.] Jane's only brother is shorter than her.
%		\item[B2.] Jane's only brother is younger than her.
%	\end{ebullet}

%\newglossaryentry{possibility}
%{
%name=joint possibility,
%text={jointly possible},
%description={A property possessed by some sentences when they are all true in a single case}
%}

%We can ask about the joint possibility of any number of sentences. For example, consider the following four sentences:
%	\begin{ebullet}
%		\item[G1.] \label{MartianGiraffes} There are at least four giraffes at the wild animal park.
%		\item[G2.] There are exactly seven gorillas at the wild animal park.
%		\item[G3.] There are not more than two martians at the wild animal park.
%		\item[G4.] Every giraffe at the wild animal park is a martian.
%	\end{ebullet}
%G1 and G4 together entail that there are at least four martian giraffes at the park. This conflicts with G3, which implies that there are no more than two martian giraffes there. So the sentences G1--G4 are jointly impossible. They cannot all be true together. (Note that the sentences G1, G3 and G4 are jointly impossible. But if sentences are already jointly impossible, adding an extra sentence to the mix cannot make them jointly possible!)

%\section[Necessary truths and falsehoods]{Necessary truths, necessary falsehoods, and contingency}

%In assessing arguments for validity, we care about what would be true \emph{if} the premises were true, but some sentences just \emph{must} be true. Consider these sentences:
%	\begin{earg}
%		\item[\ex{Acontingent}] It is raining.
%		\item[\ex{Atautology}] Either it is raining here, or it is not.
%		\item[\ex{Acontradiction}] It is both raining here and not raining here.
%	\end{earg}
%In order to know if sentence \ref{Acontingent} is true, you would need to look outside or check the weather channel. It might be true; it might be false. A sentence which is capable of being true and capable of being false (in different circumstances, of course) is called \define{contingent}.

%\newglossaryentry{contingent sentence}
%{
%name=contingent sentence,
%description={A sentence that is neither a \gls{necessary truth} nor a \gls{necessary falsehood}; a sentence that in some case is true and in some other case, false}
%}

%Sentence \ref{Atautology} is different. You do not need to look outside to know that it is true. Regardless of what the weather is like, it is either raining or it is not. That is a \define{necessary truth}.


%\end{practiceproblems}
