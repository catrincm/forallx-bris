%!TEX root = forallxbris.tex
\part{Identity}
\label{ch.identity}
\addtocontents{toc}{\protect\mbox{}\protect\hrulefill\par}

\chapter{Identity}

Consider this sentence:
\begin{earg}
\item[\ex{else1}] Pavel owes money to everyone
\end{earg}
Let the domain be people; this will allow us to symbolize `everyone' as a universal quantifier. Offering the symbolization key:
	\begin{ekey}
		\item[Oxy] \gap{x} owes money to \gap{y}
		\item[p] Pavel
	\end{ekey}
we can symbolize sentence \ref{else1} by `$\forall x Opx$'. But this has a (perhaps) odd consequence. It requires that Pavel owes money to every member of the domain (whatever the domain may be). The domain certainly includes Pavel. So this entails that Pavel owes money to himself. 

Perhaps we meant to say:
	\begin{earg}
		\item[\ex{else1b}] Pavel owes money to everyone \emph{else}
		\item[\ex{else1c}] Pavel owes money to everyone \emph{other than} Pavel
		\item[\ex{else1d}] Pavel owes money to everyone \emph{except} Pavel himself
	\end{earg}
but we do not know how to deal with the italicised words yet. The solution is to add another symbol to FOL. 

\section{Adding identity}

The new symbol we add is `$=$'. This is a symbol that we can use for \emph{identity}. 

We will then be able symbolise 
\begin{earg}
\item[\ex{superman}] Clark Kent is Superman.
\end{earg}
as $k=s$, using the symbolisation key
\begin{ekey}
\item[k] Clark Kent
\item[s] Superman
\end{ekey} 
This will also be a symbolisations of paraphrases of \ref{superman}:
\begin{earg}
\item[\ex{superman1b}] Clark Kent and Superman are the same person.
\item[\ex{superman1c}] Clark Kent is identical to Superman. 
\end{earg}

Using `$=$' we will now be able to symbolise sentences \ref{else1b}--\ref{else1d}. All of these sentences can be  paraphrased as `Everyone who is not Pavel is owed money by Pavel'. Paraphrasing some more, we get: `For all x, if x is not Pavel, then x is owed money by Pavel'. Now that we are armed with our new identity symbol, we can symbolize this as `$\forall x (\enot\, x\eid p \eif Opx)$'.

In addition to sentences that use the word `else', `other than' and `except', identity will be helpful when symbolizing some sentences that contain the words `besides' and `only.' Consider these examples:

\begin{earg}
\item[\ex{else3}] No one besides Pavel owes money to Hikaru.
\item[\ex{else4}] Only Pavel owes Hikaru money.
\end{earg}
Let `$h$' name Hikaru. Sentence \ref{else3} can be paraphrased as, `No one who is not Pavel owes money to Hikaru'. This can be symbolized by `$\enot\exists x(\enot\,x \eid p \eand Oxh)$'. Equally, sentence \ref{else3} can be paraphrased as `for all x, if x owes money to Hikaru, then x is Pavel'. It can then be symbolized as `$\forall x (Oxh \eif x \eid p)$'.

Sentence \ref{else4} can be treated similarly, but there is one subtlety here. Do either sentence \ref{else3} or \ref{else4} entail that Pavel himself owes money to Hikaru? 



%This last sentence contains the formula `$\enot \,x \eid p$'. That might look a bit strange, because the symbol that comes immediately after the `$\enot$' is a variable, rather than a predicate, but this is not a problem. We are simply negating the entire formula, `$x \eid p$'. 

%Officially, we have already given an account for FOL that doesn't allow for identity, so now we need to present an alternative: FOL-with-identity. However, we will typically say `FOL' to mean FOL-with-identity, and will simply say FOL-without-identity when we want to deal with the simpler system. 





\section{There are at least\ldots}
We will now look at more that we can do armed with our new identity symbol. 
We can also use identity to say how many things there are of a particular kind. For example, consider these sentences:
\begin{earg}
\item[\ex{atleast1}] There is at least one apple
\item[\ex{atleast2}] There are at least two apples
\item[\ex{atleast3}] There are at least three apples
\end{earg}
We will use the symbolization key:
	\begin{ekey}
		\item[Ax] \gap{x} is an apple
	\end{ekey}
Sentence \ref{atleast1} does not require identity. It can be adequately symbolized by `$\exists x Ax$': There is an apple; perhaps many, but at least one.

It might be tempting to also symbolize sentence \ref{atleast2} without identity. Yet consider the sentence `$\exists x \exists y(Ax \eand Ay)$'. Roughly, this says that there is some apple $x$ in the domain and some apple $y$ in the domain. Since nothing precludes these from being one and the same apple, this would be true even if there were only one apple. In order to make sure that we are dealing with \emph{different} apples, we need an identity predicate. Sentence \ref{atleast2} needs to say that the two apples that exist are not identical, so it can be symbolized by `$\exists x \exists y((Ax \eand Ay) \eand \enot\, x\eid y)$'.

Sentence \ref{atleast3} requires talking about three different apples. Now we need three existential quantifiers, and we need to make sure that each will pick out something different: `$\exists x \exists y\exists z[((Ax \eand Ay) \eand Az) \eand ((\enot\, x\eid y \eand \enot\, y\eid z) \eand \enot\, x\eid z)]$'.

\section{There are at most\ldots}
Now consider these sentences:
\begin{earg}
	\item[\ex{atmost1}] There is at most one apple
	\item[\ex{atmost2}] There are at most two apples
\end{earg}
Sentence \ref{atmost1} can be paraphrased as, `It is not the case that there are at least \emph{two} apples'. This is just the negation of sentence \ref{atleast2}: 
$$\enot \exists x \exists y[(Ax \eand Ay) \eand \enot\, x\eid y]$$
But sentence \ref{atmost1} can also be approached in another way. It means that if you pick out an object and it's an apple, and then you pick out an object and it's also an apple, you must have picked out the same object both times. With this in mind, it can be symbolized by
$$\forall x\forall y\bigl[(Ax \eand Ay) \eif x=y\bigr]$$
The two sentences will turn out to be logically equivalent.

In a similar way, sentence \ref{atmost2} can be approached in two equivalent ways. It can be paraphrased as, `It is not the case that there are \emph{three} or more distinct apples', so we can offer:
$$\enot \exists x \exists y\exists z(Ax \eand Ay \eand Az \eand \enot\, x\eid y \eand \enot\, y\eid z \eand \enot\, x\eid z)$$
Alternatively we can read it as saying that if you pick out an apple, and an apple, and an apple, then you will have picked out (at least) one of these objects more than once. Thus:
$$\forall x\forall y\forall z\bigl[(Ax \eand Ay \eand Az) \eif (x=y \eor x=z \eor y=z)\bigr]$$


\section{There are exactly\ldots}
We can now consider precise statements, like:
\begin{earg}
\item[\ex{exactly1}] There is exactly one apple.
\item[\ex{exactly2}] There are exactly two apples.
\item[\ex{exactly3}] There are exactly three apples.
\end{earg}
Sentence \ref{exactly1} can be paraphrased as, `There is \emph{at least} one apple and there is \emph{at most} one apple'. This is just the conjunction of sentence \ref{atleast1} and sentence \ref{atmost1}. So we can offer:
$$\exists x Ax \eand \forall x\forall y\bigl[(Ax \eand Ay) \eif x=y\bigr]$$
But it is perhaps more straightforward to paraphrase sentence \ref{exactly1} as, `There is a thing x which is an apple, and everything which is an apple is just x itself'. Thought of in this way, we offer: 
$$\exists x\bigl[Ax \eand \forall y(Ay \eif x= y)\bigr]$$
Similarly, sentence \ref{exactly2} may be paraphrased as, `There are \emph{at least} two apples, and there are \emph{at most} two apples'. Thus we could offer 
\begin{multline*}
  \exists x \exists y((Ax \eand Ay) \eand \enot\, x\eid y) \eand {}\\
  \forall x\forall y\forall z\bigl[((Ax \eand Ay) \eand Az) \eif ((x=y \eor x=z) \eor y=z)\bigr]
\end{multline*}
More efficiently, though, we can paraphrase it as `There are at least two different apples, and every apple is one of those two apples'. Then we offer:
$$\exists x\exists y\bigl[((Ax \eand Ay) \eand \enot\, x\eid y) \eand \forall z(Az \eif ( x= z \eor y \eid z)\bigr]$$
Finally, consider these sentence:
\begin{earg}
\item[\ex{exactly2things}] There are exactly two things
\item[\ex{exactly2objects}] There are exactly two objects
\end{earg}
It might be tempting to add a predicate to our symbolization key, to symbolize the English predicate `\blank\ is a thing' or `\blank\ is an object', but this is unnecessary. Words like `thing' and `object' do not sort wheat from chaff: they apply trivially to everything, which is to say, they apply trivially to every thing. So we can symbolize either sentence with either of the following:
	\begin{center}
		$\exists x \exists y \enot\, x\eid y \eand \enot \exists x \exists y \exists z ((\enot\, x\eid y \eand \enot\, y\eid z) \eand \enot\, x\eid z)$\\
		
		$\exists x \exists y \bigl[\enot\, x\eid y \eand \forall z(x=z \eor y \eid z)\bigr]$
	\end{center}

\practiceproblems

%\solutions
%\problempart
%\label{pr.FOLcandies}
%Using the following symbolization key:
%\begin{ekey}
%\item[\text{domain}] candies
%\item[Cx] \gap{x} has chocolate in it.
%\item[Mx] \gap{x} has marzipan in it.
%\item[Sx] \gap{x} has sugar in it.
%\item[Tx] Boris has tried \gap{x}.
%\item[Bxy] \gap{x} is better than \gap{y}.
%\end{ekey}
%symbolize the following English sentences in FOL:
%\begin{earg}
%\item Boris has never tried any candy.
%\item Marzipan is always made with sugar.
%\item Some candy is sugar-free.
%\item The very best candy is chocolate.
%\item No candy is better than itself.
%\item Boris has never tried sugar-free chocolate.
%\item Boris has tried marzipan and chocolate, but never together.
%%\item Boris has tried nothing that is better than sugar-free marzipan.
%\item Any candy with chocolate is better than any candy without it.
%\item Any candy with chocolate and marzipan is better than any candy that lacks both.
%\end{earg}



\problempart Explain why:
	\begin{ebullet}
		\item   `$\exists x \forall y(Ay \eiff x= y)$' is a good symbolization of `there is exactly one apple'.
		\item `$\exists x \exists y \bigl[\enot\, x\eid y \eand \forall z(Az \eiff (x= z \eor y \eid z))\bigr]$' is a good symbolization of `there are exactly two apples'.
	\end{ebullet}		


\chapter{Definite descriptions}\label{subsec.defdesc}
Consider sentences like:
	\begin{earg}
		\item[\ex{traitor1}] Nick is the traitor.
		\item[\ex{traitor2}] The traitor went to Cambridge.
		\item[\ex{traitor3}] The traitor is the deputy 
	\end{earg}
These are definite descriptions: they are meant to pick out a \emph{unique} object. They should be contrasted with \emph{indefinite} descriptions, such as `Nick  is \emph{a} traitor'. They should equally be contrasted with \emph{generics}, such as `\emph{The} whale is a mammal' (it's inappropriate to ask \emph{which} whale). The question we face is: how should we deal with definite descriptions in FOL?


\section{Treating definite descriptions as terms}
One option would be to introduce new names whenever we come across a definite description. This is probably not a great idea. We know that \emph{the} traitor---whoever it is---is indeed \emph{a} traitor. We want to preserve that information in our symbolization.

A second option would be to use a \emph{new} definite description operator, such as `$\maththe$'. The idea would be to symbolize `the F' as `$\maththe xFx$'; or to symbolize `the G' as `$\maththe xGx$', etc. Expression of the form $\maththe \meta{x} \meta{A}\meta{x}$ would then behave like names. If we followed this path, then using the following symbolization key:
	\begin{ekey}
		\item[\text{domain}] people
		\item[Tx] \gap{x} is a traitor
		\item[Dx] \gap{x} is a deputy
		\item[Cx] \gap{x} went to Cambridge
		\item[n] Nick
	\end{ekey}
We could symbolize sentence \ref{traitor1} with `$\maththe x Tx = n$', sentence \ref{traitor2} with `$C\maththe xTx$', and sentence \ref{traitor3} with `$\maththe x Tx = \maththe x Dx$'. 

However, it would be nice if we didn't have to add a new symbol to FOL. And indeed, we might be able to make do without one.

\section{Russell's analysis}
Bertrand Russell offered an analysis of definite descriptions. Very briefly put, he observed that, when we say `the F' in the context of a definite description, our aim is to pick out the \emph{one and only} thing that is F (in the appropriate context). Thus Russell analysed the notion of a definite description as follows:\footnote{Bertrand Russell, `On Denoting', 1905, \emph{Mind 14}, pp.\ 479--93; also Russell, \emph{Introduction to Mathematical Philosophy}, 1919, London: Allen and Unwin, ch.\ 16.}
	\begin{align*}
		\text{the F is G \textbf{iff} }&\text{there is at least one F, \emph{and}}\\
	&\text{there is at most one F, \emph{and}}\\	
	&\text{every F is G}
\end{align*}
Note a very important feature of this analysis: \emph{`the' does not appear on the right-side of the equivalence.} Russell is aiming to provide an understanding of definite descriptions in terms that do not presuppose them. 

Now, one might worry that we can say `the table is brown' without implying that there is one and only one table in the universe. But this is not (yet) a fantastic counterexample to Russell's analysis. The domain of discourse is likely to be restricted by context (e.g.\ to objects in my line of sight).

If we accept Russell's analysis of definite descriptions, then we can symbolize sentences of the form `the F is G' using our strategy for numerical quantification in FOL. After all, we can deal with the three conjuncts on the right-hand side of Russell's analysis as follows:
	$$\exists x Fx \eand \forall x \forall y ((Fx \eand Fy) \eif x \eid y) \eand \forall x (Fx \eif Gx)$$
In fact, we could express the same point rather more crisply, by recognizing that the first two conjuncts just amount to the claim that there is \emph{exactly} one F, and that the last conjunct tells us that that object is F. So, equivalently, we could offer:
	$$\exists x \bigl[(Fx \eand \forall y (Fy \eif x \eid y)) \eand Gx\bigr]$$
Using these sorts of techniques, we can now symbolize sentences \ref{traitor1}--\ref{traitor3} without using any new-fangled fancy operator, such as `$\maththe$'. 

Sentence \ref{traitor1} is exactly like the examples we have just considered. So we would symbolize it by `$\exists x (Tx \eand \forall y(Ty \eif x \eid y) \eand x \eid n)$'. 

Sentence \ref{traitor2} poses no problems either: `$\exists x (Tx \eand \forall y(Ty \eif x \eid y) \eand Cx)$'.

Sentence \ref{traitor3} is a little trickier, because it links two definite descriptions. But, deploying  Russell's analysis, it can be paraphrased by `there is exactly one traitor, x, and there is exactly one deputy, y, and x = y'. So we can symbolize it by: 
$$\exists x \exists y \bigl(\bigl[Tx \eand \forall z(Tz \eif x \eid z)\bigr] \eand \bigl[Dy \eand \forall z(Dz \eif y \eid z)\bigr] \eand x = y\bigr)$$
Note that we have made sure that the formula `$x \eid y$' falls within the scope of both quantifiers!

\section{Empty definite descriptions}
One of the nice features of Russell's analysis is that it allows us to handle \emph{empty} definite descriptions neatly. 

France has no king at present. Now, if we were to introduce a name, `$k$', to name the present King of France, then everything would go wrong: remember from \S\ref{s:FOLBuildingBlocks} that a name must always pick out  some object in the domain, and whatever we choose as our domain, it will contain no present kings of France. 

Russell's analysis neatly avoids this problem. Russell tells us to treat definite descriptions using predicates and quantifiers, instead of names. Since predicates can be empty (see \S\ref{s:MoreMonadic}), this means that no difficulty now arises when the definite description is empty. 

Indeed, Russell's analysis helpfully highlights two ways to go wrong in a claim involving a definite description. To adapt an example from Stephen Neale (1990),\footnote{Neale, \emph{Descriptions}, 1990, Cambridge: MIT Press.}  suppose Alex claims:
	\begin{earg}
		\item[\ex{kingdate}] I am dating the present king of France.
	\end{earg}
Using the following symbolization key:
	\begin{ekey}
		\item[a] Alex
		\item[Kx] \gap{x} is a present king of France
		\item[Dxy] \gap{x} is dating \gap{y}
	\end{ekey}
Sentence \ref{kingdate} would be symbolized by `$\exists x (\forall y(Ky \eiff  x \eid y) \eand Dax)$'. Now, this can be false in (at least) two ways, corresponding to these two different sentences:
	\begin{earg}
		\item[\ex{outernegation}] There is no one who is both the present King of France and  such that he and Alex are dating.
		\item[\ex{innernegation}] There is a unique present King of France, but Alex is not dating him.
	\end{earg}
Sentence \ref{outernegation} might be paraphrased by `It is not the case that: the present King of France and Alex are dating'. It will then be symbolized by `$\enot \exists x\bigl[(Kx \eand \forall y(Ky \eif  x \eid y)) \eand Dax \bigr]$'. We might call this \emph{outer} negation, since the negation governs the entire sentence. Note that it will be true if there is no present King of France.

Sentence \ref{innernegation} can be symbolized by `$\exists x ((Kx \eand \forall y(Ky \eif x \eid y)) \eand \enot Dax)$'. We might call this \emph{inner} negation, since the negation occurs within the scope of the definite description. Note that its truth requires that there is a present King of France, albeit one who is not dating Alex.

\section{The adequacy of Russell's analysis}
How good is Russell's analysis of definite descriptions? This question has generated a substantial philosophical literature, but we will restrict ourselves to two observations.

One worry focusses on Russell's treatment of empty definite descriptions. If there are no Fs, then on Russell's analysis, both `the F is G' is and  `the F is non-G' are false. P.F.\ Strawson suggested that such sentences should not be regarded as false, exactly.\footnote{P.F.\ Strawson, `On Referring', 1950, \emph{Mind 59}, pp.\ 320--34.} Rather, they involve presupposition failure, and need to be regarded as \emph{neither} true \emph{nor} false. 

If we agree with Strawson here, we will need to revise our logic. For, in our logic, there are only two truth values (True and False), and every sentence is assigned exactly one of these truth values. 

But there is room to disagree with Strawson. Strawson is appealing to some linguistic intuitions, but it is not clear that they are very robust. For example: isn't it just \emph{false}, not `gappy', that Tim is dating the present King of France?

Keith Donnellan raised a second sort of worry, which (very roughly) can be brought out by thinking about a case of mistaken identity.\footnote{Keith Donnellan, `Reference and Definite Descriptions', 1966, \emph{Philosophical Review 77}, pp.\ 281--304.} Two men stand in the corner: a very tall man drinking what looks like a gin martini; and a very short man drinking what looks like a pint of water. Seeing them, Malika says:
	\begin{earg}
		\item[\ex{gindrinker}] The gin-drinker is very tall!
	\end{earg}
Russell's analysis will have us render Malika's sentence as:
	\begin{earg}
		\item[\ref{gindrinker}$'$.] There is exactly one gin-drinker [in the corner], and whoever is a gin-drinker [in the corner] is very tall.
	\end{earg}
Now suppose that the very tall man is actually drinking \emph{water} from a martini glass; whereas the very short man is drinking a pint of (neat) gin. By Russell's analysis, Malika has said something false, but don't we want to say that Malika has said something \emph{true}? 

Again, one might wonder how clear our intuitions are on this case. We can all agree that Malika intended to pick out a particular man, and say something true of him (that he was tall). On Russell's analysis, she actually picked out a different man (the short one), and consequently said something false of him. But  maybe advocates of Russell's analysis only need to explain \emph{why} Malika's intentions were frustrated, and so why she said something false. This is easy enough to do:  Malika said something false because she had false beliefs about the men's drinks; if Malika's beliefs about the drinks had been true,  then she would have said something true.\footnote{Interested parties should read Saul Kripke, `Speaker Reference and Semantic Reference', 1977, in French et al (eds.), \emph{Contemporary Perspectives in the Philosophy of Language}, Minneapolis: University of Minnesota Press, pp.\ 6-27.}

To say much more here would lead us into deep philosophical waters. That would be no bad thing, but for now it would distract us from the immediate purpose of learning formal logic. So, for now, we will stick with Russell's analysis of definite descriptions, when it comes to putting things into FOL. It is certainly the best that we can offer, without significantly revising our logic, and it is quite defensible as an analysis. 

\practiceproblems

\problempart
Using the following symbolization key:
\begin{ekey}
\item[\text{domain}] people
\item[Kx] \gap{x} knows the combination to the safe.
\item[Sx] \gap{x} is a spy.
\item[Vx] \gap{x} is a vegetarian.
\item[Txy] \gap{x} trusts \gap{y}.
\item[h] Hofthor
\item[i] Ingmar
\end{ekey}
symbolize the following sentences in FOL:
\begin{earg}
\item Hofthor trusts a vegetarian.
\item Everyone who trusts Ingmar trusts a vegetarian.
\item Everyone who trusts Ingmar trusts someone who trusts a vegetarian.
\item Only Ingmar knows the combination to the safe.
\item Ingmar trusts Hofthor, but no one else.
\item The person who knows the combination to the safe is a vegetarian.
\item The person who knows the combination to the safe is not a spy.
\end{earg}


\solutions
\problempart
\label{pr.FOLcards}
Using the following symbolization key:
\begin{ekey}
\item[\text{domain}] cards in a standard deck
\item[Bx] \gap{x} is black.
\item[Cx] \gap{x} is a club.
\item[Dx] \gap{x} is a deuce.
\item[Jx] \gap{x} is a jack.
\item[Mx] \gap{x} is a man with an axe.
\item[Ox] \gap{x} is one-eyed.
\item[Wx] \gap{x} is wild.
\end{ekey}
symbolize each sentence in FOL:
\begin{earg}
\item All clubs are black cards.
\item There are no wild cards.
\item There are at least two clubs.
\item There is more than one one-eyed jack.
\item There are at most two one-eyed jacks.
\item There are two black jacks.
\item There are four deuces.
\item The deuce of clubs is a black card.
\item One-eyed jacks and the man with the axe are wild.
\item If the deuce of clubs is wild, then there is exactly one wild card.
\item The man with the axe is not a jack.
\item The deuce of clubs is not the man with the axe.
\end{earg}

\

\problempart Using the following symbolization key:
\begin{ekey}
\item[\text{domain}] animals in the world
\item[Bx] \gap{x} is in Farmer Brown's field.
\item[Hx] \gap{x} is a horse.
\item[Px] \gap{x} is a Pegasus.
\item[Wx] \gap{x} has wings.
\end{ekey}
symbolize the following sentences in FOL:
\begin{earg}
\item There are at least three horses in the world.
\item There are at least three animals in the world.
\item There is more than one horse in Farmer Brown's field.
\item There are three horses in Farmer Brown's field.
\item There is a single winged creature in Farmer Brown's field; any other creatures in the field must be wingless.
\item The Pegasus is a winged horse.
\item The animal in Farmer Brown's field is not a horse.
\item The horse in Farmer Brown's field does not have wings.
\end{earg}

\problempart
In this chapter, we symbolized `Nick is the traitor' by `$\exists x (Tx \eand \forall y(Ty \eif x \eid y) \eand x \eid n)$'. Two equally good symbolizations would be:
	\begin{ebullet}
		\item $Tn \eand \forall y(Ty \eif n \eid y)$
		\item $\forall y(Ty \eiff y \eid n)$
	\end{ebullet}
Explain why these would be equally good symbolizations.

\chapter{Semantics for FOL with identity}
FOL with identity extends FOL as we presented it earlier. 
\section{Sentences of FOL with identity}When we add the identity symbol to FOL, we add a new kind of atomic sentence, for example $a=b$. 

We simply do this by adding a new kind of atomic sentence: \factoidbox{If $\meta{a}$ and $\meta{b}$ are names, then $\meta{a}=\meta{b}$ is an atomic sentence.} 
This is added to the other clauses of what it is to be a sentence as they were given in \S\ref{s:FOLSentences}

We can now see that $\forall x (\enot\, x\eid p \eif Opx)$ is a sentence, as it could be constructed as follows:	\begin{center}
	\begin{forest}
		[$\forall x (\enot\, x\eid p \eif Opx)$
			[$(\enot\, c\eid p\eif Opc)$
				[$\enot\, c\eid p$
					[$c\eid p$]
				]
				[$Opc$]
			]
		]
	\end{forest}
	\end{center} 

Be aware that when you see $\enot\, a\eid b$ the negation is attached to the whole sentence $a\eid b$, not to $a$. So, you should not write $a\eid\enot b$. This is not a sentence. Sometimes you might see $a\neq b$, and this is short hand for $\enot\, a\eid b$. 

Sometimes you might come across cases where you might feel tempted to write something like $a\eid\enot b$. Such temptation should be avoided. For example, in our semi-formalised English we might say `for all x, if x is not p, then $Ax$'. But note that this should be $\forall x(\enot\, x\eid p\eif Ax)$, and should not be written with $x\eid\enot p$. 

\section{Semantics for identity}

%In \S\ref{ch.semantics} we described when sentences of FOL were true or false. Whereas for TFL sentences were true or false \emph{on a valuation}, in FOL sentences were true or false \emph{on an interpretation}. For FOL with identity we will also use interpretations, for example: 	\begin{ekey}
%		\item[\text{domain}] all people born before 2000\textsc{ce}
%		\item[a] Aristotle
%		\item[b] Beyonc\'e
%		\item[Px] \gap{x} is a philosopher
%		\item[Rxy] \gap{x} was born before \gap{y}
%	\end{ekey} 
%
%$=$ will stand for identity, so in our given interpretation $a=b$ is false. If we extended this interpretation with
%\begin{ekey}
%\item[c] Aristotle
%\end{ekey} then a sentence $a=c$ is true. 

Now that we have added the identity symbol to FOL, we simply need to expand our notion of truth from Part \ref{ch.semantics} to also account for atomic sentences like $a\eid b$. Our clause that we add is:
\factoidbox{
$\meta{a}\eid\meta{b}$ is true \textbf{iff} $\meta{a}$ and $\meta{b}$ name the same object in that interpretation.
}
So on our go-to interpretation from Part \ref{ch.semantics}, 
\begin{ekey}
		\item[\text{domain}] all people born before 2000\textsc{ce}
		\item[a] Aristotle
		\item[b] Beyonc\'e
		\item[Px] \gap{x} is a philosopher
		\item[Rxy] \gap{x} was born before \gap{y}
	\end{ekey} we have that $a\eid b$ is false: Aristotle and Beyonc\'e are different people, so `$a$' and `$b$' name different objects. 

Consider the following interpretation
\begin{ekey}
\item[\text{domain}] All celestial bodies
\item[e] The evening star
\item[m] The morning star
\end{ekey} It turns out that The Morning Star is \emph{the same object as} The Evening Star: they are names for Venus. So here we have two names for the same object. We thus have $e\eid m$ is true on this interpretation.

Identity becomes particularly useful when we have quantifiers. Suppose we have 
\begin{ekey}
\item[\text{domain}] Alfred, Billy, Carys
\item[a] Alfred
\item[b] Billy
\item[c] Carys
\end{ekey}
Remember, to check if $\exists x\meta{A}(\ldots x\ldots x\ldots)$ is true we first add a new name, let's use $d$, and we see if there is some way of extending the domain so that $\meta{A}(\ldots d\ldots d\ldots)$ is true. So, let's see if $\exists x \,x\eid a$ is true: we add a new name $d$ and consider the extended interpretation with 
\begin{ekey}
\item[d] Alfred
\end{ekey}
Then $d=a$ is true on this interpretation. So there is some way of interpreting `$d$' where $d=a$ is true; and thus $\exists x \, x\eid a$ is true.

On this interpretation we can also see that $\forall x(x\eid a\eor x\eid b\eor x\eid c)$ is true. Why? We add a new name `$d$'. There are three ways we can extend our original interpretation:\begin{enumerate}
\item 
\begin{ekey}
\item[d] Alfred
\end{ekey}
\item 
\begin{ekey}
\item[d] Billy
\end{ekey}
\item 
\begin{ekey}
\item[d] Carys
\end{ekey}
\end{enumerate}
On the first of these, $d=a$ is true, on the second $d=b$ is true, and on the third $d=c$ is true. So on each of these interpretations, $d\eid a\eor d\eid b\eor d\eid c$ is true, and thus $\forall x(x\eid a\eor x\eid b\eor x\eid c)$.



%Identity is a special predicate of FOL. We write it a bit differently than other two-place predicates: `$x=y$' instead of `$Ixy$' (for example). More important, though, its interpretation is fixed, once and for all. 
%


%Where \meta{a} and \meta{b} are any names, 
%	\factoidbox{
%		$\meta{a} = \meta{b}$ is true in an interpretation \textbf{iff}\\
%		 \meta{a} and \meta{b} name the very same object in that interpretation
%	}
%So in our go-to interpretation, `$a = b$' is false, since Aristotle is distinct from Beyonc\'e.

\chapter{Rules for identity}
If two names refer to the same object, then swapping one name for another will not change the truth value of any sentence. So, in particular, if `$a$' and `$b$' name the same object, then all of the following will be valid:\label{model.nonidentity}
	\begin{align*}
	 	Aa &\therefore Ab \\
	 	Ab &\therefore Aa \\
	 	Raa &\therefore Rbb\\
		Raa & \therefore Rab\\
		Rca &\therefore Rcb\\
		\forall x Rxa &\therefore \forall x Rxb
	\end{align*}

We capture this idea in our elimination rule. If you have established `$a=b$', then anything that is true of the object named by `$a$' must also be true of the object named by `$b$'. For any sentence with `$a$' in it, you can replace some or all of the occurrences of `$a$' with `$b$' and produce an equivalent sentence. For example, from `$Raa$' and `$a = b$', you are justified in inferring `$Rab$', `$Rba$' or `$Rbb$'. More generally:
\factoidbox{\begin{proof}
	\have[m]{e}{\meta{a}=\meta{b}}
	\have[n]{a}{\meta{A}(\ldots \meta{a} \ldots \meta{a}\ldots)}
	\have[\ ]{ea1}{\meta{A}(\ldots \meta{b} \ldots \meta{a}\ldots)} \by{=E}{e,a}
\end{proof}}
The notation here is as for $\exists$I. So $\meta{A}(\ldots \meta{a} \ldots \meta{a}\ldots)$ is a formula containing the name $\meta{a}$, and $\meta{A}(\ldots \meta{b} \ldots \meta{a}\ldots)$ is a formula obtained by replacing one or more instances of the name $\meta{a}$ with the name $\meta{b}$. Lines $m$ and $n$ can occur in either order, and do not need to be adjacent, but we always cite the statement of identity first. Symmetrically, we allow:
\factoidbox{\begin{proof}
	\have[m]{e}{\meta{a}=\meta{b}}
	\have[n]{a}{\meta{A}(\ldots \meta{b} \ldots \meta{b}\ldots)}
	\have[\ ]{ea2}{\meta{A}(\ldots \meta{a} \ldots \meta{b}\ldots)} \by{=E}{e,a}
\end{proof}}
This rule is sometimes called \emph{Leibniz's Law}, after Gottfried Leibniz. 	
	
Some philosophers have believed the reverse of this claim. That is, they have believed that when exactly the same sentences (not containing `$=$') are true of two objects, then they are really just one and the same object after all. This is a highly controversial philosophical claim (sometimes called the \emph{identity of indiscernibles}) and our logic will not subscribe to it; we allow that exactly the same things might be true of two \emph{distinct} objects.  

To bring this out, consider the following interpretation:
	\begin{ebullet}
		\item[\text{domain}:] P.D.\ Magnus, Tim Button
		\item[$a$:] P.D.\ Magnus
		\item[$b$:] Tim Button
		\item For every primitive predicate we care to consider, that predicate is true of \emph{nothing}.
	\end{ebullet}
Suppose `$A$' is a one-place predicate; then `$Aa$' is false and `$Ab$' is false, so `$Aa \eiff Ab$' is true. Similarly, if `$R$' is a two-place predicate, then `$Raa$' is false and `$Rab$' is false, so that `$Raa \eiff Rab$' is true. And so it goes: every atomic sentence not involving `$=$' is false, so every biconditional linking such sentences is true. For all that, Tim Button and P.D.\ Magnus are two distinct people, not one and the same!

Since we are not subscribing to the thesis of identity of indiscernibles, no matter how much you learn about two objects, we cannot prove that they are identical. That is unless, of course, you learn that the two objects are, in fact, identical, but then the proof will hardly be very illuminating.

The consequence of this, for our proof system, is that there are no sentences that do not already contain the identity predicate that could justify the conclusion `$a=b$'. This means that the identity introduction rule will not justify `$a=b$', or any other identity claim containing two different names.

However, every object is identical to itself. No premises, then, are required in order to conclude that something is identical to itself. So this will be the identity introduction rule:
\factoidbox{
\begin{proof}
	\have[\ \,\,\,]{x}{\meta{c}=\meta{c}} \by{=I}{}
\end{proof}}
Notice that like the Law of Excluded Middle, this rule does not require referring to any prior lines of the proof. For any name \meta{c}, you can write $\meta{c}=\meta{c}$ on any point, with only the {=}I rule as justification. 


To see the rules in action, we will prove some quick results. First, we will prove that identity is \emph{symmetric}:
\begin{proof}
	\open
		\hypo{ab}{a = b}
		\have{aa}{a = a}\by{=I}{}
		\have{ba}{b = a}\by{=E}{ab, aa}
	\close
	\have{abba}{a = b \eif b =a}\ci{ab-ba}
	\have{ayya}{\forall y (a \eid y \eif y \eid a)}\Ai{abba}
	\have{xyyx}{\forall x \forall y (x \eid y \eif y \eid x)}\Ai{ayya}
\end{proof}
We obtain line 3 by replacing one instance of `$a$' in line 2 with an instance of `$b$'; this is justified given `$a= b$'. 

Second, we will prove that identity is \emph{transitive}:
\begin{proof}
	\open
		\hypo{abc}{a = b \eand b = c}
		\have{ab}{a = b}\ae{abc}
		\have{bc}{b = c}\ae{abc}
		\have{ac}{a = c}\by{=E}{ab, bc}
	\close
	\have{con}{(a \eid b \eand b =c) \eif a = c}\ci{abc-ac}
	\have{conz}{\forall z((a \eid b \eand b \eid z) \eif a \eid z)}\Ai{con}
	\have{cony}{\forall y\forall z((a \eid y \eand y \eid z) \eif a \eid z)}\Ai{conz}
	\have{conx}{\forall x \forall y \forall z((x \eid y \eand y \eid z) \eif x \eid z)}\Ai{cony}
\end{proof}
We obtain line 4 by replacing `$b$' in line 3 with `$a$'; this is justified given `$a= b$'. 

\practiceproblems
\problempart
\label{pr.identity}
Provide a proof of each claim.
\begin{earg}
\item $Pa \eor Qb, Qb \eif b=c, \enot Pa \proves Qc$
\item $m=n \eor n=o, An \proves Am \eor Ao$
\item $\forall x\ x=m, Rma\proves \exists x Rxx$
\item $\forall x\forall y(Rxy \eif x=y)\proves Rab \eif Rba$
\item $\enot \exists x\enot\, x\eid m \proves \forall x\forall y (Px \eif Py)$
\item $\exists x Jx, \exists x \enot Jx\proves \exists x \exists y\ \enot\, x\eid y$
\item $\forall x(x=n \eiff Mx), \forall x(Ox \eor \enot Mx)\proves On$
\item $\exists x Dx, \forall x(x=p \eiff Dx)\proves Dp$
\item $\exists x\bigl[(Kx \eand \forall y(Ky \eif x=y)) \eand Bx\bigr], Kd\proves Bd$
\item $\proves Pa \eif \forall x(Px \eor \enot\, x\eid a)$
\end{earg}

\problempart
Show that the following are provably equivalent:
\begin{ebullet}
\item $\exists x \bigl([Fx \eand \forall y (Fy \eif x \eid y)] \eand x = n\bigr)$
\item $Fn \eand \forall y (Fy \eif n= y)$
\end{ebullet}
And hence that both have a decent claim to symbolize the English sentence `Nick is the F'.

\

\problempart
In \S\ref{sec.identity}, we claimed that the following are logically equivalent symbolizations of the English sentence `there is exactly one F':
\begin{ebullet}
\item $\exists x Fx \eand \forall x \forall y \bigl[(Fx \eand Fy) \eif x = y\bigr]$
\item $\exists x \bigl[Fx \eand \forall y (Fy \eif x \eid y)\bigr]$
\item $\exists x \forall y (Fy \eiff x \eid y)$
\end{ebullet}
Show that they are all provably equivalent. (\emph{Hint}: to show that three claims are provably equivalent, it suffices to show that the first proves the second, the second proves the third and the third proves the first; think about why.)


\
\problempart
Symbolize the following argument
	\begin{quote}
		There is exactly one F. There is exactly one G. Nothing is both F and G. So: there are exactly two things that are either F or G.
	\end{quote}
And offer a proof of it.
%\begin{ebullet}
%\item  $\exists x \bigl[Fx \eand \forall y (Fy \eif x \eid y)\bigr], \exists x \bigl[Gx \eand \forall y ( Gy \eif x \eid y)\bigr], \forall x (\enot Fx \eor \enot Gx) \proves \exists x \exists y \bigl[\enot\, x\eid y \eand \forall z ((Fz \eor Gz) \eif (x \eid y \eor x \eid z))\bigr]$
%\end{ebullet}
