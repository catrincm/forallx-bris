%!TEX root = forallxbris.tex
\part{Truth-functional logic}
\label{ch.TFL}
\addtocontents{toc}{\protect\mbox{}\protect\hrulefill\par}
 
\chapter{A Prolegomenon to TFL}
In this part the lecture notes we wish to commence our study of logical validity and logical consequence and, indeed, whenever we now talk about validity and arguments being valid we mean logical validity. In \S\ref{s:Valid} we already introduced the idea that logical validity is validity in virtue of (logical) form. To develop this idea more precisely we will look at arguments in a formal language. This will enable us to single out arguments that are valid in virtue of their form and eventually make sense of the notion of an interpretation we used in our definition of validity in \S\ref{s:Valid}. We can then give a rigorous formal definition of validity of arguments in the formal language we shall devise. This language will be the language of Truth-functional logic (TFL).

Before we introduce the language of TFL, let us take a look at why a formal language may be helpful for capturing validity of arguments, i.e., the validity of arguments in virtue of their form.

Consider this argument:
	\begin{earg}
		\prem It is raining outside.
		\prem If it is raining outside, then Jenny is miserable.
		\conc Jenny is miserable.
	\end{earg}
and another argument:
	\begin{earg}
		\prem Jenny is an anarcho-syndicalist.
		\prem If Jenny is an anarcho-syndicalist, then Dipan is an avid reader of Tolstoy.
		\conc Dipan is an avid reader of Tolstoy.
	\end{earg}
Both arguments are valid, and there is a straightforward sense in which we can say that they share a common structure. We might express the structure thus:
	\begin{earg}
		\prem A
		\prem If A, then B
		\conc B
	\end{earg}
This looks like an excellent argument \emph{structure}. Indeed, surely any argument with this \emph{structure} will be valid.

What about:
	\begin{earg}
		\prem Jenny is miserable.
		\prem If it is raining outside, then Jenny is miserable.
		\conc It is raining outside.
	\end{earg}
The form of this argument is:
\begin{earg}
\prem	$B$
\prem	If $A$ then $B$
\conc $A$
\end{earg}
Arguments of this form are generally invalid.

Be careful, though, not every argument of this form is sure to be invalid.
It’s possible to have an argument of this form that’s valid – see if you can work out how!
But most arguments of this form are invalid.

There a lot more valid argument forms. For example the 
argument form
	\begin{earg}
		\prem A or B
		\prem not-A
		\conc B
	\end{earg}
as well as the form
	\begin{earg}
		\prem not-(A and B)
		\prem A
		\conc not-B
	\end{earg}
lead to valid arguments independently of what expressions we substitute for `A' and `B': we can understand (interpret) `A' and `B' in whatever way we want, as long as we take them to be place holder for sentences the resulting arguments the resulting argument will be valid. These examples illustrate the important idea that the validity of the arguments just considered has nothing to do with the meanings of English expressions like `Jenny is miserable', `Dipan is an avid reader of Tolstoy', or any other sentence. If it has to do with meanings at all, it is with the meanings of conjunction-words like `and', `or', `not,' and `if\ldots, then\ldots'. The language of truth-functional logic is built to single out characteristic feature of these conjunction words and this will enable us to fruitfully study the idea of validity of an argument in virtue of its form.

When one introduces a language there are (at least) two task: the first is to specify the vocabulary of the language and equip the language with a grammar, that is, one has to specify how well-formed sentences of the language look like. This aspect of the language is called its \define{syntax}. The second task is to specify the \define{semantics} of the language. The semantics specifies how we are to understand the expressions of the language, what the sentences of the language mean etc. Part \ref{ch.TFL} develops both the syntax and the semantics of the language of TFL. Once this has been established we can consider how TFL may be useful for thinking about arguments in English. This will lead to the idea of symbolizing arguments in TFL and will be picked up in Part \ref{ch.symb}. 

%In Parts \ref{ch.TFL}--\ref{ch.NDTFL}, we are going to develop a formal system which allows us to symbolize many arguments in such a way as to show that they are valid in virtue of their form. That language will be \emph{truth-functional logic}, or TFL. It will have sentences like $$(A\eand (B\eif\enot C)),$$ which we will read ``$A$ and if $B$, then it is not the case that $C$''.

\chapter{Syntax of TFL}
\section{Atomic sentences}\label{sec:as}

We started isolating the form of an argument by replacing  \emph{subsentences} of sentences with individual letters. Thus in the first example of this section, `it is raining outside' is a subsentence of `If it is raining outside, then Jenny is miserable', and we replaced this subsentence with `$A$'.

Our artificial language, TFL, pursues this idea absolutely ruthlessly. We start with some \emph{atomic sentences}. These will be the basic building blocks out of which more complex sentences are built. We will use uppercase Roman letters for atomic sentences of TFL (except for $X$, $Y$, and $Z$ which we reserve for metavariables). There are only twenty-three letters $A$--$W$, but there is no limit to the number of atomic sentences that we might want to consider. By adding subscripts to letters, we obtain new atomic sentences. So, here are five different atomic sentences of TFL:
	$$A, P, P_1, P_2, A_{234}$$
You can think of atomic sentences as representing certain English sentences but for now this is simply a heuristic (in Part \ref{symb} we shall take atomic sentences to \emph{symbolize} certain English sentence).  For example, you can think of $A$ as representing the English sentence `It is raining outside', and the atomic sentence of TFL, $C$, as representing the English sentence `Jenny is miserable'. 

However, if you think of the letter $P$ as representing a particular English sentence it is important to understand that whatever structure the English sentence has, atomic sentence $P$ will not reflect this structure. From the point of view of TFL, an atomic sentence is just a letter. It can be used to build more complex sentences, but it cannot be taken apart.

\newglossaryentry{atomic sentence}
{
name=atomic sentence,
description={A sentence used to represent a basic sentence; a single letter in TFL, or a predicate symbol followed by names in FOL}
}

\section{Connectives}
\label{s:TFLConnectives}

In the previous chapter, we introduced the atomic sentences of TFL.  In TFL we have counterparts to the conjunction-words that play an important role for spelling out arguments in English, that is, in TFL we have expression that play a similar role to the role expressions like `and', `or' and `not' play in English. These are the \emph{connectives}---they can be used to form new sentences out of old ones. In TFL, we will make use of logical connectives to build complex sentences from atomic components. There are five logical connectives in TFL. This table summarises them, and they are explained throughout this section.

\newglossaryentry{connective}
{
name=connective,
description={A logical operator in TFL used to combine \glspl{atomic sentence} into larger sentences}
}
	\begin{table}[h]
	\center
	\begin{tabular}{l l l}

	\textbf{symbol}&\textbf{what it is called}&\textbf{rough meaning}\\
	\hline
	\enot&negation&`It is not the case that$\ldots$'\\
	\eand&conjunction&`$\ldots$\ and $\ldots$',\\
	\eor&disjunction&`$\ldots$\ or $\ldots$'\\
	\eif&conditional&`If $\ldots$\ then $\ldots$'\\
	%\eiff&biconditional&`$\ldots$ if and only if $\ldots$'\\

	\end{tabular}
	\end{table}

%These are not the only connectives of English of interest. Others are, e.g., `unless', `neither \dots{} nor \dots', and `because'. We will see that the first two can be expressed by the connectives we will discuss, while the last cannot. `Because', in contrast to the others, is not \emph{truth functional}. In \S\ref{??} we shall explain what this means precisely.

If we were to substitute declarative sentences for `\dots' in the right hand column of the table above, we obtain new English sentences. The language of truth-functional logic works in the same way: the connectives `\enot',`\eand',`\eor' and `\eif' combine with other sentences as introduce in \S\ref{sec:as} to form new sentences. For example, the atomic sentences `$A$' and `$C$' combine
with `$\wedge$' to form the sentence `$A\wedge C$'. If think of `$A$' and `$C$' as representing the English sentences
\begin{ekey}
		\item[A] It is raining outside
		\item[C] Jenny is miserable
	\end{ekey}
`$A\wedge C$' can be read as:
\begin{itemize}
\item It is raining \emph{and} Jenny is miserable.
\end{itemize}
Similarly, on this understanding we get the following readings:
\begin{itemize}
\item $\neg A$: It is not the case that it is raining outside. (Alternatively, it is not raining outside).
\item $A\vee C$: It is raining outside \emph{or} Jenny is miserable.
\item $A\rightarrow C$: \emph{If} it is raining outside, \emph{then} Jenny is miserable.
%\item $A\leftrightarrow B$: It is raining outside, \emph{if and only if} Jenny is miserable.
 \end{itemize}
 We shall go back to studying the connection between the connectives of truth-functional logic and various conjunction-words of English in \S\ref{sec:tt}, when we discuss the precise meaning of the connectives in truth-functional logic. For now we focus on completing the description of the formal language of truth-functional logic.
 
 To conclude our discussion of the connectives we focused on the application of the connectives to atomic sentences, but connectives can be applied to all sorts of sentences not only atomic sentences. For example, as $\neg A$ is a TFL sentence we can use `$\wedge$' to conjoin it the sentence `$C$' to form the new sentence `$(\neg A\wedge C)$'. Connectives can be applied to all TFL sentences, not only atomic sentences. It is time to say precisely what TFL sentences are.
 
\section{Sentences}\label{s:TFLSentences}
\todo{This went through a major rewrite. Check it!!!}
We have introduced the basic building blocks of truth-functional logic, the atomic sentences, and the connectives, which allow us to conjoin different sentences to form new sentences. In terms of the vocabulary of a (written) language we have introduced all the important parts save the punctuation marks. In the language of truth-functional logic we use brackets for this purpose.

What is still missing is to equip the language with a grammar. The purpose of the grammar is to distinguish wellformed sentences from nonsense, but also to avoid ambiguity.

We wish to have rules that guarantee that
$$(A\eor (B\eand C))$$ $$\enot (A\eand B)$$
are wellformed sentences of the language of truth-functional logic, while
$$A\eand\eor B\eif$$
is not. The latter expression is nonsense just as in English the sequence of words
\begin{center}
The and dog brown or is.
\end{center}
is nonsense and not a sentence of English.



The second purpose is to avoid ambiguity. In English we use commas to distinguish between two sentences
\begin{earg}
\item[\ex{engamb1}] John's tired, and Sue's tall or Rob's short.
\item[\ex{engamb2}] John's tired and Sue's tall, or Rob's short.
\end{earg}
and without a comma it would be unclear which of the two sentences we intend to convey. In TFL this job of punctuation marks is assumed by brackets, that is, we distinguish between:
$$(A\wedge(B\vee C))$$
$$((A\wedge B)\vee C)$$
You can think of the former TFL-sentence as representing the English sentence 1, whereas the latter as representing the English sentence 2.

You might know this use of brackets from mathematics:
\begin{earg}
\item[\ex{mathamb}] $9 + 3 \times 4$
\end{earg}
can either be read as:
\begin{earg}
\item[\ex{mathamb1}] $9 + (3 \times 4) \qquad(=9+12=21)$
\item[\ex{mathamb2}] $(9+3) \times 4 \qquad(=12\times 4=48)$
\end{earg}

Importantly, the language of TFL is designed to exclude any form of ambiguity. For example, $A\eand B\eor C$ will not be a sentence of TFL as it requires disambiguation. Rather the syntactic/grammatical rules of the language of TFL will be such that only expressions that have a \emph{unique} reading can be sentences of TFL. To make this precise we now provide a formal definition of what it is to be a sentence in TFL.

\factoidbox{\label{TFLsentences}
	\begin{enumerate}
		\item Every atomic sentence is a sentence.
		\item If \metaX is a sentence, then $\enot\metaX$ is a sentence.
		\item If \metaX and \metaY are sentences, then $(\metaX\eand\metaY)$ is a sentence.
		\item If \metaX and \metaY are sentences, then $(\metaX\eor\metaY)$ is a sentence.
		\item If \metaX and \metaY are sentences, then $(\metaX\eif\metaY)$ is a sentence.
		%\item If \metaX and \metaY are sentences, then $(\metaX\eiff\metaY)$ is a sentence.
		\item Nothing else is a sentence.
	\end{enumerate}
	}
	
	\newglossaryentry{sentence of TFL}
{
name=sentence of TFL,
description={A string of symbols in TFL that can be built up according to the recursive rules given on p.~\pageref{TFLsentences}}
}

The definition specifies rules according to which sentences of the language can be formed. To understand this definition let us pick it apart and consider the rules individually.



\begin{enumerate}
\item[1.] Tells us that atomic sentences as discussed in \S\ref{sec:as} are sentence of TFL.
\end{enumerate}
Recall that any uppercase Roman letters $A$--$W$, or with subscripts, e.g., $A_1, B_3, A_{100}, J_{375}$, are atomic sentences of TFL. Notice $X$, $Y$, and $Z$ are not atomic sentences. They are so-called \define{metavariables} and used as place holders for sentences of TFL (see more on metavairables in \S\ref{um}).



Our second rule says:
\begin{enumerate}
\item[2.]
If $\metaX$ is a sentence of TFL, then so is $\enot \metaX$.
\end{enumerate}
By rule 1, we know that $A$ is a sentence. Rule 2 then allows us to conclude that $\enot A$ is also a sentence. We could then apply it again and conclude that $\enot\enot A$ is also a sentence. More generally, if, by whatever, rule we have a constructed a sentence X, rule 2 tells us that $\neg X$ will also be a sentence of TFL. 

\define{Formation trees} help us keep track of this process. For the case of $\enot\enot A$ this would be:
\begin{center}
\begin{forest}
	[$\mainconnective{\enot}\, \enot A$
		[$\mainconnective{\enot}A$
			[$A$]
		]
	]
\end{forest}
\end{center}

Our third rule says:
\begin{enumerate}
\item[3.] If \metaX and \metaY are sentences, then so is $(\metaX\eand\metaY)$.
\end{enumerate}
By rule 1, $B_1$ and $D$ are both sentences. So rule 3 allows us to conclude that $(B_1\eand D)$ is a sentence. We might then apply rule 2 to conclude that $\enot(B_1\eand D)$ is also a sentence.
\begin{center}
\begin{forest}
	[$\mainconnective{\enot}\,  (B_1 \eand D)$
		[$(B_1\mainconnective{\eand} D)$
			[$B_1$]
			[$D$]
		]
	]
\end{forest}
\end{center}

The rules 4 and 5 then tell us how the $\eor$- and $\eif$-connective respectively can be used to produce new sentences of TFL. Rule 6, in contrast, tells us that sentences of TFL must be formed using the rules 1-5: if an expression cannot be obtained by consecutively applying rules 1-5, then the expression is not a sentence of TFL. Again formation trees are helpful to understand this: rule 7 tells us that all nodes of the formation tree must be sentences of TFL.

For example, consider $(A \eand (B \eor C))$ we can check this is a sentence by drawing the following formation tree:
\label{S:formationtree}
\begin{center}
\begin{forest}
	[$(A\mainconnective{\eand} (B\eor C))$
		[$A$]
		[$(B\mainconnective{\eor} C)$
			[$B$]
			[$C$]
		]
	]
\end{forest}
\end{center}
Each of the steps here tracks one of the rules of what it is to be a sentence. So we can conclude that this is a sentence of TFL. This also helps us see how to read it.
It has a different formation tree from $((A\eand B)\eor C)$:
\begin{center}
\begin{forest}
	[$((A{\eand} B)\mainconnective{\eor} C))$
		[$(A\mainconnective{\eand} B)$
			[$A$]
			[$B$]
		]
		[$C$]
	]
\end{forest}
\end{center}
The different formations will be important when we describe truth-tables for these sentences (\S\ref{sec:tt}). $((A\eand B)\eor C)$ and $((A\eand B)\eor C)$ will differ in when they are true.


One more example: consider $\enot (P \eand \enot (\enot Q \eor P))$ we can check this is a sentence by drawing the following formation tree:
\label{S:formationtree}
\begin{center}
\begin{forest}
	[$\mainconnective{\enot}\,  (P \eand \enot (\enot Q \eor P))$
		[$(P \,\mainconnective{\eand}\,  \enot (\enot Q \eor P))$
			[$P$]
			[$\mainconnective{\enot}\,   (\enot Q\eor P)$
				[$\mainconnective{\enot}\,   Q$
					[$Q$]
				]
				[$P$]
			]
		]
	]
\end{forest}
\end{center}
each of the steps here tracks one of the rules of what it is to be a sentence. So we can conclude that this is a sentence of TFL. The sentences further up the tree are formed by one of the formation rules from the sentences further down the tree.

When drawing these trees we have highlighted a particular connective on each of our nodes. We call that connective the \define{main connective} of the sentence.
\factoidbox{The \define{main connective} of sentence is the last connective that was introduced in the construction of the sentence.}

In the case of $((\enot E \eor F) \eif \enot\enot G)$, the main connective is $\eif$. Here we can see that the whole sentence can be described in the form $(\metaX\eif\metaY)$ with both $\metaX$ and $\metaY$ being complete sentences (put $\metaX=(\enot E\eor F)$ and $\metaY=\enot\enot G$). That's enough to see that $\eif$ is the main connective. In the case of $\enot\enot\enot D$, the main connective is the very first $\enot$ sign. This is because we can see the sentence as having the form $\enot\metaX$ with $\metaX$ being the complete sentence $\enot\enot D$. In the case of $(P \eand \enot (\enot Q \eor R))$, the main connective is $\eand$: it's an $(\metaX\eand\metaY)$ with $\metaX$ as $P$ and $\metaY$ as $\enot (\enot Q \eor R)$.

\newglossaryentry{main connective}
{
	name=main connective,
	description={The last connective that you add when you assemble a sentence using the recursive definition.}
}

\newglossaryentry{formation tree}
{
	name=formation tree,
	description={A tree showing the structure of a sentence and its subsentences.}
}





\subsection{Inductive Definition}
The definition of a TFL-sentence is a so-called \emph{recursive} definition. Recursive definitions begin with some specifiable base elements, and then present ways to generate indefinitely many more elements by compounding together previously established ones. To give you a better idea of what an inductive definition is, we can give an inductive definition of the idea of \emph{an ancestor of mine}. We specify a base clause.
	\begin{ebullet}
		\item My parents are ancestors of mine.
	\end{ebullet}
and then offer further clauses like:
	\begin{ebullet}
		\item If x is an ancestor of mine, then x's parents are ancestors of mine.
		\item Nothing else is an ancestor of mine.
	\end{ebullet}
Using this definition, we can easily check to see whether someone is my ancestor: just check whether she is the parent of the parent of\ldots one of my parents. And the same is true for our recursive definition of sentences of TFL. Just as the inductivedefinition allows complex sentences to be built up from simpler parts, the definition allows us to decompose sentences into their simpler parts. Once we get down to atomic sentences, then we know we are ok.

\section{Bracketing conventions}
\label{TFLconventions}


Strictly speaking, $A\eand B$ is not a sentence of TFL. When we introduce a connective $\eand,\eor$ or $\eif$, strictly speaking, we must include brackets. Only $(A\eand B)$ is strictly speaking a sentence of TFL. The reason for this rule is that we might use $(A\eand B)$ as a subsentence in a more complicated sentence. For example, we might want to negate $(A\eand B)$, obtaining $\enot(A\eand B)$. If we just had $A \eand B$ without the brackets and put a negation in front of it, we would have $\enot A \eand B$. It is most natural to read this as meaning the same thing as $(\enot A \eand B)$, but this may be very different from $\enot(A\eand B)$.

When working with TFL, however, it will make our lives easier if we are sometimes a little less than strict. So, here are two convenient conventions.
\begin{enumerate}
\item We can removep \emph{outermost} brackets of a sentence. Thus we allow ourselves to write $A\eand B$ instead of the sentence $(A\eand B)$. However, we must remember to put the brackets back in, when we want to embed the sentence into a more complicated sentence!
\item It can be a bit painful to stare at long sentences with many nested pairs of brackets. To make things a bit easier on the eyes, we will  allow ourselves to use square brackets, `[' and `]', instead of rounded ones. So there is no logical difference between $(P\eor Q)$ and $[P\eor Q]$, for example.
\end{enumerate}

Combining these two conventions, we can rewrite the unwieldy sentence
$$(((H \eif I) \eor (I \eif H)) \eand (J \eor K))$$
rather more clearly as follows:
$$\bigl[(H \eif I) \eor (I \eif H)\bigr] \eand (J \eor K)$$
The scope of each connective is now much easier to pick out.

\begin{practiceproblems}

\solutions
\problempart
\label{pr.wiffTFL}
For each of the following: (a) Is it a sentence of TFL, strictly speaking? (b) Is it a sentence of TFL, allowing for our relaxed bracketing conventions? (c) If the answer to (b) is yes, write down the formation tree of each sentence and determin the main connective at each node (if there is one). Is there a main connective for every node of the formation tree of a sentence.
\begin{earg}
\item $(A)$\hfill \myanswer{(a) no (b) no}
\item $J_{374} \eor \enot J_{374}$\hfill \myanswer{(a) no (b) yes}
\item $\enot \enot \enot \enot F$\hfill \myanswer{(a) yes (b) yes}
\item $\enot \eand S$\hfill \myanswer{(a) no (b) no}
\item $(G \eand \enot G)$\hfill \myanswer{(a) yes (b) yes}
\item $(A \eif (A \eand \enot F)) \eor (D \eiff E)$\hfill \myanswer{(a) no (b) yes}
\item $[(Z \eiff S) \eif W] \eand [J \eor X]$\hfill \myanswer{(a) no (b) yes}
\item $(F \eiff \enot D \eif J) \eor (C \eand D)$\hfill \myanswer{(a) no (b) no}
\end{earg}

\problempart
Are there any sentences of TFL that contain no atomic sentences? Explain your answer.
\\\myanswer{No. Atomic sentences contain atomic sentences (trivially). And every more complicated sentence is built up out of less complicated sentences, that were in turn built out of less complicated sentences, \ldots, that were ultimately built out of atomic sentences.}

\end{practiceproblems}


\chapter{Use and mention}\label{s:UseMention}
We have talked a lot \emph{about} sentences. So we should pause to explain an important, and very general, point.

\section{Quotation conventions}
Consider these two sentences:
	\begin{itemize}
		\item Justin Trudeau is the Prime Minister.
		\item The expression `Justin Trudeau' is composed of two uppercase letters and eleven lowercase letters
	\end{itemize}
When we want to talk about the Prime Minister, we \emph{use} his name. When we want to talk about the Prime Minister's name, we \emph{mention} that name, which we do by putting it in quotation marks.

There is a general point here. When we want to talk about things in the world, we just \emph{use} words. When we want to talk about words, we typically have to \emph{mention} those words. We need to indicate that we are mentioning them, rather than using them. To do this, some convention is needed. We can put them in quotation marks, or display them centrally in the page (say). So this sentence:
	\begin{itemize}
		\item `Justin Trudeau' is the Prime Minister.
	\end{itemize}
says that some \emph{expression} is the Prime Minister. That's false. The \emph{man} is the Prime Minister; his \emph{name} isn't. Conversely, this sentence:
	\begin{itemize}
		\item Justin Trudeau is composed of two uppercase letters and eleven lowercase letters.
	\end{itemize}
also says something false: Justin Trudeau is a man, made of flesh rather than letters. One final example:
	\begin{itemize}
		\item ``\,`Justin Trudeau'\,'' is the name of `Justin Trudeau'.
	\end{itemize} 
On the left-hand-side, here, we have the name of a name. On the right hand side, we have a name. Perhaps this kind of sentence only occurs in logic textbooks, but it is true nonetheless.

Those are just general rules for quotation, and you should observe them carefully in all your work! To be clear, the quotation-marks here do not indicate reported speech. They indicate that you are moving from talking about an object, to talking about a name of that object. 



\section{Object language and metalanguage}
These general quotation conventions are very important for us. After all, we are describing a formal language here, TFL, and so we must often \emph{mention} expressions from TFL.

When we talk about a language, the language that we are talking about is called the \define{object language}. The language that we use to talk about the object language is called the \define{metalanguage}.
\label{def.metalanguage}
\newglossaryentry{object language}
{
name=object language,
description={A language that is constructed and studied by logicians. In this textbook,
 the object languages are TFL and FOL}
}

\newglossaryentry{metalanguage}
{
name=metalanguage,
description={The language logicians use to talk about the object language. In this textbook, the metalanguage is English, supplemented by certain symbols like metavariables and technical terms like ``valid''}
}

For the most part, the object language in this chapter has been the formal language that we have been developing: TFL. The metalanguage is English. Not conversational English exactly, but English supplemented with some additional vocabulary to help us get along.

Now, we have used uppercase letters as sentence letters of TFL:
	$$A, B, C, Z, A_1, B_4, A_{25}, J_{375},\ldots$$
These are sentences of the object language (TFL). They are not sentences of English. So we must not say, for example:
	\begin{itemize}
		\item $D$ is a sentence letter of TFL.
	\end{itemize}
Obviously, we are trying to come out with an English sentence that says something about the object language (TFL), but `$D$' is a sentence of TFL, and not part of English. So the preceding is gibberish, just like:
	\begin{itemize}
		\item \foreignlanguage{german}{Schnee ist weiß} is a German sentence.
	\end{itemize}
What we surely meant to say, in this case, is:
	\begin{itemize}
		\item `\foreignlanguage{german}{Schnee ist weiß}' is a German sentence.
	\end{itemize}
Equally, what we meant to say above is just:
	\begin{itemize}
		\item `$D$' is a sentence letter of TFL.
	\end{itemize}
The general point is that, whenever we want to talk in English about some specific expression of TFL, we need to indicate that we are \emph{mentioning} the expression, rather than \emph{using} it. We can either deploy quotation marks, or we can adopt some similar convention, such as  placing it centrally in the page.


\section{Metavariables}\label{s:Metavariables}
However, we do not just want to talk about \emph{specific} expressions of TFL. We also want to be able to talk about \emph{any arbitrary} sentence of TFL. Indeed, we had to do this in \S\ref{s:TFLSentences}, when we presented the recursive definition of a sentence of TFL. We used uppercase script letters to do this, namely:
	$$\metaX, \metaY, \metaZ,\metaX_1,\metaY_1,\metaZ_1\ldots$$
These symbols do not belong to TFL. Rather, they are part of our (augmented) metalanguage that we use to talk about \emph{any} expression of TFL. To explain why we need them, recall the second clause of the recursive definition of a sentence of TFL:
	\begin{itemize}
		\item[2.] If $\metaX$ is a sentence, then $\enot \metaX$ is a sentence.
	\end{itemize}
This talks about \emph{arbitrary} sentences. If we had instead offered:
	\begin{itemize}
		\item[2$'$.] If `$A$' is a sentence, then `$\enot A$' is a sentence.
	\end{itemize}
this would not have allowed us to determine whether `$\enot B$' is a sentence. To emphasize:
	\factoidbox{
	  `$\metaX$' is a symbol (called a \define{metavariable}) in augmented English, which we use to talk about expressions of TFL. 	`$A$' is a particular sentence letter of TFL.}

        \newglossaryentry{metavariables}
{
name=metavariables,
description={A variable in the metalanguage that can represent any sentence in the object language}
}
But this last example raises a further complication, concerning quotation conventions. We did not include any quotation marks in the second clause of our inductive definition. Should we have done so?

The problem is that the expression on the right-hand-side of this rule, i.e., `$\enot\metaX$', is not a sentence of English, since it contains~`$\enot$'. So we might try to write:
	\begin{itemize}
		\item[2$''$.] If \metaX is a sentence, then `$\enot \metaX$' is a sentence.
	\end{itemize}
But this is no good: `$\enot \metaX$' is not a TFL sentence, since `$\metaX$' is a symbol of (augmented) English rather than a symbol of TFL.

What we really want to say is something like this:
	\begin{itemize}
		\item[2$'''$.] If \metaX is a sentence, then the result of concatenating the symbol `$\enot$' with the sentence \metaX is a sentence.
	\end{itemize}
This is impeccable, but rather long-winded. %Quine introduced a convention that speeds things up here. In place of (2$''$), he suggested:
%	\begin{numberlist}
%		\item[2$'''$.] If \metaX and \metaY are sentences, then $\ulcorner (\metaX\eand\metaY)\urcorner$ is a sentence
%	\end{numberlist}
%The rectangular quote-marks are sometimes called `Quine quotes', after Quine. The general interpretation of an expression like `$\ulcorner (\metaX\eand\metaY)\urcorner$' is in terms of rules for concatenation.
But we can avoid long-windedness by creating our own conventions. We can perfectly well stipulate that an expression like `$\enot \metaX$' should simply be read \emph{directly} in terms of rules for concatenation. So, \emph{officially}, the metalanguage expression `$\enot \metaX$'
simply abbreviates:
\begin{quote}
	the result of concatenating the symbol `$\enot$' with the sentence \metaX
\end{quote}
and similarly, for expressions like `$(\metaX \eand \metaY)$', `$(\metaX \eor \metaY)$', etc.


\section{Quotation conventions for arguments}
One of our main purposes for using TFL is to study arguments, and that will be our concern in \S\ref{ch.TruthTables}. In English, the premises of an argument are often expressed by individual sentences, and the conclusion by a further sentence. Since we can symbolize English sentences, we can symbolize English arguments using TFL.

Or rather, we can use TFL to symbolize each of the \emph{sentences} used in an English argument. However, TFL itself has no way to flag some of them as the \emph{premises} and another as the \emph{conclusion} of an argument.  (Contrast this with natural English, which uses words like `so', `therefore', etc., to mark that a sentence is the \emph{conclusion} of an argument.)

%So, if we want to symbolize an \emph{argument} in TFL, what are we to do? 

%An obvious thought would be to add a new symbol to the \emph{object} language of TFL itself, which we could use to separate the premises from the conclusion of an argument. However, adding a new symbol to our object language would add significant complexity to that language, since that symbol would require an official syntax.\footnote{\emph{The following footnote should be read only after you have finished the entire book!} And it would require a semantics. Here, there are deep barriers concerning the semantics. First: an object-language symbol which adequately expressed `therefore' for TFL would not be truth-functional. (\emph{Exercise}: why?) Second: a paradox known as `validity Curry' shows that FOL itself \emph{cannot} be augmented with an adequate, object-language `therefore'.} 

So, we need another bit of notation. Suppose we want to symbolize the premises of an argument with $\metaX_1$, \dots,~$\metaX_n$ and the conclusion with $\metaZ$. Then we will write:
$$\metaX_1, \ldots, \metaX_n \therefore \metaZ$$
The role of the symbol `$\therefore$' is simply to indicate which sentences are the premises and which is the conclusion.

%Strictly, this extra notation is \emph{unnecessary}. After all, we could always just write things down long-hand, saying: the premises of the argument are well symbolized by $\metaX_1, \ldots \metaX_n$, and the conclusion of the argument is well symbolized by $\metaZ$. But having some convention will save us some time. Equally, the particular convention we chose was fairly \emph{arbitrary}. After all, an equally good convention would have been to underline the conclusion of the argument. Still, this is the convention we will use.

Strictly, the symbol `$\therefore$' will not be a part of the object language, but of the \emph{metalanguage}. As such, one might think that we would need to put quote-marks around the TFL-sentences which flank it. That is a sensible thought, but adding these quote-marks would make things harder to read. Moreover---and as above---recall that \emph{we} are stipulating some new conventions. So, we can simply stipulate that these quote-marks are unnecessary. That is, we can simply write:
$$A, A \eif B \therefore B$$
\emph{without any quotation marks}, to indicate an argument whose premises are (symbolized by) `$A$' and `$A \eif B$' and whose conclusion is (symbolized by)~`$B$'.
In this Part, we have talked a lot \emph{about} sentences. So we should pause to explain an important, and very general, point.

\chapter{Semantics of TFL}
\section{Truth tables}\label{sec:tt}
We have completed introducing the syntax of the language of TFL. It is now time to turn to the semantics of TFL. The idea underlying the semantics is to specify conditions when sentences of TFL are true. In \S\ref{} we introduced the idea that an argument is logically valid if there is no interpretation on which all premises of the argument are true but the conclusion false. Accordingly, if we wish to make a start on making this idea more precise, we need to say when a sentence of TFL is true according to an interpretation and when it is false. The target of this chapter is to give precise rules for determining this. The important feature of truth functional logic is that the truth value of a complex sentence, such as $`A\eor(B\eand C)'$ is fully determined by the truths of is component parts, that is $`A'$, $`B'$ and $`C'$. If we're told whether $`A'$, $`B'$ and $C$ are true or false, then we will be able to say whether $`A\eor(B\eand C)'$ is true or false.

To be able to do this, we need to describe how the truth values of sentences are to be combined to obtain the truth value of a sentence that has been obtained via the formation rules 2-5. To do this we work through each of our connectives describing the rules governing it.

\newglossaryentry{truth value}
                 {
                   name = truth value,
                   description = {One of the two logical values sentences can have: True and False}
                   }


\subsection{Negation}
The `$\neg$'-connective is called negation. When we introduced the `$\neg$'-connective we said it should roughly be understood as `it is not the case' or, perhaps, simply `not'. Let's make that official:
\factoidbox{
If a sentence can be paraphrased as `it is not the case that \metaX'\\ it can be symbolised as $\enot\metaX$.
}
What does that mean for the truth rules? Consider:
\begin{earg}
\item[\ex{neg-f}] Bristol is not in France.
\item[\ex{neg-f}] Bristol is not in England.
\end{earg}
`Bristol is in France' is false, so `Bristol is not in France' is true.
`Bristol is in England' is true, so `Bristol is not in England' is false.

In general, to determine whether a sentence of the form $\enot \metaX$ is true. This depends on whether $\metaX$ is true or not in the way:
\begin{itemize}
\item If $\metaX$ is true, then $\enot \metaX$ is false.
\item If $\metaX$ is false, then $\enot \metaX$ is true.
\end{itemize}
We record this in shorthand:
\begin{highlighted}
\begin{center}
\begin{tabular}{ccc}
If \metaX is: &&then \enot \metaX is:\\
T &$\leadsto$& F\\
F &$\leadsto$& T
\end{tabular}
\end{center}
\end{highlighted}
We have abbreviated `True' with `T' and `False' with `F'. (But just to be clear, the two truth values are True and False; the truth values are not \emph{letters}!)
\newglossaryentry{negation}
{
name=negation,
description={The symbol \enot, used to represent words and phrases that function like the English word ``not''}
}






\subsection{Conjunction}
\label{s:ConnectiveConjunction}
The $\wedge$-connective, called conjunction, is meant to be understood of the English word `and': 
\factoidbox{
%		A sentence can be symbolized as $(\metaX\eand\metaY)$ if it can be paraphrased in English as `Both\ldots, and\ldots', or as `\ldots, but \ldots', or as `although \ldots, \ldots'.
		If a sentence can be paraphrased as `\metaX and \metaY' \\it can be symbolised as $\metaX\eand\metaY$.
	}

What is the appropriate truth rule for conjunction? Consider:
\begin{earg}
\item[\ex{conj}] She can speak German and she can speak French.
\end{earg}
If she can speak German and she can speak French, then this is true, but otherwise it is false.

More generally, the rule governing $\eand$ is:
\begin{itemize}
\item If $\metaX$ and $\metaY$ are both true, then $\metaX\eand\metaY$ is true.
\item Otherwise, $\metaX\eand \metaY$ is false.
\end{itemize}
Which we summarise
\begin{highlighted}
\begin{center}
\begin{tabular}{cccc}
If \metaX is:&and \metaY is:&&then $\metaX\eand\metaY$ is:\\
T & T &$\leadsto$& T\\
T & F &$\leadsto$& F\\
F & T &$\leadsto$& F\\
F & F &$\leadsto$& F
\end{tabular}
\end{center}
\end{highlighted}
%in the characteristic truth table:
%\begin{center}
%\begin{tabular}{c c |c}
%\metaX & \metaY & $\metaX\eand\metaY$\\
%\hline
%T & T & T\\
%T & F & F\\
%F & T & F\\
%F & F & F
%\end{tabular}
%\end{center}
Note that conjunction is \emph{symmetrical}. The truth value for $\metaX \eand \metaY$ is always the same as the truth value for $\metaY \eand \metaX$.

\newglossaryentry{conjunct}
{
name=conjunct,
description={A sentence joined to another by a \gls{conjunction}}
}


\subsection{Disjunction}
The `$\vee$'-connective is called disjunction an is meant to be understood in terms of the English `or'.
\newglossaryentry{disjunction}
{
name=disjunction,
description={The connective \eor, used to represent words and phrases that function like the English word ``or'' in its inclusive sense; or a sentence formed by using this connective}
}

\newglossaryentry{disjunct}
{
name=disjunct,
description={A sentence joined to another by a \gls{disjunction}}
}
\factoidbox{
If a sentence can be paraphrased as `\metaX or \metaY' \\it can be symbolised as $\metaX\eor\metaY$.
	}

Whereas the truth rules for the negation and conjunction were relatively straightforward, the rule for disjunction is a bit more subtle.
Consider:
\begin{earg}
\item[\ex{conj}] She can speak German or she can speak French.
\end{earg}
If she cannot speak either German or French, then this is false. If she can speak German but not French, then it is true, and if she can speak French but not German it is also true.
We have the general rules:
\begin{itemize}
\item If $\metaX$ and $\metaY$ are both false, then $\metaX\eor\metaY$ is false.
\item  If $\metaX$ is true and $\metaY$ is false, then $\metaX\eor\metaY$ is true.
\item  If $\metaX$ is false and $\metaY$ is true, then $\metaX\eor\metaY$ is true.
\end{itemize}

But what if she can speak both? Is it true or false?
It seems that in English there are two kinds of disjunctions: an \define{inclusive} and an \define{exclusive} one. For the inclusive \emph{or}, we might whisper a ``or both'' after it; whereas for the exclusive \emph{or}, we'd want to whisper a ``but not both'':
\begin{earg}
\item [\ex{inclor}] She speaks German or
she speaks French (or both).
\item [\ex{exclor}] She speaks German or
she speaks French (but not both).
\end{earg}

In logic there can be no ambiguity. We choose that $\eor$ stands for the \emph{inclusive or}.
That is, we give the final rule:
\begin{itemize}
\item If $\metaX$ and $\metaY$ are both true, then $\metaX\eor\metaY$ is true.
\end{itemize}

So, when we turn to symbolisations of arguments, one should only symbolise a sentence as $\metaX\eor \metaY$ if it is to be read as the \emph{inclusive or}. To symbolise the exclusive or, you need to use the more complex sentence: $(\metaX\eor\metaY)\eand \enot(\metaX\eand\metaY)$, which essentially makes explicit the whispered ``but not both''. (Once we have completed our presentation of the truth tables you should check that $(\metaX\eor\metaY)\eand \enot(\metaX\eand\metaY)$ can indeed be understood as exclusive or.)

%Sometimes the English is ambiguous, in which case one should point that out when symbolising and give the two alternative symbolisations.

%Consider:
%\begin{earg}
%\item[\ex{exclor}]She either ate pizza or pasta.
%\end{earg}
%Maybe this is an exclusive or, though an alternative treatment is to take this to be an inclusive or, but note that there's a implicit, or missing, premise: that she didn't eat both.

To summarise the rules for $\eor$:
%\begin{center}
%\begin{tabular}{c c |c}
%\metaX & \metaY & $\metaX\eor\metaY$\\
%\hline
%T & T & T\\
%T & F & T\\
%F & T & T\\
%F & F & F
%\end{tabular}
%\end{center}
\begin{highlighted}
\begin{center}
\begin{tabular}{cccc}
If \metaX is:&and \metaY is:&&then $\metaX\eor\metaY$ is:\\
T & T &$\leadsto$& T\\
T & F &$\leadsto$& T\\
F & T &$\leadsto$& T\\
F & F &$\leadsto$& F
\end{tabular}
\end{center}
\end{highlighted}
Like conjunction, disjunction is symmetrical.

\subsection{Conditional}\todo{Say more about ``only if''! Students thought it was $\eiff$}
The `$\rightarrow$'-connective is called the conditional-connective and it is meant to be related to our understanding of \emph{if\ldots, then\ldots} sentences. Here, $P$ is called the \define{antecedent} of the conditional $(P \eif Q)$, and $Q$ is called the \define{consequent}.
	\factoidbox{
	 If a sentence can be paraphrased as \\`If \metaX, then \metaY' it can be symbolised as $\metaX\eif\metaY$.
	}




\newglossaryentry{conditional}
{
name=conditional,
description={The symbol \eif, used to represent words and phrases that function like the English phrase ``if \dots{} then \dots''; a sentence formed by using this symbol}
}
What are the truth rules for the conditional? Consider the sentence:
\begin{earg}
\item[\ex{bartender}] If she is drinking a beer, then she is over eighteen.
\end{earg}
What are the circumstances under which this conditional is false?
Here is what we'll say:
\begin{itemize}
\item If she's drinking beer and is under age, then it is false.
\item It is true in all other circumstances.
\end{itemize}
The understand the rationale for this, let us think about when a bartender would get into trouble (clearly the conditional should be true for everyone drinking beer in a bar). That is the case if an under age woman is drinking beer. If she's drinking beer and is over 18 years old, the barkeeper has done their job correctly. The conditional is true. What about if she is not drinking beer but, say, coke? Then it is irrelevant whether she is 18 or not. The barkeeper doesn't have to check her age. The conditional is true for trivial reasons.


%We get the characteristic truth table:
%\begin{center}
%\begin{tabular}{c c|c}
%\metaX & \metaY & $\metaX\eif\metaY$\\
%\hline
%T & T & T\\
%T & F & F\\
%F & T & T\\
%F & F & T
%\end{tabular}
%\end{center}
We summarise these rules:
\begin{highlighted}
\begin{center}
\begin{tabular}{cccc}
If \metaX is:&and \metaY is:&&then $\metaX\eif\metaY$ is:\\
T & T &$\leadsto$& T\\
T & F &$\leadsto$& F\\
F & T &$\leadsto$& T\\
F & F &$\leadsto$& T
\end{tabular}
\end{center}
\end{highlighted}
In this case, it's very important to remember which way around it goes. The TF-line is different to the FT-line.

This is why the terms `antecedent' and `consequent' are so useful.  \begin{highlighted}
In $\metaX\eif\metaY$, \metaX is called the \define{antecedent}, and \metaY the \define{consequent}.
\end{highlighted}
We can redescribe this rule:
If the antecedent is true and the consequent false, then the conditional sentence is false, otherwise it is true.
\newglossaryentry{antecedent}
{
name=antecedent,
description={The sentence on the left side of a \gls{conditional}}
}


\newglossaryentry{consequent}
{
name=consequent,
description={The sentence on the right side of a \gls{conditional}}
}

The TFL connective $\eif$ is \emph{stipulated} to be governed by these rules. This sometimes marked by calling it the \define{material conditional}. But the truth rules of the $\eif$-connective only tell us part of the story of \emph{if\ldots, then\ldots} sentences in English, as the truth rules do not seem to work well for all these sentences. For example, truth rules for material implication do not seem to work well with our understanding of the sentence
\begin{earg}
\item[\ex{kangaroo}] If Kangaroos had no tails, they would topple over.
\end{earg}
We will discuss conditional-sentences of this kind in \S\ref{s:IndicativeSubjunctive} and will look at some problems arising due to understanding $\eif$ in terms of material implication in \S\ref{s:ParadoxesOfMaterialConditional}. However, for our purposes $\eif$-connective will be understood in terms of truth rules given in this section.


\section{Truth}
In the previous section we have learned how truth values of costitutent sentences determine the truth value of the complex sentences. This means that if we are presented with the truth value of the relevant atomic sentences that appear in a complex sentence, we can determine whether that sentence is true or false.

But how do we determine whether a given atomic sentence is true? This is where the notion of an \emph{interpretation} comes into the picture. An interpretation will stipulate (assign) truth values of particular atomic sentences. Let `$B$' stand for the English sentence `Ben is happy'. Then there is one interpretation according to which this is true, that is, $B$ will be assigned the value ``True'' on this interpretation. There is another interpretation according to which Ben is not happy, that is, $B$ is assigned the value ``False'' on this interpretation. In TFL an interpretation of the atomic sentences is called a \define{valuation}:

\factoidbox{
		A \define{valuation} is any assignment of truth values to the atomic sentences of TFL.
	}

        \newglossaryentry{valuation}
{
name=valuation,
description={An assignment of \glspl{truth value} to particular atomic \glspl{sentence of TFL}}
}

To better grasp what a valuation is it makes sense to look at the truth table of the sentence `$A\wedge B$', which looks as follows (from now on, in contrast to \S\ref{sec:tt}, we no longer display the explanatory text, and replace $\leadsto$ by a vertical line):

\begin{center}
\begin{tabular}{ccc|c}
$A$&$B$&&$A\wedge B$\\\hline
T & T && T\\
T & F && F\\
F & T && T\\
F & F && T
\end{tabular}
\end{center}

Now the first two rows of every horizontal line in the above table give us a (different) valuation for the atomic sentence $A$ and $B$: according to the valuation given by the first line both $A$ and $B$ are true, according to the valuation given by the second line $A$ is true but $B$ is false, and so on. Let use $v_1,v_2,\ldots$ as names for different valuations, then we can make valuations explicit in the truth table above:
\begin{center}
\begin{tabular}{c|ccc|c}
Valuation&$A$&$B$&&$A\wedge B$\\\hline
$v_1$&T & T && T\\
$v_2$&T & F && F\\
$v_3$&F & T && T\\
$v_4$&F & F && T
\end{tabular}
\end{center}
 
 The truth table then tells us that `$A\wedge B$' is true relative to the valuation $v_1$, but false relative to the $v_2,v_3$, and $v_4$. We can generalize idea and give a definition of when a sentence of TFL is true relative to a given valuation. The definition will be recursive again. Indeed it will give us rules to compute the truth of sentence for every formation rule of the definition of a TFL sentence.
 
 \factoidbox{\label{Truthint}Let $v$ be a valuation. Then
	\begin{enumerate}
		\item An atomic sentence $X$ is true relative to $v$, if and only if $v$ assigns the value T to $X$.
		\item a sentence $\enot\metaX$ is true relative to $v$, if and only if $\metaX$ is not true relative to $v$.
		\item a sentence $(\metaX\eand\metaY)$ is true relative to $v$, if and only if $\metaX$ and $\metaY$ are both true relative to $v$.
		\item a sentence $(\metaX\eor\metaY)$ is true relative to $v$, if and only if $\metaX$ or $\metaY$is true relative to $v$.
		\item a sentence $(\metaX\eif\metaY)$  is true relative to $v$, if and only if $\metaX$ is not true or $\metaY$ is true relative to $v$
		%\item If \metaX and \metaY are sentences, then $(\metaX\eiff\metaY)$ is a sentence.
		%\item Nothing else is a sentence.
	\end{enumerate}
	}
	
	\newglossaryentry{Truthval}
{
name=Truth relative to a valuation,
description={a list of recursive rules that allow us to determine whether a sentence of TFL is true given a particular valuation, p.~\pageref{Truthval}}
}

Let's go through the five rules of the definition step by step. The first rule should be relatively immediate: an atomic sentence is true relative to a valuation, if and only if the valuation says it has value ``True''. For rule 2, we look at the truth table for negation: the truth table tells us that a sentence $\neg X$ is true whenever $X$ is false (not true). That is, $\neg X$ is true relative to a valuation, if $X$ is not true (false) relative to that valuation. For the case of conjunction, we can revisit the discussion we used to motivate talking about truth relative to a valuation. We say that the conjunction `$A\wedge B$' was only true relative to the valuation $v_1$, that is, the valuation relative to which both `$A$' and `$B$' were true. Our reasoning was not specific to the specific conjunction `$A\wedge B$' but applicable to all sentences of the form $X\wedge Y$ with arbitrary conjuncts $X$ and $Y$.

With this definition in place we can when a sentence is true relative to some interpretation $v$. Let us consider the sentence:


For the fourth rule we need to look at the truth table for disjunction:
\begin{center}
\begin{tabular}{ccc|c}
$X$&$Y$&&$X\vee Y$\\\hline
T & T && T\\
T & F && T\\
F & T && T\\
F & F && F
\end{tabular}
\end{center}
According to the truth table $X\vee Y$ is true on lines 1-3, that is, inspecting these three lines one can see that it suffices for $X$ or $Y$ to be true for $X\vee Y$ to be true. This is precisely what the third rule says.

For the conditional (rule 5) we also reexamine the truth table $X\rightarrow Y$:
\begin{center}
\begin{tabular}{ccc|c}
$X$&$Y$&&$X\rightarrow Y$\\\hline
T & T && T\\
T & F && F\\
F & T && T\\
F & F && T
\end{tabular}
\end{center}
According to the truth table a sentence $X\rightarrow Y$ is true on line 1, 3, and 4. Let's check whether the rule is correct: $X\rightarrow Y$ has to be true, if and only if $X$ is false or $Y$ is true. If $X$ is false, we are either in line 3 or in line 4, but in both lines $X\rightarrow Y$ turns out true. If $Y$ is true, we are either in line 1 or line 3 of the truth table and $X\rightarrow Y$ is true in both lines. So rule 5 is correct.

With the definition in place we can now say when an arbitrary sentence of TFL is true relative to a valuation $v$. This requires giving the complete truth table for this sentence. For example, consider the sentence $A\wedge(B\vee C)$. To give the complete truth table of this sentence we first need to give the truth table for $B\vee C$ and then combine it with the truth values for $A$ to produce the complete truth table. We get the following truthtable:

\begin{center}
\begin{tabular}{ccc|c||c}
$A$&$B$&$C$&$B\vee C$&$A\wedge(B\vee C)$\\\hline
\textbf{T} & \textbf{T} &\textbf{T}& T & \textbf{T}\\
\textbf{T} & \textbf{T} & F& T & \textbf{T}\\
\textbf{T} & F & \textbf{T}& T& \textbf{T}\\
T & F &F&F& F\\
F & T &T&T&F\\
F& T & F & T &F\\
F & F & T &T &F\\
F & F & F& F& F
\end{tabular}
\end{center}

We discuss constructing complete truth tables for sentences of TFL in more detail in \S\ref{sec:ctt}. But inspecting the truth table we see that `$A\wedge(B\vee C)$' is true relative relative to an interpretation $v$ if and only if `$A$' is true relative relative to v and at least one of `$B$' and `$C$' is also true relative to $v$. To this effect we only need to check the lines 1,2, and 3 of the truth table, as these are the only lines in which `$A\wedge(B\vee C)$' is true. 
 


 
 \section{Validity and other logical notions}
 In \S\ref{??} we said that an argument was valid, if and only if there is no interpretation such that all premises are true but the conclusion false. At that point the definition was suggestive, but we lacked a clear understanding of `interpretation' and when a sentence is true relative to an interpretation. But for the language of TFL we can now turn the informal definition in a precise and rigorous definition.
 
 Recall that an argument consisted of a number of premises together with a conclusion. A TFL-argument then can be written as $X_1,\ldots,X_n\therefore Y$ where $X_1,\ldots,X_n, Y$ are sentences of TFL, and $X_1,\ldots,X_n$ are the premises and $Y$ the conclusion of the argument.
 
 \factoidbox{\label{TFLvalid}A TFL-argument $X_1,\ldots,X_n\therefore Y$ is \define{valid} if and only if there is no valuation $v$ such that $X_1$ is true relative to $v$ and\ldots and $X_n$ is true relative to $v$, but $Y$ is false relative to $v$.}
 
 We can now investigate whether a TFL-argument is valid or not. The difficult bit is to show that there is no valuation on which all premises are true but the conclusion false. After all there are many different valuations. Fortunately, only the atomic sentences that occur in the premises and the conclusion of the argument will be relevant for deciding whether an argument is valid or not. Consider the argument:
 $$\neg B,A\rightarrow B\therefore\neg A$$
 In this case there are only four different ways to assign truth values to $A$ and $B$, that is, we only need to consider four valuation. This means that we can check whether the argument is valid via the following truth table:
 
 \begin{center}
\begin{tabular}{c|cc|c|c||c}
Valuation&$A$&$B$&$\neg B$&$A\rightarrow B$&$\neg A$\\\hline
$v_1$&T & T & F & T & F\\
$v_2$&T & F & T & F & F\\
$v_3$&F & T & F & T & T\\
$v_4$&F & F & \textbf{T}& \textbf{T} & \textbf{T}
\end{tabular}
\end{center}

In the truth table we have used the truth table for negation to compute the truth value of $\neg B$ ($\neg A$) from $B$ ($A$) and the truth value of the conditional to obtain the value of $A\rightarrow B$ from the values of $A$ and $B$. By inspecting the truth table we see that only relative to valuation $v_4$ all the premises of the argument ($\neg B$ and $A\rightarrow B$) are true. But relative to $v_4$ the conclusion $\neg A$ is true likewise. There is no interpretation relative to which all premises are true, but the conclusion false. The argument is valid.

Unfortunately, when we consider arguments that involve more atomic sentences we need to consider more valuations: for $n$ atomic sentences there are $2^n$ different options for assigning truth values to these atomic sentences, that is we need to consider $2^n$ different valuations. Hence, the more atomic sentences one needs to consider the longer the truth tables will be. 

It is worth highlighting a peculiar feature of the definition of validity: there are valid argument without a premise. Consider the argument
$$\therefore A\vee\neg A.$$
The argument is valid if there is no interpretation such that all premises are true but the conclusion false, that is if there is no interpretation such that $A\vee\neg A$ is false. This can be quickly verified as there are only two valuations to consider:
 
 \begin{center}
\begin{tabular}{c|c|c||c}
Valuation&$A$&$\neg A$&$\neg A\vee A$\\\hline
$v_1$&T & F & \textbf{T}\\
$v_2$&F & T & \textbf{T}
\end{tabular}
\end{center} 

$A\vee\neg A$ is true on both $v_1$ and $v_2$. The argument is valid. The conclusions of arguments with no premises are called \define{logical truths} or \define{tautologies}.
\factoidbox{\label{logical truth}A TFL-sentence $X$ is called a \define{logical truth} or \define{tautology} if and only if the TFL-argument $\therefore X$ is logically valid.}
A logical truth is a TFL-sentence that follows from every other TFL-sentence (Exercise: explain why). A sentence $Y$ \define{follows from} the sentences $X_1,\ldots,X_n$, that is when $Y$ is a \define{(logical) consequence} of $X_1,\ldots,X_n$:
\factoidbox{\label{TFLcons}$Y$ is a \define{(logical) consequence} of $X_1,\ldots,X_n$ if and only if the argument $X_1,\ldots,X_n\therefore Y$ is valid.}
Notice one important  difference between validity and consequence: the former is a property of TFL-arguments while the latter is a property of TFL-sentences!

A logical truth follows from every sentence. In contrast, every sentence  follows from a \define{logical contradiction}.
\factoidbox{\label{TFLcontra}$X$ is a \define{logical contradiction} if there is no valuation $v$ such that X is true relative to $v$.}
Every sentence $Y$ follows from a contradiction $X$, since $X\therefore Y$ is valid: there is no valuation on which $X$ is true (and $Y$ false).

We end this section by introducing two further important logical notions: \define{consistency} and \define{logical equivalence}.
\factoidbox{\label{TFLconsist}A collection of TFL-sentences $X_1,\ldots,X_n$ is \define{consistent} iff there is a valuation relative to which $X_1,\ldots,X_n$ are true. Otherwise the collection is \define{inconsistent}.}
Any collection of sentences that contains a logical contradiction is inconsistent. But there are of course many collections of sentences that are consistent. Give some examples!

The final notion we introduce is that of logical equivalence.
\factoidbox{\label{TFLequiv}$X$ and $Y$ are \define{logically equivalent} if and only if $X$ follows from $Y$ and $Y$ follows from $X$.}
In some sense if two sentences are logically equivalent they mean the same thing from the perspective of TFL. At least semantically TFL cannot tell the two sentences apart. Examples of logically equivalent sentences include $A$ and $A\wedge A$, of $\neg A\vee B$ and $A\rightarrow B$. This can be checked by means of truth tables (Exercise!).



\section{Constructing Truth Tables}\label{sec:ctt}
It's time to get our hands dirty. So far we have given a lot of abstract definitions, but have not really discussed how to do things and, in particular, how to construct truth tables in a systematic way. That's what we will do now!


Consider the sentence $(\enot I\eand H)\eif H$. We will give a \emph{truth table} which lists all the valuations and says whether this sentence is true or false on each of them.
The valuations assign either the value ``True'' or ``False'' to each atomic sentence. In this case we have two atomic sentences, $I$ and $H$, so we have four ($2^2$) valuations ($v_1,\dots,v_4$) each of which is a line in the truth table:
\begin{center}
\begin{tabular}{c|cc|c}
Valuation&$I$&$H$&$(\enot I\eand H)\eif H$\\\hline
$v_1$&T&T&\\
$v_2$&T&F&\\
$v_3$&F&T&\\
$v_4$&F&F&
\end{tabular}
\end{center}
Our job is to fill out the truth values of $(\enot I\eand H)\eif H$.

Here the formation tree will help us know what to do (see \S\ref{TFLsentences}):
\begin{center}
\begin{forest}
[$(\enot I\eand H)\eif H$
	[$(\enot I\eand H)$
		[$\enot I$
			[$I$]
		]
		[$H$]
	]
	[$H$]
]
\end{forest}
\end{center}
The idea is that we work ourselves from \define{leaves} of the tree (the atomic sentence) to the \define{root} of the formation tree (the sentence that has been constructed). The truth rule for $\enot$ tells us how the truth value of $\enot I$ depends on the truth of $I$. Then the rule for $\eand$ tells us how the truth value of $\enot I\eand H$ depends on the truths of $\enot I$ and $H$; and finally, the rule for $\eif$ tells us how the truth value of $(\enot I\eand H)\eif H$ depends on those of $\enot I\eand H$ and $H$. 

So to work out the truth values of $(\enot I\eand H)\eif H$ we first need to work out the truth values of $\enot I$ and $\enot I\eand H$. We expand our truth table with columns for each of these.
\begin{center}
\begin{tabular}{c|cc|c|c||c}
Valuation&$I$&$H$&$\enot I$&$(\enot I\eand H)$&$(\enot I\eand H)\eif H$\\\hline
$v_1$&T&T&&\\
$v_2$&T&F&&\\
$v_3$&F&T&&\\
$v_4$&F&F&&
\end{tabular}
\end{center}
The first step is $\enot I$. We use the truth table (rule) for negation:
\begin{center}
\begin{tabular}{ccc}
If \metaX is & & then $\enot X$ is\\
T&$\leadsto$&F\\
F&$\leadsto$&T
\end{tabular}
\end{center}
Now, we can fill out:

\begin{center}
\begin{tabular}{c|cc|c|c||c}
Valuation&$I$&$H$&$\enot I$&$(\enot I\eand H)$&$(\enot I\eand H)\eif H$\\\hline
$v_1$&T&T&F&\\
$v_2$&T&F&F&\\
$v_3$&F&T&T&\\
$v_4$&F&F&T&
\end{tabular}
\end{center}
We worked these out using the following instructions:
\begin{itemize}
\item Go to the column for $I$ and for every valuation do the following:
\begin{itemize}
\item If the value of $I$ is T, then put F into the column of $\neg I$ at the line of the valuation.
\item If the value of $I$ is F, then put T into the column of $\neg I$ at the line of the valuation.
 \end{itemize}
 \end{itemize}
The next step is to consider $\enot I\eand H$. For this we will use the truth rule for $\eand$:
\begin{center}
\begin{tabular}{ccccc}
If \metaX is&and \metaY is  && then $\metaX\eand \metaY$ is\\
T&T&$\leadsto$&T\\
T&F&$\leadsto$&F\\
F&T&$\leadsto$&F\\
F&F&$\leadsto$&F
\end{tabular}
\end{center}
Now, we can fill out:
\begin{center}
\begin{tabular}{c|cc|c|c||c}
Valuation&$I$&$H$&$\enot I$&$(\enot I\eand H)$&$(\enot I\eand H)\eif H$\\\hline
$v_1$&T&T&F&F\\
$v_2$&T&F&F&F\\
$v_3$&F&T&T&T\\
$v_4$&F&F&T&F
\end{tabular}
\end{center}
We worked these using the following instructions:
\begin{itemize}
\item  For every valuation go to column $H$ and do the following:
\begin{itemize}
\item[$\dagger$]  If the value of $H$ is F, put F into column $(\enot I\eand H)$ at the line of the valuation.
\item If the value of $H$ is T, go to column of $\neg I$:
\begin{itemize}
\item if the value of $\neg I$ is T, put T into column $(\enot I\eand H)$ and at the line of the valuation.
\item if the value of $\neg I$ is F, put F into column $(\enot I\eand H)$ and at the line of the valuation.
 \end{itemize}
 \end{itemize}
 \end{itemize}
The instruction $\dagger$ is justified by the truth rules of the conjunction: if one of the conjuncts has value F, the conjunction will also have value F.

Now, finally, we need to look at $(\enot I\eand H)\eif H$, and will use the truth rule for $\eif$:

\begin{center}
\begin{tabular}{cccc}
If \metaX is&and \metaY is  && then $\metaX\eif \metaY$ is\\
T&T&$\leadsto$&T\\
T&F&$\leadsto$&F\\
F&T&$\leadsto$&T\\
F&F&$\leadsto$&T
\end{tabular}
\end{center}
Now, we can fill out:
\begin{center}
\begin{tabular}{c|cc|c|c||c}
Valuation&$I$&$H$&$\enot I$&$(\enot I\eand H)$&$(\enot I\eand H)\eif H$\\\hline
$v_1$&T&T&F&F&T\\
$v_2$&T&F&F&F&T\\
$v_3$&F&T&T&T&T\\
$v_4$&F&F&T&F&Y
\end{tabular}
\end{center}
We worked these out by the following procedure:
\begin{itemize}
\item  For every valuation go to column $(\enot I\eand H)$ and do the following:
\begin{itemize}
\item[$\star$]  If the value of $(\enot I\eand H)$ is F, put T into column $(\enot I\eand H)\eif H$ at the line of the valuation.
\item If the value of $(\enot I\eand H)$ is T go to column $H$ and:
\begin{itemize}
\item if the value of $H$ is T, put T into column $(\enot I\eand H)\eif H$ at the line of the valuation.
\item if the value of $H$ is F, put F into column $(\enot I\eand H)\eif H$ at the line of the valuation.
\end{itemize}
\end{itemize}
\end{itemize}
The instruction $\star$ is justified by the truth rules of the conditional: if the antecedent of the conditional has value F, then the conditional will also value T.

With this example in mind let us try to give a general instruction for constructing a truth table for a sentence $X$.
\subsection{How to do truth tables}
\factoidbox{
\begin{enumerate}
\item Write down the formation tree of $X$.
\item Find all atomic sentences on the tree. These will be the leaves of the tree.
\item If you have found all atomic sentences, you can start the truth table:
\begin{itemize}
\item You will need a column for every atomic sentence.
\item If there are $n$ atomic sentences, you will need $2^n$ valuations, that is, $2^n$ horizontal lines.
\item Make sure you have correctly written down all the different valuations!
\end{itemize}
\item From the atomic sentences (the leaves of the tree) move upwards to the root (the sentence $X$) and
\begin{itemize}
\item for each node (constituent sentence) create a column in the truth table;
\item make sure you correctly identify the main connective of the sentence heading the column;
\item the root, that is, the sentence $X$ should be the last column of the truth table
\end{itemize}
\item Moving from left to right compute the truth values of each column
\begin{itemize}
\item Make sure you use the truth rule associated with the main connective of the sentence heading the column.
\item Stay in one and the same valuation (horizontal line) when you compute the truth value of the column.
\item Compute the truth value for every valuation.
\end{itemize}
\item You are done when you have computed the truth values of the column of $X$ \emph{and} there are no gaps in the truth table.
\end{enumerate}}

If you follow these outlines, you should be able to construct truth tables for arbitrary TFL sentences. Of course, the more complicated the sentences are, and the more atomic sentence letters they contain the longer and tedious the truth table---but the more important it becomes to painstakingly stick to the guidelines we have given. 

We have already seen that truth tables can be used to find out relative to which valuations a sentence is true or whether an argument is valid. They can also be used to determine whether sentences follow from each other or whether they are consistent or inconsistent. Let sum how to check for the various logical notions using truth tables:

\begin{description}
\item[Validity:]Construct a truth table with columns for all atomic sentences occurring in the premises and the conclusion; columns for all subsentences of the premises and conclusion, columns for all the premises and the conclusion. If in all valuation (horizontal line) in which all premises are true the conclusion is true too, then the argument is valid. Otherwise it is invalid.
\item[Consequence]To check that $Y$ is a consequence of $X_1,\ldots,X_n$, we need to check whether $X_1,\ldots,X_n\therefore Y$ is valid (see above).
\item[Logical Equivalence] To check whether $X$ and $Y$ are logically equivalent, we need to check whether $X\therefore Y$ and $Y\therefore X$ are valid (see above). This will be the case if in a truth table for both $X$ and $Y$ whenever $X$ is true relative to valuation so is $Y$ and vice versa.
\item[Logical truth/Tautology] $X$ is a tautology if and only if $\therefore X$ is valid, that is, if in the truth table for $X$, $X$ receives value T relative to all valuations.
\item[Logical conrradiction]  $X$ is a logical contradiction if and only if in the truth table for $X$, $X$ receives value F relative to all valuations.
\item[Consistency]To check whether $X_1,\ldots,X_n$ are consistent construct a truth table with columns for all atomic sentences occurring in $X_1,\ldots,X_n$, columns for all subsentences of $X_1,\ldots,X_n$, and columns for $X_1,\ldots,X_n$. If there is a valuation (horizontal line) relative to which all of $X_1,\ldots,X_n$ receive value T, they are consistent. Otherwise they are inconsistent.
\end{description}

\begin{practiceproblems}\label{pr.TT.TTorC}
\problempart
\label{pr.TT.TTorC}
\label{pr.TT.TTorC}
Complete truth tables for each of the following:
\begin{earg}
\item $A \eif A$ %taut
\myanswer{\begin{center}
\begin{tabular}{c | def}
$A$ & $A$&$\eif$&$A$\\
\hline
 T & T&\TTbf{T}&T\\
F & F&\TTbf{T}&F
\end{tabular}
\end{center}
}

\item $C \eif\enot C$ %contingent
\myanswer{\begin{center}
\begin{tabular}{c | d e e f}
$C$ & $C$&$\eif$&$\enot$&$C$\\
\hline
 T & T & \TTbf{F}& F& T\\
F & F & \TTbf{T}& T& F\\
\end{tabular}
\end{center}}
\item $(A \eif B) \eif (\enot A\vee B)$ %tautology

%\myanswer{\begin{center}
%\begin{tabular}{c c | d e e e e e e e f}
%$A$ & $B$&$(A$&$\eif$&$B)$&$\eif$&$(\enot A$&$\eor$&$B)$ \\
%\hline
%T & T & T & T & T & \TTbf{T} & F & T & T \\
%T & F & T & F & F & \TTbf{T} & F & T & T\\
%F & T & F & F & T & \TTbf{T} & F & F & T\\
%F & F & F & T & F & \TTbf{T} & T & F & F & T & F
 %\end{tabular}
%\end{center}}
\item $(A \eif B) \eor (B \eif A)$ % taut

\myanswer{
\begin{center}
\begin{tabular}{c c | d e e e e e f}
$A$ & $B$&$(A$&$\eif$&$B)$&$\eor$&$(B$&$\eif$&$A)$ \\
\hline
T & T & T & T & T & \TTbf{T} & T & T & T\\
T & F & T & F & F & \TTbf{T} & F & T & T\\
F & T & F & T & T & \TTbf{T} & T & F & F \\
F & F & F & T & F & \TTbf{T} & F & T & F
 \end{tabular}
\end{center}}
\item $(A \eand B) \eif (B \eor A)$  %taut

\myanswer{
\begin{center}
\begin{tabular}{c c | d e e e e e f}
$A$ & $B$&$(A$&$\eand$&$B)$&$\eif$&$(B$&$\eor$&$A)$ \\
\hline
T & T & T & T & T & \TTbf{T} & T & T & T\\
T & F & T & F & F & \TTbf{T} & F & T & T\\
F & T & F & F & T & \TTbf{T} & T & T & F \\
F & F & F & F & F & \TTbf{T} & F & F & F
 \end{tabular}
\end{center}}
\item $\enot(A \eor B) \eif (\enot A \eand \enot B)$ %taut

\myanswer{\begin{center}
\begin{tabular}{c c | d e e e e e e e e f}
$A$ & $B$&$\enot$&$(A$&$\eor$&$B)$&$\eif$&$(\enot$&$A$&$\eand$&$\enot$&$B)$\\
\hline
T & T & F & T & T & T & \TTbf{T} & F & T & F & F & T\\
T & F & F & T& T & F & \TTbf{T} & F & T & F & T & F\\
F & T & F & F & T & T & \TTbf{T} & T & F & F & F & T\\
F & F & T & F & F & F & \TTbf{T} & T & F & T & T & F
 \end{tabular}
\end{center}}
\item $(\enot A \eand \enot B) \eif\enot(A \eor B) $ %taut

\myanswer{\begin{center}
\begin{tabular}{c c | d e e e e e e e e f}
$A$ & $B$&$\enot$&$(A$&$\eor$&$B)$&$\leftarrow$&$(\enot$&$A$&$\eand$&$\enot$&$B)$\\
\hline
T & T & F & T & T & T & \TTbf{T} & F & T & F & F & T\\
T & F & F & T& T & F & \TTbf{T} & F & T & F & T & F\\
F & T & F & F & T & T & \TTbf{T} & T & F & F & F & T\\
F & F & T & F & F & F & \TTbf{T} & T & F & T & T & F
 \end{tabular}
\end{center}}


\item $\bigl[(A\eand B) \eand\enot(A\eand B)\bigr] \eand C$ %contradiction

\myanswer{\begin{center}
\begin{tabular}{c c c | d e e e e e e e e f}
$A$ & $B$&$C$&$\bigl[(A$&$\eand$&$B)$&$ \eand$&$\enot$&$(A$&$\eand$&$B)\bigr]$&$\eand$&$C$\\
\hline
T & T & T & T & T & T & F & F & T & T & T & \TTbf{F} & T\\
T & T & F & T& T & T & F & F & T & T & T & \TTbf{F}& F\\
T & F & T & T & F & F & F & T & T & F & F & \TTbf{F} & T\\
T & F & F & T & F & F & F & T & T & F & F & \TTbf{F} & F\\
F & T & T & F & F & T & F & T & F & F & T & \TTbf{F} & T\\
F & T & F & F & F & T & F & T & F & F & T & \TTbf{F} & F\\
F & F & T & F & F & F & F & T & F & F & F & \TTbf{F} & T\\
F & F & F & F & F & F & F & T & F & F & F & \TTbf{F} & F
\end{tabular}
\end{center}}
\item $[(A \eand B) \eand C] \eif B$ %taut

\myanswer{\begin{center}
\begin{tabular}{c c c | d e e e e e f}
$A$ & $B$&$C$&$[(A$&$\eand$&$B)$&$\eand$&$C]$&$\eif$&$B$\\
\hline
T & T & T & T & T & T & T & T & \TTbf{T} & T\\
T & T & F & T & T & T & F & F & \TTbf{T} & T\\
T & F & T & T & F & F & F & T & \TTbf{T} & F\\
T & F & F & T & F & F & F & F & \TTbf{T} & F\\
F & T & T & F & F & T & F & T & \TTbf{T} & T\\
F & T & F & F & F & T & F & F & \TTbf{T} & T\\
F & F & T & F & F & F & F & T & \TTbf{T} & F\\
F & F & F & F & F & F & F & F & \TTbf{T} & F\\
\end{tabular}
\end{center}}
\item $\enot\bigl[(C\eor A) \eor B\bigr]$ %contingent

\myanswer{\begin{center}
\begin{tabular}{c c c | d e e e e f}
$A$ & $B$&$C$&$\enot\bigl[($&$C$&$\eor$&$A)$&$\eor$&$B\bigr]$\\
\hline
T & T & T & \TTbf{F} & T & T & T & T & T\\
T & T & F & \TTbf{F} & F & T & T & T & T\\
T & F & T & \TTbf{F} & T & T & T & T & F\\
T & F & F & \TTbf{F} & F & T & T & T & F\\
F & T & T & \TTbf{F} & T & T & F & T & T\\
F & T & F & \TTbf{F} & F & F & F & T & T\\
F & F & T & \TTbf{F} & T & T & F & T & F\\
F & F & F & \TTbf{T} & F & F & F & F & F
\end{tabular}
\end{center}}
\end{earg}


\problempart
Some brackets are redundant. Which ones? To find out check the claims below and eventually propose further conventions for omitting some brackets.\begin{earg}

	\item `$((A \eand B) \eand C)$' and `$(A \eand (B \eand C))$' have the same truth table
\myanswer{\begin{center}
\begin{tabular}{c c c | d e e e f | d e e e f }
$A$ & $B$ & $C$ & $(A$&$\eand$& $B)$ &$ \eand$ & $C$ & $A$ & $\eand$ & $(B$&$\eand$&$C)$\\
\hline
T & T & T & T & T & T &  \TTbf{T} & T &T & \TTbf{T} & T & T& T \\
T & T & F & T& T & T &  \TTbf{F} & F & T& \TTbf{F} & T & F& F\\
T & F & T & T & F & F &  \TTbf{F} & T & T & \TTbf{F} & F & F & T \\
T & F & F &  T & F & F &  \TTbf{F} & F & T& \TTbf{F} & F & F & F\\
F & T & T & F & F & T & \TTbf{F} & T & F& \TTbf{F} & T & T & T\\
F & T & F & F & F & T & \TTbf{F} & F & F& \TTbf{F} &  T & F & F\\
F & F & T & F & F & F & \TTbf{F} & T & F& \TTbf{F} & F& F & T\\
F & F & F & F & F & F & \TTbf{F} & F & F& \TTbf{F} &  F& F & F\\
\end{tabular}
\end{center}}

	\item `$((A \eor B) \eor C)$' and `$(A \eor (B \eor C))$' have the same truth table
\myanswer{\begin{center}
\begin{tabular}{c c c | d e e e f | d e e e f }
$A$ & $B$ & $C$ & $(A$&$\eor$& $B)$ &$ \eor$ & $C$ & $A$ & $\eor$ & $(B$&$\eor$&$C)$\\
\hline
T & T & T & T & T & T &  \TTbf{T} & T &T & \TTbf{T} & T & T& T \\
T & T & F & T& T & T &  \TTbf{T} & F & T& \TTbf{T} & T & T& F\\
T & F & T & T & T & F &  \TTbf{T} & T & T & \TTbf{T} & F & T & T \\
T & F & F &  T & T& F &  \TTbf{T} & F & T& \TTbf{T} & F & F & F\\
F & T & T & F & T & T & \TTbf{T} & T & F& \TTbf{T} & T & T & T\\
F & T & F & F & T & T & \TTbf{T} & F & F& \TTbf{T} &  T & T & F\\
F & F & T & F & F & F & \TTbf{T} & T & F& \TTbf{T} & F& T & T\\
F & F & F & F & F & F & \TTbf{F} & F & F& \TTbf{F} &  F& F & F\\
\end{tabular}
\end{center}}

	\item `$((A \eor B) \eand C)$' and `$(A \eor (B \eand C))$' do not have the same truth table
\myanswer{\begin{center}
\begin{tabular}{c c c | d e e e f | d e e e f }
$A$ & $B$ & $C$ & $(A$&$\eor$& $B)$ &$ \eand$ & $C$ & $A$ & $\eor$ & $(B$&$\eand$&$C)$\\
\hline
T & T & T & T & T & T &  \TTbf{T} & T &T & \TTbf{T} & T & T& T \\
T & T & F & T& T & T &  \TTbf{F} & F & T& \TTbf{T} & T & F& F\\
T & F & T & T & T & F &  \TTbf{T} & T & T & \TTbf{T} & F & F & T \\
T & F & F &  T & T& F &  \TTbf{F} & F & T& \TTbf{T} & F & F & F\\
F & T & T & F & T & T & \TTbf{T} & T & F& \TTbf{T} & T & T & T\\
F & T & F & F & T & T & \TTbf{F} & F & F& \TTbf{F} &  T & F & F\\
F & F & T & F & F & F & \TTbf{F} & T & F& \TTbf{F} & F& F & T\\
F & F & F & F & F & F & \TTbf{F} & F & F& \TTbf{F} &  F& F & F\\
\end{tabular}
\end{center}}

	\item `$((A \eif B) \eif C)$' and `$(A \eif (B \eif C))$' do not have the same truth table
\myanswer{\begin{center}
\begin{tabular}{c c c | d e e e f | d e e e f }
$A$ & $B$ & $C$ & $(A$&$\eif$& $B)$ &$ \eif$ & $C$ & $A$ & $\eif$ & $(B$&$\eif$&$C)$\\
\hline
T & T & T & T & T & T &  \TTbf{T} & T &T & \TTbf{T} & T & T& T \\
T & T & F & T& T & T &  \TTbf{F} & F & T& \TTbf{F} & T & F& F\\
T & F & T & T & F & F &  \TTbf{T} & T & T & \TTbf{T} & F & T & T \\
T & F & F &  T & F& F &  \TTbf{T} & F & T& \TTbf{T} & F & T & F\\
F & T & T & F & T & T & \TTbf{T} & T & F& \TTbf{T} & T & T & T\\
F & T & F & F & T & T & \TTbf{F} & F & F& \TTbf{T} &  T & F & F\\
F & F & T & F & T & F & \TTbf{T} & T & F& \TTbf{T} & F& T & T\\
F & F & F & F & T & F & \TTbf{F} & F & F& \TTbf{T} &  F& T & F\\
\end{tabular}
\end{center}}
\end{earg}

\problempart
Write complete truth tables for the following sentences and mark the column that represents the truth values for the whole sentence.

\begin{earg}


 \item $\enot [(X \eand Y) \eor (X \eor Y)]$

\myanswer{
\begin{center}
\begin{tabular}{c|c|ccccccc}
\cline{2-2}
~	&	\enot	&	 [(X 	&	\eand& 	Y) 	&	\eor 	&	(X 	&	\eor 	&	Y)] \\
\cline{2-9}
	&	F	&	T	&	T	&	T	&	T	&	T	&	T	&	T	\\
	&	F	&	T	&	F	&	F	&	T	&	T	&	T	&	F	\\
	&	F	&	F	&	F	&	T	&	T	&	F	&	T	&	T	\\
	&	T	&	F	&	F	&	F	&	F	&	F	&	F	&	F	\\
\cline{2-2}
\end{tabular}
\end{center}
}


\item $(A \eif B) \eif (\enot B\eif \enot A)$

\myanswer{
\begin{center}
\begin{tabular}{cccc|c|ccccc}
\cline{5-5}
~	&	(A 	&	\eif	&	B)	&	 \eif 	&	(\enot&	B 	&	\eif 	&	 \enot 	& 	 A) \\
\cline{2-10}
	&	T	&	T	&	T	&	T		&	F	 &	T	&	T	&	F		&	T	\\	
	&	T	&	F	&	F	&	T		&	T	 &	F	&	F	&	F		&	T	\\
	&	F	&	T	&	T	&	T		&	F	 &	T	&	T	&	T		&	F	\\
	&	F	&	T	&	F	&	T		&	T	 &	F	&	T	&	T		&	F	\\
\cline{5-5}
\end{tabular}
\end{center}
}

\item $[C \eiff (D \eor E)] \eand \enot C$

\myanswer{
\begin{center}
\begin{tabular}{cccccc|c|cc}
\cline{7-7}
~	&	[C 	&	\eiff 	&	(D 	&	\eor 	&	E)] 	&	\eand 	&	 \enot 	&	 C \\
\cline{2-9}
	&	T	&	T	&	T	&	T	&	T	&	F		&	F		&	T	\\
	&	T	&	T	&	T	&	T	&	F	&	F		&	F		&	T	\\
	&	T	&	T	&	F	&	T	&	T	&	F		&	F		&	T	\\
	&	T	&	F	&	F	&	F	&	F	&	F		&	F		&	T	\\
	&	F	&	F	&	T	&	T	&	T	&	F		&	T		&	F	\\
	&	F	&	F	&	T	&	T	&	F	&	F		&	T		&	F	\\
	&	F	&	F	&	F	&	T	&	T	&	F		&	T		&	F	\\
	&	F	&	T	&	F	&	F	&	F	&	T		&	T		&	F	\\
\cline{7-7}
\end{tabular}
\end{center}
}

\item $\enot(G \eand (B \eand H)) \eiff (G \eor (B \eor H))$

\myanswer{
\begin{center}
\begin{tabular}{ccccccc|c|ccccc}
\cline{8-8}
~	&\enot&	(G 	&\eand &	(B 	&	 \eand 	&	 H))	&	\eiff 	&	(G 	& \eor 	& (B 	& \eor	& H))	\\
\cline{2-13}
	&F	   &	T	&	  T &	T	&	T		&	T	&	F	&	T	&	T	&	T	&	T	&	T	\\
	&T	   &	T	&	  F &	T	&	F		&	F	&	T	&	T	&	T	&	T	&	T	&	F	\\	
	&T	   &	T	&	 F  &	F	&	F		&	T	&	T	&	T	&	T	&	F	&	T	&	T	\\
	&T	   &	T	&	 F  &	F	&	F		&	F	&	T	&	T	&	T	&	F	&	F	&	F	\\
	&T	   &	F	&	F   &	T	&	T		&	T	&	T	&	F	&	T	&	T	&	T	&	T	\\
	&T	   &	F	&	F   &	T	&	F		&	F	&	T	&	F	&	T	&	T	&	T	&	F	\\
	&T	   &	F	&	F   &	F	&	F		&	T	&	T	&	F	&	T	&	F	&	T	&	T	\\
	&T	   &	F	&	F   &	F	&	F		&	F	&	F	&	F	&	F	&	F	&	F	&	F	\\
\cline{8-8}
\end{tabular}
\end{center}
}

\vspace{1em}

\end{earg}

\problempart
Write complete truth tables for the following sentences and mark the column that represents the possible truth values for the whole sentence.

\begin{earg}

\item	$(D \eand \enot D) \eif G $


\myanswer{\vspace{1em}
\begin{center}
\begin{tabular}{ccccc|c|c}
\cline{6-6}
	&	(D 	&	 \eand 	& 	 \enot	&	 D) 	&	 \eif 	&	 G \\
 \hline
	&	T	&	F		&	F		&	T	&	T	&	T	\\
	&	T	&	F		&	F		&	T	&	T	&	F	\\
	&	F	&	F		&	T		&	F	&	T	&	T	\\
	&	F	&	F		&	T		&	F	&	T	&	F	\\
\cline{6-6}
\end{tabular}
\end{center}
}
\vspace{1em}


\item	$(\enot P \eor \enot M) \eiff M $

\myanswer{
\begin{center}
\begin{tabular}{cccccc|c|c}
\cline{7-7}
	&	(\enot 	&	P 	&	\eor 	&	\enot 	& 	 M) 	& 	\eiff 	&	 M \\
 \hline
	&	F		&	T	&	F	&	F		&	T	&	F	&	T	\\
	&	F		&	T	&	T	&	T		&	F	&	F	&	F	\\
	&	T		&	F	&	T	&	F		&	T	&	T	&	T	\\
	&	T		&	F	&	T	&	T		&	F	&	F	&	F	\\
\cline{7-7}
\end{tabular}
\end{center}
}
\vspace{1em}



\item	$\enot \enot (\enot A \eand \enot B)  $

\myanswer{
\begin{center}
\begin{tabular}{c|c|cccccc}
\cline{2-2}
	&	\enot		&	 \enot 	&	(\enot 	& 	 A 	& \eand 	& 	\enot 	&	 B)  \\
 \hline
	&	F		&	T		&	F		&	T	&	F	&	F		&	T	\\
	&	F		&	T		&	F		&	T	&	F	&	T		&	F	\\
	&	F		&	T		&	T		&	F	&	F	&	F		&	T	\\
	&	T		&	F		&	T		&	F	&	T	&	T		&	F	\\
\cline{2-2}
\end{tabular}
\end{center}
}
\vspace{1em}



\item 	$[(D \eand R) \eif I] \eif \enot(D \eor R) $

\myanswer{
\begin{center}
\begin{tabular}{cccccc|c|cccc}
\cline{7-7}
	&	[(D 	& 	 \eand 	& 	 R)	& 	\eif 	&	I] 	&	\eif 	&	 \enot 	&	(D 	&	 \eor 	& R) \\
	 \hline
	&	T	&	T		&	T	&	T	&	T	&	F	&	F		&	T	&	T		&T	\\
	&	T	&	T		&	T	&	F	&	F	&	T	&	F		&	T	&	T		&T	\\
	&	T	&	F		&	F	&	T	&	T	&	F	&	F		&	T	&	T		&F	\\
	&	T	&	F		&	F	&	T	&	F	&	F	&	F		&	T	&	T		&F	\\
	&	F	&	F		&	T	&	T	&	T	&	F	&	F		&	F	&	T		&T	\\
	&	F	&	F		&	T	&	T	&	F	&	F	&	F		&	F	&	T		&T	\\
	&	F	&	F		&	F	&	T	&	T	&	T	&	T		&	F	&	F		&F	\\
	&	F	&	F		&	F	&	T	&	F	&	T	&	T		&	F	&	F		&F	\\
\cline{7-7}
\end{tabular}
\end{center}
}
	
\vspace{1em}


\item	$\enot [(D \eiff O) \eiff A] \eif (\enot D \eand O) $

\myanswer{
\begin{center}
\begin{tabular}{ccccccc|c|cccc}
\cline{8-8}
	&	\enot 	&	[(D 	&	\eiff 	&	O) 	&	\eiff 	&	 A]	& 	\eif 	 &	(\enot 	& 	D 	 & 	 \eand &O) \\ 
	\hline
	&	F		&	T	&	T	&	T	&	T	&	T	&	T	&	F		&	T	&	F	&T	\\
	&	T		&	T	&	T	&	T	&	F	&	F	&	F	&	F		&	T	&	F	&T	\\
	&	T		&	T	&	F	&	F	&	F	&	T	&	F	&	F		&	T	&	F	&F	\\
	&	F		&	T	&	F	&	F	&	T	&	F	&	T	&	F		&	T	&	F	&F	\\
	&	T		&	F	&	F	&	T	&	F	&	T	&	T	&	T		&	F	&	T	&T	\\
	&	F		&	F	&	F	&	T	&	T	&	F	&	T	&	T		&	F	&	T	&T	\\
	&	F		&	F	&	T	&	F	&	T	&	T	&	T	&	T		&	F	&	F	&F	\\
	&	T		&	F	&	T	&	F	&	F	&	F	&	F	&	T		&	F	&	F	&F	\\
\cline{8-8}
\end{tabular}
\end{center}
}
\vspace{1em}
\end{earg}



\problempart
 Can you think of sentences with the following truth table:
\begin{enumerate}
\item \begin{tabular}{cc|c}
$A$&$B$&?\\\hline
T&T&T\\
T&F&T\\
F&T&T\\
F&F&T
\end{tabular}\myanswer{$A\eor\enot A$}
\item \begin{tabular}{cc|c}
$A$&$B$&?\\\hline
T&T&F\\
T&F&F\\
F&T&T\\
F&F&F
\end{tabular}\myanswer{$\enot A\eand B$}
\item \begin{tabular}{cc|c}
$A$&$B$&?\\\hline
T&T&T\\
T&F&F\\
F&T&T\\
F&F&T
\end{tabular}\myanswer{\quad\begin{minipage}{.75\textwidth}
$\enot( A\eand \enot B)$;\\ or the more systematic answer following G.2.: $$(A\eand B)\eor (\enot A\eand B)\eor (\enot A\eand\enot B)$$
\end{minipage}}

\end{enumerate}
\problempart
Suppose $X$ is TFL sentence containing two atomic sentences. Then there are in fact sixteen different possible columns for $X$.
\begin{enumerate}
\item Can you explain why?\myanswer{\\ The first line might have T or F, the second line might have T or F, etc. There are 4 valuations, so there are $2\times 2\times 2\times 2=2^4=16$ different ways of putting Ts and Fs to these 4 lines. I.e.~16 different columns. }
\item Can you show for each of these combinations that there is a sentence of TFL with that column describing its truth.
\myanswer{\\
See section 37.2 of forall$x$:Bristol for the general argument for this.
}
\item {} Can you show there's always a formula just using $\enot$ and $\eand$ with that column describing its truth.
\myanswer{\\
The previous answer had that every truth table can be given by a sentence with $\eand, \eor$ and $\enot$. We can replace any instances of $\eor$ by $\eand$ using: $$\metaX\eor\metaY\text{ is logically equivalent to }\enot(\enot\metaX\eand\enot\metaY)$$
See 37.4 of forall$x$:Bristol for more details.
}
\end{enumerate}

If you want additional practice, you can construct truth tables for any of the sentences and arguments in the exercises for the previous chapter.

\problempart
\label{pr.TT.valid}
\label{pr.TT.valid}
Use truth tables to determine whether each argument is valid or invalid.
\begin{earg}
\item $A\eif A \therefore A$  \hfill \myanswer{Invalid (see line 2)}
\myanswer{\begin{center}
\begin{tabular}{c | d e f | c}
$A$ &$A$&$\eif$&$A$&$A$\\
\hline
 T & T & \TTbf{T} & T & T\\
 F & F & \TTbf{T} & F & F
 \end{tabular}
\end{center}}
\item $A\eif(A\eand\enot A) \therefore \enot A$  \hfill \myanswer{Valid}
\myanswer{\begin{center}
\begin{tabular}{c | d e e e e f | df}
$A$&$A$&$\eif$&$(A$&$\eand$&$\enot$&$A)$&$\enot$&$A$\\
\hline
 T & T & \TTbf{F} & T & F& F&T&\TTbf{F}&T\\
 F & F & \TTbf{T} & F & F&T&F&\TTbf{T}&F
 \end{tabular}
\end{center}}
\item $A\eor(B\eif A) \therefore \enot A \eif \enot B$  \hfill \myanswer{Valid}
\myanswer{\begin{center}
\begin{tabular}{c c | d e e e f | d e e e f}
$A$ & $B$ & $A$&$\eor$&$(B$&$\eif$&$A)$&$\enot$&$A$&$\eif$&$\enot$&$B$\\
\hline
T & T & T & \TTbf{T} & T & T & T & F & T & \TTbf{T} & F & T \\
T & F & T & \TTbf{T} & F & T & T & F & T & \TTbf{T} & T & F \\
F & T & F & \TTbf{F} & T & F & F & T & F & \TTbf{F} & F & T \\
F & F & F & \TTbf{T} & F & T & F & T & F & \TTbf{T} & T & F
\end{tabular}
\end{center}}
\item $A\eor B, B\eor C, \enot A \therefore B \eand C$  \hfill \myanswer{Invalid (see line 6)}
\myanswer{\begin{center}
\begin{tabular}{c c c | d e f | d e f | d f | d e f}
$A$ & $B$ & $C$ & $A$&$\eor$&$B$&$B$&$\eor$&$C$&$\enot$&$A$&$B$&$\eand$&$C$\\
\hline
T & T & T & T & \TTbf{T} & T & T & \TTbf{T} & T & \TTbf{F} & T & T & \TTbf{T} & T \\
T & T & F & T & \TTbf{T} & T & T & \TTbf{T} & F & \TTbf{F} & T & T &\TTbf{F} & F \\
T & F & T & T & \TTbf{T} & F & F & \TTbf{T} & T & \TTbf{F} & T & F & \TTbf{F} & T \\
T & F & F & T & \TTbf{T} & F & F & \TTbf{F} & F & \TTbf{F} & T & F & \TTbf{F} & F\\
T & T & T & F & \TTbf{T} & T & T & \TTbf{T} & T & \TTbf{T} & F & T & \TTbf{T} & T \\
T & T & F & F & \TTbf{T} & T & T & \TTbf{T} & F & \TTbf{T} & F & T &\TTbf{F} & F \\
T & F & T & F & \TTbf{F} & F & F & \TTbf{T} & T & \TTbf{T} & F & F & \TTbf{F} & T \\
T & F & F & F & \TTbf{F} & F & F & \TTbf{F} & F & \TTbf{T} & F & F & \TTbf{F} & F
\end{tabular}
\end{center}}
\item $(B\eand A)\eif C, (C\eand A)\eif B \therefore (C\eand B)\eif A$  \hfill \myanswer{Invalid (see line 5)}
\myanswer{\begin{center}
\begin{tabular}{c c c | d e e e f | d e e e f | d e e e f}
$A$ & $B$ & $C$ & $(B$&$\eand$&$A)$&$\eif$&$C$&$(C$&$\eand$&$A)$&$\eif$&$B$&$(C$&$\eand$&$ B)$&$\eif$&$A$\\
\hline
T & T & T & T & T & T & \TTbf{T} & T & T & T & T & \TTbf{T} & T & T & T & T & \TTbf{T} & T\\
T & T & F & T & T & T & \TTbf{F} & F & F & F & T & \TTbf{T} & T & F & F & T & \TTbf{T} & T\\
T & F & T & F & F & T & \TTbf{T} & T & T & T & T & \TTbf{F} & F & T & F & F & \TTbf{T} & T\\
T & F & F & F & F & T & \TTbf{T} & F & F & F & T & \TTbf{T} & F & F & F & F & \TTbf{T} & T\\
F & T & T & T & F & F & \TTbf{T} & T & T & F & F & \TTbf{T} & T & T & T & T & \TTbf{F} & F\\
F & T & F & T & F & F & \TTbf{T} & F & F & F & F & \TTbf{T} & T & F & F & T & \TTbf{T} & F\\
F & F & T & F & F & F & \TTbf{T} & T & T & F & F & \TTbf{T} & F & T & F & F & \TTbf{T} & F\\
F & F & F & F & F & F & \TTbf{T} & F & F & F & F & \TTbf{T} & F & F & F & F & \TTbf{T} & F
\end{tabular}
\end{center}}
\end{earg}

\problempart Determine whether each sentence is a tautology, a contradiction, or a contingent sentence, using a complete truth table.
\begin{earg}
\item $\enot B \eand B$ \vspace{.5ex} \hfill \myanswer{Contradiction}


\item $\enot D \eor D$ \vspace{.5ex} \hfill \myanswer{Tautology}


\item $(A\eand B) \eor (B\eand A)$\vspace{.5ex} \hfill \myanswer{Contingent}


\item $\enot[A \eif (B \eif A)]$\vspace{.5ex} \hfill \myanswer{Contradiction}


\item $A \eiff [A \eif (B \eand \enot B)]$ \vspace{.5ex} \hfill \myanswer{Contradiction}


\item $[(A \eand B) \eiff B] \eif (A \eif B)$ \vspace{.5ex} \hfill \myanswer{Contingent}


\end{earg}

\noindent\problempart
\label{pr.TT.equiv}
Determine whether each the following sentences are logically equivalent using complete truth tables. If the two sentences really are logically equivalent, write ``equivalent.'' Otherwise write, ``Not equivalent.'' 
\begin{earg}
\item $A$ and $\enot A$
\item $A \eand \enot A$ and $\enot B \eiff B$
\item $[(A \eor B) \eor C]$ and $[A \eor (B \eor C)]$
\item $A \eor (B \eand C)$ and $(A \eor B) \eand (A \eor C)$
\item $[A \eand (A \eor B)] \eif B$ and $A \eif B$\end{earg}


\problempart
\label{pr.TT.equiv2}
Determine whether each the following sentences are logically equivalent using complete truth tables. If the two sentences really are equivalent, write ``equivalent.'' Otherwise write, ``not equivalent.''
\begin{earg}
\item $A\eif A$ and $A \eiff A$
\item $\enot(A \eif B)$ and $\enot A \eif \enot B$
\item $A \eor B$ and $\enot A \eif B$
\item$(A \eif B) \eif C$ and $A \eif (B \eif C)$
\item $A \eiff (B \eiff C)$ and $A \eand (B \eand C)$
\end{earg}

\problempart
\label{pr.TT.satisfiable2}
Determine whether each collection of sentences is jointly satisfiable or jointly unsatisfiable using a complete truth table.

\begin{earg}

\item $A \eand \enot B$, $\enot(A \eif B)$, $B \eif A$ %Consistent

\myanswer{
\begin{center}
\begin{tabular}{ccccccccccccccc} 
~ 	&	A 	& \eand	&  \enot & B &  & \enot &  (A &  \eif & B)	 & 	 & 	 B	 & 	\eif  & A  & Consistent \\ 
\cline{2-5} \cline{7-10}\cline{12-14} 
	& 	T   & F     &   F	 & T &  &  F	& 	T &   T	  & T 	 & 	 & 	 T	 & 	 T	  & T  &	  \\ 
\cline{2-14}
	& \multicolumn{1}{|r}{T}& 	\textbf{T}	 & T	 & F & & \textbf{T}	 & 	 T	 & 	 F	 	 & 	 F	 	 & 	 & 	 F	 	 & 	 \textbf{T}	 	 & 	 \multicolumn{1}{r|}{T}	 	 & 	  \\ 
\cline{2-14}
	& 	 F	 				 & 	 F	 & 	 F	 & T & 	& 	 F	 & 	 F	 & 	 T	 	 & 	 T	 	 & 	  & 	 T	 	 & 	 F	 	 & 	 F	 	 & 	  \\ 
	& 	 F	  				& 	 F	 & 	 T	 & 	F&  & 	 F	 & 	 F	 & 	 T	 	 & 	 F	 	 & 	  & 	 F	 	 & 	 T	 	 & 	 F	 	 & 	  \\ 
\end{tabular}
\end{center}
}

\item $A \eor B$, $A \eif \enot A$, $B \eif \enot B$ %unsatisfiable.

\myanswer{
\begin{center}
\begin{tabular}{ccccccccccccccc} 
  & A	 & \eor 	 & B 	 & 	 	 & A 	 & \eif 	 & 	\enot & A 	 & 	 	 & B 	 & \eif 	 & \enot	 & 	B 	 & 	Insatisfiable \\ 
\cline{2-4}\cline{6- 9} \cline{11-14}
   &	 T	 & 	 T	 &T  	 & 	 	 & T	 & 	 F	 & 	F 	 & T 	 & 	 	 & 	T 	 & 	F 	 & 	 F	 & 	T 	 & 	 \\ 
   &	 T	& 	 T	 & F 	 & 	 	 & 	T 	 & 	 F	 & 	 F	 & 	 T	 & 	 	 & 	F 	 & 	 T	 & 	 T	 & 	 F	 & 	 \\ 
   &	 F	& 	 T	 & 	 T	 & 	 	 & 	F 	 & 	 T	 & 	 T	 & 	F 	 & 	 	 & 	 T	 & 	 F	 & 	 F	 & 	 T	 & 	 \\ 
   &	 F	& 	 F	 & 	 F	 & 	 	 & 	 F	 & 	 T	 & 	 T	 & 	 F	 & 	 	 & 	 F	 & 	 T	 & 	 T	 & 	 F	 & 	 \\ 
\end{tabular}
\end{center}
}

\item $\enot(\enot A \eor B) $, $A \eif \enot C$, $A \eif (B \eif C)$ \hfill \myanswer{Consistent}

\myanswer{
\begin{center}
\begin{tabular}{ccccccccccccccccc}
   \enot & (\enot & A & \eor & B) &  & A  & \eif 	 & \enot 	 & C & 	 & A & \eif 	& (B & \eif & C) &  \\ 
 \cline{1-5}\cline{7-10} \cline{12-16} 
	F 	& 	F	 & 	T & T	 & T & 	  & T & F	 & 	 F & T 	 & 	 & T & T	 & T	 & T 	 & T 	 & \\ 
   	 F	& 	F	 & 	T & T	 & T & 	  & T & T	 & 	 T & F	 & 	 & T & F	 & T	 & F	 & F 	 & \\ 
   	 T & 	F 	& 	T & F	 & F & 	  & T & F	 & 	 F & T	 & 	 & T & T	 & F	 & T	 & T 	 & \\ 
\cline{1-16}
   	 \multicolumn{1}{|r}{\TTbf{T}}		&  F	 & 	T & F	 & 	F &  & 	T & \TTbf{T}	 & 	 T & F 	& 	 & T & \TTbf{T}	 & F	 & T	 & \multicolumn{1}{r|}{F} 	 & \\ 
\cline{1-16}
   	 F	& 	T	 & 	F & T	 & 	T &  & 	F & T	 & 	 F & T	 & 	 & F	 & F	 & T	 & T	 & T 	 & \\ 
   	 F	& 	 T	& 	F & T	 & 	T &  & 	F & T	 & 	T & F 	& 	 & F	 & T	 & T	 & F 	 & F 	 & \\ 
   	 F	& 	 T	& 	F & T	 & 	F &  & 	F & T	 & 	F & T	 & 	 & F	 & T	 & F	 & T	 & T 	 & \\ 
   	 F	& 	 T	& 	F & T	 & 	F &  & 	F & T	 & 	T & F	 & 	 & F	 & T	 & F	 & T	 & F 	 & \\ 
\end{tabular}
\end{center}
}



\item $A \eif B$, $A \eand \enot B$ \hfill \myanswer{Insatisfiable}

\item $A \eif (B \eif C)$, $(A \eif B) \eif C$, $A \eif C$ \hfill \myanswer{ Consistent} 

\end{earg}

\noindent\problempart
\label{pr.TT.satisfiable3}
Determine whether each collection of sentences is consistent or inconsistent, using a complete truth table.
\begin{earg}
\item $\enot B$, $A \eif B$, $A$ \vspace{.5ex} \hfill \myanswer{Insatisfiable}
\item $\enot(A \eor B)$, $A \eiff B$, $B \eif A$\vspace{.5ex} \hfill \myanswer{Consistent}
\item $A \eor B$, $\enot B$, $\enot B \eif \enot A$\vspace{.5ex} \hfill \myanswer{Insatisfiable}
\item $A \eiff B$, $\enot B \eor \enot A$, $A \eif B$\vspace{.5ex} \hfill \myanswer{Consistent} 
\item $(A \eor B) \eor C$, $\enot A \eor \enot B$, $\enot C \eor \enot B$\vspace{.5ex} \hfill \myanswer{Consistent}
\end{earg}

\noindent\problempart
\label{pr.TT.valid2}
Determine whether each argument is valid or invalid, using a complete truth table.
\begin{earg}
\item $A\eif B$, $B \therefore  A$ \hfill \myanswer{Invalid}

\item $A\eiff B$, $B\eiff C \therefore A\eiff C$ \hfill \myanswer{Valid}

\item $A \eif B$, $A \eif C\therefore B \eif C$ \hfill \myanswer{Invalid}.

\item $A \eif B$, $B \eif A\therefore A \eiff B$ \hfill \myanswer{Valid} 
\end{earg}

\noindent\problempart
\label{pr.TT.valid3}
Determine whether each argument is valid or invalid, using a complete truth table.
\begin{earg}
\item $A\eor\bigl[A\eif(A\eiff A)\bigr] \therefore  A $\vspace{.5ex} \hfill \myanswer{Invalid}
\item $A\eor B$, $B\eor C$, $\enot B \therefore A \eand C$\vspace{.5ex} \hfill \myanswer{Valid}
\item $A \eif B$, $\enot A\therefore \enot B$ \vspace{.5ex} \hfill \myanswer{Invalid}
\item $A$, $B\therefore \enot(A\eif \enot B)$ \vspace{.5ex} \hfill \myanswer{Valid}
\item $\enot(A \eand B)$, $A \eor B$, $A \eiff B\therefore C$ \vspace{.5ex} \hfill \myanswer{Valid}
\end{earg}

\noindent\problempart
\label{pr.TT.meta}
Are the following statements true? Why?
\begin{itemize}
\item if $Y$ is a logical consequence of $X$, then $X\rightarrow Y$ is a logical truth.\myanswer{Yes}
\item if $X\rightarrow Y$ is a logical truth, then $Y$ is a logical consequence of $X$.\myanswer{Yes}
\item if $X\rightarrow Y\wedge\neg Y$ is a logical truth, then $X$ is a logical contradiction.\myanswer{Yes}
\item if $X\rightarrow Y\wedge\neg Y$ is true relative to a valuation, then $X$ is a logical contradiction.\myanswer{No}
\item if $X\vee\neg X\rightarrow Y$ is a logical truth, then $Y$ is a tautology.\myanswer{Yes}
\item if $X\vee\neg X\rightarrow Y$ is true relative to a valuation, then $Y$ is a tautology.\myanswer{No}
\end{itemize}
\end{practiceproblems}




